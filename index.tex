% Options for packages loaded elsewhere
% Options for packages loaded elsewhere
\PassOptionsToPackage{unicode}{hyperref}
\PassOptionsToPackage{hyphens}{url}
\PassOptionsToPackage{dvipsnames,svgnames,x11names}{xcolor}
%
\documentclass[
  letterpaper,
  DIV=11,
  numbers=noendperiod]{scrreprt}
\usepackage{xcolor}
\usepackage{amsmath,amssymb}
\setcounter{secnumdepth}{5}
\usepackage{iftex}
\ifPDFTeX
  \usepackage[T1]{fontenc}
  \usepackage[utf8]{inputenc}
  \usepackage{textcomp} % provide euro and other symbols
\else % if luatex or xetex
  \usepackage{unicode-math} % this also loads fontspec
  \defaultfontfeatures{Scale=MatchLowercase}
  \defaultfontfeatures[\rmfamily]{Ligatures=TeX,Scale=1}
\fi
\usepackage{lmodern}
\ifPDFTeX\else
  % xetex/luatex font selection
\fi
% Use upquote if available, for straight quotes in verbatim environments
\IfFileExists{upquote.sty}{\usepackage{upquote}}{}
\IfFileExists{microtype.sty}{% use microtype if available
  \usepackage[]{microtype}
  \UseMicrotypeSet[protrusion]{basicmath} % disable protrusion for tt fonts
}{}
\makeatletter
\@ifundefined{KOMAClassName}{% if non-KOMA class
  \IfFileExists{parskip.sty}{%
    \usepackage{parskip}
  }{% else
    \setlength{\parindent}{0pt}
    \setlength{\parskip}{6pt plus 2pt minus 1pt}}
}{% if KOMA class
  \KOMAoptions{parskip=half}}
\makeatother
% Make \paragraph and \subparagraph free-standing
\makeatletter
\ifx\paragraph\undefined\else
  \let\oldparagraph\paragraph
  \renewcommand{\paragraph}{
    \@ifstar
      \xxxParagraphStar
      \xxxParagraphNoStar
  }
  \newcommand{\xxxParagraphStar}[1]{\oldparagraph*{#1}\mbox{}}
  \newcommand{\xxxParagraphNoStar}[1]{\oldparagraph{#1}\mbox{}}
\fi
\ifx\subparagraph\undefined\else
  \let\oldsubparagraph\subparagraph
  \renewcommand{\subparagraph}{
    \@ifstar
      \xxxSubParagraphStar
      \xxxSubParagraphNoStar
  }
  \newcommand{\xxxSubParagraphStar}[1]{\oldsubparagraph*{#1}\mbox{}}
  \newcommand{\xxxSubParagraphNoStar}[1]{\oldsubparagraph{#1}\mbox{}}
\fi
\makeatother

\usepackage{color}
\usepackage{fancyvrb}
\newcommand{\VerbBar}{|}
\newcommand{\VERB}{\Verb[commandchars=\\\{\}]}
\DefineVerbatimEnvironment{Highlighting}{Verbatim}{commandchars=\\\{\}}
% Add ',fontsize=\small' for more characters per line
\usepackage{framed}
\definecolor{shadecolor}{RGB}{241,243,245}
\newenvironment{Shaded}{\begin{snugshade}}{\end{snugshade}}
\newcommand{\AlertTok}[1]{\textcolor[rgb]{0.68,0.00,0.00}{#1}}
\newcommand{\AnnotationTok}[1]{\textcolor[rgb]{0.37,0.37,0.37}{#1}}
\newcommand{\AttributeTok}[1]{\textcolor[rgb]{0.40,0.45,0.13}{#1}}
\newcommand{\BaseNTok}[1]{\textcolor[rgb]{0.68,0.00,0.00}{#1}}
\newcommand{\BuiltInTok}[1]{\textcolor[rgb]{0.00,0.23,0.31}{#1}}
\newcommand{\CharTok}[1]{\textcolor[rgb]{0.13,0.47,0.30}{#1}}
\newcommand{\CommentTok}[1]{\textcolor[rgb]{0.37,0.37,0.37}{#1}}
\newcommand{\CommentVarTok}[1]{\textcolor[rgb]{0.37,0.37,0.37}{\textit{#1}}}
\newcommand{\ConstantTok}[1]{\textcolor[rgb]{0.56,0.35,0.01}{#1}}
\newcommand{\ControlFlowTok}[1]{\textcolor[rgb]{0.00,0.23,0.31}{\textbf{#1}}}
\newcommand{\DataTypeTok}[1]{\textcolor[rgb]{0.68,0.00,0.00}{#1}}
\newcommand{\DecValTok}[1]{\textcolor[rgb]{0.68,0.00,0.00}{#1}}
\newcommand{\DocumentationTok}[1]{\textcolor[rgb]{0.37,0.37,0.37}{\textit{#1}}}
\newcommand{\ErrorTok}[1]{\textcolor[rgb]{0.68,0.00,0.00}{#1}}
\newcommand{\ExtensionTok}[1]{\textcolor[rgb]{0.00,0.23,0.31}{#1}}
\newcommand{\FloatTok}[1]{\textcolor[rgb]{0.68,0.00,0.00}{#1}}
\newcommand{\FunctionTok}[1]{\textcolor[rgb]{0.28,0.35,0.67}{#1}}
\newcommand{\ImportTok}[1]{\textcolor[rgb]{0.00,0.46,0.62}{#1}}
\newcommand{\InformationTok}[1]{\textcolor[rgb]{0.37,0.37,0.37}{#1}}
\newcommand{\KeywordTok}[1]{\textcolor[rgb]{0.00,0.23,0.31}{\textbf{#1}}}
\newcommand{\NormalTok}[1]{\textcolor[rgb]{0.00,0.23,0.31}{#1}}
\newcommand{\OperatorTok}[1]{\textcolor[rgb]{0.37,0.37,0.37}{#1}}
\newcommand{\OtherTok}[1]{\textcolor[rgb]{0.00,0.23,0.31}{#1}}
\newcommand{\PreprocessorTok}[1]{\textcolor[rgb]{0.68,0.00,0.00}{#1}}
\newcommand{\RegionMarkerTok}[1]{\textcolor[rgb]{0.00,0.23,0.31}{#1}}
\newcommand{\SpecialCharTok}[1]{\textcolor[rgb]{0.37,0.37,0.37}{#1}}
\newcommand{\SpecialStringTok}[1]{\textcolor[rgb]{0.13,0.47,0.30}{#1}}
\newcommand{\StringTok}[1]{\textcolor[rgb]{0.13,0.47,0.30}{#1}}
\newcommand{\VariableTok}[1]{\textcolor[rgb]{0.07,0.07,0.07}{#1}}
\newcommand{\VerbatimStringTok}[1]{\textcolor[rgb]{0.13,0.47,0.30}{#1}}
\newcommand{\WarningTok}[1]{\textcolor[rgb]{0.37,0.37,0.37}{\textit{#1}}}

\usepackage{longtable,booktabs,array}
\usepackage{calc} % for calculating minipage widths
% Correct order of tables after \paragraph or \subparagraph
\usepackage{etoolbox}
\makeatletter
\patchcmd\longtable{\par}{\if@noskipsec\mbox{}\fi\par}{}{}
\makeatother
% Allow footnotes in longtable head/foot
\IfFileExists{footnotehyper.sty}{\usepackage{footnotehyper}}{\usepackage{footnote}}
\makesavenoteenv{longtable}
\usepackage{graphicx}
\makeatletter
\newsavebox\pandoc@box
\newcommand*\pandocbounded[1]{% scales image to fit in text height/width
  \sbox\pandoc@box{#1}%
  \Gscale@div\@tempa{\textheight}{\dimexpr\ht\pandoc@box+\dp\pandoc@box\relax}%
  \Gscale@div\@tempb{\linewidth}{\wd\pandoc@box}%
  \ifdim\@tempb\p@<\@tempa\p@\let\@tempa\@tempb\fi% select the smaller of both
  \ifdim\@tempa\p@<\p@\scalebox{\@tempa}{\usebox\pandoc@box}%
  \else\usebox{\pandoc@box}%
  \fi%
}
% Set default figure placement to htbp
\def\fps@figure{htbp}
\makeatother


% definitions for citeproc citations
\NewDocumentCommand\citeproctext{}{}
\NewDocumentCommand\citeproc{mm}{%
  \begingroup\def\citeproctext{#2}\cite{#1}\endgroup}
\makeatletter
 % allow citations to break across lines
 \let\@cite@ofmt\@firstofone
 % avoid brackets around text for \cite:
 \def\@biblabel#1{}
 \def\@cite#1#2{{#1\if@tempswa , #2\fi}}
\makeatother
\newlength{\cslhangindent}
\setlength{\cslhangindent}{1.5em}
\newlength{\csllabelwidth}
\setlength{\csllabelwidth}{3em}
\newenvironment{CSLReferences}[2] % #1 hanging-indent, #2 entry-spacing
 {\begin{list}{}{%
  \setlength{\itemindent}{0pt}
  \setlength{\leftmargin}{0pt}
  \setlength{\parsep}{0pt}
  % turn on hanging indent if param 1 is 1
  \ifodd #1
   \setlength{\leftmargin}{\cslhangindent}
   \setlength{\itemindent}{-1\cslhangindent}
  \fi
  % set entry spacing
  \setlength{\itemsep}{#2\baselineskip}}}
 {\end{list}}
\usepackage{calc}
\newcommand{\CSLBlock}[1]{\hfill\break\parbox[t]{\linewidth}{\strut\ignorespaces#1\strut}}
\newcommand{\CSLLeftMargin}[1]{\parbox[t]{\csllabelwidth}{\strut#1\strut}}
\newcommand{\CSLRightInline}[1]{\parbox[t]{\linewidth - \csllabelwidth}{\strut#1\strut}}
\newcommand{\CSLIndent}[1]{\hspace{\cslhangindent}#1}



\setlength{\emergencystretch}{3em} % prevent overfull lines

\providecommand{\tightlist}{%
  \setlength{\itemsep}{0pt}\setlength{\parskip}{0pt}}



 


\usepackage{booktabs}
\usepackage{caption}
\usepackage{longtable}
\usepackage{colortbl}
\usepackage{array}
\usepackage{anyfontsize}
\usepackage{multirow}
\usepackage{wrapfig}
\usepackage{float}
\usepackage{pdflscape}
\usepackage{tabu}
\usepackage{threeparttable}
\usepackage{threeparttablex}
\usepackage[normalem]{ulem}
\usepackage{makecell}
\usepackage{xcolor}
\KOMAoption{captions}{tableheading}
\makeatletter
\@ifpackageloaded{bookmark}{}{\usepackage{bookmark}}
\makeatother
\makeatletter
\@ifpackageloaded{caption}{}{\usepackage{caption}}
\AtBeginDocument{%
\ifdefined\contentsname
  \renewcommand*\contentsname{Table of contents}
\else
  \newcommand\contentsname{Table of contents}
\fi
\ifdefined\listfigurename
  \renewcommand*\listfigurename{List of Figures}
\else
  \newcommand\listfigurename{List of Figures}
\fi
\ifdefined\listtablename
  \renewcommand*\listtablename{List of Tables}
\else
  \newcommand\listtablename{List of Tables}
\fi
\ifdefined\figurename
  \renewcommand*\figurename{Figure}
\else
  \newcommand\figurename{Figure}
\fi
\ifdefined\tablename
  \renewcommand*\tablename{Table}
\else
  \newcommand\tablename{Table}
\fi
}
\@ifpackageloaded{float}{}{\usepackage{float}}
\floatstyle{ruled}
\@ifundefined{c@chapter}{\newfloat{codelisting}{h}{lop}}{\newfloat{codelisting}{h}{lop}[chapter]}
\floatname{codelisting}{Listing}
\newcommand*\listoflistings{\listof{codelisting}{List of Listings}}
\makeatother
\makeatletter
\makeatother
\makeatletter
\@ifpackageloaded{caption}{}{\usepackage{caption}}
\@ifpackageloaded{subcaption}{}{\usepackage{subcaption}}
\makeatother
\usepackage{bookmark}
\IfFileExists{xurl.sty}{\usepackage{xurl}}{} % add URL line breaks if available
\urlstyle{same}
\hypersetup{
  pdftitle={STP-BRIOCHE : A priori MIPD - Amoxicillin},
  colorlinks=true,
  linkcolor={blue},
  filecolor={Maroon},
  citecolor={Blue},
  urlcolor={Blue},
  pdfcreator={LaTeX via pandoc}}


\title{STP-BRIOCHE : A priori MIPD - Amoxicillin}
\author{}
\date{2026-02-12}
\begin{document}
\maketitle

\renewcommand*\contentsname{Table of contents}
{
\hypersetup{linkcolor=}
\setcounter{tocdepth}{2}
\tableofcontents
}

\bookmarksetup{startatroot}

\chapter*{Preface}\label{preface}
\addcontentsline{toc}{chapter}{Preface}

\markboth{Preface}{Preface}

This book contains all that is related to the a priori MIPD developped
in STP-BRIOCHE

\part{Validation of implementation from articles}

\chapter{Validation PKpop Amox Carlier
2013}\label{validation-pkpop-amox-carlier-2013}

\chapter{Paper}\label{paper}

(Carlier et al. 2013)

\chapter{Model description}\label{model-description}

The same model describe both amoxicillin and clavulanic acid
concentrations.

\begin{longtable}[]{@{}
  >{\raggedright\arraybackslash}p{(\linewidth - 12\tabcolsep) * \real{0.1218}}
  >{\raggedright\arraybackslash}p{(\linewidth - 12\tabcolsep) * \real{0.1090}}
  >{\raggedright\arraybackslash}p{(\linewidth - 12\tabcolsep) * \real{0.1346}}
  >{\raggedright\arraybackslash}p{(\linewidth - 12\tabcolsep) * \real{0.2179}}
  >{\raggedleft\arraybackslash}p{(\linewidth - 12\tabcolsep) * \real{0.1218}}
  >{\raggedright\arraybackslash}p{(\linewidth - 12\tabcolsep) * \real{0.1859}}
  >{\raggedleft\arraybackslash}p{(\linewidth - 12\tabcolsep) * \real{0.1090}}@{}}

\caption{\label{tbl-model-desc}Model description}

\tabularnewline

\toprule\noalign{}
\begin{minipage}[b]{\linewidth}\raggedright
DOI
\end{minipage} & \begin{minipage}[b]{\linewidth}\raggedright
Structural model
\end{minipage} & \begin{minipage}[b]{\linewidth}\raggedright
Variability model
\end{minipage} & \begin{minipage}[b]{\linewidth}\raggedright
Covariates effects
\end{minipage} & \begin{minipage}[b]{\linewidth}\raggedleft
Number of patients
\end{minipage} & \begin{minipage}[b]{\linewidth}\raggedright
Free or total concentrations
\end{minipage} & \begin{minipage}[b]{\linewidth}\raggedleft
Unbound fraction
\end{minipage} \\
\midrule\noalign{}
\endhead
\bottomrule\noalign{}
\endlastfoot
10.1093/jac/dkt240 & 2-comp & log-normal clearance & creatinine
clearance on clearance & 13 & free & 0.83 \\

\end{longtable}

\section{Model parameters}\label{model-parameters}

\begin{longtable}[]{@{}
  >{\raggedright\arraybackslash}p{(\linewidth - 6\tabcolsep) * \real{0.1172}}
  >{\raggedright\arraybackslash}p{(\linewidth - 6\tabcolsep) * \real{0.7734}}
  >{\raggedright\arraybackslash}p{(\linewidth - 6\tabcolsep) * \real{0.0703}}
  >{\raggedleft\arraybackslash}p{(\linewidth - 6\tabcolsep) * \real{0.0391}}@{}}

\caption{\label{tbl-model-param}Model parameters for amoxicillin}

\tabularnewline

\toprule\noalign{}
\begin{minipage}[b]{\linewidth}\raggedright
parameter\_name
\end{minipage} & \begin{minipage}[b]{\linewidth}\raggedright
parameter\_description
\end{minipage} & \begin{minipage}[b]{\linewidth}\raggedright
unit
\end{minipage} & \begin{minipage}[b]{\linewidth}\raggedleft
mean
\end{minipage} \\
\midrule\noalign{}
\endhead
\bottomrule\noalign{}
\endlastfoot
CLpop & Typical clearance (for a patient with creatinin clearance of 102
mL/min) & L/h & 10.0 \\
Q & Intercompartmental clearance & L/h & 15.6 \\
Vcpop & Typical central volume of distribution (for a patient with
creatinin clearance of 102 mL/min) & L & 13.7 \\
Vp & Peripheral volume of distribution & L & 13.7 \\
cv\_iiv\_CL & Coefficient of variation of the inter individual
variability on clearance (\%) & unitless & 39.9 \\
cv\_iiv\_Vc & Coefficient of variation of the inter individual
variability on central volume of distribution (\%) & unitless & 38.7 \\
ruv & Coefficient of variation of the residual variability & unitless &
22.0 \\

\end{longtable}

\section{Inter-individual variability and covariate
effects}\label{inter-individual-variability-and-covariate-effects}

Covariate effect : \[
TVCL_{i} = CLpop*\frac{CLCR_{i}}{102}*e^{\eta_{i}}
\] With :\\
\(TVCL_{i}\) : Amoxicillin clearance for individual \emph{i}\\
\(CLpop\) : Typical amoxicillin clearance (L/h)\\
\(CLCR_{i}\) : 24h urinary creatinin clearance (mL/min) for individual
\emph{i}\\
\(\eta_{i}\) : Normal variable with mean 0 and variance
\(\omega^2_{CL}\)\\
\(102\) : population's median urinary creatinine clearance in mL/min

\chapter{Validation of the mrgsolve implementation of the amoxicillin
model}\label{validation-of-the-mrgsolve-implementation-of-the-amoxicillin-model}

To validate that our implementation of the model is correct we will
attempt to reproduce figure 2 and table 5 of the original article.

\section{Preliminary testing}\label{preliminary-testing}

First we check that simulations for a single dosing regimen and
creatinine clearance value make sense.

\pandocbounded{\includegraphics[keepaspectratio]{Model_implementation_validation/quarto/valid_amox_carlier_2013_files/figure-pdf/perform_pta_sim_clcr_30_ci_6g_q24h-1.pdf}}

\begin{longtable}[]{@{}rrrrr@{}}
\toprule\noalign{}
MIC & mean\_ft\_over\_mic & sd\_ft\_over\_mic & median\_ft\_over\_mic &
PTA\_0.5 \\
\midrule\noalign{}
\endhead
\bottomrule\noalign{}
\endlastfoot
4 & 1 & NA & 1 & 1 \\
\end{longtable}

Looks ok

\section{Table 5}\label{table-5}

This table indicate wether the target is attained for a typical patient
receiving amoxicillin, for various dosing regimen and various creatinin
clearances.

\begin{longtable}[]{@{}
  >{\raggedright\arraybackslash}p{(\linewidth - 18\tabcolsep) * \real{0.1207}}
  >{\raggedright\arraybackslash}p{(\linewidth - 18\tabcolsep) * \real{0.1466}}
  >{\raggedright\arraybackslash}p{(\linewidth - 18\tabcolsep) * \real{0.1121}}
  >{\raggedleft\arraybackslash}p{(\linewidth - 18\tabcolsep) * \real{0.0431}}
  >{\raggedleft\arraybackslash}p{(\linewidth - 18\tabcolsep) * \real{0.1897}}
  >{\raggedleft\arraybackslash}p{(\linewidth - 18\tabcolsep) * \real{0.0862}}
  >{\raggedleft\arraybackslash}p{(\linewidth - 18\tabcolsep) * \real{0.1983}}
  >{\raggedleft\arraybackslash}p{(\linewidth - 18\tabcolsep) * \real{0.0259}}
  >{\raggedleft\arraybackslash}p{(\linewidth - 18\tabcolsep) * \real{0.0431}}
  >{\raggedleft\arraybackslash}p{(\linewidth - 18\tabcolsep) * \real{0.0345}}@{}}

\caption{\label{tbl-sim-dosing-reg}Simulated dosing regimen}

\tabularnewline

\toprule\noalign{}
\begin{minipage}[b]{\linewidth}\raggedright
Infusion type
\end{minipage} & \begin{minipage}[b]{\linewidth}\raggedright
Loading dose ?
\end{minipage} & \begin{minipage}[b]{\linewidth}\raggedright
Dose regimen
\end{minipage} & \begin{minipage}[b]{\linewidth}\raggedleft
time
\end{minipage} & \begin{minipage}[b]{\linewidth}\raggedleft
Infusion duration (h)
\end{minipage} & \begin{minipage}[b]{\linewidth}\raggedleft
Dose (mg)
\end{minipage} & \begin{minipage}[b]{\linewidth}\raggedleft
Interdose interval (h)
\end{minipage} & \begin{minipage}[b]{\linewidth}\raggedleft
ss
\end{minipage} & \begin{minipage}[b]{\linewidth}\raggedleft
evid
\end{minipage} & \begin{minipage}[b]{\linewidth}\raggedleft
cmt
\end{minipage} \\
\midrule\noalign{}
\endhead
\bottomrule\noalign{}
\endlastfoot
intermittent & no & ii 0.5g q4h & 0.0 & 0.5 & 500 & 4 & 0 & 1 & 1 \\
intermittent & no & ii 0.5g q6h & 0.0 & 0.5 & 500 & 6 & 0 & 1 & 1 \\
intermittent & no & ii 0.5g q8h & 0.0 & 0.5 & 500 & 8 & 0 & 1 & 1 \\
intermittent & no & ii 1g q4h & 0.0 & 0.5 & 1000 & 4 & 0 & 1 & 1 \\
intermittent & no & ii 1g q6h & 0.0 & 0.5 & 1000 & 6 & 0 & 1 & 1 \\
intermittent & no & ii 1g q8h & 0.0 & 0.5 & 1000 & 8 & 0 & 1 & 1 \\
intermittent & no & ii 2g q6h & 0.0 & 0.5 & 2000 & 6 & 0 & 1 & 1 \\
intermittent & no & ii 2g q8h & 0.0 & 0.5 & 2000 & 8 & 0 & 1 & 1 \\
extended & no & ei 0.5g q4h & 0.0 & 2.0 & 500 & 4 & 0 & 1 & 1 \\
extended & no & ei 0.5g q6h & 0.0 & 3.0 & 500 & 6 & 0 & 1 & 1 \\
extended & no & ei 0.5g q8h & 0.0 & 4.0 & 500 & 8 & 0 & 1 & 1 \\
extended & no & ei 1g q4h & 0.0 & 2.0 & 1000 & 4 & 0 & 1 & 1 \\
extended & no & ei 1g q6h & 0.0 & 3.0 & 1000 & 6 & 0 & 1 & 1 \\
extended & no & ei 1g q8h & 0.0 & 4.0 & 1000 & 8 & 0 & 1 & 1 \\
extended & no & ei 2g q6h & 0.0 & 3.0 & 2000 & 6 & 0 & 1 & 1 \\
extended & no & ei 2g q8h & 0.0 & 4.0 & 2000 & 8 & 0 & 1 & 1 \\
extended & no & ei 3g q6h & 0.0 & 3.0 & 3000 & 6 & 0 & 1 & 1 \\
continuous & yes 1g over 0.5h & ci 6g q24h & 0.0 & 0.5 & 1000 & 0 & 0 &
1 & 1 \\
continuous & yes 1g over 0.5h & ci 6g q24h & 0.5 & 24.0 & 6000 & 24 & 0
& 1 & 1 \\
continuous & yes 1g over 0.5h & ci 4g q24h & 0.0 & 0.5 & 1000 & 0 & 0 &
1 & 1 \\
continuous & yes 1g over 0.5h & ci 4g q24h & 0.5 & 24.0 & 4000 & 24 & 0
& 1 & 1 \\
continuous & yes 1g over 0.5h & ci 3g q24h & 0.0 & 0.5 & 1000 & 0 & 0 &
1 & 1 \\
continuous & yes 1g over 0.5h & ci 3g q24h & 0.5 & 24.0 & 3000 & 24 & 0
& 1 & 1 \\
continuous & yes 1g over 0.5h & ci 8g q24h & 0.0 & 0.5 & 1000 & 0 & 0 &
1 & 1 \\
continuous & yes 1g over 0.5h & ci 8g q24h & 0.5 & 24.0 & 8000 & 24 & 0
& 1 & 1 \\
continuous & yes 1g over 0.5h & ci 12g q24h & 0.0 & 0.5 & 1000 & 0 & 0 &
1 & 1 \\
continuous & yes 1g over 0.5h & ci 12g q24h & 0.5 & 24.0 & 12000 & 24 &
0 & 1 & 1 \\

\end{longtable}

Simulations were performed for 4 creatinine clearance values (30 mL/min,
50 mL/min, 130 mL/min, 190 mL/min).

Two different fT\textgreater MIC were studied (50\%, 100\%) They were
computed for a range of MICs (4, 8, 16 mg/L).

First, \(fT>MIC\) was calculated at steady state. Amoxicillin
concentrations were considered at steady state after nine days.

\begin{figure}

\centering{

\includegraphics[width=1\linewidth,height=\textheight,keepaspectratio]{Model_implementation_validation/quarto/C:/Users/Vincent/Documents/travail/git_repositories/stp-brioche/Amoxicillin/Model_implementation_validation/published_figures_tables/amoxicillin/carlier_2013/table_5.png}

}

\caption{\label{fig-original-table-5}Original paper table 5}

\end{figure}%

\begin{table}

\caption{\label{tbl-at-ss}Reproduction of table 5 at steady state}

\centering{

\fontsize{12.0pt}{14.4pt}\selectfont
\begin{tabular*}{\linewidth}{@{\extracolsep{\fill}}l|ccccccccccccccccccccc}
\toprule
 & \multicolumn{7}{c}{MIC = 4 mg/L} & \multicolumn{7}{c}{MIC = 8 mg/L} & \multicolumn{7}{c}{MIC = 16 mg/L} \\ 
\cmidrule(lr){2-8} \cmidrule(lr){9-15} \cmidrule(lr){16-22}
 & \multicolumn{3}{c}{ fT>MIC > 0.5} & \multicolumn{3}{c}{ fT>MIC > 1} & sim & \multicolumn{3}{c}{ fT>MIC > 0.5} & \multicolumn{3}{c}{ fT>MIC > 1} & sim & \multicolumn{3}{c}{ fT>MIC > 0.5} & \multicolumn{3}{c}{ fT>MIC > 1} & sim \\ 
\cmidrule(lr){2-4} \cmidrule(lr){5-7} \cmidrule(lr){8-8} \cmidrule(lr){9-11} \cmidrule(lr){12-14} \cmidrule(lr){15-15} \cmidrule(lr){16-18} \cmidrule(lr){19-21} \cmidrule(lr){22-22}
 & agreement & paper & sim & agreement & paper & sim & ftovermic & agreement & paper & sim & agreement & paper & sim & ftovermic & agreement & paper & sim & agreement & paper & sim & ftovermic \\ 
\midrule\addlinespace[2.5pt]
\multicolumn{22}{l}{Creatinine clearance 30 mL/min} \\[2.5pt] 
\midrule\addlinespace[2.5pt]
ii 0.5g q8h & ✔ & + & + & ✔ & + & + & 100.0\% & ✔ & + & + & ✔ & + & + & 100.0\% & ✔ & + & + & ✔ & - & - & 77.8\% \\ 
ii 1g q8h & ✔ & + & + & ✔ & + & + & 100.0\% & ✔ & + & + & ✔ & + & + & 100.0\% & ✔ & + & + & ✔ & + & + & 100.0\% \\ 
ii 1g q6h & ✔ & + & + & ✔ & + & + & 100.0\% & ✔ & + & + & ✔ & + & + & 100.0\% & ✔ & + & + & ✔ & + & + & 100.0\% \\ 
\midrule\addlinespace[2.5pt]
\multicolumn{22}{l}{Creatinine clearance 50 mL/min} \\[2.5pt] 
\midrule\addlinespace[2.5pt]
ii 0.5g q6h & ✔ & + & + & ✔ & + & + & 100.0\% & ✔ & + & + & ✔ & + & + & 100.0\% & ✘ & + & - & ✔ & - & - & 46.0\% \\ 
ii 1g q8h & ✔ & + & + & ✔ & + & + & 100.0\% & ✔ & + & + & ✔ & + & + & 100.0\% & ✔ & + & + & ✔ & - & - & 76.0\% \\ 
ei 1g q8h & ✔ & + & + & ✔ & + & + & 100.0\% & ✔ & + & + & ✔ & + & + & 100.0\% & ✔ & + & + & ✘ & + & - & 99.6\% \\ 
ii 1g q6h & ✔ & + & + & ✔ & + & + & 100.0\% & ✔ & + & + & ✔ & + & + & 100.0\% & ✔ & + & + & ✔ & + & + & 100.0\% \\ 
ci 4g q24h & ✔ & + & + & ✔ & + & + & 100.0\% & ✔ & + & + & ✔ & + & + & 100.0\% & ✔ & + & + & ✔ & + & + & 100.0\% \\ 
\midrule\addlinespace[2.5pt]
\multicolumn{22}{l}{Creatinine clearance 130 mL/min} \\[2.5pt] 
\midrule\addlinespace[2.5pt]
ii 1g q8h & ✔ & + & + & ✔ & - & - & 62.7\% & ✔ & - & - & ✔ & - & - & 39.2\% & ✔ & - & - & ✔ & - & - & 18.1\% \\ 
ii 1g q6h & ✔ & + & + & ✘ & + & - & 86.5\% & ✔ & + & + & ✔ & - & - & 55.2\% & ✔ & - & - & ✔ & - & - & 26.0\% \\ 
ei 1g q6h & ✔ & + & + & ✔ & + & + & 100.0\% & ✔ & + & + & ✔ & - & - & 76.4\% & ✘ & + & - & ✔ & - & - & 36.1\% \\ 
ci 4g q24h & ✔ & + & + & ✔ & + & + & 100.0\% & ✔ & + & + & ✔ & + & + & 100.0\% & ✔ & - & - & ✔ & - & - & 0.0\% \\ 
ii 1g q4h & ✔ & + & + & ✔ & + & + & 100.0\% & ✔ & + & + & ✘ & + & - & 93.0\% & ✘ & + & - & ✔ & - & - & 47.0\% \\ 
ci 6g q24h & ✔ & + & + & ✔ & + & + & 100.0\% & ✔ & + & + & ✔ & + & + & 100.0\% & ✔ & + & + & ✔ & + & + & 100.0\% \\ 
\midrule\addlinespace[2.5pt]
\multicolumn{22}{l}{Creatinine clearance 190 mL/min} \\[2.5pt] 
\midrule\addlinespace[2.5pt]
ii 1g q6h & ✔ & + & + & ✔ & - & - & 56.3\% & ✔ & - & - & ✔ & - & - & 33.2\% & ✔ & - & - & ✔ & - & - & 16.9\% \\ 
ei 1g q6h & ✔ & + & + & ✘ & + & - & 79.1\% & ✔ & + & + & ✔ & - & - & 54.1\% & ✔ & - & - & ✔ & - & - & 0.0\% \\ 
ci 4g q24h & ✔ & + & + & ✔ & + & + & 100.0\% & ✔ & + & + & ✘ & - & + & 100.0\% & ✔ & - & - & ✔ & - & - & 0.0\% \\ 
ii 1g q4h & ✔ & + & + & ✘ & + & - & 89.7\% & ✔ & + & + & ✔ & - & - & 54.5\% & ✔ & - & - & ✔ & - & - & 27.6\% \\ 
ci 6g q24h & ✔ & + & + & ✔ & + & + & 100.0\% & ✔ & + & + & ✔ & + & + & 100.0\% & ✘ & + & - & ✔ & - & - & 0.0\% \\ 
ei 2g q6h & ✔ & + & + & ✔ & + & + & 100.0\% & ✔ & + & + & ✔ & - & - & 79.1\% & ✔ & + & + & ✔ & - & - & 54.1\% \\ 
ci 8g q24h & ✔ & + & + & ✔ & + & + & 100.0\% & ✔ & + & + & ✔ & + & + & 100.0\% & ✔ & + & + & ✔ & + & + & 100.0\% \\ 
ei 3g q6h & ✔ & + & + & ✔ & + & + & 100.0\% & ✔ & + & + & ✘ & + & - & 93.8\% & ✔ & + & + & ✔ & - & - & 68.6\% \\ 
\bottomrule
\end{tabular*}

}

\end{table}%

Table~\ref{tbl-at-ss-disagrement} below shows creatinine clearances and
dosing regimens for which there is at least one disagreement with the
paper.

\begin{table}

\caption{\label{tbl-at-ss-disagrement}Table of disagreements}

\centering{

\fontsize{12.0pt}{14.4pt}\selectfont
\begin{tabular*}{\linewidth}{@{\extracolsep{\fill}}l|ccccccccccccccccccccc}
\toprule
 & \multicolumn{7}{c}{MIC = 4 mg/L} & \multicolumn{7}{c}{MIC = 8 mg/L} & \multicolumn{7}{c}{MIC = 16 mg/L} \\ 
\cmidrule(lr){2-8} \cmidrule(lr){9-15} \cmidrule(lr){16-22}
 & \multicolumn{3}{c}{ fT>MIC > 0.5} & \multicolumn{3}{c}{ fT>MIC > 1} & sim & \multicolumn{3}{c}{ fT>MIC > 0.5} & \multicolumn{3}{c}{ fT>MIC > 1} & sim & \multicolumn{3}{c}{ fT>MIC > 0.5} & \multicolumn{3}{c}{ fT>MIC > 1} & sim \\ 
\cmidrule(lr){2-4} \cmidrule(lr){5-7} \cmidrule(lr){8-8} \cmidrule(lr){9-11} \cmidrule(lr){12-14} \cmidrule(lr){15-15} \cmidrule(lr){16-18} \cmidrule(lr){19-21} \cmidrule(lr){22-22}
 & agreement & paper & sim & agreement & paper & sim & ftovermic & agreement & paper & sim & agreement & paper & sim & ftovermic & agreement & paper & sim & agreement & paper & sim & ftovermic \\ 
\midrule\addlinespace[2.5pt]
\multicolumn{22}{l}{Creatinine clearance 130 mL/min} \\[2.5pt] 
\midrule\addlinespace[2.5pt]
ii 1g q6h & ✔ & + & + & ✘ & + & - & 86.5\% &  &  &  &  &  &  &  &  &  &  &  &  &  &  \\ 
ii 1g q4h &  &  &  &  &  &  &  & ✔ & + & + & ✘ & + & - & 93.0\% & ✘ & + & - & ✔ & - & - & 47.0\% \\ 
ei 1g q6h &  &  &  &  &  &  &  &  &  &  &  &  &  &  & ✘ & + & - & ✔ & - & - & 36.1\% \\ 
\midrule\addlinespace[2.5pt]
\multicolumn{22}{l}{Creatinine clearance 190 mL/min} \\[2.5pt] 
\midrule\addlinespace[2.5pt]
ei 1g q6h & ✔ & + & + & ✘ & + & - & 79.1\% &  &  &  &  &  &  &  &  &  &  &  &  &  &  \\ 
ii 1g q4h & ✔ & + & + & ✘ & + & - & 89.7\% &  &  &  &  &  &  &  &  &  &  &  &  &  &  \\ 
ci 4g q24h &  &  &  &  &  &  &  & ✔ & + & + & ✘ & - & + & 100.0\% &  &  &  &  &  &  &  \\ 
ei 3g q6h &  &  &  &  &  &  &  & ✔ & + & + & ✘ & + & - & 93.8\% &  &  &  &  &  &  &  \\ 
ci 6g q24h &  &  &  &  &  &  &  &  &  &  &  &  &  &  & ✘ & + & - & ✔ & - & - & 0.0\% \\ 
\midrule\addlinespace[2.5pt]
\multicolumn{22}{l}{Creatinine clearance 50 mL/min} \\[2.5pt] 
\midrule\addlinespace[2.5pt]
ii 0.5g q6h &  &  &  &  &  &  &  &  &  &  &  &  &  &  & ✘ & + & - & ✔ & - & - & 46.0\% \\ 
ei 1g q8h &  &  &  &  &  &  &  &  &  &  &  &  &  &  & ✔ & + & + & ✘ & + & - & 99.6\% \\ 
\bottomrule
\end{tabular*}

}

\end{table}%

Aside from one dosing regimen (ci 4g q24h for a CLCR of 190 mL/minutes),
disagreements between table 5 and our simulations are underestimations
of the fT\textgreater MIC.

A lot of these disagreements imply differences between the
fT\textgreater MIC that we simulated and table 5 close to 10\%.

\begin{itemize}
\tightlist
\item
  The dosing regimens for wich there is more difference are :

  \begin{itemize}
  \tightlist
  \item
    for a creatinin clearance of 130 mL/min and a MIC of 4 mg/L, ``ii 1g
    q6h'' ;
  \item
    for a creatinin clearance of 130 mL/min and a MIC of 16 mg/L, ``ei1g
    q6h'' ;
  \item
    for a creatinin clearance of 190 mL/min and a MIC of 4 mg/L, ``ei 1g
    q6h'' ;
  \item
    for a creatinin clearance of 190 mL/min and a MIC of 8 mg/L, ``ci 4g
    q24h'' and ``ei 3g q6h'' ;
  \item
    for a creatinin clearance of 190 mL/min and a MIC of 16 mg/L, ``ci
    6g q24h''.
  \end{itemize}
\end{itemize}

We seemed to underestimate fT\textgreater MIC for higher creatinin
clearances more than lower ones.

No kind of dosing regimen (continuous, extended or intermittent
infusion) looked more susceptible to disagreements between our
simulations and table 5.

For continuous infusions, we obtained fT\textgreater MIC equal to 0\% or
close or equal to 100\%. However, in table 5, for a CLCR of 190 mg/L and
a MIC of 16 mg/L, we notice that for he dosing regimen ``Ci 6g q24h'',
the target of fT \textgreater{} MIC \textgreater= 50\% is atteined but
not a fT \textgreater{} MIC of 100\%. This may indicate that the
article's author didn't perform calculate fT \textgreater{} MIC at
steady state only. Instead, they may have included the loading dose in
the time interval used to calculate fT \textgreater{} MIC.

We noticed that the authors simulated 7 days of treatment to create
figure 2. We hypothesized that they also simulated 7 days of treatment
and calculated fT \textgreater{} MIC on all the simulation duration in
order to create table 5. We then calculated fT \textgreater{} MIC on
seven days of treatment (results not shown here). This yielded results
similar to the one obtained on 24 hours at steady state : we observed
disagreements for the same dosing regimens, MIC and CLCR and fT
\textgreater{} MIC equals or slightly inferiors. In particular, for a
CLCR of 190 mg/L and a MIC of 16 mg/L, we still obtained a fT
\textgreater{} MIC equal to 0\%. We didn't confirm our hypothesis.

We also calculated fT \textgreater{} MIC on the first 24 hours of
treatment. However, the fT\textgreater MIC were underestimated compared
to the results of the article and there were a lot more disagreements
between our simulations and table 5. For a CLCR of 190 mg/L and a MIC of
16 mg/L, fT \textgreater{} MIC was only equal to 9\%.

\section{Figure 2}\label{figure-2}

While it would seem that Figure 2 could also be used for validation of
our model implementation, careful consideration of the plot makes us
suspect that technical errors happened while preparing the figure.
Therefore, no attempt at reproducing the figure will be made.

\begin{figure}

\centering{

\includegraphics[width=1\linewidth,height=\textheight,keepaspectratio]{Model_implementation_validation/quarto/C:/Users/Vincent/Documents/travail/git_repositories/stp-brioche/Amoxicillin/Model_implementation_validation/published_figures_tables/amoxicillin/carlier_2013/fig_2a.png}

}

\caption{\label{fig-original-fig-2a}Article figure 2a}

\end{figure}%

\begin{figure}

\centering{

\includegraphics[width=1\linewidth,height=\textheight,keepaspectratio]{Model_implementation_validation/quarto/C:/Users/Vincent/Documents/travail/git_repositories/stp-brioche/Amoxicillin/Model_implementation_validation/published_figures_tables/amoxicillin/carlier_2013/fig_2a_zoom.png}

}

\caption{\label{fig-original-fig-2a-zoom}Zoom on a strange section of
figure 2a}

\end{figure}%

\section{Conclusion}\label{conclusion}

We think our mrgsolve model is an adequate representation of the
published model since :

\begin{enumerate}
\def\labelenumi{\arabic{enumi}.}
\tightlist
\item
  The equations and parameters values of our mrgsolve model are
  identical to the reported ones
\item
  Most fT\textgreater MIC were accurately reproduced
\end{enumerate}

The slight discrepencies could be due to different simulation settings
(\emph{e.g.} different time points) and/or missing correlations between
variability parameters which were not reported.

\chapter{Validation PKpop Amox Fournier
2018}\label{validation-pkpop-amox-fournier-2018}

\chapter{Paper}\label{paper-1}

Fournier et al. (2018)

\chapter{Model description}\label{model-description-1}

\begin{longtable}[]{@{}
  >{\raggedright\arraybackslash}p{(\linewidth - 12\tabcolsep) * \real{0.1329}}
  >{\raggedright\arraybackslash}p{(\linewidth - 12\tabcolsep) * \real{0.1076}}
  >{\raggedright\arraybackslash}p{(\linewidth - 12\tabcolsep) * \real{0.1329}}
  >{\raggedright\arraybackslash}p{(\linewidth - 12\tabcolsep) * \real{0.2152}}
  >{\raggedleft\arraybackslash}p{(\linewidth - 12\tabcolsep) * \real{0.1203}}
  >{\raggedright\arraybackslash}p{(\linewidth - 12\tabcolsep) * \real{0.1835}}
  >{\raggedleft\arraybackslash}p{(\linewidth - 12\tabcolsep) * \real{0.1076}}@{}}

\caption{\label{tbl-model-desc}Model description}

\tabularnewline

\toprule\noalign{}
\begin{minipage}[b]{\linewidth}\raggedright
DOI
\end{minipage} & \begin{minipage}[b]{\linewidth}\raggedright
Structural model
\end{minipage} & \begin{minipage}[b]{\linewidth}\raggedright
Variability model
\end{minipage} & \begin{minipage}[b]{\linewidth}\raggedright
Covariates effects
\end{minipage} & \begin{minipage}[b]{\linewidth}\raggedleft
Number of patients
\end{minipage} & \begin{minipage}[b]{\linewidth}\raggedright
Free or total concentrations
\end{minipage} & \begin{minipage}[b]{\linewidth}\raggedleft
Unbound fraction
\end{minipage} \\
\midrule\noalign{}
\endhead
\bottomrule\noalign{}
\endlastfoot
10.1128/AAC.00505-18 & 2-comp & log-normal clearance & creatinine
clearance on clearance & 21 & total & 0.82 \\

\end{longtable}

\section{Model parameters}\label{model-parameters-1}

\begin{longtable}[]{@{}
  >{\raggedright\arraybackslash}p{(\linewidth - 8\tabcolsep) * \real{0.1389}}
  >{\raggedright\arraybackslash}p{(\linewidth - 8\tabcolsep) * \real{0.6574}}
  >{\raggedright\arraybackslash}p{(\linewidth - 8\tabcolsep) * \real{0.0833}}
  >{\raggedleft\arraybackslash}p{(\linewidth - 8\tabcolsep) * \real{0.0741}}
  >{\raggedleft\arraybackslash}p{(\linewidth - 8\tabcolsep) * \real{0.0463}}@{}}

\caption{\label{tbl-model-param}Model parameters}

\tabularnewline

\toprule\noalign{}
\begin{minipage}[b]{\linewidth}\raggedright
parameter\_name
\end{minipage} & \begin{minipage}[b]{\linewidth}\raggedright
parameter\_description
\end{minipage} & \begin{minipage}[b]{\linewidth}\raggedright
unit
\end{minipage} & \begin{minipage}[b]{\linewidth}\raggedleft
mean
\end{minipage} & \begin{minipage}[b]{\linewidth}\raggedleft
rse
\end{minipage} \\
\midrule\noalign{}
\endhead
\bottomrule\noalign{}
\endlastfoot
CLpop & Typical clearance & L/h & 13.6000 & 0.08 \\
theta\_CRCL\_CL & Covariate effect of creatinine clearance on
amoxicillin clearance & unitless & 0.0057 & 0.25 \\
Vcpop & Typical central volume of distribution & L & 9.7300 & 0.20 \\
Q & Intercompartimental clearance & L/h & 20.1000 & 0.24 \\
Vp & Peripheral volume of distribution & L & 17.6000 & 0.14 \\
fup & Plasma unbound fraction & unitless & 0.8200 & NA \\
cv\_iiv\_CL & Coefficient of variation of the inter individual
variability on cleara & unitless & 0.3730 & 0.19 \\
ruv\_prop & Proportional residual variability & unitless & 0.3700 &
0.19 \\
ruv\_add & Additive residual variability & mg/L & 0.0800 & 0.10 \\

\end{longtable}

\textbf{Note:} \textsuperscript{a}theta\_CLCR\_CL was wrongly reported
as 0.57 in the publication, email exchanges with the authors confirmed
that the actual value was 0.0057

\section{Inter-individual variability and covariate
effects}\label{inter-individual-variability-and-covariate-effects-1}

\[
CL_{i} = CL_{pop} \times [1 + (\theta_{CLCR-CL} \times (CLCR_{i} - 110))] \times e^{\eta_{i}}
\]

With :

\begin{itemize}
\item
  \(CL_{i}\) : Amoxicillin clearance for individual \emph{i}
\item
  \(CLCR_i\) : Creatinine clearance for individual \emph{i}
\item
  \(\eta_{i}\) : Normal variable with mean 0 and variance
  \(\omega^2_{CL}\)
\item
  Other parameters defined in Table~\ref{tbl-model-param}
\end{itemize}

Central volume of distribution is allometrically scaled to body weight :
\[
V1_{i} = V1_{pop} \times \frac{BW}{70}
\] With :

\begin{itemize}
\item
  \(V1_{i}\) : Central volume of distribution for individual \emph{i}
\item
  \(BW\) : Body weight of the individual \emph{i} in kilograms
\end{itemize}

\chapter{Validation of the mrgsolve
implementation}\label{validation-of-the-mrgsolve-implementation}

To validate that our implementation of the model is correct we will
attempt to reproduce figure 4 and table 4 of the original article

\section{Preliminary testing}\label{preliminary-testing-1}

First we check that simulations for a single dosing regimen and
creatinine clearance value make sense.

\pandocbounded{\includegraphics[keepaspectratio]{Model_implementation_validation/quarto/valid_amox_fournier_2018_files/figure-pdf/perform_pta_sim_clcr_200_q4h-1.pdf}}

\begin{longtable}[]{@{}rrrrr@{}}
\toprule\noalign{}
MIC & mean\_ft\_over\_mic & sd\_ft\_over\_mic & median\_ft\_over\_mic &
PTA\_0.5 \\
\midrule\noalign{}
\endhead
\bottomrule\noalign{}
\endlastfoot
1 & 0.9862974 & 0.0561477 & 1 & 1 \\
\end{longtable}

Profiles seem to make sense, the implementation looks ok so far.

\section{Reproduction of Figure 4}\label{reproduction-of-figure-4}

The figure represents the probability of target attainment for a
population of 500 virtual patients of various weight and creatinine
clearance, receiving various dosing regimens of amoxicillin. These
dosing regimens are represented in the following table :

\begin{longtable}[]{@{}
  >{\raggedleft\arraybackslash}p{(\linewidth - 6\tabcolsep) * \real{0.1429}}
  >{\raggedleft\arraybackslash}p{(\linewidth - 6\tabcolsep) * \real{0.3143}}
  >{\raggedleft\arraybackslash}p{(\linewidth - 6\tabcolsep) * \real{0.3286}}
  >{\raggedright\arraybackslash}p{(\linewidth - 6\tabcolsep) * \real{0.2143}}@{}}

\caption{\label{tbl-sim-dosing-reg}Simulated dosing regimen}

\tabularnewline

\toprule\noalign{}
\begin{minipage}[b]{\linewidth}\raggedleft
Dose (mg)
\end{minipage} & \begin{minipage}[b]{\linewidth}\raggedleft
Infusion duration (h)
\end{minipage} & \begin{minipage}[b]{\linewidth}\raggedleft
Interdose interval (h)
\end{minipage} & \begin{minipage}[b]{\linewidth}\raggedright
Dosing regimen
\end{minipage} \\
\midrule\noalign{}
\endhead
\bottomrule\noalign{}
\endlastfoot
2000 & 0.5 & 4 & 2g q4h \\
2000 & 2.0 & 6 & 2g q6h IT = 2h \\
2000 & 2.0 & 4 & 2g q4h IT = 2h \\
2000 & 0.5 & 8 & 2g q8h \\
1000 & 0.5 & 4 & 1g q4h \\
1000 & 2.0 & 4 & 1g q4h IT = 2h \\
1000 & 0.5 & 6 & 1g q6h \\
1000 & 0.5 & 8 & 1g q8h \\
1000 & 0.5 & 12 & 1g q12h \\
500 & 0.5 & 4 & 0.5g q4h \\
500 & 0.5 & 6 & 0.5g q6h \\
500 & 0.5 & 8 & 0.5g q8h \\
500 & 0.5 & 12 & 0.5g q12h \\
2000 & 2.0 & 8 & 2g q8h IT = 2h \\
1000 & 2.0 & 6 & 1g q6h IT = 2h \\

\end{longtable}

Patients weights were sampled from a lognormal distribution with mean 89
kg and standard deviation of 0.184 ln(kg).

Simualtions were performed for 6 creatinine clearance values(15 mL/min,
30 mL/min, 60 mL/min, 100 mL/min, 150 mL/min, 200 mL/min)

Two different fT\textgreater MIC (with unbound fraction = 0.82) were
studied (50\%, 100\%). They were computed for a range of MICs (0.25,
0.5, 1, 2, 4, 8, 16 mg/L).

The PTA when the target is 100\% of the time above MIC calculated
between 10 and 11 simulated days of treatment are inferior to those
calculated between 9 and 10 days or 11 and 12 days. In order to
reproduce figure 4, PTA were calculated between the 11 an 12th days of
therapy.

\begin{figure}

\centering{

\pandocbounded{\includegraphics[keepaspectratio]{Model_implementation_validation/quarto/valid_amox_fournier_2018_files/figure-pdf/fig-reprod-fig-4-top-row-1.pdf}}

}

\caption{\label{fig-reprod-fig-4-top-row}Attempt at reproducing the
first row of figure 4}

\end{figure}%

\begin{figure}

\centering{

\includegraphics[width=1\linewidth,height=\textheight,keepaspectratio]{Model_implementation_validation/quarto/C:/Users/Vincent/Documents/travail/git_repositories/stp-brioche/Amoxicillin/Model_implementation_validation/published_figures_tables/amoxicillin/fournier_2018/original_figure_4_first_row.jpeg}

}

\caption{\label{fig-original-fig-4-top-row}Original paper first row of
figure 4}

\end{figure}%

\begin{figure}

\centering{

\pandocbounded{\includegraphics[keepaspectratio]{Model_implementation_validation/quarto/valid_amox_fournier_2018_files/figure-pdf/fig-reprod-fig-4-second-row-sim_12-days-1.pdf}}

}

\caption{\label{fig-reprod-fig-4-second-row-sim_12-days}Attempt at
reproducing the second row of figure 4, twelve days of simulation}

\end{figure}%

\begin{figure}

\centering{

\includegraphics[width=1\linewidth,height=\textheight,keepaspectratio]{Model_implementation_validation/quarto/C:/Users/Vincent/Documents/travail/git_repositories/stp-brioche/Amoxicillin/Model_implementation_validation/published_figures_tables/amoxicillin/fournier_2018/original_figure_4_second_row.png}

}

\caption{\label{fig-original-fig-4-second-row}Original paper second row
of figure 4}

\end{figure}%

\begin{figure}

\centering{

\pandocbounded{\includegraphics[keepaspectratio]{Model_implementation_validation/quarto/valid_amox_fournier_2018_files/figure-pdf/fig-reprod-fig-4-third-row-sim_12-days-1.pdf}}

}

\caption{\label{fig-reprod-fig-4-third-row-sim_12-days}Attempt at
reproducing the third row of figure 4, twelve days of simulation}

\end{figure}%

\begin{figure}

\centering{

\includegraphics[width=1\linewidth,height=\textheight,keepaspectratio]{Model_implementation_validation/quarto/C:/Users/Vincent/Documents/travail/git_repositories/stp-brioche/Amoxicillin/Model_implementation_validation/published_figures_tables/amoxicillin/fournier_2018/original_figure_4_third_row.png}

}

\caption{\label{fig-original-fig-4-third-row}Original paper third row of
figure 4}

\end{figure}%

\begin{figure}

\centering{

\pandocbounded{\includegraphics[keepaspectratio]{Model_implementation_validation/quarto/valid_amox_fournier_2018_files/figure-pdf/fig-reprod-fig-4-fourth-row-sim_12-days-1.pdf}}

}

\caption{\label{fig-reprod-fig-4-fourth-row-sim_12-days}Attempt at
reproducing the fourth row of figure 4, twelve days of simulation}

\end{figure}%

\begin{figure}

\centering{

\includegraphics[width=1\linewidth,height=\textheight,keepaspectratio]{Model_implementation_validation/quarto/C:/Users/Vincent/Documents/travail/git_repositories/stp-brioche/Amoxicillin/Model_implementation_validation/published_figures_tables/amoxicillin/fournier_2018/original_figure_4_fourth_row.png}

}

\caption{\label{fig-original-fig-4-fourth-row}Original paper fourth row
of figure 4}

\end{figure}%

\begin{figure}

\centering{

\pandocbounded{\includegraphics[keepaspectratio]{Model_implementation_validation/quarto/valid_amox_fournier_2018_files/figure-pdf/fig-reprod-fig-4-fifth-row-sim_12-days-1.pdf}}

}

\caption{\label{fig-reprod-fig-4-fifth-row-sim_12-days}Attempt at
reproducing the fifth row of figure 4, twelve days of simulation}

\end{figure}%

\begin{figure}

\centering{

\includegraphics[width=1\linewidth,height=\textheight,keepaspectratio]{Model_implementation_validation/quarto/C:/Users/Vincent/Documents/travail/git_repositories/stp-brioche/Amoxicillin/Model_implementation_validation/published_figures_tables/amoxicillin/fournier_2018/original_figure_4_fifth_row.png}

}

\caption{\label{fig-original-fig-4-fifth-row}Original paper fifth row of
figure 4}

\end{figure}%

\begin{figure}

\centering{

\pandocbounded{\includegraphics[keepaspectratio]{Model_implementation_validation/quarto/valid_amox_fournier_2018_files/figure-pdf/fig-reprod-fig-4-sixth-row-sim_12-days-1.pdf}}

}

\caption{\label{fig-reprod-fig-4-sixth-row-sim_12-days}Attempt at
reproducing the sixth row of figure 4, twelve days of simulation}

\end{figure}%

\begin{figure}

\centering{

\includegraphics[width=1\linewidth,height=\textheight,keepaspectratio]{Model_implementation_validation/quarto/C:/Users/Vincent/Documents/travail/git_repositories/stp-brioche/Amoxicillin/Model_implementation_validation/published_figures_tables/amoxicillin/fournier_2018/original_figure_4_sixth_row.png}

}

\caption{\label{fig-original-fig-4-sixth-row}Original paper sixth row of
figure 4}

\end{figure}%

Regarding over or under estimation of PTA in relation to MIC, compared
with PTAs of the original article, no trend can be distinguished.
Differences are of the order of 5\%. Our simulations line up well with
the published figures.

To facilitate comparison between our simulations and the article figure
4, results of our simulations were plotted with article data. Data from
the original figure 4 of the article were digitalized. They are the
dashed lines in the following figures.

\begin{figure}

\centering{

\pandocbounded{\includegraphics[keepaspectratio]{Model_implementation_validation/quarto/valid_amox_fournier_2018_files/figure-pdf/fig-reprod-top-row-1.pdf}}

}

\caption{\label{fig-reprod-top-row}Reproduction of the first row of
figure 4}

\end{figure}%

\begin{figure}

\centering{

\pandocbounded{\includegraphics[keepaspectratio]{Model_implementation_validation/quarto/valid_amox_fournier_2018_files/figure-pdf/fig-reprod-second-row-1.pdf}}

}

\caption{\label{fig-reprod-second-row}Reproduction of the 2nd row of
figure 4}

\end{figure}%

\begin{figure}

\centering{

\pandocbounded{\includegraphics[keepaspectratio]{Model_implementation_validation/quarto/valid_amox_fournier_2018_files/figure-pdf/fig-reprod-third-row-1.pdf}}

}

\caption{\label{fig-reprod-third-row}Reproduction of the 3rd row of
figure 4}

\end{figure}%

\begin{figure}

\centering{

\pandocbounded{\includegraphics[keepaspectratio]{Model_implementation_validation/quarto/valid_amox_fournier_2018_files/figure-pdf/fig-reprod-fourth-row-1.pdf}}

}

\caption{\label{fig-reprod-fourth-row}Reproduction of the 4th row of
figure 4}

\end{figure}%

\begin{figure}

\centering{

\pandocbounded{\includegraphics[keepaspectratio]{Model_implementation_validation/quarto/valid_amox_fournier_2018_files/figure-pdf/fig-reprod-fifth-row-1.pdf}}

}

\caption{\label{fig-reprod-fifth-row}Reproduction of the 5th row of
figure 4}

\end{figure}%

\begin{figure}

\centering{

\pandocbounded{\includegraphics[keepaspectratio]{Model_implementation_validation/quarto/valid_amox_fournier_2018_files/figure-pdf/fig-reprod-sixth-row-1.pdf}}

}

\caption{\label{fig-reprod-sixth-row}Reproduction of the 6th row of
figure 4}

\end{figure}%

Regarding over or under estimation of PTA in relation to MIC, compared
with PTAs of the original article, no trend can be distinguished.
Differences are of the order of 5\%. Our simulations line up well with
the published figures.

\section{Reproduction of table 4}\label{reproduction-of-table-4}

Table 4 represents the influence of bodyweight and creatinin clearance
on fT\textgreater MIC means and PTAs for two targets. Four bodyweight
(50, 70, 100, 150 kg) and 3 creatinin clearance were studied. Dosing
regimen was 1g infused over 30 minutes every eight hours and MIC was 8
mg/L. For each condition, 500 patients were simulated.

Simulated results are rounded to three significant digits.
fT\textgreater MIC and PTA were computed between the 11th and 12th day
of simulation.

\begin{table}
\fontsize{12.0pt}{14.4pt}\selectfont
\begin{tabular*}{\linewidth}{@{\extracolsep{\fill}}rrrrrrrrrr}
\toprule
 &  & \multicolumn{2}{c}{fT\textgreater{}MIC, mean} & \multicolumn{2}{c}{fT\textgreater{}MIC, sd} & \multicolumn{2}{c}{PTA fT\textgreater{}MIC ≥ 50\%} & \multicolumn{2}{c}{PTA fT\textgreater{}MIC ≥ 100\%} \\ 
\cmidrule(lr){3-4} \cmidrule(lr){5-6} \cmidrule(lr){7-8} \cmidrule(lr){9-10}
CRCL & BW & fT>MIC, mean & Article fT>MIC, mean & fT>MIC, SD & Article fT>MIC, SD & PTA fT>MIC ≥ 50\% & Article PTA fT>MIC ≥ 50\% & PTA fT>MIC ≥ 100\% & Article PTA fT>MIC ≥ 100\% \\ 
\midrule\addlinespace[2.5pt]
\multicolumn{10}{l}{CRCL 30 mL/min} \\[2.5pt] 
\midrule\addlinespace[2.5pt]
30 & 50 & 0.630 & 0.65 & 0.235 & 0.24 & 0.662 & 0.690 & 0.132 & 0.170 \\ 
30 & 70 & 0.667 & 0.68 & 0.236 & 0.24 & 0.718 & 0.730 & 0.178 & 0.210 \\ 
30 & 100 & 0.701 & 0.72 & 0.243 & 0.24 & 0.750 & 0.770 & 0.244 & 0.260 \\ 
30 & 150 & 0.765 & 0.76 & 0.237 & 0.23 & 0.812 & 0.820 & 0.338 & 0.320 \\ 
\midrule\addlinespace[2.5pt]
\multicolumn{10}{l}{CRCL 100 mL/min} \\[2.5pt] 
\midrule\addlinespace[2.5pt]
100 & 50 & 0.331 & 0.34 & 0.162 & 0.18 & 0.140 & 0.160 & 0.004 & 0.008 \\ 
100 & 70 & 0.332 & 0.36 & 0.171 & 0.19 & 0.144 & 0.190 & 0.002 & 0.010 \\ 
100 & 100 & 0.371 & 0.38 & 0.185 & 0.20 & 0.198 & 0.220 & 0.016 & 0.020 \\ 
100 & 150 & 0.407 & 0.42 & 0.195 & 0.21 & 0.250 & 0.270 & 0.016 & 0.040 \\ 
\midrule\addlinespace[2.5pt]
\multicolumn{10}{l}{CRCL 200 mL/min} \\[2.5pt] 
\midrule\addlinespace[2.5pt]
200 & 50 & 0.176 & 0.18 & 0.095 & 0.10 & 0.012 & 0.012 & 0.000 & 0.000 \\ 
200 & 70 & 0.190 & 0.19 & 0.100 & 0.10 & 0.024 & 0.018 & 0.000 & 0.000 \\ 
200 & 100 & 0.195 & 0.21 & 0.087 & 0.11 & 0.010 & 0.030 & 0.000 & 0.000 \\ 
200 & 150 & 0.213 & 0.23 & 0.103 & 0.12 & 0.018 & 0.040 & 0.002 & 0.000 \\ 
\bottomrule
\end{tabular*}
\end{table}

We reproduced the results very well !

\chapter{Validation of mrgsolve implementation of Rambaud amoxicillin
PopPK model developed on a population of endocarditis
patients}\label{validation-of-mrgsolve-implementation-of-rambaud-amoxicillin-poppk-model-developed-on-a-population-of-endocarditis-patients}

\section{Article}\label{article}

Rambaud A, Gaborit BJ, Deschanvres C, Le Turnier P, Lecomte R,
Asseray-Madani N, Leroy AG, Deslandes G, Dailly É, Jolliet P, Boutoille
D, Bellouard R, Gregoire M; Nantes Anti-Microbial Agents PK/PD (NAMAP)
study group. Development and validation of a dosing nomogram for
amoxicillin in infective endocarditis. J Antimicrob Chemother. 2020 Oct
1;75(10):2941-2950. doi: 10.1093/jac/dkaa232. PMID: 32601687.

\section{Model description}\label{model-description-2}

\textbf{Route of administration:} Continuous infusion.\\
\textbf{Population size:} 107 patients for model development and 53 for
validation. There are no significant differences between the
characteristics of the two populations.\\
\textbf{Free or total concentrations:} Total concentrations.\\
\textbf{Structural model:} 2 compartment model with linear elimination,
inter-individual variability (IIV) on all of the parameters\\
\textbf{Error model} The original residual error model where the
standard deviation of each observation modelled by a polynomial equation
is weighted by process noise is approximated with a combined
proportional and additive error model.\\
\textbf{Covariate model:} Creatinine clearance (CRCL) estimated using
the CKD-EPI formula on the elimination constant (\(K_e\))\\
\[ 
K\_{e_i} = TVKe \cdot \left( \frac{CLCR_{i}}{64.92} \right)^{\beta} \cdot e^{\eta*{i}} \
\]with\\
\(Ke{i}\) : Amoxicillin elimination constant for subject \emph{i}
(\(h^{-1}\))\\
\(TVKe\) : Typical amoxicillin elimination constant (\(h^{-1}\))\\
\(\beta\) : Typical effect of CRCL on \(Ke_{i}\)\\
\(\eta_{i}\) : individual deviation from typical value with normal
distribution centered on 0\\
\(64.92\) : population's median creatinine clearance (\(mL.min^{-1}\))

\section{Conversion from non-parametric to
parametric}\label{conversion-from-non-parametric-to-parametric}

The original model is non-parametric. Inter-individual variability given
in the form of mean absolute weighted deviation (MAWD) of parameters was
converted to log normal inter-individual variability by \[
\sigma \approx MAWD \cdot 1.4826
\] \[
CV = \frac{\sigma}{\theta}
\] \[
\omega^2 = ln(CV^2 + 1)
\]with\\
\(\sigma\): standard deviation\\
CV : coefficient of variation\\
\(\theta\): typical parameter value\\
\(\omega^2\): variance of \(\eta_{i}\)

\section{Parameters and mrgsolve
implementation}\label{parameters-and-mrgsolve-implementation}

\textbf{Table 1.} Model parameter values where \(K_{12}\) is the
diffusion rate constant from the central to the peripheral compartment,
\(K_{21}\) is the diffusion rate constant from the peripheral to the
central compartment, \(K_e\) is the elimination constant, \(V_1\) is the
volume of distribution of the central compartment, \(\beta\) is the
effect of CRCL on the elimination constant.

\begin{longtable}[]{@{}ccc@{}}
\toprule\noalign{}
\textbf{Parameter} & \(\theta\) & \(\omega^2\) \\
\midrule\noalign{}
\endhead
\bottomrule\noalign{}
\endlastfoot
\(V_1\) (L) & 5.70 & 0.0615 \\
\(K_{12}\) (\(h^{-1}\)) & 0.247 & 1.11 \\
\(K_{21}\) (\(h^{-1}\)) & 11.43 & 0.194 \\
\(K_e\) (\(h^{-1}\)) & 1.69 & 0.0367 \\
\(\beta\) & 0.946 & 0.0964 \\
\end{longtable}

A standard value of 0.05 is used for proportional residual error and 1
\(mg.L^{-1}\) for additive residual unexplained variability (RUV).

\section{Subject and observation
simulation}\label{subject-and-observation-simulation}

107 subjects with the characteristics of the model development dataset
are simulated. 4 steady-state concentrations are simulated for each
subject with the time grid of 72, 96, 120, and 144 h.

\section{Covariate simulation}\label{covariate-simulation}

Creatinine clearance was simulated with a log-normal distribution using
the median (80 \(mL.min^{-1}\)) and interquantile range (52.6 - 106). As
the obtained distribution is too narrow compared to the minimum and
maximum values read from the figures (\(\approx\) 10 and 160) a small
proportion of extra values are simulated in the ranges 10-30 and
140-160. The simulations are iterated until the desired interquantile
range is obtained with a maximum difference of 2.

\pandocbounded{\includegraphics[keepaspectratio]{Model_implementation_validation/quarto/valid_rambaud_2020_files/figure-pdf/CRCL-simulation-1.pdf}}

\section{Dose simulation}\label{dose-simulation}

Doses are simulated with a normal distribution between 2000 mg and 20000
mg with a median of 12000 mg. The doses are rounded to the nearest 500
mg.

\pandocbounded{\includegraphics[keepaspectratio]{Model_implementation_validation/quarto/valid_rambaud_2020_files/figure-pdf/dose-simulation-1.pdf}}

The events table is generated with adding the times, CMT and EVID

\section{Model validation by figure
reproduction}\label{model-validation-by-figure-reproduction}

In the article there are three figures representing the results of the
PopPK analysis. The goodness of fit plots of observed concentrations as
a function of PRED or IPRED are less relevant to be reproduced without
the observed values. There is not enough information concerning RUV to
reproduce the plots with weighted residual errors. Figure 4 is selected
to reproduce which represents VPCs of amoxicillin concentrations against
aGFR (CRCL). To reproduce the figure, the previsously described 4
concentrations per subject are simulated and plotted agains the
respective CRCL values. The data is binned (using 4 bins) as on the
original figure. The 5th, 50th and 95th percentiles are calculated.

\begin{figure}[H]

{\centering \pandocbounded{\includegraphics[keepaspectratio]{Model_implementation_validation/quarto/valid_rambaud_2020_files/figure-pdf/figure-reproduction-1.pdf}}

}

\caption{Figure 4 - reproduced. Open circles are observed amoxicillin
concentrations at steady-state. Solid lines represent the 5th, 50th and
95th percentiles for simulated concentrations.}

\end{figure}%

\begin{figure}[H]

{\centering \includegraphics[width=1\linewidth,height=\textheight,keepaspectratio]{Model_implementation_validation/quarto/C:/Users/Vincent/Documents/travail/git_repositories/stp-brioche/Amoxicillin/Model_implementation_validation/published_figures_tables/amoxicillin/rambaud_2020/Figure_4_original.png}

}

\caption{Figure 4 - original. Open circles are observed amoxicillin
concentrations at steady-state. Solid lines represent the 5th, 50th and
95th percentiles for observed concentrations. Dashed lines represent the
5th, 50th and 95th percentiles for simulated concentrations.}

\end{figure}%

\section{Checking parameter
distributions}\label{checking-parameter-distributions}

Since we relied on approximations to convert the MAWD to log-normal
standard deviations, we will check whether the MAWD computed from
simulated parameter distributions are close to the ones reported in the
paper

\begin{longtable}[]{@{}lrrl@{}}
\toprule\noalign{}
name & Simulated MAWD & Publised MAWD & Relative difference \\
\midrule\noalign{}
\endhead
\bottomrule\noalign{}
\endlastfoot
K12 & 0.1524385 & 0.237 & 35.7 \% \\
K21 & 3.1753618 & 3.570 & 11.1 \% \\
CCRCL & 0.1947902 & 0.203 & 4 \% \\
KE1 & 0.2197522 & 0.220 & 0.1 \% \\
Vc & 0.9723197 & 0.968 & -0.4 \% \\
\end{longtable}

The relative difference is below 10\% for all parameters but K12 (KCP in
the paper). This is not of concern because K12 and K21 are not of
importance in this specific case.

Indeed when we compute the peripheral volume of this bi-compartmental
model we realize that it is negligible when compared with the central
compartment volume. \[
Vp = K12/K21 \times Vc = 0.247 / 11.427 \times 5.698 = 0.123 L \]

Thus having a bad representation of transfer to (and from) the
peripheral compartment will not cause any problems.

It is worthwhile to note that the data used in the paper only includes
concentrations measured after continuous infusions of amoxicillin. In
this case, the PK parameters describing peripheral distribution are not
identifiable which explains the very low \(Vp\) value.

\chapter{Validation PKpop Amox Mellon
2020}\label{validation-pkpop-amox-mellon-2020}

\chapter{Paper}\label{paper-2}

Mellon et al. (2020)

\begin{longtable}[]{@{}
  >{\raggedright\arraybackslash}p{(\linewidth - 12\tabcolsep) * \real{0.1081}}
  >{\raggedright\arraybackslash}p{(\linewidth - 12\tabcolsep) * \real{0.0766}}
  >{\raggedright\arraybackslash}p{(\linewidth - 12\tabcolsep) * \real{0.4189}}
  >{\raggedright\arraybackslash}p{(\linewidth - 12\tabcolsep) * \real{0.1171}}
  >{\raggedleft\arraybackslash}p{(\linewidth - 12\tabcolsep) * \real{0.0721}}
  >{\raggedright\arraybackslash}p{(\linewidth - 12\tabcolsep) * \real{0.1306}}
  >{\raggedleft\arraybackslash}p{(\linewidth - 12\tabcolsep) * \real{0.0766}}@{}}

\caption{\label{tbl-model-desc}Model description}

\tabularnewline

\toprule\noalign{}
\begin{minipage}[b]{\linewidth}\raggedright
DOI
\end{minipage} & \begin{minipage}[b]{\linewidth}\raggedright
structural\_model
\end{minipage} & \begin{minipage}[b]{\linewidth}\raggedright
variability\_model
\end{minipage} & \begin{minipage}[b]{\linewidth}\raggedright
covariate\_effects
\end{minipage} & \begin{minipage}[b]{\linewidth}\raggedleft
patients\_number
\end{minipage} & \begin{minipage}[b]{\linewidth}\raggedright
free\_or\_total\_concentrations
\end{minipage} & \begin{minipage}[b]{\linewidth}\raggedleft
unbound fraction
\end{minipage} \\
\midrule\noalign{}
\endhead
\bottomrule\noalign{}
\endlastfoot
doi:10.1093/jac/dkaa368 & 2-comp & interindividual variability for all
parameters, log-distribution for all parameters except F & weight on
absorption rate & 27 & total & 0.8 \\

\end{longtable}

\section{Model parameters}\label{model-parameters-2}

\begin{longtable}[]{@{}
  >{\raggedright\arraybackslash}p{(\linewidth - 10\tabcolsep) * \real{0.1504}}
  >{\raggedright\arraybackslash}p{(\linewidth - 10\tabcolsep) * \real{0.1128}}
  >{\raggedright\arraybackslash}p{(\linewidth - 10\tabcolsep) * \real{0.5865}}
  >{\raggedright\arraybackslash}p{(\linewidth - 10\tabcolsep) * \real{0.0677}}
  >{\raggedleft\arraybackslash}p{(\linewidth - 10\tabcolsep) * \real{0.0526}}
  >{\raggedleft\arraybackslash}p{(\linewidth - 10\tabcolsep) * \real{0.0301}}@{}}

\caption{\label{tbl-model-param}Model parameters}

\tabularnewline

\toprule\noalign{}
\begin{minipage}[b]{\linewidth}\raggedright
DOI
\end{minipage} & \begin{minipage}[b]{\linewidth}\raggedright
parameter\_name
\end{minipage} & \begin{minipage}[b]{\linewidth}\raggedright
parameter\_description
\end{minipage} & \begin{minipage}[b]{\linewidth}\raggedright
unit
\end{minipage} & \begin{minipage}[b]{\linewidth}\raggedleft
mean
\end{minipage} & \begin{minipage}[b]{\linewidth}\raggedleft
rse
\end{minipage} \\
\midrule\noalign{}
\endhead
\bottomrule\noalign{}
\endlastfoot
10.1093/jac/dkaa368 & Fpop & Typical oral bioavailability & unitless &
0.797 & 5 \\
10.1093/jac/dkaa369 & Mtt\_pop & Typical mean transit time & h & 1.000 &
6 \\
10.1093/jac/dkaa370 & Ktr\_pop & Typical transit rate & h-1 & 1.600 &
11 \\
10.1093/jac/dkaa371 & Ka\_pop & Typical absorption rate constant & h-1 &
1.700 & 10 \\
10.1093/jac/dkaa372 & beta\_weight & Relation between weight and
absorption rate constant & log h-1 & -3.100 & 19 \\
10.1093/jac/dkaa373 & CLpop & Typical clearance & L/h & 14.600 & 5 \\
10.1093/jac/dkaa374 & Vcpop & Typical central volume of distribution & L
& 9.000 & 11 \\
10.1093/jac/dkaa375 & Qpop & Typical intercompartimental clearance & L/h
& 4.200 & 12 \\
10.1093/jac/dkaa376 & Vppop & Typical peripheral volume of distribution
& L & 6.400 & 8 \\
10.1093/jac/dkaa377 & omega\_F & Standard deviation of F random effect &
unitless & 1.000 & 20 \\
10.1093/jac/dkaa378 & omega\_Mtt & Standard deviation of Mtt random
effect & unitless & 0.250 & 19 \\
10.1093/jac/dkaa379 & omega\_Ktr & Standard deviation of Ktr random
effect & unitless & 0.490 & 17 \\
10.1093/jac/dkaa380 & omega\_Ka & Standard deviation of Ka random effect
& unitless & 0.250 & 40 \\
10.1093/jac/dkaa381 & omega\_V1 & Standard deviation of V1 random effect
& unitless & 0.500 & 17 \\
10.1093/jac/dkaa382 & omega\_CL & Standard deviation of CL random effect
& unitless & 0.270 & 16 \\
10.1093/jac/dkaa383 & omega\_V2 & Standard deviation of V2 random effect
& unitless & 0.300 & 23 \\
10.1093/jac/dkaa384 & omega\_Q & Standard deviation of Q random effect &
unitless & 0.590 & 15 \\
10.1093/jac/dkaa385 & correlation & Correlation between random effects
of CL and Q & unitless & 0.900 & 7 \\
10.1093/jac/dkaa386 & SD\_a & Standard deviation of the additive
component for the residual error model & mg/L & 0.383 & 15 \\
10.1093/jac/dkaa387 & SD\_b & Standard deviation of the proportional
component for the residual error model & mg/L & 0.131 & 7 \\

\end{longtable}

Given both the concentration's values and the values of standard
deviations of the additive and proportional components of the residual
error model for the basic model, we choose to divide by 100 the
component's values announced for the final model with covariates.

\section{Interindividual variability and covariate
effects}\label{interindividual-variability-and-covariate-effects}

The only covariate retained by the article's authors is weight.

\[ 
Ka_{i} = Ka_{pop} \times (BW/109.3)^{beta_{weight}} \times exp(\eta^{Ka})
\]

With :

\begin{itemize}
\tightlist
\item
  \(Ka_{i}\) : absorption rate for individual i (\(h^{-1}\))
\item
  \(BW\) : bodyweight of the individual i (kg)
\item
  \(\eta_{Ka}\) : normal variable with mean 0 and variance
  \({\omega_{Ka}}^2\)
\end{itemize}

Other parameters defined in Table~\ref{tbl-model-param}.

According to the article, all model parameters are log-normally
distributed, except F wich is logit transformed :

\[
F = log(F_{pop}/(1-F_{pop})) + \eta^F
\] With :

\begin{itemize}
\tightlist
\item
  \(EF\) : normal variable with mean 0 and variance \({\omega_{F}}^2\)
\end{itemize}

\chapter{Validation of the model
implementation}\label{validation-of-the-model-implementation}

To validate the model's implementation, simulations were compared with
the article's observations of figure 1 and 3.

For the simulations appearing later in the article, ``body weight was
randomly drawn from a uniform distribution {[}85--155 kg{]}, which
corresponds to the range of weights observed at the inclusion of
participating subjects''. To create ``patients'' for our simulations,
weights were sampled from a uniform distribution with that density.

\chapter{Figure 1}\label{figure-1}

First, we created 27 ``patients'' by sampling weights from the uniform
distribution described above. We then simulated concentrations for 27
``patients'' using typical parameters values and inter individual
variability. This was intended as a control.

The weights of the patients included in the study are unknown. Moreover,
our simulations were taking inter individual variability in account. As
a result, we were not expecting to reproduce the observed concentrations
profiles described in figure 1. This was intended as a control of the
overall time-concentration profile.

\begin{figure}

\begin{minipage}{0.50\linewidth}

\begin{figure}[H]

{\centering \pandocbounded{\includegraphics[keepaspectratio]{Model_implementation_validation/quarto/valid_amox_mellon_hammas_2020_files/figure-pdf/fig1a-1.pdf}}

}

\subcaption{Article's figure 1a, IV administration, compared with
simulations}

\end{figure}%

\end{minipage}%
%
\begin{minipage}{0.50\linewidth}

\begin{figure}[H]

{\centering \includegraphics[width=1.68in,height=\textheight,keepaspectratio]{Model_implementation_validation/quarto/C:/Users/Vincent/Documents/travail/git_repositories/stp-brioche/Amoxicillin/Model_implementation_validation/published_figures_tables/amoxicillin/mellon_hammas_2020/fig_1_a_iv.png}

}

\subcaption{Article's figure 1a, IV administration, compared with
simulations}

\end{figure}%

\end{minipage}%

\end{figure}%

For amoxicillin IV administration, Cmax and the most of the
time-concentration profiles look similar. However, for a lot of the
simulated ``patients'' concentrations decrease quicker or to lower
values than the real patient's of the article.

\begin{figure}

\begin{minipage}{0.50\linewidth}

\begin{figure}[H]

{\centering \pandocbounded{\includegraphics[keepaspectratio]{Model_implementation_validation/quarto/valid_amox_mellon_hammas_2020_files/figure-pdf/fig1b-1.pdf}}

}

\subcaption{Article's figure 1b, oral administration, compared with
simulations}

\end{figure}%

\end{minipage}%
%
\begin{minipage}{0.50\linewidth}

\begin{figure}[H]

{\centering \includegraphics[width=1.68in,height=\textheight,keepaspectratio]{Model_implementation_validation/quarto/C:/Users/Vincent/Documents/travail/git_repositories/stp-brioche/Amoxicillin/Model_implementation_validation/published_figures_tables/amoxicillin/mellon_hammas_2020/fig_1_b_oral.png}

}

\subcaption{Article's figure 1b, oral administration, compared with
simulations}

\end{figure}%

\end{minipage}%

\end{figure}%

For amoxicillin oral administration, Tmax and Cmax of our simulations
and the original article looks the same and overall profile seems
similar. However, like for IV administration, some simulated patients'
concentrations decrease more and quicker than the real patients'.

Those differences may be caused by interindividual variability and
differences between real and simulate dpatients' weights.

\chapter{Figure 3}\label{figure-3}

In the article, 500 simulations were used in order to generate visual
predictive check (VPC) for the final model. We simulated 500
``patients'' with different weights to reproduce figure 3.

According to the article, amoxicillin plasmatic concentration
measurements were done before administration and 15, 30, 45, 60, 120,
240 and 480 minutes after. Given these facts and figure 3, it looks like
5 bins were done to compute VPCs for amoxicillin oral administration :
between 15 and 30 minutes, between 45 and 60 minutes, at 2, 4 and 8
hours. We tried to reproduce that binning.

We plotted 10th, 50th and 90th percentiles of our simulations with the
corresponding percentiles of the siulations in figure 3. Percentiles of
the simulation were retrieved from the article's figures by digitizing
the middle of the 95\% confidence interval of the simulations.

\pandocbounded{\includegraphics[keepaspectratio]{Model_implementation_validation/quarto/valid_amox_mellon_hammas_2020_files/figure-pdf/plot to reproduce fig 3b total concentration displayed-1.pdf}}

The median of our simulations line up quite well with what was found in
the paper. There is more variability in our simulation than in the paper
but this could be attributed to a different distribution of patient
weights in the real dataset, indeed we simulated using a uniform
distribution between the minimum and maximum weights reported in the
paper while it would be plausible that the real weights were more
concentrated towards the center of the distribution. Also the binning
will impact the closeness of our simulations with the published VPC.

Total concentrations :

\pandocbounded{\includegraphics[keepaspectratio]{Model_implementation_validation/quarto/valid_amox_mellon_hammas_2020_files/figure-pdf/plot fig 3 b iv total concentrations-1.pdf}}

Overall our simulations line up quite well with what was found in the
paper. Slight differences were expected due to binning.

\chapter{Figure 4}\label{figure-4}

\begin{figure}

\begin{minipage}{0.50\linewidth}

\begin{figure}[H]

\centering{

\pandocbounded{\includegraphics[keepaspectratio]{Model_implementation_validation/quarto/valid_amox_mellon_hammas_2020_files/figure-pdf/fig-4a-1.pdf}}

}

\caption{\label{fig-4a-1}Figure 4a and reproduction}

\end{figure}%

\end{minipage}%
%
\begin{minipage}{0.50\linewidth}

\begin{figure}[H]

\centering{

\includegraphics[width=2.44in,height=\textheight,keepaspectratio]{Model_implementation_validation/quarto/C:/Users/Vincent/Documents/travail/git_repositories/stp-brioche/Amoxicillin/Model_implementation_validation/published_figures_tables/amoxicillin/mellon_hammas_2020/fig_4_a_iv.png}

}

\caption{\label{fig-4a-2}Figure 4a and reproduction}

\end{figure}%

\end{minipage}%

\end{figure}%

\begin{figure}

\begin{minipage}{0.50\linewidth}

\begin{figure}[H]

\centering{

\pandocbounded{\includegraphics[keepaspectratio]{Model_implementation_validation/quarto/valid_amox_mellon_hammas_2020_files/figure-pdf/fig-4b-1.pdf}}

}

\caption{\label{fig-4b-1}Figure 4b and reproduction}

\end{figure}%

\end{minipage}%
%
\begin{minipage}{0.50\linewidth}

\begin{figure}[H]

\centering{

\includegraphics[width=2.61in,height=\textheight,keepaspectratio]{Model_implementation_validation/quarto/C:/Users/Vincent/Documents/travail/git_repositories/stp-brioche/Amoxicillin/Model_implementation_validation/published_figures_tables/amoxicillin/mellon_hammas_2020/fig_4_b_oral.png}

}

\caption{\label{fig-4b-2}Figure 4b and reproduction}

\end{figure}%

\end{minipage}%

\end{figure}%

Our simulations seems to match figure 4 well.

\part{Simulations}

\chapter{Covariate simulation}\label{covariate-simulation-1}

\begin{Shaded}
\begin{Highlighting}[]
\FunctionTok{library}\NormalTok{(tidyverse)}
\FunctionTok{library}\NormalTok{(here)}
\FunctionTok{library}\NormalTok{(knitr)}
\FunctionTok{library}\NormalTok{(rriskDistributions) }\CommentTok{\# to fit distributions}
\FunctionTok{library}\NormalTok{(RColorBrewer)}
\FunctionTok{library}\NormalTok{(msm) }\CommentTok{\# for truncated normal }
\FunctionTok{library}\NormalTok{(gt) }\CommentTok{\# for complex tables}
\FunctionTok{library}\NormalTok{(MASS) }\CommentTok{\# for colinarity}
\FunctionTok{library}\NormalTok{(tibble)}
\FunctionTok{library}\NormalTok{(dplyr)}
\FunctionTok{library}\NormalTok{(tidyr)}
\FunctionTok{library}\NormalTok{(MASS)}
\FunctionTok{library}\NormalTok{(ggcorrplot)}
\FunctionTok{library}\NormalTok{(patchwork)}
\NormalTok{conflicted}\SpecialCharTok{::}\FunctionTok{conflicts\_prefer}\NormalTok{(dplyr}\SpecialCharTok{::}\NormalTok{filter)}
\end{Highlighting}
\end{Shaded}

\begin{Shaded}
\begin{Highlighting}[]
\FunctionTok{set.seed}\NormalTok{(}\DecValTok{1991}\NormalTok{)}
\end{Highlighting}
\end{Shaded}

\section{Objective}\label{objective}

\begin{Shaded}
\begin{Highlighting}[]
\NormalTok{n\_patient }\OtherTok{\textless{}{-}} \DecValTok{625} \CommentTok{\# number of patients simulated per cohort}
\end{Highlighting}
\end{Shaded}

\begin{Shaded}
\begin{Highlighting}[]
\NormalTok{here}\SpecialCharTok{::}\FunctionTok{i\_am}\NormalTok{(}\StringTok{"a\_priori/For\_publication/Simulations/covariate\_simulation.qmd"}\NormalTok{)}

\CommentTok{\# Create folder to store published figures}
\ControlFlowTok{if}\NormalTok{ (}\SpecialCharTok{!}\FunctionTok{dir.exists}\NormalTok{(}\FunctionTok{here}\NormalTok{(}\StringTok{"a\_priori/For\_publication/Figures"}\NormalTok{))) \{}
  \FunctionTok{dir.create}\NormalTok{(}\FunctionTok{here}\NormalTok{(}\StringTok{"a\_priori/For\_publication/Figures"}\NormalTok{))}
\NormalTok{\}}

\NormalTok{cov\_data }\OtherTok{\textless{}{-}} \FunctionTok{read\_csv}\NormalTok{(}\FunctionTok{here}\NormalTok{(}\StringTok{"a\_priori/For\_publication/Simulations/cov\_distrib\_from\_papers.csv"}\NormalTok{))}
\end{Highlighting}
\end{Shaded}

From each identified paper, we want to simulate a cohort of patients.
For each cohort we want to simulate 625 patients. We define a patient by
a vector of covariate values. The covariates of interest are :

\begin{itemize}
\tightlist
\item
  WT : Body weight (kg)
\item
  CRCL : Creatinine clearance (mL/min)
\item
  ICU : Is the patient an ICU patient (=1) or not (=0)
\item
  BURN : Is the patient a burn patient (=1) or not (=0)
\item
  OBESE : Is the patient a obese - BMI \textgreater{} 30 kg/m\^{}2 (=1)
  or not (=0)
\item
  SEX : Male (=0) or female (=1)
\item
  HT : height (cm)
\item
  AGE : age (years)
\end{itemize}

To simulate covariate values we use information on the distribution of
the covariates taken from the included papers.

We included 4 papers :

\begin{itemize}
\item
  Carlier et al. (2013)
\item
  Fournier et al. (2018)
\item
  Mellon et al. (2020)
\item
  Rambaud et al. (2020)
\end{itemize}

One issue with the covariate data reported in these papers is that
correlations between covariates are not reported and simulating without
correlations runs the risk of simulating improbable covariate values. We
thus obtained a clinical dataset composed of more than 16000 ICU
patients (Johnson et al. (2023)) from which we could derive correlations
between covariates of interest. Age, serum creatinine, body weight, sex,
height and BMI were obtained for these patients and the creatinine
clearance was calculated (in mL/min for Cockroft and Gault and CKD-EPI
and in mL/min/1.73 m2 for MDRD (to correspond to Mellon)). A correlation
matrix was calculated for each of the respective model development
cohorts separately for females and males.

This correlation matrix is used to simulate correlations between the
simulated WT, CRCL, AGE, and HT/BMI if necessary to convert CRCL to
serum creatinine. (CRCL\_MDRD\_norm means CRCL in mL/min/1.73 m² used
for Mellon and CRCL\_MDRD in mL/min.). Different correlation matrices
are used based on sex. This way two different mutltivariate
distributions are simulated based on the two matrices with a number of
patients that respects the original article proportions.

We can find correlations for the log-transforms of certain variables.
For the covariates where a log-normal distribution fits better the log
mean and standard deviation of this distribution will be used to sample
log values with a multivariate normal distribution. At the end, the log
variables will be retransformed to normal ones.

\section{Carlier et al. (2013)}\label{carlier2013-bq}

\subsection{Covariate distribution}\label{covariate-distribution}

All patients are ICU patients and we will assume that they are not burn
and not obese patients. Creatinine clearance was obtained using the
measured urinary equation. Thus for all patients ICU = 1, BURN = 0.

As discussed in the article, renal replacement therapy is an exclusion
criterion, therefore the minimum value of creatinine clearance is set to
10 mL/min, any simulated value below 10 mL/min will be resampled. CRCL
was calculated using the measured urinary equation.

For WT and CRCL here are the data from the paper :

\begin{Shaded}
\begin{Highlighting}[]
\NormalTok{cov\_data\_carlier }\OtherTok{\textless{}{-}}\NormalTok{ cov\_data }\SpecialCharTok{|\textgreater{}} 
  \FunctionTok{filter}\NormalTok{(Paper }\SpecialCharTok{==} \StringTok{"Carlier\_2013"}\NormalTok{)}

\NormalTok{cov\_data\_carlier }\SpecialCharTok{|\textgreater{}} 
  \FunctionTok{filter}\NormalTok{(Covariate }\SpecialCharTok{\%in\%} \FunctionTok{c}\NormalTok{(}\StringTok{"WT"}\NormalTok{,}\StringTok{"CRCL"}\NormalTok{,}\StringTok{"AGE"}\NormalTok{,}\StringTok{"BMI"}\NormalTok{)) }\SpecialCharTok{|\textgreater{}} 
  \FunctionTok{kable}\NormalTok{()}
\end{Highlighting}
\end{Shaded}

\begin{longtable}[]{@{}lllrrrrr@{}}
\toprule\noalign{}
Paper & Covariate & Unit & Median & Q1 & Q3 & Min & Max \\
\midrule\noalign{}
\endhead
\bottomrule\noalign{}
\endlastfoot
Carlier\_2013 & WT & kg & 75 & 70 & 79 & NA & NA \\
Carlier\_2013 & CRCL & mL/min & 102 & 50 & 157 & 10 & NA \\
Carlier\_2013 & BMI & kg/m\^{}\{2\} & 24 & 21 & 25 & NA & NA \\
Carlier\_2013 & AGE & year & 62 & 58 & 72 & NA & NA \\
\end{longtable}

\begin{Shaded}
\begin{Highlighting}[]
\NormalTok{cor\_matrix\_Carlier\_F }\OtherTok{\textless{}{-}} \FunctionTok{readRDS}\NormalTok{(}\FunctionTok{here}\NormalTok{(}\StringTok{"a\_priori/For\_publication/Simulations/cor\_matrix\_Carlier\_F.rds"}\NormalTok{))}
\FunctionTok{print}\NormalTok{(cor\_matrix\_Carlier\_F)}
\end{Highlighting}
\end{Shaded}

\begin{verbatim}
                    CREAT   CREAT_log         AGE          WT      WT_log
CREAT          1.00000000  0.93719806  0.17043554  0.07114115  0.07024583
CREAT_log      0.93719806  1.00000000  0.22904749  0.09016458  0.09019928
AGE            0.17043554  0.22904749  1.00000000 -0.11058359 -0.10874794
WT             0.07114115  0.09016458 -0.11058359  1.00000000  0.99663272
WT_log         0.07024583  0.09019928 -0.10874794  0.99663272  1.00000000
HT            -0.02615290 -0.03452481 -0.22145176  0.36158540  0.35423946
HT_log        -0.02656759 -0.03481661 -0.22200991  0.36341580  0.35641748
BMI            0.09120381  0.11653457  0.01416104  0.83213862  0.83590568
BMI_log        0.08905860  0.11477057  0.01748252  0.82644900  0.83409794
CRCL_MDRD     -0.73311938 -0.90200963 -0.35625197  0.05193539  0.05083450
CRCL_MDRD_log -0.91894029 -0.98323371 -0.33894619  0.05521443  0.05501799
                       HT      HT_log         BMI     BMI_log   CRCL_MDRD
CREAT         -0.02615290 -0.02656759  0.09120381  0.08905860 -0.73311938
CREAT_log     -0.03452481 -0.03481661  0.11653457  0.11477057 -0.90200963
AGE           -0.22145176 -0.22200991  0.01416104  0.01748252 -0.35625197
WT             0.36158540  0.36341580  0.83213862  0.82644900  0.05193539
WT_log         0.35423946  0.35641748  0.83590568  0.83409794  0.05083450
HT             1.00000000  0.99908709 -0.21054270 -0.21983890  0.15405745
HT_log         0.99908709  1.00000000 -0.20907388 -0.21810288  0.15428482
BMI           -0.21054270 -0.20907388  1.00000000  0.99655760 -0.03784624
BMI_log       -0.21983890 -0.21810288  0.99655760  1.00000000 -0.03799028
CRCL_MDRD      0.15405745  0.15428482 -0.03784624 -0.03799028  1.00000000
CRCL_MDRD_log  0.14873339  0.14934183 -0.03146339 -0.03070224  0.92275219
              CRCL_MDRD_log
CREAT           -0.91894029
CREAT_log       -0.98323371
AGE             -0.33894619
WT               0.05521443
WT_log           0.05501799
HT               0.14873339
HT_log           0.14934183
BMI             -0.03146339
BMI_log         -0.03070224
CRCL_MDRD        0.92275219
CRCL_MDRD_log    1.00000000
\end{verbatim}

\begin{Shaded}
\begin{Highlighting}[]
\FunctionTok{ggcorrplot}\NormalTok{(cor\_matrix\_Carlier\_F, }\AttributeTok{lab =} \ConstantTok{TRUE}\NormalTok{, }\AttributeTok{title =} \StringTok{"Sex: female"}\NormalTok{)}
\end{Highlighting}
\end{Shaded}

\pandocbounded{\includegraphics[keepaspectratio]{Simulations/covariate_simulation_files/figure-pdf/carlier-cor-matrix-1.pdf}}

\begin{Shaded}
\begin{Highlighting}[]
\NormalTok{cor\_matrix\_Carlier\_M }\OtherTok{\textless{}{-}} \FunctionTok{readRDS}\NormalTok{(}\FunctionTok{here}\NormalTok{(}\StringTok{"a\_priori/For\_publication/Simulations/cor\_matrix\_Carlier\_M.rds"}\NormalTok{))}
\FunctionTok{print}\NormalTok{(cor\_matrix\_Carlier\_M)}
\end{Highlighting}
\end{Shaded}

\begin{verbatim}
                    CREAT   CREAT_log          AGE          WT      WT_log
CREAT          1.00000000  0.93656904  0.129584751  0.03477888  0.03489511
CREAT_log      0.93656904  1.00000000  0.191015858  0.06256888  0.06255796
AGE            0.12958475  0.19101586  1.000000000 -0.05978080 -0.05792048
WT             0.03477888  0.06256888 -0.059780801  1.00000000  0.99691085
WT_log         0.03489511  0.06255796 -0.057920475  0.99691085  1.00000000
HT            -0.01562634 -0.02015099 -0.099371429  0.33471772  0.33277207
HT_log        -0.01514335 -0.01943840 -0.098654543  0.33778296  0.33606644
BMI            0.04448518  0.07478234  0.002442579  0.78722872  0.78753911
BMI_log        0.04503665  0.07581809  0.004260007  0.79489820  0.79912510
CRCL_MDRD     -0.73662552 -0.90639303 -0.327115476  0.04508957  0.04511819
CRCL_MDRD_log -0.92158994 -0.98613183 -0.287263519  0.06492902  0.06504475
                       HT      HT_log          BMI      BMI_log   CRCL_MDRD
CREAT         -0.01562634 -0.01514335  0.044485175  0.045036646 -0.73662552
CREAT_log     -0.02015099 -0.01943840  0.074782343  0.075818092 -0.90639303
AGE           -0.09937143 -0.09865454  0.002442579  0.004260007 -0.32711548
WT             0.33471772  0.33778296  0.787228725  0.794898201  0.04508957
WT_log         0.33277207  0.33606644  0.787539107  0.799125098  0.04511819
HT             1.00000000  0.99918214 -0.312182156 -0.300458114  0.12810278
HT_log         0.99918214  1.00000000 -0.309805459 -0.297640862  0.12729396
BMI           -0.31218216 -0.30980546  1.000000000  0.996020588 -0.03549251
BMI_log       -0.30045811 -0.29764086  0.996020588  1.000000000 -0.03551703
CRCL_MDRD      0.12810278  0.12729396 -0.035492514 -0.035517032  1.00000000
CRCL_MDRD_log  0.11829734  0.11791657 -0.010057809 -0.009333342  0.92462410
              CRCL_MDRD_log
CREAT          -0.921589942
CREAT_log      -0.986131825
AGE            -0.287263519
WT              0.064929019
WT_log          0.065044752
HT              0.118297341
HT_log          0.117916569
BMI            -0.010057809
BMI_log        -0.009333342
CRCL_MDRD       0.924624105
CRCL_MDRD_log   1.000000000
\end{verbatim}

\begin{Shaded}
\begin{Highlighting}[]
\FunctionTok{ggcorrplot}\NormalTok{(cor\_matrix\_Carlier\_M, }\AttributeTok{lab =} \ConstantTok{TRUE}\NormalTok{, }\AttributeTok{title =} \StringTok{"Sex: male"}\NormalTok{)}
\end{Highlighting}
\end{Shaded}

\pandocbounded{\includegraphics[keepaspectratio]{Simulations/covariate_simulation_files/figure-pdf/carlier-cor-matrix-2.pdf}}

We are going to fit normal and log-normal distributions to the observed
quantiles and select the best fitting one

\begin{Shaded}
\begin{Highlighting}[]
\NormalTok{compute\_cv\_from\_sd\_lnorm }\OtherTok{\textless{}{-}} \ControlFlowTok{function}\NormalTok{(sdlog)\{}
  \CommentTok{\#\textquotesingle{} Compute the geometric coefficient of variation from the standard deviation of a lognormal distribution}
  \CommentTok{\#\textquotesingle{}}
  \CommentTok{\#\textquotesingle{} @param sdlog numeric. standard deviation of the log transformed variable}
  \CommentTok{\#\textquotesingle{} }
  \CommentTok{\#\textquotesingle{} @return numeric. geometric coefficient of variation}
  \FunctionTok{sqrt}\NormalTok{(}\FunctionTok{exp}\NormalTok{(sdlog}\SpecialCharTok{\^{}}\DecValTok{2}\NormalTok{)}\SpecialCharTok{{-}}\DecValTok{1}\NormalTok{)}
\NormalTok{\}}

\NormalTok{extract\_cov\_param }\OtherTok{\textless{}{-}} \ControlFlowTok{function}\NormalTok{(cov\_data, cov\_name) \{}
  
  \CommentTok{\#\textquotesingle{} Get covariate quantiles from covariate characteristics table}
  \CommentTok{\#\textquotesingle{}}
  \CommentTok{\#\textquotesingle{} @param cov\_data tibble containing the covariate information}
  \CommentTok{\#\textquotesingle{} @param cov\_name character. name of the covariate to select, comes from the Covariate column of cov\_data.}
  \CommentTok{\#\textquotesingle{} @return list with :}
  \CommentTok{\#\textquotesingle{}  {-} p : numeric vector of probabilities}
  \CommentTok{\#\textquotesingle{}  {-} q : numeric vector of quantiles}
  \CommentTok{\#\textquotesingle{}  {-} cov\_name : character string containing the covariate name}
  \CommentTok{\#\textquotesingle{}  {-} cov\_unit : character string containing the unit of the covariate}
  
  
\NormalTok{  cov\_filtered\_df }\OtherTok{\textless{}{-}}\NormalTok{ cov\_data }\SpecialCharTok{|\textgreater{}}
    \FunctionTok{filter}\NormalTok{(Covariate }\SpecialCharTok{==}\NormalTok{ cov\_name) }\SpecialCharTok{|\textgreater{}}
    \FunctionTok{pivot\_longer}\NormalTok{(}
      \AttributeTok{cols =} \FunctionTok{c}\NormalTok{(Median, Q1, Q3, Min, Max), }\CommentTok{\# if the format of the csv columns ever changes, this will need to change}
      \AttributeTok{names\_to =} \StringTok{"p"}\NormalTok{,}
      \AttributeTok{values\_to =} \StringTok{"q"}
\NormalTok{    ) }\SpecialCharTok{|\textgreater{}}
    \FunctionTok{mutate}\NormalTok{(}\AttributeTok{p =} \FunctionTok{case\_match}\NormalTok{(p, }\StringTok{"Median"} \SpecialCharTok{\textasciitilde{}} \FloatTok{0.5}\NormalTok{, }\StringTok{"Q1"} \SpecialCharTok{\textasciitilde{}} \FloatTok{0.25}\NormalTok{, }\StringTok{"Q3"} \SpecialCharTok{\textasciitilde{}} \FloatTok{0.75}\NormalTok{, }\StringTok{"Min"} \SpecialCharTok{\textasciitilde{}} \FloatTok{0.01}\NormalTok{, }\StringTok{"Max"} \SpecialCharTok{\textasciitilde{}} \FloatTok{0.99}\NormalTok{)) }\SpecialCharTok{|\textgreater{}}
    \CommentTok{\# if the format of the csv columns ever changes, the code above will need to change}
    \FunctionTok{filter}\NormalTok{(}\SpecialCharTok{!}\FunctionTok{is.na}\NormalTok{(q)) }\SpecialCharTok{|\textgreater{}}
    \FunctionTok{group\_by}\NormalTok{(Paper) }\SpecialCharTok{|\textgreater{}}
    \FunctionTok{arrange}\NormalTok{(p, }\AttributeTok{.by\_group =} \ConstantTok{TRUE}\NormalTok{) }\SpecialCharTok{|\textgreater{}}
    \FunctionTok{ungroup}\NormalTok{()}
  
  \FunctionTok{list}\NormalTok{(}
    \AttributeTok{p =}\NormalTok{ cov\_filtered\_df}\SpecialCharTok{$}\NormalTok{p,}
    \AttributeTok{q =}\NormalTok{ cov\_filtered\_df}\SpecialCharTok{$}\NormalTok{q,}
    \AttributeTok{cov\_name =} \FunctionTok{unique}\NormalTok{(cov\_filtered\_df}\SpecialCharTok{$}\NormalTok{Covariate),}
    \AttributeTok{cov\_unit =} \FunctionTok{unique}\NormalTok{(cov\_filtered\_df}\SpecialCharTok{$}\NormalTok{Unit)}
\NormalTok{  )}
  
  
\NormalTok{\}}

\NormalTok{evaluate\_distrib }\OtherTok{\textless{}{-}} \ControlFlowTok{function}\NormalTok{(p, q, cov\_name, cov\_unit, min\_plot, max\_plot,}\AttributeTok{tol =} \FloatTok{0.001}\NormalTok{) \{}
  \CommentTok{\#\textquotesingle{} Fit normal and lognormal distribution to observed quantiles}
  \CommentTok{\#\textquotesingle{} @param p numeric vector of probabilities}
  \CommentTok{\#\textquotesingle{} @param q numeric vector of quantiles}
  \CommentTok{\#\textquotesingle{} @param cov\_name character string containing the covariate name}
  \CommentTok{\#\textquotesingle{} @param cov\_unit character string containing the unit of the covariate}
  \CommentTok{\#\textquotesingle{} @param min\_plot minimal value of the covariate for which the distribution should be plotted}
  \CommentTok{\#\textquotesingle{} @param max\_plot maximal value of the covariate for which the distribution should be plotted}
  \CommentTok{\#\textquotesingle{} @param tol numeric. tolerance value for the fitting algorithm, default is 0.001}
  \CommentTok{\#\textquotesingle{} @return list with :}
  \CommentTok{\#\textquotesingle{}  {-} plot : ggplot2 showing the fitted distributions and the observed covariate}
  \CommentTok{\#\textquotesingle{}  {-} lnorm\_par : named numeric vector with the lognormal distribution parameters}
  \CommentTok{\#\textquotesingle{}  {-} norm\_par : named numeric vector with the normal distribution parameters}
  
  
\NormalTok{  lnorm\_par }\OtherTok{\textless{}{-}} \FunctionTok{get.lnorm.par}\NormalTok{(}\AttributeTok{p =}\NormalTok{ p, }\AttributeTok{q =}\NormalTok{ q, }\AttributeTok{plot =} \ConstantTok{FALSE}\NormalTok{, }\AttributeTok{show.output =} \ConstantTok{FALSE}\NormalTok{,}\AttributeTok{tol =}\NormalTok{ tol)}
\NormalTok{  norm\_par }\OtherTok{\textless{}{-}} \FunctionTok{get.norm.par}\NormalTok{(}\AttributeTok{p =}\NormalTok{ p, }\AttributeTok{q =}\NormalTok{ q, }\AttributeTok{plot =} \ConstantTok{FALSE}\NormalTok{, }\AttributeTok{show.output =} \ConstantTok{FALSE}\NormalTok{,}\AttributeTok{tol=}\NormalTok{tol)}
  
  \CommentTok{\#create a dataset with the true percentiles to be able to show them on the plot}
\NormalTok{  true\_percentiles }\OtherTok{\textless{}{-}} \FunctionTok{tibble}\NormalTok{(p, q) }\SpecialCharTok{|\textgreater{}} 
    \FunctionTok{mutate}\NormalTok{(}\AttributeTok{label =} \FunctionTok{case\_match}\NormalTok{(p, }
                              \DecValTok{0} \SpecialCharTok{\textasciitilde{}} \StringTok{"True min"}\NormalTok{,}
                              \FloatTok{0.25} \SpecialCharTok{\textasciitilde{}} \StringTok{"True Q1"}\NormalTok{,}
                              \FloatTok{0.5} \SpecialCharTok{\textasciitilde{}} \StringTok{"True median"}\NormalTok{,}
                              \FloatTok{0.75} \SpecialCharTok{\textasciitilde{}} \StringTok{"True Q3"}\NormalTok{,}
                              \DecValTok{1} \SpecialCharTok{\textasciitilde{}} \StringTok{"True max"}\NormalTok{)) }
  
  \CommentTok{\#create a dataset with the percentiles of the fitted distributiosn to be able to show them on the plot}
\NormalTok{  distrib\_percentiles }\OtherTok{\textless{}{-}}  \FunctionTok{tibble}\NormalTok{(}
    \AttributeTok{probability =}\NormalTok{ p,}
    \AttributeTok{lnorm =} \FunctionTok{qlnorm}\NormalTok{(probability, }\AttributeTok{meanlog =}\NormalTok{ lnorm\_par[}\StringTok{"meanlog"}\NormalTok{], }\AttributeTok{sdlog =}
\NormalTok{                     lnorm\_par[}\StringTok{"sdlog"}\NormalTok{]),}
    \AttributeTok{norm =} \FunctionTok{qnorm}\NormalTok{(probability, }\AttributeTok{mean =}\NormalTok{ norm\_par[}\StringTok{"mean"}\NormalTok{], }\AttributeTok{sd =}\NormalTok{ norm\_par[}\StringTok{"sd"}\NormalTok{])}
\NormalTok{  ) }\SpecialCharTok{|\textgreater{}}
    \FunctionTok{pivot\_longer}\NormalTok{(}\AttributeTok{cols =} \FunctionTok{c}\NormalTok{(lnorm, norm), }\AttributeTok{names\_to =} \StringTok{"distrib"}\NormalTok{) }\SpecialCharTok{|\textgreater{}}
    \FunctionTok{mutate}\NormalTok{(}\AttributeTok{density =} \FunctionTok{case\_match}\NormalTok{(}
\NormalTok{      distrib,}
      \StringTok{"norm"} \SpecialCharTok{\textasciitilde{}} \FunctionTok{dnorm}\NormalTok{(value, }\AttributeTok{mean =}\NormalTok{ norm\_par[}\StringTok{"mean"}\NormalTok{], }\AttributeTok{sd =}
\NormalTok{                       norm\_par[}\StringTok{"sd"}\NormalTok{]),}
      \StringTok{"lnorm"} \SpecialCharTok{\textasciitilde{}} \FunctionTok{dlnorm}\NormalTok{(value, }\AttributeTok{meanlog =}\NormalTok{ lnorm\_par[}\StringTok{"meanlog"}\NormalTok{], }\AttributeTok{sdlog =}
\NormalTok{                         lnorm\_par[}\StringTok{"sdlog"}\NormalTok{])}
\NormalTok{    )) }\SpecialCharTok{|\textgreater{}} 
    \FunctionTok{mutate}\NormalTok{(}\AttributeTok{label =} \FunctionTok{case\_match}\NormalTok{(probability, }
                              \DecValTok{0} \SpecialCharTok{\textasciitilde{}} \StringTok{"Min"}\NormalTok{,}
                              \FloatTok{0.25} \SpecialCharTok{\textasciitilde{}} \StringTok{"Q1"}\NormalTok{,}
                              \FloatTok{0.5} \SpecialCharTok{\textasciitilde{}} \StringTok{"Median"}\NormalTok{,}
                              \FloatTok{0.75} \SpecialCharTok{\textasciitilde{}} \StringTok{"Q3"}\NormalTok{,}
                              \DecValTok{1} \SpecialCharTok{\textasciitilde{}} \StringTok{"Max"}\NormalTok{)) }
  
  
  \CommentTok{\#simulate 1000 datapoints from the fitted distributions to show on the plot}
\NormalTok{  param\_data }\OtherTok{\textless{}{-}} \FunctionTok{tibble}\NormalTok{(}
    \AttributeTok{variable =} \FunctionTok{seq}\NormalTok{(min\_plot, max\_plot, }\AttributeTok{length.out =} \DecValTok{1000}\NormalTok{),}
    \AttributeTok{lnorm =} \FunctionTok{dlnorm}\NormalTok{(variable, }\AttributeTok{meanlog =}\NormalTok{ lnorm\_par[}\StringTok{"meanlog"}\NormalTok{], }\AttributeTok{sdlog =}\NormalTok{ lnorm\_par[}\StringTok{"sdlog"}\NormalTok{]),}
    \AttributeTok{norm =} \FunctionTok{dnorm}\NormalTok{(variable, }\AttributeTok{mean =}\NormalTok{ norm\_par[}\StringTok{"mean"}\NormalTok{], }\AttributeTok{sd =}\NormalTok{ norm\_par[}\StringTok{"sd"}\NormalTok{])}
\NormalTok{  ) }\SpecialCharTok{|\textgreater{}}
    \FunctionTok{pivot\_longer}\NormalTok{(}\AttributeTok{cols =} \FunctionTok{c}\NormalTok{(lnorm, norm), }\AttributeTok{names\_to =} \StringTok{"distrib"}\NormalTok{)}
  
  
\NormalTok{  plot }\OtherTok{\textless{}{-}} \FunctionTok{ggplot}\NormalTok{(param\_data, }\FunctionTok{aes}\NormalTok{(}\AttributeTok{x =}\NormalTok{ variable, }\AttributeTok{y =}\NormalTok{ value, }\AttributeTok{color =}\NormalTok{ distrib)) }\SpecialCharTok{+}
    \FunctionTok{geom\_vline}\NormalTok{(}\AttributeTok{data =}\NormalTok{ true\_percentiles, }\AttributeTok{mapping =} \FunctionTok{aes}\NormalTok{(}\AttributeTok{xintercept =}\NormalTok{ q),}\AttributeTok{lty =} \StringTok{"dashed"}\NormalTok{) }\SpecialCharTok{+}
    \FunctionTok{geom\_line}\NormalTok{() }\SpecialCharTok{+}
    \FunctionTok{geom\_segment}\NormalTok{(}\AttributeTok{data =}\NormalTok{ distrib\_percentiles, }\FunctionTok{aes}\NormalTok{(}\AttributeTok{x =}\NormalTok{ value, }\AttributeTok{y =} \DecValTok{0}\NormalTok{, }\AttributeTok{yend =}
\NormalTok{                                                   density)) }\SpecialCharTok{+}
    \FunctionTok{theme\_bw}\NormalTok{() }\SpecialCharTok{+}
    \FunctionTok{scale\_color\_brewer}\NormalTok{(}\StringTok{"Distribution"}\NormalTok{, }\AttributeTok{palette =} \StringTok{"Dark2"}\NormalTok{) }\SpecialCharTok{+}
    \FunctionTok{ylab}\NormalTok{(}\StringTok{"Density"}\NormalTok{) }\SpecialCharTok{+}
    \FunctionTok{xlab}\NormalTok{(}\FunctionTok{paste0}\NormalTok{(cov\_name, }\StringTok{" ("}\NormalTok{, cov\_unit, }\StringTok{")"}\NormalTok{)) }\SpecialCharTok{+}
    \FunctionTok{annotate}\NormalTok{(}
      \AttributeTok{geom =} \StringTok{"text"}\NormalTok{,}
      \AttributeTok{label =} \FunctionTok{paste0}\NormalTok{(}
        \StringTok{"N \textasciitilde{} ("}\NormalTok{,}
        \FunctionTok{signif}\NormalTok{(norm\_par[}\StringTok{"mean"}\NormalTok{], }\DecValTok{3}\NormalTok{),}
        \StringTok{", "}\NormalTok{,}
        \FunctionTok{signif}\NormalTok{(norm\_par[}\StringTok{"sd"}\NormalTok{], }\DecValTok{3}\NormalTok{),}
        \StringTok{")}\SpecialCharTok{\textbackslash{}n}\StringTok{"}\NormalTok{,}
        \StringTok{"LN \textasciitilde{} ("}\NormalTok{,}
        \FunctionTok{signif}\NormalTok{(}\FunctionTok{exp}\NormalTok{(lnorm\_par[}\StringTok{"meanlog"}\NormalTok{]), }\DecValTok{3}\NormalTok{),}
        \StringTok{", "}\NormalTok{,}
        \FunctionTok{signif}\NormalTok{(}\FunctionTok{compute\_cv\_from\_sd\_lnorm}\NormalTok{(lnorm\_par[}\StringTok{"sdlog"}\NormalTok{]), }\DecValTok{3}\NormalTok{) }\SpecialCharTok{*} \DecValTok{100}\NormalTok{,}
        \StringTok{" \%)"}
\NormalTok{      ),}
      \AttributeTok{x =} \FunctionTok{min}\NormalTok{(param\_data}\SpecialCharTok{$}\NormalTok{variable) }\SpecialCharTok{+} \FloatTok{0.99} \SpecialCharTok{*} \FunctionTok{diff}\NormalTok{(}\FunctionTok{range}\NormalTok{(param\_data}\SpecialCharTok{$}\NormalTok{variable)),}
      \AttributeTok{y =} \FunctionTok{min}\NormalTok{(param\_data}\SpecialCharTok{$}\NormalTok{value) }\SpecialCharTok{+} \FloatTok{0.99} \SpecialCharTok{*} \FunctionTok{diff}\NormalTok{(}\FunctionTok{range}\NormalTok{(param\_data}\SpecialCharTok{$}\NormalTok{value)),}
      \AttributeTok{hjust =} \DecValTok{1}\NormalTok{,}
      \AttributeTok{vjust =} \DecValTok{1}
\NormalTok{    )}\SpecialCharTok{+}
    \FunctionTok{geom\_text}\NormalTok{(}
      \AttributeTok{data =}\NormalTok{ true\_percentiles,}
      \AttributeTok{mapping =} \FunctionTok{aes}\NormalTok{(}\AttributeTok{x =}\NormalTok{ q,}\AttributeTok{label =}\NormalTok{ label,}\AttributeTok{color=}\ConstantTok{NULL}\NormalTok{),}
      \AttributeTok{show.legend =} \ConstantTok{FALSE}\NormalTok{,}
      \AttributeTok{y =} \FunctionTok{min}\NormalTok{(param\_data}\SpecialCharTok{$}\NormalTok{value) }\SpecialCharTok{+} \FloatTok{0.95} \SpecialCharTok{*} \FunctionTok{diff}\NormalTok{(}\FunctionTok{range}\NormalTok{(param\_data}\SpecialCharTok{$}\NormalTok{value)),}
      \AttributeTok{hjust =} \DecValTok{1}\NormalTok{,}
      \AttributeTok{vjust =} \SpecialCharTok{{-}}\DecValTok{1}\NormalTok{,}
      \AttributeTok{angle =} \DecValTok{90}
\NormalTok{    )}\SpecialCharTok{+}
    \FunctionTok{geom\_text}\NormalTok{(}
      \AttributeTok{data =}\NormalTok{ distrib\_percentiles,}
      \AttributeTok{mapping =} \FunctionTok{aes}\NormalTok{(}\AttributeTok{x =}\NormalTok{ value, }\AttributeTok{y=}\NormalTok{density,}\AttributeTok{label =}\NormalTok{ label),}
      \AttributeTok{show.legend =} \ConstantTok{FALSE}\NormalTok{,}
      \AttributeTok{hjust =} \DecValTok{1}\NormalTok{,}
      \AttributeTok{vjust =} \SpecialCharTok{{-}}\DecValTok{1}\NormalTok{,}
      \AttributeTok{angle =} \DecValTok{90}
\NormalTok{    )}
  
  \FunctionTok{list}\NormalTok{(}\AttributeTok{plot =}\NormalTok{ plot,}
       \AttributeTok{lnorm\_par =}\NormalTok{ lnorm\_par,}
       \AttributeTok{norm\_par =}\NormalTok{ norm\_par)}
  
\NormalTok{\}}

\NormalTok{fit\_cov\_wrapper }\OtherTok{\textless{}{-}} \ControlFlowTok{function}\NormalTok{(cov\_data, cov\_name, min\_plot, max\_plot, }\AttributeTok{tol =} \FloatTok{0.001}\NormalTok{) \{}
  
  \CommentTok{\#\textquotesingle{} Wrapper to get covariate quantiles from covariate characteristics table and fit normal and lognormal distribution}
  \CommentTok{\#\textquotesingle{} to observed quantiles }
  \CommentTok{\#\textquotesingle{}}
  \CommentTok{\#\textquotesingle{} @param cov\_data tibble containing the covariate information}
  \CommentTok{\#\textquotesingle{} @param cov\_name character. name of the covariate to select, comes from the Covariate column of cov\_data.}
  \CommentTok{\#\textquotesingle{} @param min\_plot minimal value of the covariate for which the distribution should be plotted}
  \CommentTok{\#\textquotesingle{} @param max\_plot maximal value of the covariate for which the distribution should be plotted}
  \CommentTok{\#\textquotesingle{} @param tol numeric. tolerance value for the fitting algorithm, default is 0.001}
  \CommentTok{\#\textquotesingle{} @return list with :}
  \CommentTok{\#\textquotesingle{}  {-} plot : ggplot2 showing the fitted distributions and the observed covariate}
  \CommentTok{\#\textquotesingle{}  {-} lnorm\_par : named numeric vector with the lognormal distribution parameters}
  \CommentTok{\#\textquotesingle{}  {-} norm\_par : named numeric vector with the normal distribution parameters}
  
  
\NormalTok{  cov\_df }\OtherTok{\textless{}{-}} \FunctionTok{extract\_cov\_param}\NormalTok{(cov\_data, cov\_name)}
  
  \FunctionTok{evaluate\_distrib}\NormalTok{(}
    \AttributeTok{p =}\NormalTok{ cov\_df}\SpecialCharTok{$}\NormalTok{p,}
    \AttributeTok{q =}\NormalTok{ cov\_df}\SpecialCharTok{$}\NormalTok{q,}
    \AttributeTok{min\_plot =}\NormalTok{ min\_plot,}
    \AttributeTok{max\_plot =}\NormalTok{ max\_plot,}
    \AttributeTok{cov\_name =}\NormalTok{ cov\_df}\SpecialCharTok{$}\NormalTok{cov\_name,}
    \AttributeTok{cov\_unit =}\NormalTok{ cov\_df}\SpecialCharTok{$}\NormalTok{cov\_unit,}
    \AttributeTok{tol =}\NormalTok{ tol}
\NormalTok{  )}
  
\NormalTok{\}}
\end{Highlighting}
\end{Shaded}

\begin{Shaded}
\begin{Highlighting}[]
\NormalTok{carlier\_crcl }\OtherTok{\textless{}{-}} \FunctionTok{fit\_cov\_wrapper}\NormalTok{(}
  \AttributeTok{cov\_data =}\NormalTok{ cov\_data\_carlier,}
  \AttributeTok{cov\_name =} \StringTok{"CRCL"}\NormalTok{,}
  \AttributeTok{min\_plot =} \DecValTok{0}\NormalTok{,}
  \AttributeTok{max\_plot =} \DecValTok{250}
\NormalTok{)}

\NormalTok{carlier\_crcl}\SpecialCharTok{$}\NormalTok{plot}
\end{Highlighting}
\end{Shaded}

\pandocbounded{\includegraphics[keepaspectratio]{Simulations/covariate_simulation_files/figure-pdf/fit-carlier-crcl-1.pdf}}

Normal distribution seems better than log-normal.

\begin{Shaded}
\begin{Highlighting}[]
\NormalTok{carlier\_wt }\OtherTok{\textless{}{-}} \FunctionTok{fit\_cov\_wrapper}\NormalTok{(}
  \AttributeTok{cov\_data =}\NormalTok{ cov\_data\_carlier,}
  \AttributeTok{cov\_name =} \StringTok{"WT"}\NormalTok{,}
  \AttributeTok{min\_plot =} \DecValTok{40}\NormalTok{,}
  \AttributeTok{max\_plot =} \DecValTok{120}
\NormalTok{)}

\NormalTok{carlier\_wt}\SpecialCharTok{$}\NormalTok{plot}
\end{Highlighting}
\end{Shaded}

\pandocbounded{\includegraphics[keepaspectratio]{Simulations/covariate_simulation_files/figure-pdf/fit-carlier-wt-1.pdf}}

No discernable difference in fit. We have to use the log-normal
distribution to be able to calculate height.

\begin{Shaded}
\begin{Highlighting}[]
\NormalTok{carlier\_age }\OtherTok{\textless{}{-}} \FunctionTok{fit\_cov\_wrapper}\NormalTok{(}
  \AttributeTok{cov\_data =}\NormalTok{ cov\_data\_carlier,}
  \AttributeTok{cov\_name =} \StringTok{"AGE"}\NormalTok{,}
  \AttributeTok{min\_plot =} \DecValTok{0}\NormalTok{,}
  \AttributeTok{max\_plot =} \DecValTok{120}
\NormalTok{)}

\NormalTok{carlier\_age}\SpecialCharTok{$}\NormalTok{plot}
\end{Highlighting}
\end{Shaded}

\pandocbounded{\includegraphics[keepaspectratio]{Simulations/covariate_simulation_files/figure-pdf/fit-carlier-age-1.pdf}}

No discernible difference in fit. We will use the normal distribution as
generally, age is normally distributed.

\begin{Shaded}
\begin{Highlighting}[]
\NormalTok{carlier\_bmi }\OtherTok{\textless{}{-}} \FunctionTok{fit\_cov\_wrapper}\NormalTok{(}
  \AttributeTok{cov\_data =}\NormalTok{ cov\_data\_carlier,}
  \AttributeTok{cov\_name =} \StringTok{"BMI"}\NormalTok{,}
  \AttributeTok{min\_plot =} \DecValTok{0}\NormalTok{,}
  \AttributeTok{max\_plot =} \DecValTok{40}\NormalTok{,}
  \AttributeTok{tol=}\FloatTok{0.002}
\NormalTok{)}

\NormalTok{carlier\_bmi}\SpecialCharTok{$}\NormalTok{plot}
\end{Highlighting}
\end{Shaded}

\pandocbounded{\includegraphics[keepaspectratio]{Simulations/covariate_simulation_files/figure-pdf/fit-carlier-bmi-1.pdf}}

Tolerance was increased to 0.002. No discernible difference in fit. We
have to use the log-normal distribution to be able to obtain height.

\subsection{Simulation of covariates}\label{simulation-of-covariates}

Covariates will be sampled from the following distributions :

\begin{itemize}
\tightlist
\item
  CRCL : Normal distribution with a minimum of 10 mL/min, with mean 105
  mL/min and standard deviation 68.5 mL/min
\item
  WT : Lognormal distribution with mean 74.6 kg, and coefficient of
  variation 9.04 \%
\item
  AGE : Normal distribution with mean 63.7 mL/min and standard deviation
  11 years and with a minimum of 18 years
\item
  BMI : Log-normal distribution with mean 3.15 mL/min and standard
  deviation 0.141 kg/m\^{}\{2\}
\end{itemize}

\begin{Shaded}
\begin{Highlighting}[]
\NormalTok{sim\_norm\_without\_negatives }\OtherTok{\textless{}{-}} \ControlFlowTok{function}\NormalTok{(n,mean,sd)\{}
      
  \CommentTok{\#\textquotesingle{} Simulate values from normal distribution but resample negative values}
  \CommentTok{\#\textquotesingle{}}
  \CommentTok{\#\textquotesingle{} @param n numeric. number of samples to simulate}
  \CommentTok{\#\textquotesingle{} @param mean numeric. mean of the normal distribution}
  \CommentTok{\#\textquotesingle{} @param sd numeric. standard deviation of the normal distribtuion}
  \CommentTok{\#\textquotesingle{} @return numeric vector of n values}
\NormalTok{  sim\_values }\OtherTok{\textless{}{-}} \FunctionTok{rnorm}\NormalTok{(n,mean,sd)}
\NormalTok{  neg\_values }\OtherTok{\textless{}{-}}\NormalTok{ sim\_values[sim\_values}\SpecialCharTok{\textless{}=}\DecValTok{0}\NormalTok{]}
\NormalTok{  pos\_values }\OtherTok{\textless{}{-}}\NormalTok{ sim\_values[sim\_values}\SpecialCharTok{\textgreater{}}\DecValTok{0}\NormalTok{]}

  \ControlFlowTok{while}\NormalTok{(}\FunctionTok{length}\NormalTok{(neg\_values)}\SpecialCharTok{\textgreater{}}\DecValTok{0}\NormalTok{)\{}
\NormalTok{    new\_values }\OtherTok{\textless{}{-}} \FunctionTok{rnorm}\NormalTok{(}\FunctionTok{length}\NormalTok{(neg\_values),mean,sd)}
\NormalTok{    sim\_values }\OtherTok{\textless{}{-}} \FunctionTok{c}\NormalTok{(pos\_values,new\_values)}
\NormalTok{    neg\_values }\OtherTok{\textless{}{-}}\NormalTok{ sim\_values[sim\_values}\SpecialCharTok{\textless{}=}\DecValTok{0}\NormalTok{]}
\NormalTok{    pos\_values }\OtherTok{\textless{}{-}}\NormalTok{ sim\_values[sim\_values}\SpecialCharTok{\textgreater{}}\DecValTok{0}\NormalTok{]}
\NormalTok{    \}}
\NormalTok{  sim\_values}
\NormalTok{\}}

\CommentTok{\# Only for normal distribution when the minimum is 10 ml/min, resample below 10}
\NormalTok{sim\_norm\_without\_dialysis }\OtherTok{\textless{}{-}} \ControlFlowTok{function}\NormalTok{(n,mean,sd)\{}
      
  \CommentTok{\#\textquotesingle{} Simulate values from normal distribution but resample negative values}
  \CommentTok{\#\textquotesingle{}}
  \CommentTok{\#\textquotesingle{} @param n numeric. number of samples to simulate}
  \CommentTok{\#\textquotesingle{} @param mean numeric. mean of the normal distribution}
  \CommentTok{\#\textquotesingle{} @param sd numeric. standard deviation of the normal distribtuion}
  \CommentTok{\#\textquotesingle{} @return numeric vector of n values}
\NormalTok{  sim\_values }\OtherTok{\textless{}{-}} \FunctionTok{rnorm}\NormalTok{(n,mean,sd)}
\NormalTok{  neg\_values }\OtherTok{\textless{}{-}}\NormalTok{ sim\_values[sim\_values}\SpecialCharTok{\textless{}=}\DecValTok{10}\NormalTok{]}
\NormalTok{  pos\_values }\OtherTok{\textless{}{-}}\NormalTok{ sim\_values[sim\_values}\SpecialCharTok{\textgreater{}}\DecValTok{10}\NormalTok{]}

  \ControlFlowTok{while}\NormalTok{(}\FunctionTok{length}\NormalTok{(neg\_values)}\SpecialCharTok{\textgreater{}}\DecValTok{0}\NormalTok{)\{}
\NormalTok{    new\_values }\OtherTok{\textless{}{-}} \FunctionTok{rnorm}\NormalTok{(}\FunctionTok{length}\NormalTok{(neg\_values),mean,sd)}
\NormalTok{    sim\_values }\OtherTok{\textless{}{-}} \FunctionTok{c}\NormalTok{(pos\_values,new\_values)}
\NormalTok{    neg\_values }\OtherTok{\textless{}{-}}\NormalTok{ sim\_values[sim\_values}\SpecialCharTok{\textless{}=}\DecValTok{10}\NormalTok{]}
\NormalTok{    pos\_values }\OtherTok{\textless{}{-}}\NormalTok{ sim\_values[sim\_values}\SpecialCharTok{\textgreater{}}\DecValTok{10}\NormalTok{]}
\NormalTok{    \}}
\NormalTok{  sim\_values}
\NormalTok{\}}
\CommentTok{\# Resimulate subjects younger than 18 years for normally distributed age}
\NormalTok{sim\_norm\_without\_minors }\OtherTok{\textless{}{-}} \ControlFlowTok{function}\NormalTok{(n,mean,sd)\{}
      
  \CommentTok{\#\textquotesingle{} Simulate values from normal distribution but resample negative values}
  \CommentTok{\#\textquotesingle{}}
  \CommentTok{\#\textquotesingle{} @param n numeric. number of samples to simulate}
  \CommentTok{\#\textquotesingle{} @param mean numeric. mean of the normal distribution}
  \CommentTok{\#\textquotesingle{} @param sd numeric. standard deviation of the normal distribtuion}
  \CommentTok{\#\textquotesingle{} @return numeric vector of n values}
\NormalTok{  sim\_values }\OtherTok{\textless{}{-}} \FunctionTok{rnorm}\NormalTok{(n,mean,sd)}
\NormalTok{  neg\_values }\OtherTok{\textless{}{-}}\NormalTok{ sim\_values[sim\_values}\SpecialCharTok{\textless{}}\DecValTok{18}\NormalTok{]}
\NormalTok{  pos\_values }\OtherTok{\textless{}{-}}\NormalTok{ sim\_values[sim\_values}\SpecialCharTok{\textgreater{}=}\DecValTok{18}\NormalTok{]}

  \ControlFlowTok{while}\NormalTok{(}\FunctionTok{length}\NormalTok{(neg\_values)}\SpecialCharTok{\textgreater{}}\DecValTok{0}\NormalTok{)\{}
\NormalTok{    new\_values }\OtherTok{\textless{}{-}} \FunctionTok{rnorm}\NormalTok{(}\FunctionTok{length}\NormalTok{(neg\_values),mean,sd)}
\NormalTok{    sim\_values }\OtherTok{\textless{}{-}} \FunctionTok{c}\NormalTok{(pos\_values,new\_values)}
\NormalTok{    neg\_values }\OtherTok{\textless{}{-}}\NormalTok{ sim\_values[sim\_values}\SpecialCharTok{\textless{}}\DecValTok{18}\NormalTok{]}
\NormalTok{    pos\_values }\OtherTok{\textless{}{-}}\NormalTok{ sim\_values[sim\_values}\SpecialCharTok{\textgreater{}=}\DecValTok{18}\NormalTok{]}
\NormalTok{    \}}
\NormalTok{  sim\_values}
\NormalTok{\}}
\end{Highlighting}
\end{Shaded}

The relevant part of the correlation matrix is converted to a covariance
matrix by multiplying the already calculated standard deviations of the
variables with the diagonal of the correlation matrix. More information
can be found at
\href{https://blogs.sas.com/content/iml/2010/12/10/converting-between-correlation-and-covariance-matrices.html}{SAS}
and at
\href{https://stats.stackexchange.com/questions/62850/obtaining-covariance-matrix-from-correlation-matrix}{stats}.
A multivariate normal distribution is fitted with resimulation of CRCL
below 10 ml/min, AGE below 18 years, and other variables below 0.

The proportion of males is 0.85 in the original article, so the number
of males is 531 and the number of females is 94 to add up to 625.

\begin{Shaded}
\begin{Highlighting}[]
\NormalTok{mean\_WT\_log }\OtherTok{\textless{}{-}} \FunctionTok{as.numeric}\NormalTok{(carlier\_wt}\SpecialCharTok{$}\NormalTok{lnorm\_par[}\StringTok{"meanlog"}\NormalTok{])}
\NormalTok{sd\_WT\_log }\OtherTok{\textless{}{-}} \FunctionTok{as.numeric}\NormalTok{(carlier\_wt}\SpecialCharTok{$}\NormalTok{lnorm\_par[}\StringTok{"sdlog"}\NormalTok{])}
\NormalTok{mean\_BMI\_log }\OtherTok{\textless{}{-}} \FunctionTok{as.numeric}\NormalTok{(carlier\_bmi}\SpecialCharTok{$}\NormalTok{lnorm\_par[}\StringTok{"meanlog"}\NormalTok{])}
\NormalTok{sd\_BMI\_log }\OtherTok{\textless{}{-}} \FunctionTok{as.numeric}\NormalTok{(carlier\_bmi}\SpecialCharTok{$}\NormalTok{lnorm\_par[}\StringTok{"sdlog"}\NormalTok{])}
\NormalTok{r\_M }\OtherTok{\textless{}{-}} \FloatTok{0.34} \CommentTok{\# Pearson correlation between HT\_log and WT\_log for males}
\NormalTok{r\_F }\OtherTok{\textless{}{-}} \FloatTok{0.36} \CommentTok{\# Pearson correlation for females}

\CommentTok{\# sd is different based on sex}
\NormalTok{mean\_HT\_log }\OtherTok{=}\NormalTok{ (mean\_WT\_log }\SpecialCharTok{{-}}\NormalTok{ mean\_BMI\_log) }\SpecialCharTok{/} \DecValTok{2}
\NormalTok{sd\_HT\_M\_log }\OtherTok{=}\NormalTok{ (}\SpecialCharTok{{-}}\NormalTok{r\_M }\SpecialCharTok{*}\NormalTok{ sd\_WT\_log }\SpecialCharTok{+} \FunctionTok{sqrt}\NormalTok{(r\_M}\SpecialCharTok{\^{}}\DecValTok{2} \SpecialCharTok{*}\NormalTok{ sd\_WT\_log}\SpecialCharTok{\^{}}\DecValTok{2} \SpecialCharTok{{-}}\NormalTok{ sd\_WT\_log}\SpecialCharTok{\^{}}\DecValTok{2} \SpecialCharTok{+}\NormalTok{ sd\_BMI\_log}\SpecialCharTok{\^{}}\DecValTok{2}\NormalTok{))}\SpecialCharTok{/}\DecValTok{2} \CommentTok{\# quadratic equation with positive result}
\NormalTok{sd\_HT\_F\_log }\OtherTok{=}\NormalTok{ (}\SpecialCharTok{{-}}\NormalTok{r\_F }\SpecialCharTok{*}\NormalTok{ sd\_WT\_log }\SpecialCharTok{+} \FunctionTok{sqrt}\NormalTok{(r\_F}\SpecialCharTok{\^{}}\DecValTok{2} \SpecialCharTok{*}\NormalTok{ sd\_WT\_log}\SpecialCharTok{\^{}}\DecValTok{2} \SpecialCharTok{{-}}\NormalTok{ sd\_WT\_log}\SpecialCharTok{\^{}}\DecValTok{2} \SpecialCharTok{+}\NormalTok{ sd\_BMI\_log}\SpecialCharTok{\^{}}\DecValTok{2}\NormalTok{))}\SpecialCharTok{/}\DecValTok{2} \CommentTok{\# quadratic equation with positive}

\NormalTok{correlated\_simulation }\OtherTok{\textless{}{-}} \ControlFlowTok{function}\NormalTok{(n, means, cov\_matrix) \{}
  \FunctionTok{set.seed}\NormalTok{(}\DecValTok{1991}\NormalTok{)}
\NormalTok{  collected }\OtherTok{\textless{}{-}} \FunctionTok{matrix}\NormalTok{(}\ConstantTok{NA}\NormalTok{, }\DecValTok{0}\NormalTok{, }\FunctionTok{length}\NormalTok{(means))}
  \FunctionTok{colnames}\NormalTok{(collected) }\OtherTok{\textless{}{-}} \FunctionTok{names}\NormalTok{(means)}

  \ControlFlowTok{while}\NormalTok{ (}\FunctionTok{nrow}\NormalTok{(collected) }\SpecialCharTok{\textless{}}\NormalTok{ n) \{}
\NormalTok{    batch }\OtherTok{\textless{}{-}}\NormalTok{ MASS}\SpecialCharTok{::}\FunctionTok{mvrnorm}\NormalTok{(}\AttributeTok{n =}\NormalTok{ n, }\AttributeTok{mu =}\NormalTok{ means, }\AttributeTok{Sigma =}\NormalTok{ cov\_matrix)}
    \FunctionTok{colnames}\NormalTok{(batch) }\OtherTok{\textless{}{-}} \FunctionTok{names}\NormalTok{(means)}

\NormalTok{    valid }\OtherTok{\textless{}{-}}\NormalTok{ batch[, }\StringTok{"AGE"}\NormalTok{] }\SpecialCharTok{\textgreater{}=} \DecValTok{18} \SpecialCharTok{\&}
\NormalTok{             batch[, }\StringTok{"CRCL\_MDRD"}\NormalTok{] }\SpecialCharTok{\textgreater{}=} \DecValTok{10}

\NormalTok{    batch\_valid }\OtherTok{\textless{}{-}}\NormalTok{ batch[valid, , drop }\OtherTok{=} \ConstantTok{FALSE}\NormalTok{]}
\NormalTok{    collected }\OtherTok{\textless{}{-}} \FunctionTok{rbind}\NormalTok{(collected, batch\_valid)}
\NormalTok{  \}}

\NormalTok{  collected[}\DecValTok{1}\SpecialCharTok{:}\NormalTok{n, , drop }\OtherTok{=} \ConstantTok{FALSE}\NormalTok{]}
\NormalTok{\}}

\NormalTok{means }\OtherTok{\textless{}{-}} \FunctionTok{c}\NormalTok{(}
  \AttributeTok{CRCL\_MDRD =}\NormalTok{ carlier\_crcl}\SpecialCharTok{$}\NormalTok{norm\_par[}\StringTok{"mean"}\NormalTok{],}
  \AttributeTok{WT\_log      =}\NormalTok{ carlier\_wt}\SpecialCharTok{$}\NormalTok{lnorm\_par[}\StringTok{"meanlog"}\NormalTok{],}
  \AttributeTok{AGE     =}\NormalTok{ carlier\_age}\SpecialCharTok{$}\NormalTok{norm\_par[}\StringTok{"mean"}\NormalTok{],}
  \AttributeTok{HT\_log     =}\NormalTok{ mean\_HT\_log}
\NormalTok{)}

\NormalTok{sds\_M }\OtherTok{\textless{}{-}} \FunctionTok{c}\NormalTok{(}
  \AttributeTok{CRCL\_MDRD =}\NormalTok{ carlier\_crcl}\SpecialCharTok{$}\NormalTok{norm\_par[}\StringTok{"sd"}\NormalTok{],}
  \AttributeTok{WT\_log      =}\NormalTok{ carlier\_wt}\SpecialCharTok{$}\NormalTok{lnorm\_par[}\StringTok{"sdlog"}\NormalTok{], }
  \AttributeTok{AGE     =}\NormalTok{ carlier\_age}\SpecialCharTok{$}\NormalTok{norm\_par[}\StringTok{"sd"}\NormalTok{],}
  \AttributeTok{HT\_log     =}\NormalTok{ sd\_HT\_M\_log}
\NormalTok{)}

\NormalTok{sds\_F }\OtherTok{\textless{}{-}} \FunctionTok{c}\NormalTok{(}
  \AttributeTok{CRCL\_MDRD =}\NormalTok{ carlier\_crcl}\SpecialCharTok{$}\NormalTok{norm\_par[}\StringTok{"sd"}\NormalTok{],}
  \AttributeTok{WT\_log      =}\NormalTok{ carlier\_wt}\SpecialCharTok{$}\NormalTok{lnorm\_par[}\StringTok{"sdlog"}\NormalTok{], }
  \AttributeTok{AGE     =}\NormalTok{ carlier\_age}\SpecialCharTok{$}\NormalTok{norm\_par[}\StringTok{"sd"}\NormalTok{],}
  \AttributeTok{HT\_log     =}\NormalTok{ sd\_HT\_F\_log}
\NormalTok{)}

\NormalTok{covariates }\OtherTok{\textless{}{-}} \FunctionTok{c}\NormalTok{(}\StringTok{"CRCL\_MDRD"}\NormalTok{, }\StringTok{"WT\_log"}\NormalTok{, }\StringTok{"AGE"}\NormalTok{, }\StringTok{"HT\_log"}\NormalTok{)}

\FunctionTok{names}\NormalTok{(means) }\OtherTok{\textless{}{-}}\NormalTok{ covariates}

\CommentTok{\# Correlation and Covariance Matrix}
\NormalTok{cor\_carlier\_M }\OtherTok{\textless{}{-}}\NormalTok{ cor\_matrix\_Carlier\_M[covariates, covariates]}
\NormalTok{cov\_matrix\_M }\OtherTok{\textless{}{-}} \FunctionTok{diag}\NormalTok{(sds\_M) }\SpecialCharTok{\%*\%}\NormalTok{ cor\_carlier\_M }\SpecialCharTok{\%*\%} \FunctionTok{diag}\NormalTok{(sds\_M)}
\FunctionTok{eigen}\NormalTok{(cov\_matrix\_M)}\SpecialCharTok{$}\NormalTok{values}
\end{Highlighting}
\end{Shaded}

\begin{verbatim}
[1] 4.705706e+03 1.077951e+02 8.326780e-03 1.446538e-03
\end{verbatim}

\begin{Shaded}
\begin{Highlighting}[]
\NormalTok{cor\_carlier\_F }\OtherTok{\textless{}{-}}\NormalTok{ cor\_matrix\_Carlier\_F[covariates, covariates]}
\NormalTok{cov\_matrix\_F }\OtherTok{\textless{}{-}} \FunctionTok{diag}\NormalTok{(sds\_F) }\SpecialCharTok{\%*\%}\NormalTok{ cor\_carlier\_F }\SpecialCharTok{\%*\%} \FunctionTok{diag}\NormalTok{(sds\_F)}
\FunctionTok{eigen}\NormalTok{(cov\_matrix\_F)}\SpecialCharTok{$}\NormalTok{values}
\end{Highlighting}
\end{Shaded}

\begin{verbatim}
[1] 4.708164e+03 1.053367e+02 8.259148e-03 1.338558e-03
\end{verbatim}

\begin{Shaded}
\begin{Highlighting}[]
\CommentTok{\# Simulate}
\NormalTok{sim\_data\_M }\OtherTok{\textless{}{-}} \FunctionTok{correlated\_simulation}\NormalTok{(}\AttributeTok{n =} \DecValTok{531}\NormalTok{, }\AttributeTok{means =}\NormalTok{ means, }\AttributeTok{cov\_matrix =}\NormalTok{ cov\_matrix\_M)}
\NormalTok{sim\_data\_M }\OtherTok{\textless{}{-}} \FunctionTok{as\_tibble}\NormalTok{(sim\_data\_M) }\SpecialCharTok{\%\textgreater{}\%}
  \FunctionTok{mutate}\NormalTok{(}\AttributeTok{SEX =} \DecValTok{0}\NormalTok{)}

\NormalTok{sim\_data\_F }\OtherTok{\textless{}{-}} \FunctionTok{correlated\_simulation}\NormalTok{(}\AttributeTok{n =} \DecValTok{94}\NormalTok{, }\AttributeTok{means =}\NormalTok{ means, }\AttributeTok{cov\_matrix =}\NormalTok{ cov\_matrix\_F)}
\NormalTok{sim\_data\_F }\OtherTok{\textless{}{-}} \FunctionTok{as\_tibble}\NormalTok{(sim\_data\_F) }\SpecialCharTok{\%\textgreater{}\%}
  \FunctionTok{mutate}\NormalTok{(}\AttributeTok{SEX =} \DecValTok{1}\NormalTok{)}

\NormalTok{sim\_data }\OtherTok{\textless{}{-}} \FunctionTok{rbind}\NormalTok{(sim\_data\_M, sim\_data\_F)}

\CommentTok{\# Put the covariates together}
\NormalTok{sim\_cov\_carlier }\OtherTok{\textless{}{-}} \FunctionTok{tibble}\NormalTok{(}
  \AttributeTok{Paper =} \StringTok{"Carlier\_2013"}\NormalTok{,}
  \AttributeTok{ID\_within\_paper =} \DecValTok{1}\SpecialCharTok{:}\NormalTok{n\_patient,}
  \AttributeTok{ICU =} \DecValTok{1}\NormalTok{,}
  \AttributeTok{BURN =} \DecValTok{0}\NormalTok{,}
  \AttributeTok{OBESE =} \DecValTok{0}\NormalTok{,}
  \AttributeTok{CRCL =}\NormalTok{ sim\_data}\SpecialCharTok{$}\NormalTok{CRCL\_MDRD,}
  \AttributeTok{WT\_log   =}\NormalTok{ sim\_data}\SpecialCharTok{$}\NormalTok{WT\_log,}
  \AttributeTok{AGE  =}\NormalTok{ sim\_data}\SpecialCharTok{$}\NormalTok{AGE,}
  \AttributeTok{HT\_log  =}\NormalTok{ sim\_data}\SpecialCharTok{$}\NormalTok{HT\_log,}
  \AttributeTok{SEX =}\NormalTok{ sim\_data}\SpecialCharTok{$}\NormalTok{SEX}
\NormalTok{)}

\CommentTok{\# Add OBESE variable and reconvert logWT to WT}
\NormalTok{sim\_cov\_carlier }\OtherTok{\textless{}{-}}\NormalTok{ sim\_cov\_carlier }\SpecialCharTok{\%\textgreater{}\%}
  \FunctionTok{mutate}\NormalTok{(}\AttributeTok{WT =} \FunctionTok{exp}\NormalTok{(WT\_log),}
         \AttributeTok{HT =}\NormalTok{ (}\FunctionTok{exp}\NormalTok{(HT\_log))}\SpecialCharTok{*}\DecValTok{100}\NormalTok{,}
         \AttributeTok{BMI =}\NormalTok{ WT }\SpecialCharTok{/}\NormalTok{ (HT}\SpecialCharTok{/}\DecValTok{100}\NormalTok{)}\SpecialCharTok{\^{}}\DecValTok{2}\NormalTok{,}
         \AttributeTok{OBESE =} \FunctionTok{ifelse}\NormalTok{(BMI }\SpecialCharTok{\textgreater{}} \DecValTok{30}\NormalTok{, }\DecValTok{1}\NormalTok{, }\DecValTok{0}\NormalTok{))}

\NormalTok{summary\_sim\_cov\_carlier }\OtherTok{\textless{}{-}}\NormalTok{ sim\_cov\_carlier }\SpecialCharTok{|\textgreater{}} 
  \FunctionTok{pivot\_longer}\NormalTok{(}\AttributeTok{cols =} \FunctionTok{c}\NormalTok{(ICU,BURN,OBESE,CRCL,WT,AGE,BMI),}
               \AttributeTok{names\_to =} \StringTok{"Covariate"}\NormalTok{) }\SpecialCharTok{|\textgreater{}} 
  \FunctionTok{summarise}\NormalTok{(}\AttributeTok{.by =} \FunctionTok{c}\NormalTok{(Covariate,Paper),}
            \AttributeTok{Min =} \FunctionTok{min}\NormalTok{(value),}
            \AttributeTok{Q1 =} \FunctionTok{quantile}\NormalTok{(value, }\FloatTok{0.25}\NormalTok{),}
            \AttributeTok{Median =} \FunctionTok{quantile}\NormalTok{(value, }\FloatTok{0.5}\NormalTok{),}
            \AttributeTok{Q3 =} \FunctionTok{quantile}\NormalTok{(value, }\FloatTok{0.75}\NormalTok{),}
            \AttributeTok{Max =} \FunctionTok{max}\NormalTok{(value))}
\end{Highlighting}
\end{Shaded}

\begin{Shaded}
\begin{Highlighting}[]
\NormalTok{cov\_carlier\_compare }\OtherTok{\textless{}{-}}\NormalTok{ summary\_sim\_cov\_carlier }\SpecialCharTok{|\textgreater{}}
  \FunctionTok{rename\_with}\NormalTok{(}\SpecialCharTok{\textasciitilde{}} \FunctionTok{paste0}\NormalTok{(.x, }\StringTok{"\_sim"}\NormalTok{), }\SpecialCharTok{{-}}\FunctionTok{one\_of}\NormalTok{(}\StringTok{"Covariate"}\NormalTok{, }\StringTok{"Paper"}\NormalTok{)) }\SpecialCharTok{|\textgreater{}}
  \FunctionTok{left\_join}\NormalTok{(}\FunctionTok{rename\_with}\NormalTok{(}
\NormalTok{    cov\_data\_carlier,}
    \SpecialCharTok{\textasciitilde{}} \FunctionTok{paste0}\NormalTok{(.x, }\StringTok{"\_true"}\NormalTok{),}
    \SpecialCharTok{{-}}\FunctionTok{one\_of}\NormalTok{(}\StringTok{"Covariate"}\NormalTok{, }\StringTok{"Paper"}\NormalTok{, }\StringTok{"Unit"}\NormalTok{)}
\NormalTok{  )) }\SpecialCharTok{|\textgreater{}} 
  \FunctionTok{relocate}\NormalTok{(Covariate, }\AttributeTok{.after =}\NormalTok{ Paper) }\SpecialCharTok{|\textgreater{}} 
  \FunctionTok{relocate}\NormalTok{(Unit, }\AttributeTok{.after =}\NormalTok{ Covariate)}

\NormalTok{cov\_carlier\_compare }\SpecialCharTok{|\textgreater{}}
  \FunctionTok{gt}\NormalTok{() }\SpecialCharTok{|\textgreater{}}
  \FunctionTok{fmt\_scientific}\NormalTok{() }\SpecialCharTok{|\textgreater{}} 
    \FunctionTok{tab\_spanner}\NormalTok{(}\AttributeTok{columns =} \FunctionTok{starts\_with}\NormalTok{(}\StringTok{"Min"}\NormalTok{),}
              \AttributeTok{label =} \StringTok{"Min"}\NormalTok{) }\SpecialCharTok{|\textgreater{}} 
      \FunctionTok{tab\_spanner}\NormalTok{(}\AttributeTok{columns =} \FunctionTok{starts\_with}\NormalTok{(}\StringTok{"Q1"}\NormalTok{),}
              \AttributeTok{label =} \StringTok{"Q1"}\NormalTok{) }\SpecialCharTok{|\textgreater{}} 
      \FunctionTok{tab\_spanner}\NormalTok{(}\AttributeTok{columns =} \FunctionTok{starts\_with}\NormalTok{(}\StringTok{"Median"}\NormalTok{),}
              \AttributeTok{label =} \StringTok{"Median"}\NormalTok{) }\SpecialCharTok{|\textgreater{}} 
      \FunctionTok{tab\_spanner}\NormalTok{(}\AttributeTok{columns =} \FunctionTok{starts\_with}\NormalTok{(}\StringTok{"Q3"}\NormalTok{),}
              \AttributeTok{label =} \StringTok{"Q3"}\NormalTok{) }\SpecialCharTok{|\textgreater{}} 
  \FunctionTok{tab\_spanner}\NormalTok{(}\AttributeTok{columns =} \FunctionTok{starts\_with}\NormalTok{(}\StringTok{"Max"}\NormalTok{),}
              \AttributeTok{label =} \StringTok{"Max"}\NormalTok{)}
\end{Highlighting}
\end{Shaded}

\begin{table}
\fontsize{12.0pt}{14.4pt}\selectfont
\begin{tabular*}{\linewidth}{@{\extracolsep{\fill}}lllrrrrrrrrrr}
\toprule
 &  &  & \multicolumn{2}{c}{Min} & \multicolumn{2}{c}{Q1} & \multicolumn{2}{c}{Median} & \multicolumn{2}{c}{Q3} & \multicolumn{2}{c}{Max} \\ 
\cmidrule(lr){4-5} \cmidrule(lr){6-7} \cmidrule(lr){8-9} \cmidrule(lr){10-11} \cmidrule(lr){12-13}
Paper & Covariate & Unit & Min\_sim & Min\_true & Q1\_sim & Q1\_true & Median\_sim & Median\_true & Q3\_sim & Q3\_true & Max\_sim & Max\_true \\ 
\midrule\addlinespace[2.5pt]
Carlier\_2013 & ICU & Unitless & 1.00 & 1.00 & 1.00 & 1.00 & 1.00 & 1.00 & 1.00 & 1.00 & 1.00 & 1.00 \\ 
Carlier\_2013 & BURN & Unitless & 0.00 & 0.00 & 0.00 & 0.00 & 0.00 & 0.00 & 0.00 & 0.00 & 0.00 & 0.00 \\ 
Carlier\_2013 & OBESE & Unitless & 0.00 & 0.00 & 0.00 & 0.00 & 0.00 & 0.00 & 0.00 & 0.00 & 1.00 & 0.00 \\ 
Carlier\_2013 & CRCL & mL/min & 1.18 $\times$ 10\textsuperscript{1} & 1.00 $\times$ 10\textsuperscript{1} & 6.68 $\times$ 10\textsuperscript{1} & 5.00 $\times$ 10\textsuperscript{1} & 1.15 $\times$ 10\textsuperscript{2} & 1.02 $\times$ 10\textsuperscript{2} & 1.64 $\times$ 10\textsuperscript{2} & 1.57 $\times$ 10\textsuperscript{2} & 3.64 $\times$ 10\textsuperscript{2} & NA \\ 
Carlier\_2013 & WT & kg & 5.89 $\times$ 10\textsuperscript{1} & NA & 7.02 $\times$ 10\textsuperscript{1} & 7.00 $\times$ 10\textsuperscript{1} & 7.47 $\times$ 10\textsuperscript{1} & 7.50 $\times$ 10\textsuperscript{1} & 7.98 $\times$ 10\textsuperscript{1} & 7.90 $\times$ 10\textsuperscript{1} & 1.01 $\times$ 10\textsuperscript{2} & NA \\ 
Carlier\_2013 & AGE & year & 2.84 $\times$ 10\textsuperscript{1} & NA & 5.38 $\times$ 10\textsuperscript{1} & 5.80 $\times$ 10\textsuperscript{1} & 6.30 $\times$ 10\textsuperscript{1} & 6.20 $\times$ 10\textsuperscript{1} & 7.00 $\times$ 10\textsuperscript{1} & 7.20 $\times$ 10\textsuperscript{1} & 9.82 $\times$ 10\textsuperscript{1} & NA \\ 
Carlier\_2013 & BMI & kg/m\textasciicircum{}\{2\} & 1.66 $\times$ 10\textsuperscript{1} & NA & 2.19 $\times$ 10\textsuperscript{1} & 2.10 $\times$ 10\textsuperscript{1} & 2.32 $\times$ 10\textsuperscript{1} & 2.40 $\times$ 10\textsuperscript{1} & 2.50 $\times$ 10\textsuperscript{1} & 2.50 $\times$ 10\textsuperscript{1} & 3.14 $\times$ 10\textsuperscript{1} & NA \\ 
\bottomrule
\end{tabular*}
\end{table}

\begin{Shaded}
\begin{Highlighting}[]
\NormalTok{evaluate\_cov\_sim }\OtherTok{\textless{}{-}} \ControlFlowTok{function}\NormalTok{(p, q, cov\_name, cov\_unit, cov\_sim) \{}
  \CommentTok{\#\textquotesingle{} Plot simulated covariate distribution versus observed quantiles from the paper}
  \CommentTok{\#\textquotesingle{} @param p numeric vector of probabilities}
  \CommentTok{\#\textquotesingle{} @param q numeric vector of quantiles}
  \CommentTok{\#\textquotesingle{} @param cov\_name character string containing the covariate name}
  \CommentTok{\#\textquotesingle{} @param cov\_unit character string containing the unit of the covariate}
  \CommentTok{\#\textquotesingle{} @param cov\_sim tibble containing the simulated values for the covariate}
  \CommentTok{\#\textquotesingle{} @return ggplot2 showing the distributions of the simulated covariate and the observed covariate quantiles}
  
  
  \CommentTok{\#create a dataset with the true percentiles to be able to show them on the plot}
\NormalTok{  true\_percentiles }\OtherTok{\textless{}{-}} \FunctionTok{tibble}\NormalTok{(p, q) }\SpecialCharTok{|\textgreater{}} 
    \FunctionTok{mutate}\NormalTok{(}\AttributeTok{label =} \FunctionTok{case\_match}\NormalTok{(p, }
                              \DecValTok{0} \SpecialCharTok{\textasciitilde{}} \StringTok{"True min"}\NormalTok{,}
                              \FloatTok{0.25} \SpecialCharTok{\textasciitilde{}} \StringTok{"True Q1"}\NormalTok{,}
                              \FloatTok{0.5} \SpecialCharTok{\textasciitilde{}} \StringTok{"True median"}\NormalTok{,}
                              \FloatTok{0.75} \SpecialCharTok{\textasciitilde{}} \StringTok{"True Q3"}\NormalTok{,}
                              \DecValTok{1} \SpecialCharTok{\textasciitilde{}} \StringTok{"True max"}\NormalTok{)) }
  
  \CommentTok{\#create a dataset with the percentiles of the fitted distributions to be able to show them on the plot}
\NormalTok{   sim\_percentiles }\OtherTok{\textless{}{-}}  \FunctionTok{tibble}\NormalTok{(}
    \AttributeTok{probability =}\NormalTok{ p,}
    \AttributeTok{q =} \FunctionTok{quantile}\NormalTok{(}\FunctionTok{pull}\NormalTok{(cov\_sim,cov\_name),p) }
\NormalTok{  ) }\SpecialCharTok{|\textgreater{}} 
        \FunctionTok{mutate}\NormalTok{(}\AttributeTok{label =} \FunctionTok{case\_match}\NormalTok{(probability, }
                              \DecValTok{0} \SpecialCharTok{\textasciitilde{}} \StringTok{"Sim min"}\NormalTok{,}
                              \FloatTok{0.25} \SpecialCharTok{\textasciitilde{}} \StringTok{"Sim Q1"}\NormalTok{,}
                              \FloatTok{0.5} \SpecialCharTok{\textasciitilde{}} \StringTok{"Sim median"}\NormalTok{,}
                              \FloatTok{0.75} \SpecialCharTok{\textasciitilde{}} \StringTok{"Sim Q3"}\NormalTok{,}
                              \DecValTok{1} \SpecialCharTok{\textasciitilde{}} \StringTok{"Sim max"}\NormalTok{))}
   
\CommentTok{\# create an initial histogram in order to be able to extract statistics of the binned covariate}
\NormalTok{hist\_ini }\OtherTok{\textless{}{-}} \FunctionTok{ggplot}\NormalTok{(cov\_sim, }\FunctionTok{aes}\NormalTok{(}\AttributeTok{x =}\NormalTok{ .data[[cov\_name]])) }\SpecialCharTok{+}
  \FunctionTok{geom\_histogram}\NormalTok{(}\AttributeTok{bins=}\DecValTok{15}\NormalTok{)}
   
\NormalTok{hist\_data }\OtherTok{\textless{}{-}} \FunctionTok{layer\_data}\NormalTok{(hist\_ini)}
   
\NormalTok{plot }\OtherTok{\textless{}{-}} \FunctionTok{ggplot}\NormalTok{(hist\_data) }\SpecialCharTok{+}
  \FunctionTok{geom\_rect}\NormalTok{(}\FunctionTok{aes}\NormalTok{(}\AttributeTok{xmin=}\NormalTok{xmin,}\AttributeTok{xmax=}\NormalTok{xmax, }\AttributeTok{ymin=}\DecValTok{0}\NormalTok{, }\AttributeTok{ymax=}\NormalTok{count),}
            \AttributeTok{alpha =} \FloatTok{0.5}\NormalTok{) }\SpecialCharTok{+}
     \FunctionTok{geom\_vline}\NormalTok{(}\AttributeTok{data =}\NormalTok{ true\_percentiles,}
                \AttributeTok{mapping =} \FunctionTok{aes}\NormalTok{(}\AttributeTok{xintercept =}\NormalTok{ q,}\AttributeTok{color =} \StringTok{"True"}\NormalTok{),}
                \AttributeTok{lty =} \StringTok{"dashed"}\NormalTok{) }\SpecialCharTok{+}
     \FunctionTok{geom\_vline}\NormalTok{(}\AttributeTok{data =}\NormalTok{ sim\_percentiles,}
                \AttributeTok{mapping =} \FunctionTok{aes}\NormalTok{(}\AttributeTok{xintercept =}\NormalTok{ q,}\AttributeTok{color =} \StringTok{"Sim"}\NormalTok{),}
                \AttributeTok{lty =} \StringTok{"solid"}\NormalTok{) }\SpecialCharTok{+}
     \FunctionTok{theme\_bw}\NormalTok{() }\SpecialCharTok{+}
     \FunctionTok{ylab}\NormalTok{(}\StringTok{"Count"}\NormalTok{) }\SpecialCharTok{+}
     \FunctionTok{xlab}\NormalTok{(}\FunctionTok{paste0}\NormalTok{(cov\_name, }\StringTok{" ("}\NormalTok{, cov\_unit, }\StringTok{")"}\NormalTok{))  }\SpecialCharTok{+}

     \FunctionTok{geom\_text}\NormalTok{(}
       \AttributeTok{data =}\NormalTok{ true\_percentiles,}
       \AttributeTok{mapping =} \FunctionTok{aes}\NormalTok{(}\AttributeTok{x =}\NormalTok{ q, }\AttributeTok{label =}\NormalTok{ label, }\AttributeTok{color =} \StringTok{"True"}\NormalTok{),}
       \AttributeTok{show.legend =} \ConstantTok{FALSE}\NormalTok{,}
       \AttributeTok{y =} \FloatTok{0.95}\SpecialCharTok{*}\FunctionTok{max}\NormalTok{(hist\_data}\SpecialCharTok{$}\NormalTok{ymax),}
       \AttributeTok{hjust =} \DecValTok{1}\NormalTok{,}
       \AttributeTok{vjust =} \SpecialCharTok{{-}}\DecValTok{1}\NormalTok{,}
       \AttributeTok{angle =} \DecValTok{90}
\NormalTok{     ) }\SpecialCharTok{+}
     \FunctionTok{geom\_text}\NormalTok{(}
       \AttributeTok{data =}\NormalTok{ sim\_percentiles,}
       \AttributeTok{mapping =} \FunctionTok{aes}\NormalTok{(}\AttributeTok{x =}\NormalTok{ q, }\AttributeTok{label =}\NormalTok{ label, }\AttributeTok{color =} \StringTok{"Sim"}\NormalTok{),}
       \AttributeTok{show.legend =} \ConstantTok{FALSE}\NormalTok{,}
       \AttributeTok{y =} \FloatTok{0.95}\SpecialCharTok{*}\FunctionTok{max}\NormalTok{(hist\_data}\SpecialCharTok{$}\NormalTok{ymax),}
       \AttributeTok{hjust =} \DecValTok{1}\NormalTok{,}
       \AttributeTok{vjust =} \SpecialCharTok{{-}}\DecValTok{1}\NormalTok{,}
       \AttributeTok{angle =} \DecValTok{90}
\NormalTok{     ) }\SpecialCharTok{+}
     \FunctionTok{scale\_color\_brewer}\NormalTok{(}\AttributeTok{name =} \StringTok{"Quantiles"}\NormalTok{,}
                        \AttributeTok{palette =}\StringTok{"Dark2"}\NormalTok{)}
\NormalTok{   plot}
\NormalTok{\}}

\NormalTok{plot\_sim\_cov\_wrapper }\OtherTok{\textless{}{-}} \ControlFlowTok{function}\NormalTok{(cov\_data,cov\_name,cov\_sim)\{}
\NormalTok{  cov\_df }\OtherTok{\textless{}{-}} \FunctionTok{extract\_cov\_param}\NormalTok{(cov\_data, cov\_name)}

\FunctionTok{evaluate\_cov\_sim}\NormalTok{(cov\_df}\SpecialCharTok{$}\NormalTok{p,cov\_df}\SpecialCharTok{$}\NormalTok{q,cov\_df}\SpecialCharTok{$}\NormalTok{cov\_name,cov\_df}\SpecialCharTok{$}\NormalTok{cov\_unit,}
\NormalTok{                 cov\_sim)}
\NormalTok{\}}
\end{Highlighting}
\end{Shaded}

\begin{Shaded}
\begin{Highlighting}[]
\FunctionTok{plot\_sim\_cov\_wrapper}\NormalTok{(cov\_data\_carlier,}\StringTok{"CRCL"}\NormalTok{,sim\_cov\_carlier)}
\end{Highlighting}
\end{Shaded}

\pandocbounded{\includegraphics[keepaspectratio]{Simulations/covariate_simulation_files/figure-pdf/plot-cov-carlier-1.pdf}}

\begin{Shaded}
\begin{Highlighting}[]
\FunctionTok{plot\_sim\_cov\_wrapper}\NormalTok{(cov\_data\_carlier,}\StringTok{"WT"}\NormalTok{,sim\_cov\_carlier)}
\end{Highlighting}
\end{Shaded}

\pandocbounded{\includegraphics[keepaspectratio]{Simulations/covariate_simulation_files/figure-pdf/plot-cov-carlier-2.pdf}}

\begin{Shaded}
\begin{Highlighting}[]
\FunctionTok{plot\_sim\_cov\_wrapper}\NormalTok{(cov\_data\_carlier,}\StringTok{"AGE"}\NormalTok{,sim\_cov\_carlier)}
\end{Highlighting}
\end{Shaded}

\pandocbounded{\includegraphics[keepaspectratio]{Simulations/covariate_simulation_files/figure-pdf/plot-cov-carlier-3.pdf}}

\begin{Shaded}
\begin{Highlighting}[]
\FunctionTok{plot\_sim\_cov\_wrapper}\NormalTok{(cov\_data\_carlier,}\StringTok{"BMI"}\NormalTok{,sim\_cov\_carlier)}
\end{Highlighting}
\end{Shaded}

\pandocbounded{\includegraphics[keepaspectratio]{Simulations/covariate_simulation_files/figure-pdf/plot-cov-carlier-4.pdf}}

Then, based on BMI and WT, height (HT) is calculated to be able to
calculate the body surface area (BSA). CRCL was calculated using the
measured urinary equation, and as we have know information about the
urinary volume or urinary creatinine concentration, we cannot reconvert
CRCL to CREAT based on this equation. The most recommended equation is
CKD-EPI, however as different exponents are used based on the value of
CREAT, it can't be used for reconversion of CRCL to CREAT. Thus, the
second best equation, MDRD is used.

\begin{Shaded}
\begin{Highlighting}[]
\NormalTok{reconvert\_MDRD }\OtherTok{\textless{}{-}} \ControlFlowTok{function}\NormalTok{(AGE, CRCL, SEX) \{}
    \CommentTok{\# Mutliply by 0.742 for females}
\NormalTok{    sex\_factor }\OtherTok{\textless{}{-}} \FunctionTok{ifelse}\NormalTok{(SEX }\SpecialCharTok{==} \DecValTok{0}\NormalTok{, }\DecValTok{1}\NormalTok{, }\FloatTok{0.742}\NormalTok{)}
    
    \CommentTok{\# Calculate DFG}
\NormalTok{   CREAT }\OtherTok{\textless{}{-}}\NormalTok{ (CRCL }\SpecialCharTok{/}\NormalTok{ (}\DecValTok{175} \SpecialCharTok{*}\NormalTok{ (AGE}\SpecialCharTok{\^{}}\NormalTok{(}\SpecialCharTok{{-}}\FloatTok{0.203}\NormalTok{)) }\SpecialCharTok{*}\NormalTok{ sex\_factor))}\SpecialCharTok{\^{}}\NormalTok{(}\SpecialCharTok{{-}}\DecValTok{1} \SpecialCharTok{/} \FloatTok{1.154}\NormalTok{)}
  
  \FunctionTok{return}\NormalTok{(CREAT)}
\NormalTok{\}}

\NormalTok{sim\_cov\_carlier }\OtherTok{\textless{}{-}}\NormalTok{ sim\_cov\_carlier }\SpecialCharTok{\%\textgreater{}\%}
  \FunctionTok{mutate}\NormalTok{(}
    \AttributeTok{BSA =}\NormalTok{ (WT}\SpecialCharTok{\^{}}\FloatTok{0.425} \SpecialCharTok{*}\NormalTok{ (HT}\SpecialCharTok{\^{}}\FloatTok{0.725}\NormalTok{) }\SpecialCharTok{*} \FloatTok{0.007184}\NormalTok{), }\CommentTok{\# Calculate body surface area based on Dubois\&Dubois formula}
    \AttributeTok{CRCL =}\NormalTok{ CRCL }\SpecialCharTok{*}\NormalTok{ (}\FloatTok{1.73} \SpecialCharTok{/}\NormalTok{BSA) }\CommentTok{\# convert mL/min to mL/min/1.73 m2}
\NormalTok{  )}
\NormalTok{sim\_cov\_carlier}\SpecialCharTok{$}\NormalTok{CREAT }\OtherTok{\textless{}{-}} \FunctionTok{mapply}\NormalTok{(reconvert\_MDRD, sim\_cov\_carlier}\SpecialCharTok{$}\NormalTok{AGE, sim\_cov\_carlier}\SpecialCharTok{$}\NormalTok{CRCL, sim\_cov\_carlier}\SpecialCharTok{$}\NormalTok{SEX)}
\NormalTok{sim\_cov\_carlier }\OtherTok{\textless{}{-}}\NormalTok{ sim\_cov\_carlier }\SpecialCharTok{\%\textgreater{}\%}
\NormalTok{  dplyr}\SpecialCharTok{::}\FunctionTok{select}\NormalTok{(Paper, ID\_within\_paper, ICU, BURN, OBESE, CREAT, WT, BSA, BMI, AGE, SEX, HT)}

\CommentTok{\# Calculate the simulated correlation matrices separately for the two sexes to compare with the original}
\NormalTok{sim\_cov\_carlier\_M }\OtherTok{\textless{}{-}}\NormalTok{ sim\_cov\_carlier }\SpecialCharTok{\%\textgreater{}\%} \FunctionTok{filter}\NormalTok{(SEX }\SpecialCharTok{==} \DecValTok{0}\NormalTok{)}
\NormalTok{sim\_cov\_carlier\_F }\OtherTok{\textless{}{-}}\NormalTok{ sim\_cov\_carlier }\SpecialCharTok{\%\textgreater{}\%} \FunctionTok{filter}\NormalTok{(SEX }\SpecialCharTok{==} \DecValTok{1}\NormalTok{)}

\NormalTok{cor\_sim\_M }\OtherTok{\textless{}{-}} \FunctionTok{cor}\NormalTok{(sim\_cov\_carlier\_M }\SpecialCharTok{\%\textgreater{}\%}\NormalTok{ dplyr}\SpecialCharTok{::}\FunctionTok{select}\NormalTok{(WT, CREAT, AGE, HT, BMI), }\AttributeTok{use =} \StringTok{"complete.obs"}\NormalTok{)}
\NormalTok{cor\_sim\_F }\OtherTok{\textless{}{-}} \FunctionTok{cor}\NormalTok{(sim\_cov\_carlier\_F }\SpecialCharTok{\%\textgreater{}\%}\NormalTok{ dplyr}\SpecialCharTok{::}\FunctionTok{select}\NormalTok{(WT, CREAT, AGE, HT, BMI), }\AttributeTok{use =} \StringTok{"complete.obs"}\NormalTok{)}

\FunctionTok{ggcorrplot}\NormalTok{(cor\_matrix\_Carlier\_F, }\AttributeTok{lab =} \ConstantTok{TRUE}\NormalTok{, }\AttributeTok{title =} \StringTok{"Carlier original (females)"}\NormalTok{)}
\end{Highlighting}
\end{Shaded}

\pandocbounded{\includegraphics[keepaspectratio]{Simulations/covariate_simulation_files/figure-pdf/reconvert_creat_carlier-1.pdf}}

\begin{Shaded}
\begin{Highlighting}[]
\NormalTok{Carlier\_F }\OtherTok{\textless{}{-}} \FunctionTok{ggcorrplot}\NormalTok{(cor\_sim\_F, }\AttributeTok{lab =} \ConstantTok{TRUE}\NormalTok{, }\AttributeTok{title =} \StringTok{"Carlier simulated (females)"}\NormalTok{)}
\NormalTok{Carlier\_F }
\end{Highlighting}
\end{Shaded}

\pandocbounded{\includegraphics[keepaspectratio]{Simulations/covariate_simulation_files/figure-pdf/reconvert_creat_carlier-2.pdf}}

\begin{Shaded}
\begin{Highlighting}[]
\FunctionTok{ggcorrplot}\NormalTok{(cor\_matrix\_Carlier\_M, }\AttributeTok{lab =} \ConstantTok{TRUE}\NormalTok{, }\AttributeTok{title =} \StringTok{"Carlier original (males)"}\NormalTok{)}
\end{Highlighting}
\end{Shaded}

\pandocbounded{\includegraphics[keepaspectratio]{Simulations/covariate_simulation_files/figure-pdf/reconvert_creat_carlier-3.pdf}}

\begin{Shaded}
\begin{Highlighting}[]
\NormalTok{Carlier\_M }\OtherTok{\textless{}{-}} \FunctionTok{ggcorrplot}\NormalTok{(cor\_sim\_M, }\AttributeTok{lab =} \ConstantTok{TRUE}\NormalTok{, }\AttributeTok{title =} \StringTok{"Carlier simulated (males)"}\NormalTok{)}
\NormalTok{Carlier\_M}
\end{Highlighting}
\end{Shaded}

\pandocbounded{\includegraphics[keepaspectratio]{Simulations/covariate_simulation_files/figure-pdf/reconvert_creat_carlier-4.pdf}}

\section{Fournier et al. (2018)}\label{fournier2018}

\section{General considerations about the
paper}\label{general-considerations-about-the-paper}

This study includes only 21 patients, thus deriving distributions from
the reported covariates values might be difficult

\subsection{Covariate distribution}\label{covariate-distribution-1}

All patients are ICU patients with burns who are assumed to be non
obese. Creatinine clearance is calculated based on the Cockcroft \&
Gault equation, whose output is in mL/min, so there is no need to
calculate BSA. Thus for all patients ICU = 1, BURN = 1 and OBESE=0.

It is not mentioned in the article if certain patients are on renal
replacement therapy, therefore we assume that they are not, and set the
minimum value of creatinine clearance to 10 mL/min. CRCL was estimated
using the Cockroft and Gault equation.

For AGE, not the median and IQR is given but the mean (50.1) and
standard deviation (24.3). As we cannot compare the simulated
distribution to the true one, so we suppose that the distribution
follows a normal distribution.

The proportion of males in the original article is 0.762.

Neither BMI nor HT or BSA are provided in the article, thus the height
is simulated based on the height obtained from the MIMIC clinical
dataset (normal distribution with mean of 169 and sd of 10.3).

For WT, AGE and CRCL here are the data from the paper :

\begin{Shaded}
\begin{Highlighting}[]
\NormalTok{cov\_data\_fournier }\OtherTok{\textless{}{-}}\NormalTok{ cov\_data }\SpecialCharTok{|\textgreater{}} 
  \FunctionTok{filter}\NormalTok{(Paper }\SpecialCharTok{==} \StringTok{"Fournier\_2018"}\NormalTok{)}

\NormalTok{cov\_data\_fournier }\SpecialCharTok{|\textgreater{}} 
  \FunctionTok{filter}\NormalTok{(Covariate }\SpecialCharTok{\%in\%} \FunctionTok{c}\NormalTok{(}\StringTok{"WT"}\NormalTok{,}\StringTok{"CRCL"}\NormalTok{,}\StringTok{"AGE"}\NormalTok{)) }\SpecialCharTok{|\textgreater{}} 
  \FunctionTok{kable}\NormalTok{()}
\end{Highlighting}
\end{Shaded}

\begin{longtable}[]{@{}lllrrrrr@{}}
\toprule\noalign{}
Paper & Covariate & Unit & Median & Q1 & Q3 & Min & Max \\
\midrule\noalign{}
\endhead
\bottomrule\noalign{}
\endlastfoot
Fournier\_2018 & WT & kg & 72.4 & 67 & 83.6 & NA & NA \\
Fournier\_2018 & CRCL & mL/min & 128.0 & 65 & 150.0 & 10 & NA \\
Fournier\_2018 & AGE & year & 50.1 & NA & NA & NA & NA \\
\end{longtable}

\begin{Shaded}
\begin{Highlighting}[]
\NormalTok{cor\_matrix\_Fournier\_F }\OtherTok{\textless{}{-}} \FunctionTok{readRDS}\NormalTok{(}\FunctionTok{here}\NormalTok{(}\StringTok{"a\_priori/For\_publication/Simulations/cor\_matrix\_Fournier\_F.rds"}\NormalTok{))}
\FunctionTok{print}\NormalTok{(cor\_matrix\_Fournier\_F)}
\end{Highlighting}
\end{Shaded}

\begin{verbatim}
              CREAT         AGE         WT     WT_log          HT      HT_log
CREAT    1.00000000  0.11092068  0.1240980  0.1236777 -0.01115272 -0.01163038
AGE      0.11092068  1.00000000 -0.1411092 -0.1377268 -0.21810675 -0.21757157
WT       0.12409798 -0.14110923  1.0000000  0.9910295  0.26291857  0.26261857
WT_log   0.12367767 -0.13772678  0.9910295  1.0000000  0.27210890  0.27214801
HT      -0.01115272 -0.21810675  0.2629186  0.2721089  1.00000000  0.99936169
HT_log  -0.01163038 -0.21757157  0.2626186  0.2721480  0.99936169  1.00000000
BMI      0.13219801 -0.06085883  0.9199729  0.9108563 -0.12595561 -0.12641807
CRCL_CG -0.57789581 -0.60548019  0.3271195  0.3230419  0.20870474  0.20811864
                BMI    CRCL_CG
CREAT    0.13219801 -0.5778958
AGE     -0.06085883 -0.6054802
WT       0.91997291  0.3271195
WT_log   0.91085635  0.3230419
HT      -0.12595561  0.2087047
HT_log  -0.12641807  0.2081186
BMI      1.00000000  0.2518084
CRCL_CG  0.25180837  1.0000000
\end{verbatim}

\begin{Shaded}
\begin{Highlighting}[]
\FunctionTok{ggcorrplot}\NormalTok{(cor\_matrix\_Fournier\_F, }\AttributeTok{lab =} \ConstantTok{TRUE}\NormalTok{, }\AttributeTok{title =} \StringTok{"Sex: female"}\NormalTok{)}
\end{Highlighting}
\end{Shaded}

\pandocbounded{\includegraphics[keepaspectratio]{Simulations/covariate_simulation_files/figure-pdf/Fournier-cor-matrix-1.pdf}}

\begin{Shaded}
\begin{Highlighting}[]
\NormalTok{cor\_matrix\_Fournier\_M }\OtherTok{\textless{}{-}} \FunctionTok{readRDS}\NormalTok{(}\FunctionTok{here}\NormalTok{(}\StringTok{"a\_priori/For\_publication/Simulations/cor\_matrix\_Fournier\_M.rds"}\NormalTok{))}
\FunctionTok{print}\NormalTok{(cor\_matrix\_Fournier\_M)}
\end{Highlighting}
\end{Shaded}

\begin{verbatim}
              CREAT         AGE          WT      WT_log          HT      HT_log
CREAT    1.00000000  0.08421679  0.07887415  0.07757763 -0.01796103 -0.01782471
AGE      0.08421679  1.00000000 -0.11683417 -0.11039865 -0.13268497 -0.13114182
WT       0.07887415 -0.11683417  1.00000000  0.99316445  0.36730646  0.36578272
WT_log   0.07757763 -0.11039865  0.99316445  1.00000000  0.36989373  0.36884368
HT      -0.01796103 -0.13268497  0.36730646  0.36989373  1.00000000  0.99929302
HT_log  -0.01782471 -0.13114182  0.36578272  0.36884368  0.99929302  1.00000000
BMI      0.09421423 -0.05819520  0.87199105  0.86717872 -0.12604976 -0.12807159
CRCL_CG -0.60813562 -0.55803721  0.29302773  0.28921272  0.18756329  0.18568377
                BMI    CRCL_CG
CREAT    0.09421423 -0.6081356
AGE     -0.05819520 -0.5580372
WT       0.87199105  0.2930277
WT_log   0.86717872  0.2892127
HT      -0.12604976  0.1875633
HT_log  -0.12807159  0.1856838
BMI      1.00000000  0.2145950
CRCL_CG  0.21459503  1.0000000
\end{verbatim}

\begin{Shaded}
\begin{Highlighting}[]
\FunctionTok{ggcorrplot}\NormalTok{(cor\_matrix\_Fournier\_M, }\AttributeTok{lab =} \ConstantTok{TRUE}\NormalTok{, }\AttributeTok{title =} \StringTok{"Sex: male"}\NormalTok{)}
\end{Highlighting}
\end{Shaded}

\pandocbounded{\includegraphics[keepaspectratio]{Simulations/covariate_simulation_files/figure-pdf/Fournier-cor-matrix-2.pdf}}

\begin{Shaded}
\begin{Highlighting}[]
\NormalTok{fournier\_crcl }\OtherTok{\textless{}{-}} \FunctionTok{fit\_cov\_wrapper}\NormalTok{(}
  \AttributeTok{cov\_data =}\NormalTok{ cov\_data\_fournier,}
  \AttributeTok{cov\_name =} \StringTok{"CRCL"}\NormalTok{,}
  \AttributeTok{min\_plot =} \DecValTok{0}\NormalTok{,}
  \AttributeTok{max\_plot =} \DecValTok{250}\NormalTok{,}
  \AttributeTok{tol=}\FloatTok{0.002}
\NormalTok{)}

\NormalTok{fournier\_crcl}\SpecialCharTok{$}\NormalTok{plot}
\end{Highlighting}
\end{Shaded}

\pandocbounded{\includegraphics[keepaspectratio]{Simulations/covariate_simulation_files/figure-pdf/fit-fournier-crcl-1.pdf}}

Tolerance was increased to 0.002 to have a fit for log-normal. Normal
distribution is better suited.

\begin{Shaded}
\begin{Highlighting}[]
\NormalTok{fournier\_wt }\OtherTok{\textless{}{-}} \FunctionTok{fit\_cov\_wrapper}\NormalTok{(}
  \AttributeTok{cov\_data =}\NormalTok{ cov\_data\_fournier,}
  \AttributeTok{cov\_name =} \StringTok{"WT"}\NormalTok{,}
  \AttributeTok{min\_plot =} \DecValTok{40}\NormalTok{,}
  \AttributeTok{max\_plot =} \DecValTok{120}
\NormalTok{)}

\NormalTok{fournier\_wt}\SpecialCharTok{$}\NormalTok{plot}
\end{Highlighting}
\end{Shaded}

\pandocbounded{\includegraphics[keepaspectratio]{Simulations/covariate_simulation_files/figure-pdf/fit-fournier-wt-1.pdf}}

In this case the log-normal distribution quantiles are closer to the
true quantiles than the normal ones. We will take the log-normal ones.

\subsection{Simulation of covariates}\label{simulation-of-covariates-1}

Covariates will be sampled from the following distributions :

\begin{itemize}
\tightlist
\item
  CRCL : Normal distribution with mean 117.016 mL/min and standard
  deviation 63.3 mL/min
\item
  WT : Lognormal distribution with mean 73.8 kg, and coefficient of
  variation 17 \%
\item
  AGE : Supposed normal distribution with mean of 50.1 years and
  standard deviation of 24.3
\end{itemize}

The proportion of males is 0.762 in the original article, so the number
of males is 476 and the number of females is 149 to add up to 625.

\begin{Shaded}
\begin{Highlighting}[]
\NormalTok{correlated\_simulation }\OtherTok{\textless{}{-}} \ControlFlowTok{function}\NormalTok{(n, means, cov\_matrix) \{}
  \FunctionTok{set.seed}\NormalTok{(}\DecValTok{1991}\NormalTok{)}
\NormalTok{  collected }\OtherTok{\textless{}{-}} \FunctionTok{matrix}\NormalTok{(}\ConstantTok{NA}\NormalTok{, }\DecValTok{0}\NormalTok{, }\FunctionTok{length}\NormalTok{(means))}
  \FunctionTok{colnames}\NormalTok{(collected) }\OtherTok{\textless{}{-}} \FunctionTok{names}\NormalTok{(means)}

  \ControlFlowTok{while}\NormalTok{ (}\FunctionTok{nrow}\NormalTok{(collected) }\SpecialCharTok{\textless{}}\NormalTok{ n) \{}
\NormalTok{    batch }\OtherTok{\textless{}{-}}\NormalTok{ MASS}\SpecialCharTok{::}\FunctionTok{mvrnorm}\NormalTok{(}\AttributeTok{n =}\NormalTok{ n, }\AttributeTok{mu =}\NormalTok{ means, }\AttributeTok{Sigma =}\NormalTok{ cov\_matrix)}
    \FunctionTok{colnames}\NormalTok{(batch) }\OtherTok{\textless{}{-}} \FunctionTok{names}\NormalTok{(means)}

\NormalTok{    valid }\OtherTok{\textless{}{-}}\NormalTok{ batch[, }\StringTok{"AGE"}\NormalTok{] }\SpecialCharTok{\textgreater{}=} \DecValTok{18} \SpecialCharTok{\&}
\NormalTok{             batch[, }\StringTok{"CRCL\_CG"}\NormalTok{] }\SpecialCharTok{\textgreater{}=} \DecValTok{10}

\NormalTok{    batch\_valid }\OtherTok{\textless{}{-}}\NormalTok{ batch[valid, , drop }\OtherTok{=} \ConstantTok{FALSE}\NormalTok{]}
\NormalTok{    collected }\OtherTok{\textless{}{-}} \FunctionTok{rbind}\NormalTok{(collected, batch\_valid)}
\NormalTok{  \}}

\NormalTok{  collected[}\DecValTok{1}\SpecialCharTok{:}\NormalTok{n, , drop }\OtherTok{=} \ConstantTok{FALSE}\NormalTok{]}
\NormalTok{\}}

\NormalTok{means }\OtherTok{\textless{}{-}} \FunctionTok{c}\NormalTok{(}
  \AttributeTok{CRCL\_CG =}\NormalTok{ fournier\_crcl}\SpecialCharTok{$}\NormalTok{norm\_par[}\StringTok{"mean"}\NormalTok{],}
  \AttributeTok{WT\_log      =}\NormalTok{ fournier\_wt}\SpecialCharTok{$}\NormalTok{lnorm\_par[}\StringTok{"meanlog"}\NormalTok{],}
  \AttributeTok{AGE     =} \FloatTok{50.1}\NormalTok{,}
  \AttributeTok{HT =} \DecValTok{169}
\NormalTok{)}

\NormalTok{sds }\OtherTok{\textless{}{-}} \FunctionTok{c}\NormalTok{(}
  \AttributeTok{CRCL\_CG =}\NormalTok{ fournier\_crcl}\SpecialCharTok{$}\NormalTok{norm\_par[}\StringTok{"sd"}\NormalTok{],}
  \AttributeTok{WT\_log      =}\NormalTok{ fournier\_wt}\SpecialCharTok{$}\NormalTok{lnorm\_par[}\StringTok{"sdlog"}\NormalTok{],}
  \AttributeTok{AGE     =} \FloatTok{24.3}\NormalTok{,}
  \AttributeTok{HT =} \FloatTok{10.3}
\NormalTok{)}

\NormalTok{covariates }\OtherTok{\textless{}{-}} \FunctionTok{c}\NormalTok{(}\StringTok{"CRCL\_CG"}\NormalTok{, }\StringTok{"WT\_log"}\NormalTok{, }\StringTok{"AGE"}\NormalTok{, }\StringTok{"HT"}\NormalTok{)}

\FunctionTok{names}\NormalTok{(means) }\OtherTok{\textless{}{-}}\NormalTok{ covariates}

\CommentTok{\# Correlation and Covariance Matrix}
\NormalTok{cor\_fournier\_M }\OtherTok{\textless{}{-}}\NormalTok{ cor\_matrix\_Fournier\_M[covariates, covariates]}
\NormalTok{cov\_matrix\_M }\OtherTok{\textless{}{-}} \FunctionTok{diag}\NormalTok{(sds) }\SpecialCharTok{\%*\%}\NormalTok{ cor\_fournier\_M }\SpecialCharTok{\%*\%} \FunctionTok{diag}\NormalTok{(sds)}
\FunctionTok{eigen}\NormalTok{(cov\_matrix\_M)}\SpecialCharTok{$}\NormalTok{values}
\end{Highlighting}
\end{Shaded}

\begin{verbatim}
[1] 4218.3820076  387.0322287  102.1295861    0.0230053
\end{verbatim}

\begin{Shaded}
\begin{Highlighting}[]
\NormalTok{cor\_fournier\_F }\OtherTok{\textless{}{-}}\NormalTok{ cor\_matrix\_Fournier\_F[covariates, covariates]}
\NormalTok{cov\_matrix\_F }\OtherTok{\textless{}{-}} \FunctionTok{diag}\NormalTok{(sds) }\SpecialCharTok{\%*\%}\NormalTok{ cor\_fournier\_F }\SpecialCharTok{\%*\%} \FunctionTok{diag}\NormalTok{(sds)}
\FunctionTok{eigen}\NormalTok{(cov\_matrix\_F)}\SpecialCharTok{$}\NormalTok{values}
\end{Highlighting}
\end{Shaded}

\begin{verbatim}
[1] 4.253434e+03 3.545807e+02 9.952763e+01 2.398513e-02
\end{verbatim}

\begin{Shaded}
\begin{Highlighting}[]
\CommentTok{\# Simulate}
\NormalTok{sim\_data\_M }\OtherTok{\textless{}{-}} \FunctionTok{correlated\_simulation}\NormalTok{(}\AttributeTok{n =} \DecValTok{476}\NormalTok{, }\AttributeTok{means =}\NormalTok{ means, }\AttributeTok{cov\_matrix =}\NormalTok{ cov\_matrix\_M)}
\NormalTok{sim\_data\_M }\OtherTok{\textless{}{-}} \FunctionTok{as\_tibble}\NormalTok{(sim\_data\_M) }\SpecialCharTok{\%\textgreater{}\%}
  \FunctionTok{mutate}\NormalTok{(}\AttributeTok{SEX =} \DecValTok{0}\NormalTok{)}

\NormalTok{sim\_data\_F }\OtherTok{\textless{}{-}} \FunctionTok{correlated\_simulation}\NormalTok{(}\AttributeTok{n =} \DecValTok{149}\NormalTok{, }\AttributeTok{means =}\NormalTok{ means, }\AttributeTok{cov\_matrix =}\NormalTok{ cov\_matrix\_F)}
\NormalTok{sim\_data\_F }\OtherTok{\textless{}{-}} \FunctionTok{as\_tibble}\NormalTok{(sim\_data\_F) }\SpecialCharTok{\%\textgreater{}\%}
  \FunctionTok{mutate}\NormalTok{(}\AttributeTok{SEX =} \DecValTok{1}\NormalTok{)}

\NormalTok{sim\_data }\OtherTok{\textless{}{-}} \FunctionTok{rbind}\NormalTok{(sim\_data\_M, sim\_data\_F)}

\CommentTok{\# Put the covariates together}
\NormalTok{sim\_cov\_fournier }\OtherTok{\textless{}{-}} \FunctionTok{tibble}\NormalTok{(}
  \AttributeTok{Paper =} \StringTok{"Fournier\_2018"}\NormalTok{,}
  \AttributeTok{ID\_within\_paper =} \DecValTok{1}\SpecialCharTok{:}\NormalTok{n\_patient,}
  \AttributeTok{ICU =} \DecValTok{1}\NormalTok{,}
  \AttributeTok{BURN =} \DecValTok{1}\NormalTok{,}
  \AttributeTok{OBESE =} \DecValTok{0}\NormalTok{,}
  \AttributeTok{CRCL =}\NormalTok{ sim\_data}\SpecialCharTok{$}\NormalTok{CRCL\_CG,}
  \AttributeTok{WT\_log   =}\NormalTok{ sim\_data}\SpecialCharTok{$}\NormalTok{WT\_log,}
  \AttributeTok{AGE  =}\NormalTok{ sim\_data}\SpecialCharTok{$}\NormalTok{AGE,}
  \AttributeTok{HT  =}\NormalTok{ sim\_data}\SpecialCharTok{$}\NormalTok{HT,}
  \AttributeTok{SEX =}\NormalTok{ sim\_data}\SpecialCharTok{$}\NormalTok{SEX)}

\NormalTok{sim\_cov\_fournier }\OtherTok{\textless{}{-}}\NormalTok{ sim\_cov\_fournier }\SpecialCharTok{\%\textgreater{}\%}
  \FunctionTok{mutate}\NormalTok{(}\AttributeTok{WT =} \FunctionTok{exp}\NormalTok{(WT\_log))}

\NormalTok{summary\_sim\_cov\_fournier }\OtherTok{\textless{}{-}}\NormalTok{ sim\_cov\_fournier }\SpecialCharTok{|\textgreater{}} 
  \FunctionTok{pivot\_longer}\NormalTok{(}\AttributeTok{cols =} \FunctionTok{c}\NormalTok{(ICU,BURN,OBESE,CRCL,WT,AGE),}
               \AttributeTok{names\_to =} \StringTok{"Covariate"}\NormalTok{) }\SpecialCharTok{|\textgreater{}} 
  \FunctionTok{summarise}\NormalTok{(}\AttributeTok{.by =} \FunctionTok{c}\NormalTok{(Covariate,Paper),}
            \AttributeTok{Min =} \FunctionTok{min}\NormalTok{(value),}
            \AttributeTok{Q1 =} \FunctionTok{quantile}\NormalTok{(value, }\FloatTok{0.25}\NormalTok{),}
            \AttributeTok{Median =} \FunctionTok{quantile}\NormalTok{(value, }\FloatTok{0.5}\NormalTok{),}
            \AttributeTok{Q3 =} \FunctionTok{quantile}\NormalTok{(value, }\FloatTok{0.75}\NormalTok{),}
            \AttributeTok{Max =} \FunctionTok{max}\NormalTok{(value))}

\FunctionTok{paste0}\NormalTok{(}\StringTok{"The proportion of males is "}\NormalTok{, }\FunctionTok{round}\NormalTok{(}\FunctionTok{mean}\NormalTok{(sim\_cov\_fournier}\SpecialCharTok{$}\NormalTok{SEX }\SpecialCharTok{==} \DecValTok{0}\NormalTok{), }\DecValTok{3}\NormalTok{), }\StringTok{" compared to the original 0.762."}\NormalTok{)}
\end{Highlighting}
\end{Shaded}

\begin{verbatim}
[1] "The proportion of males is 0.762 compared to the original 0.762."
\end{verbatim}

\begin{Shaded}
\begin{Highlighting}[]
\NormalTok{cov\_fournier\_compare }\OtherTok{\textless{}{-}}\NormalTok{ summary\_sim\_cov\_fournier }\SpecialCharTok{|\textgreater{}}
  \FunctionTok{rename\_with}\NormalTok{(}\SpecialCharTok{\textasciitilde{}} \FunctionTok{paste0}\NormalTok{(.x, }\StringTok{"\_sim"}\NormalTok{), }\SpecialCharTok{{-}}\FunctionTok{one\_of}\NormalTok{(}\StringTok{"Covariate"}\NormalTok{, }\StringTok{"Paper"}\NormalTok{)) }\SpecialCharTok{|\textgreater{}}
  \FunctionTok{left\_join}\NormalTok{(}\FunctionTok{rename\_with}\NormalTok{(}
\NormalTok{    cov\_data\_fournier,}
    \SpecialCharTok{\textasciitilde{}} \FunctionTok{paste0}\NormalTok{(.x, }\StringTok{"\_true"}\NormalTok{),}
    \SpecialCharTok{{-}}\FunctionTok{one\_of}\NormalTok{(}\StringTok{"Covariate"}\NormalTok{, }\StringTok{"Paper"}\NormalTok{, }\StringTok{"Unit"}\NormalTok{)}
\NormalTok{  )) }\SpecialCharTok{|\textgreater{}} 
  \FunctionTok{relocate}\NormalTok{(Covariate, }\AttributeTok{.after =}\NormalTok{ Paper) }\SpecialCharTok{|\textgreater{}} 
  \FunctionTok{relocate}\NormalTok{(Unit, }\AttributeTok{.after =}\NormalTok{ Covariate)}

\NormalTok{cov\_fournier\_compare }\SpecialCharTok{|\textgreater{}}
  \FunctionTok{gt}\NormalTok{() }\SpecialCharTok{|\textgreater{}}
  \FunctionTok{fmt\_scientific}\NormalTok{() }\SpecialCharTok{|\textgreater{}} 
    \FunctionTok{tab\_spanner}\NormalTok{(}\AttributeTok{columns =} \FunctionTok{starts\_with}\NormalTok{(}\StringTok{"Min"}\NormalTok{),}
              \AttributeTok{label =} \StringTok{"Min"}\NormalTok{) }\SpecialCharTok{|\textgreater{}} 
      \FunctionTok{tab\_spanner}\NormalTok{(}\AttributeTok{columns =} \FunctionTok{starts\_with}\NormalTok{(}\StringTok{"Q1"}\NormalTok{),}
              \AttributeTok{label =} \StringTok{"Q1"}\NormalTok{) }\SpecialCharTok{|\textgreater{}} 
      \FunctionTok{tab\_spanner}\NormalTok{(}\AttributeTok{columns =} \FunctionTok{starts\_with}\NormalTok{(}\StringTok{"Median"}\NormalTok{),}
              \AttributeTok{label =} \StringTok{"Median"}\NormalTok{) }\SpecialCharTok{|\textgreater{}} 
      \FunctionTok{tab\_spanner}\NormalTok{(}\AttributeTok{columns =} \FunctionTok{starts\_with}\NormalTok{(}\StringTok{"Q3"}\NormalTok{),}
              \AttributeTok{label =} \StringTok{"Q3"}\NormalTok{) }\SpecialCharTok{|\textgreater{}} 
  \FunctionTok{tab\_spanner}\NormalTok{(}\AttributeTok{columns =} \FunctionTok{starts\_with}\NormalTok{(}\StringTok{"Max"}\NormalTok{),}
              \AttributeTok{label =} \StringTok{"Max"}\NormalTok{)}
\end{Highlighting}
\end{Shaded}

\begin{table}
\fontsize{12.0pt}{14.4pt}\selectfont
\begin{tabular*}{\linewidth}{@{\extracolsep{\fill}}lllrrrrrrrrrr}
\toprule
 &  &  & \multicolumn{2}{c}{Min} & \multicolumn{2}{c}{Q1} & \multicolumn{2}{c}{Median} & \multicolumn{2}{c}{Q3} & \multicolumn{2}{c}{Max} \\ 
\cmidrule(lr){4-5} \cmidrule(lr){6-7} \cmidrule(lr){8-9} \cmidrule(lr){10-11} \cmidrule(lr){12-13}
Paper & Covariate & Unit & Min\_sim & Min\_true & Q1\_sim & Q1\_true & Median\_sim & Median\_true & Q3\_sim & Q3\_true & Max\_sim & Max\_true \\ 
\midrule\addlinespace[2.5pt]
Fournier\_2018 & ICU & Unitless & 1.00 & 1.00 & 1.00 & 1.00 & 1.00 & 1.00 & 1.00 & 1.00 & 1.00 & 1.00 \\ 
Fournier\_2018 & BURN & Unitless & 1.00 & 1.00 & 1.00 & 1.00 & 1.00 & 1.00 & 1.00 & 1.00 & 1.00 & 1.00 \\ 
Fournier\_2018 & OBESE & Unitless & 0.00 & 0.00 & 0.00 & 0.00 & 0.00 & 0.00 & 0.00 & 0.00 & 0.00 & 0.00 \\ 
Fournier\_2018 & CRCL & mL/min & 1.01 $\times$ 10\textsuperscript{1} & 1.00 $\times$ 10\textsuperscript{1} & 7.32 $\times$ 10\textsuperscript{1} & 6.50 $\times$ 10\textsuperscript{1} & 1.16 $\times$ 10\textsuperscript{2} & 1.28 $\times$ 10\textsuperscript{2} & 1.55 $\times$ 10\textsuperscript{2} & 1.50 $\times$ 10\textsuperscript{2} & 2.88 $\times$ 10\textsuperscript{2} & NA \\ 
Fournier\_2018 & WT & kg & 4.53 $\times$ 10\textsuperscript{1} & NA & 6.66 $\times$ 10\textsuperscript{1} & 6.70 $\times$ 10\textsuperscript{1} & 7.45 $\times$ 10\textsuperscript{1} & 7.24 $\times$ 10\textsuperscript{1} & 8.40 $\times$ 10\textsuperscript{1} & 8.36 $\times$ 10\textsuperscript{1} & 1.18 $\times$ 10\textsuperscript{2} & NA \\ 
Fournier\_2018 & AGE & year & 1.80 $\times$ 10\textsuperscript{1} & NA & 3.49 $\times$ 10\textsuperscript{1} & NA & 5.03 $\times$ 10\textsuperscript{1} & 5.01 $\times$ 10\textsuperscript{1} & 6.72 $\times$ 10\textsuperscript{1} & NA & 1.31 $\times$ 10\textsuperscript{2} & NA \\ 
\bottomrule
\end{tabular*}
\end{table}

\begin{Shaded}
\begin{Highlighting}[]
\FunctionTok{plot\_sim\_cov\_wrapper}\NormalTok{(cov\_data\_fournier,}\StringTok{"CRCL"}\NormalTok{,sim\_cov\_fournier)}
\end{Highlighting}
\end{Shaded}

\pandocbounded{\includegraphics[keepaspectratio]{Simulations/covariate_simulation_files/figure-pdf/plot-cov-fournier-1.pdf}}

\begin{Shaded}
\begin{Highlighting}[]
\FunctionTok{plot\_sim\_cov\_wrapper}\NormalTok{(cov\_data\_fournier,}\StringTok{"WT"}\NormalTok{,sim\_cov\_fournier)}
\end{Highlighting}
\end{Shaded}

\pandocbounded{\includegraphics[keepaspectratio]{Simulations/covariate_simulation_files/figure-pdf/plot-cov-fournier-2.pdf}}

\begin{Shaded}
\begin{Highlighting}[]
\FunctionTok{plot\_sim\_cov\_wrapper}\NormalTok{(cov\_data\_fournier,}\StringTok{"AGE"}\NormalTok{,sim\_cov\_fournier)}
\end{Highlighting}
\end{Shaded}

\pandocbounded{\includegraphics[keepaspectratio]{Simulations/covariate_simulation_files/figure-pdf/plot-cov-fournier-3.pdf}}

Then, based on the BMI and WT, the height (HT) is calculated to be able
to calculate the body surface area (BSA). CRCL is reconverted to serum
creatinine (CREAT) based on the Cockroft and Gault equation.

\begin{Shaded}
\begin{Highlighting}[]
\NormalTok{reconvert\_CG }\OtherTok{\textless{}{-}} \ControlFlowTok{function}\NormalTok{(AGE, CRCL, SEX, WT) \{}
    \CommentTok{\# Mutliply by 0.85 for females}
\NormalTok{    sex\_factor }\OtherTok{\textless{}{-}} \FunctionTok{ifelse}\NormalTok{(SEX }\SpecialCharTok{==} \DecValTok{0}\NormalTok{, }\DecValTok{1}\NormalTok{, }\FloatTok{0.85}\NormalTok{)}
    
    \CommentTok{\# Calculate DFG}
\NormalTok{    CREAT }\OtherTok{=}\NormalTok{ ((}\DecValTok{140} \SpecialCharTok{{-}}\NormalTok{ AGE) }\SpecialCharTok{*}\NormalTok{ WT }\SpecialCharTok{*}\NormalTok{ sex\_factor) }\SpecialCharTok{/}\NormalTok{ (CRCL }\SpecialCharTok{*} \DecValTok{72}\NormalTok{)}
  
  \FunctionTok{return}\NormalTok{(CREAT)}
\NormalTok{\}}

\NormalTok{sim\_cov\_fournier}\SpecialCharTok{$}\NormalTok{CREAT }\OtherTok{\textless{}{-}} \FunctionTok{mapply}\NormalTok{(reconvert\_CG, sim\_cov\_fournier}\SpecialCharTok{$}\NormalTok{AGE, sim\_cov\_fournier}\SpecialCharTok{$}\NormalTok{CRCL, sim\_cov\_fournier}\SpecialCharTok{$}\NormalTok{SEX, sim\_cov\_fournier}\SpecialCharTok{$}\NormalTok{WT)}
\NormalTok{sim\_cov\_fournier }\OtherTok{\textless{}{-}}\NormalTok{ sim\_cov\_fournier }\SpecialCharTok{\%\textgreater{}\%}
  \FunctionTok{mutate}\NormalTok{(}
     \AttributeTok{BSA =}\NormalTok{ (WT}\SpecialCharTok{\^{}}\FloatTok{0.425} \SpecialCharTok{*}\NormalTok{ (HT}\SpecialCharTok{\^{}}\FloatTok{0.725}\NormalTok{) }\SpecialCharTok{*} \FloatTok{0.007184}\NormalTok{),}
     \AttributeTok{BMI =}\NormalTok{ WT }\SpecialCharTok{/}\NormalTok{ (HT}\SpecialCharTok{/}\DecValTok{100}\NormalTok{)}\SpecialCharTok{\^{}}\DecValTok{2}\NormalTok{,}
     \AttributeTok{OBESE =} \FunctionTok{ifelse}\NormalTok{(BMI }\SpecialCharTok{\textgreater{}} \DecValTok{30}\NormalTok{, }\DecValTok{1}\NormalTok{, }\DecValTok{0}\NormalTok{)}
\NormalTok{  ) }\SpecialCharTok{\%\textgreater{}\%}
\NormalTok{  dplyr}\SpecialCharTok{::}\FunctionTok{select}\NormalTok{(Paper, ID\_within\_paper, ICU, BURN, OBESE, CREAT, WT, BSA, BMI, AGE, SEX, HT)}
\CommentTok{\# Calculate the simulated correlation matrices separately for the two sexes to compare with the original}
\NormalTok{sim\_cov\_fournier\_M }\OtherTok{\textless{}{-}}\NormalTok{ sim\_cov\_fournier }\SpecialCharTok{\%\textgreater{}\%} \FunctionTok{filter}\NormalTok{(SEX }\SpecialCharTok{==} \DecValTok{0}\NormalTok{)}
\NormalTok{sim\_cov\_fournier\_F }\OtherTok{\textless{}{-}}\NormalTok{ sim\_cov\_fournier }\SpecialCharTok{\%\textgreater{}\%} \FunctionTok{filter}\NormalTok{(SEX }\SpecialCharTok{==} \DecValTok{1}\NormalTok{)}

\NormalTok{cor\_sim\_M }\OtherTok{\textless{}{-}} \FunctionTok{cor}\NormalTok{(sim\_cov\_fournier\_M }\SpecialCharTok{\%\textgreater{}\%}\NormalTok{ dplyr}\SpecialCharTok{::}\FunctionTok{select}\NormalTok{(WT, CREAT, AGE, HT, BMI), }\AttributeTok{use =} \StringTok{"complete.obs"}\NormalTok{)}
\NormalTok{cor\_sim\_F }\OtherTok{\textless{}{-}} \FunctionTok{cor}\NormalTok{(sim\_cov\_fournier\_F }\SpecialCharTok{\%\textgreater{}\%}\NormalTok{ dplyr}\SpecialCharTok{::}\FunctionTok{select}\NormalTok{(WT, CREAT, AGE, HT, BMI), }\AttributeTok{use =} \StringTok{"complete.obs"}\NormalTok{)}

\FunctionTok{ggcorrplot}\NormalTok{(cor\_matrix\_Fournier\_F, }\AttributeTok{lab =} \ConstantTok{TRUE}\NormalTok{, }\AttributeTok{title =} \StringTok{"Fournier original (females)"}\NormalTok{)}
\end{Highlighting}
\end{Shaded}

\pandocbounded{\includegraphics[keepaspectratio]{Simulations/covariate_simulation_files/figure-pdf/reconvert_creat_fournier-1.pdf}}

\begin{Shaded}
\begin{Highlighting}[]
\NormalTok{Fournier\_F }\OtherTok{\textless{}{-}} \FunctionTok{ggcorrplot}\NormalTok{(cor\_sim\_F, }\AttributeTok{lab =} \ConstantTok{TRUE}\NormalTok{, }\AttributeTok{title =} \StringTok{"Fournier simulated (females)"}\NormalTok{)}
\NormalTok{Fournier\_F}
\end{Highlighting}
\end{Shaded}

\pandocbounded{\includegraphics[keepaspectratio]{Simulations/covariate_simulation_files/figure-pdf/reconvert_creat_fournier-2.pdf}}

\begin{Shaded}
\begin{Highlighting}[]
\FunctionTok{ggcorrplot}\NormalTok{(cor\_matrix\_Fournier\_M, }\AttributeTok{lab =} \ConstantTok{TRUE}\NormalTok{, }\AttributeTok{title =} \StringTok{"Fournier original (males)"}\NormalTok{)}
\end{Highlighting}
\end{Shaded}

\pandocbounded{\includegraphics[keepaspectratio]{Simulations/covariate_simulation_files/figure-pdf/reconvert_creat_fournier-3.pdf}}

\begin{Shaded}
\begin{Highlighting}[]
\NormalTok{Fournier\_M }\OtherTok{\textless{}{-}} \FunctionTok{ggcorrplot}\NormalTok{(cor\_sim\_M, }\AttributeTok{lab =} \ConstantTok{TRUE}\NormalTok{, }\AttributeTok{title =} \StringTok{"Fournier simulated (males)"}\NormalTok{)}
\NormalTok{Fournier\_M}
\end{Highlighting}
\end{Shaded}

\pandocbounded{\includegraphics[keepaspectratio]{Simulations/covariate_simulation_files/figure-pdf/reconvert_creat_fournier-4.pdf}}

\section{Mellon et al. (2020)}\label{mellon2020}

\subsection{Covariate distribution}\label{covariate-distribution-2}

Patients are obese, but otherwise healthy, meaning that they are not in
intensive care and not burn victims. They are all obese (BMI
\textgreater{} 30 kg/m\^{}2). Creatinine clearance is calculated based
on the MDRD equation. Thus for all patients ICU = 0, BURN = 0, OBESE=1.

For WT, BMI and CRCL here are the data from the paper :

\begin{Shaded}
\begin{Highlighting}[]
\NormalTok{cov\_data\_mellon }\OtherTok{\textless{}{-}}\NormalTok{ cov\_data }\SpecialCharTok{|\textgreater{}} 
  \FunctionTok{filter}\NormalTok{(Paper }\SpecialCharTok{==} \StringTok{"Mellon\_2020"}\NormalTok{)}

\NormalTok{cov\_data\_mellon }\SpecialCharTok{|\textgreater{}} 
  \FunctionTok{filter}\NormalTok{(Covariate }\SpecialCharTok{\%in\%} \FunctionTok{c}\NormalTok{(}\StringTok{"WT"}\NormalTok{,}\StringTok{"CRCL"}\NormalTok{, }\StringTok{"AGE"}\NormalTok{, }\StringTok{"BMI"}\NormalTok{, }\StringTok{"BMI\_log"}\NormalTok{)) }\SpecialCharTok{|\textgreater{}} 
  \FunctionTok{kable}\NormalTok{()}
\end{Highlighting}
\end{Shaded}

\begin{longtable}[]{@{}
  >{\raggedright\arraybackslash}p{(\linewidth - 14\tabcolsep) * \real{0.1846}}
  >{\raggedright\arraybackslash}p{(\linewidth - 14\tabcolsep) * \real{0.1538}}
  >{\raggedright\arraybackslash}p{(\linewidth - 14\tabcolsep) * \real{0.2615}}
  >{\raggedleft\arraybackslash}p{(\linewidth - 14\tabcolsep) * \real{0.1077}}
  >{\raggedleft\arraybackslash}p{(\linewidth - 14\tabcolsep) * \real{0.0462}}
  >{\raggedleft\arraybackslash}p{(\linewidth - 14\tabcolsep) * \real{0.0462}}
  >{\raggedleft\arraybackslash}p{(\linewidth - 14\tabcolsep) * \real{0.0923}}
  >{\raggedleft\arraybackslash}p{(\linewidth - 14\tabcolsep) * \real{0.1077}}@{}}
\toprule\noalign{}
\begin{minipage}[b]{\linewidth}\raggedright
Paper
\end{minipage} & \begin{minipage}[b]{\linewidth}\raggedright
Covariate
\end{minipage} & \begin{minipage}[b]{\linewidth}\raggedright
Unit
\end{minipage} & \begin{minipage}[b]{\linewidth}\raggedleft
Median
\end{minipage} & \begin{minipage}[b]{\linewidth}\raggedleft
Q1
\end{minipage} & \begin{minipage}[b]{\linewidth}\raggedleft
Q3
\end{minipage} & \begin{minipage}[b]{\linewidth}\raggedleft
Min
\end{minipage} & \begin{minipage}[b]{\linewidth}\raggedleft
Max
\end{minipage} \\
\midrule\noalign{}
\endhead
\bottomrule\noalign{}
\endlastfoot
Mellon\_2020 & WT & kg & 109.3 & NA & NA & 88.00 & 151.50 \\
Mellon\_2020 & CRCL & mL/min/1.73m\^{}\{2\} & 94.0 & NA & NA & 62.00 &
133.00 \\
Mellon\_2020 & BMI & kg/m\^{}\{2\} & 40.6 & NA & NA & 35.20 & 67.30 \\
Mellon\_2020 & BMI\_log & kg/m\^{}\{2\} & 3.7 & NA & NA & 3.56 & 4.21 \\
Mellon\_2020 & CRCL & ml/min & NA & NA & NA & NA & NA \\
Mellon\_2020 & AGE & year & 51.7 & NA & NA & 22.90 & 62.90 \\
\end{longtable}

\begin{Shaded}
\begin{Highlighting}[]
\NormalTok{cor\_matrix\_Mellon\_F }\OtherTok{\textless{}{-}} \FunctionTok{readRDS}\NormalTok{(}\FunctionTok{here}\NormalTok{(}\StringTok{"a\_priori/For\_publication/Simulations/cor\_matrix\_Mellon\_F.rds"}\NormalTok{))}
\FunctionTok{print}\NormalTok{(cor\_matrix\_Mellon\_F)}
\end{Highlighting}
\end{Shaded}

\begin{verbatim}
                      CREAT          AGE          WT       WT_log          HT
CREAT           1.000000000  0.022643469  0.03125540  0.031968067  0.05795702
AGE             0.022643469  1.000000000 -0.07488881 -0.077065515 -0.11168121
WT              0.031255400 -0.074888809  1.00000000  0.997431167  0.43988528
WT_log          0.031968067 -0.077065515  0.99743117  1.000000000  0.45364368
HT              0.057957020 -0.111681207  0.43988528  0.453643683  1.00000000
HT_log          0.057716460 -0.110775924  0.43501172  0.448949779  0.99935506
BMI            -0.006435320 -0.004011043  0.76869713  0.758978914 -0.23142245
BMI_log        -0.006781559 -0.004085614  0.77244329  0.765197692 -0.22615779
CRCL_MDRD_norm -0.936664533 -0.319596180 -0.00640208 -0.006556334 -0.02317188
                    HT_log          BMI      BMI_log CRCL_MDRD_norm
CREAT           0.05771646 -0.006435320 -0.006781559   -0.936664533
AGE            -0.11077592 -0.004011043 -0.004085614   -0.319596180
WT              0.43501172  0.768697132  0.772443287   -0.006402080
WT_log          0.44894978  0.758978914  0.765197692   -0.006556334
HT              0.99935506 -0.231422452 -0.226157792   -0.023171879
HT_log          1.00000000 -0.237225021 -0.231732528   -0.023241184
BMI            -0.23722502  1.000000000  0.997187131    0.009471973
BMI_log        -0.23173253  0.997187131  1.000000000    0.009607333
CRCL_MDRD_norm -0.02324118  0.009471973  0.009607333    1.000000000
\end{verbatim}

\begin{Shaded}
\begin{Highlighting}[]
\FunctionTok{ggcorrplot}\NormalTok{(cor\_matrix\_Mellon\_F, }\AttributeTok{lab =} \ConstantTok{TRUE}\NormalTok{, }\AttributeTok{title =} \StringTok{"Sex: female"}\NormalTok{)}
\end{Highlighting}
\end{Shaded}

\pandocbounded{\includegraphics[keepaspectratio]{Simulations/covariate_simulation_files/figure-pdf/Mellon-cor-matrix-1.pdf}}

\begin{Shaded}
\begin{Highlighting}[]
\NormalTok{cor\_matrix\_Mellon\_M }\OtherTok{\textless{}{-}} \FunctionTok{readRDS}\NormalTok{(}\FunctionTok{here}\NormalTok{(}\StringTok{"a\_priori/For\_publication/Simulations/cor\_matrix\_Mellon\_M.rds"}\NormalTok{))}
\FunctionTok{print}\NormalTok{(cor\_matrix\_Mellon\_M)}
\end{Highlighting}
\end{Shaded}

\begin{verbatim}
                      CREAT          AGE           WT       WT_log           HT
CREAT           1.000000000 -0.079420098 -0.005155970 -0.002021643  0.062039693
AGE            -0.079420098  1.000000000 -0.020403922 -0.018224696 -0.002835067
WT             -0.005155970 -0.020403922  1.000000000  0.997694029  0.617330996
WT_log         -0.002021643 -0.018224696  0.997694029  1.000000000  0.634363996
HT              0.062039693 -0.002835067  0.617330996  0.634363996  1.000000000
HT_log          0.061429587 -0.001756480  0.613093576  0.631018266  0.999227265
BMI            -0.070072797 -0.019345942  0.535836189  0.518641232 -0.328796339
BMI_log        -0.069673338 -0.020297428  0.545596150  0.528796578 -0.319827206
CRCL_MDRD_norm -0.940158336 -0.215895006  0.007018922  0.003387753 -0.059982333
                    HT_log         BMI     BMI_log CRCL_MDRD_norm
CREAT           0.06142959 -0.07007280 -0.06967334   -0.940158336
AGE            -0.00175648 -0.01934594 -0.02029743   -0.215895006
WT              0.61309358  0.53583619  0.54559615    0.007018922
WT_log          0.63101827  0.51864123  0.52879658    0.003387753
HT              0.99922726 -0.32879634 -0.31982721   -0.059982333
HT_log          1.00000000 -0.33401211 -0.32475168   -0.059647213
BMI            -0.33401211  1.00000000  0.99775044    0.069173500
BMI_log        -0.32475168  0.99775044  1.00000000    0.069388818
CRCL_MDRD_norm -0.05964721  0.06917350  0.06938882    1.000000000
\end{verbatim}

\begin{Shaded}
\begin{Highlighting}[]
\FunctionTok{ggcorrplot}\NormalTok{(cor\_matrix\_Mellon\_M, }\AttributeTok{lab =} \ConstantTok{TRUE}\NormalTok{, }\AttributeTok{title =} \StringTok{"Sex: male"}\NormalTok{)}
\end{Highlighting}
\end{Shaded}

\pandocbounded{\includegraphics[keepaspectratio]{Simulations/covariate_simulation_files/figure-pdf/Mellon-cor-matrix-2.pdf}}

Mellon is different from the other papers as CRCL is given as
ml/min/1.73m\^{}\{2\} based on the MDRD equation, and instead of Q1 and
Q3, the minimum and maxiumum of the covariates is given (apart from
BMI). Another particularity is that due to the particularity of the
subjects (obesity), the distribution of body weight is positively
(right) skewed. Positively skewed distribution is relatively rare, so it
is not easy to find a distribution function to describe it. Beta
distribution needs scaling the data between 0 and 1. and 4 parameter
stable distribution et gamma distribution have several parameters that
need to be defined, and in the absence of the original data, it is not
possible to estimate these parameters.

The proportion of males is 0.148.

\begin{Shaded}
\begin{Highlighting}[]
\NormalTok{mellon\_crcl }\OtherTok{\textless{}{-}} \FunctionTok{fit\_cov\_wrapper}\NormalTok{(}
  \AttributeTok{cov\_data =}\NormalTok{ cov\_data\_mellon,}
  \AttributeTok{cov\_name =} \StringTok{"CRCL"}\NormalTok{,}
  \AttributeTok{min\_plot =} \DecValTok{0}\NormalTok{,}
  \AttributeTok{max\_plot =} \DecValTok{250}
\NormalTok{)}

\NormalTok{mellon\_crcl}\SpecialCharTok{$}\NormalTok{plot}
\end{Highlighting}
\end{Shaded}

\pandocbounded{\includegraphics[keepaspectratio]{Simulations/covariate_simulation_files/figure-pdf/fit-mellon-crcl-1.pdf}}

Similar fits; normal distribution is selected.

\begin{Shaded}
\begin{Highlighting}[]
\NormalTok{mellon\_wt }\OtherTok{\textless{}{-}} \FunctionTok{fit\_cov\_wrapper}\NormalTok{(}
  \AttributeTok{cov\_data =}\NormalTok{ cov\_data\_mellon,}
  \AttributeTok{cov\_name =} \StringTok{"WT"}\NormalTok{,}
  \AttributeTok{min\_plot =} \DecValTok{40}\NormalTok{,}
  \AttributeTok{max\_plot =} \DecValTok{170}
\NormalTok{)}

\NormalTok{mellon\_wt}\SpecialCharTok{$}\NormalTok{plot}
\end{Highlighting}
\end{Shaded}

\pandocbounded{\includegraphics[keepaspectratio]{Simulations/covariate_simulation_files/figure-pdf/fit-mellon-wt-1.pdf}}

Log-normal distribution fits slightly better than normal.

\begin{Shaded}
\begin{Highlighting}[]
\NormalTok{mellon\_age }\OtherTok{\textless{}{-}} \FunctionTok{fit\_cov\_wrapper}\NormalTok{(}
  \AttributeTok{cov\_data =}\NormalTok{ cov\_data\_mellon,}
  \AttributeTok{cov\_name =} \StringTok{"AGE"}\NormalTok{,}
  \AttributeTok{min\_plot =} \DecValTok{0}\NormalTok{,}
  \AttributeTok{max\_plot =} \DecValTok{120}
\NormalTok{)}

\NormalTok{mellon\_age}\SpecialCharTok{$}\NormalTok{plot}
\end{Highlighting}
\end{Shaded}

\pandocbounded{\includegraphics[keepaspectratio]{Simulations/covariate_simulation_files/figure-pdf/fit-mellon-age-1.pdf}}

Normal distribution fits a bit better.

\begin{Shaded}
\begin{Highlighting}[]
\NormalTok{mellon\_bmi }\OtherTok{\textless{}{-}} \FunctionTok{fit\_cov\_wrapper}\NormalTok{(}
  \AttributeTok{cov\_data =}\NormalTok{ cov\_data\_mellon,}
  \AttributeTok{cov\_name =} \StringTok{"BMI"}\NormalTok{,}
  \AttributeTok{min\_plot =} \DecValTok{0}\NormalTok{,}
  \AttributeTok{max\_plot =} \DecValTok{100}\NormalTok{, }
  \AttributeTok{tol =} \FloatTok{0.001}
\NormalTok{)}

\NormalTok{mellon\_bmi}\SpecialCharTok{$}\NormalTok{plot}
\end{Highlighting}
\end{Shaded}

\pandocbounded{\includegraphics[keepaspectratio]{Simulations/covariate_simulation_files/figure-pdf/fit-mellon-bmi-1.pdf}}

\begin{Shaded}
\begin{Highlighting}[]
\NormalTok{mellon\_bmi\_log }\OtherTok{\textless{}{-}} \FunctionTok{fit\_cov\_wrapper}\NormalTok{(}
  \AttributeTok{cov\_data =}\NormalTok{ cov\_data\_mellon,}
  \AttributeTok{cov\_name =} \StringTok{"BMI\_log"}\NormalTok{,}
  \AttributeTok{min\_plot =} \FloatTok{2.5}\NormalTok{,}
  \AttributeTok{max\_plot =} \DecValTok{5}\NormalTok{, }
  \AttributeTok{tol =} \FloatTok{0.001}
\NormalTok{)}

\NormalTok{mellon\_bmi\_log}\SpecialCharTok{$}\NormalTok{plot}
\end{Highlighting}
\end{Shaded}

\pandocbounded{\includegraphics[keepaspectratio]{Simulations/covariate_simulation_files/figure-pdf/fit-mellon-bmi-log-1.pdf}}

Log-normal distribution fits a bit better.

\subsection{Simulation of covariates}\label{simulation-of-covariates-2}

Covariates will be sampled from the following distributions :

\begin{itemize}
\tightlist
\item
  CRCL : Normal distribution from which negative values have been
  resampled, with mean 94 mL/min and standard deviation 14.3 mL/min
\item
  WT : Log-normal distribution with mean 109 kg, and coefficient of
  variation 9.41 \%
\item
  AGE : Normal distribution with mean \ensuremath{2.84\times 10^{22}}
  years, and standard deviation 4.81 \%
\item
  BMI : Log-normal distribution with mean 40.6 \(kg/m^{2}\), and
  coefficient of variation 6.14 \%
\end{itemize}

The proportion of males is 0.148 in the original article, so the number
of males is 93 and the number of females is 532 to add up to 625.

\begin{Shaded}
\begin{Highlighting}[]
\NormalTok{correlated\_simulation }\OtherTok{\textless{}{-}} \ControlFlowTok{function}\NormalTok{(n, means, cov\_matrix) \{}
  \FunctionTok{set.seed}\NormalTok{(}\DecValTok{1991}\NormalTok{)}
\NormalTok{  collected }\OtherTok{\textless{}{-}} \FunctionTok{matrix}\NormalTok{(}\ConstantTok{NA}\NormalTok{, }\DecValTok{0}\NormalTok{, }\FunctionTok{length}\NormalTok{(means))}
  \FunctionTok{colnames}\NormalTok{(collected) }\OtherTok{\textless{}{-}} \FunctionTok{names}\NormalTok{(means)}

  \ControlFlowTok{while}\NormalTok{ (}\FunctionTok{nrow}\NormalTok{(collected) }\SpecialCharTok{\textless{}}\NormalTok{ n) \{}
\NormalTok{    batch }\OtherTok{\textless{}{-}}\NormalTok{ MASS}\SpecialCharTok{::}\FunctionTok{mvrnorm}\NormalTok{(}\AttributeTok{n =}\NormalTok{ n, }\AttributeTok{mu =}\NormalTok{ means, }\AttributeTok{Sigma =}\NormalTok{ cov\_matrix)}
    \FunctionTok{colnames}\NormalTok{(batch) }\OtherTok{\textless{}{-}} \FunctionTok{names}\NormalTok{(means)}

\NormalTok{    valid }\OtherTok{\textless{}{-}}\NormalTok{ batch[, }\StringTok{"AGE"}\NormalTok{] }\SpecialCharTok{\textgreater{}=} \DecValTok{18} \SpecialCharTok{\&}
\NormalTok{             batch[, }\StringTok{"CRCL\_MDRD\_norm"}\NormalTok{] }\SpecialCharTok{\textgreater{}=} \DecValTok{10} \SpecialCharTok{\&}
\NormalTok{             batch[, }\StringTok{"BMI\_log"}\NormalTok{] }\SpecialCharTok{\textgreater{}} \FloatTok{3.401197} \CommentTok{\# log(30)}

\NormalTok{    batch\_valid }\OtherTok{\textless{}{-}}\NormalTok{ batch[valid, , drop }\OtherTok{=} \ConstantTok{FALSE}\NormalTok{]}
\NormalTok{    collected }\OtherTok{\textless{}{-}} \FunctionTok{rbind}\NormalTok{(collected, batch\_valid)}
\NormalTok{  \}}

\NormalTok{  collected[}\DecValTok{1}\SpecialCharTok{:}\NormalTok{n, , drop }\OtherTok{=} \ConstantTok{FALSE}\NormalTok{]}
\NormalTok{\}}

\NormalTok{means }\OtherTok{\textless{}{-}} \FunctionTok{c}\NormalTok{(}
  \AttributeTok{CRCL\_MDRD\_norm =}\NormalTok{ mellon\_crcl}\SpecialCharTok{$}\NormalTok{norm\_par[}\StringTok{"mean"}\NormalTok{],}
  \AttributeTok{WT\_log      =}\NormalTok{ mellon\_wt}\SpecialCharTok{$}\NormalTok{lnorm\_par[}\StringTok{"meanlog"}\NormalTok{],}
  \AttributeTok{AGE     =}\NormalTok{ mellon\_age}\SpecialCharTok{$}\NormalTok{norm\_par[}\StringTok{"mean"}\NormalTok{],}
  \AttributeTok{BMI\_log     =}\NormalTok{ mellon\_bmi}\SpecialCharTok{$}\NormalTok{lnorm\_par[}\StringTok{"meanlog"}\NormalTok{]}
\NormalTok{)}

\NormalTok{sds }\OtherTok{\textless{}{-}} \FunctionTok{c}\NormalTok{(}
  \AttributeTok{CRCL\_MDRD\_norm =}\NormalTok{ mellon\_crcl}\SpecialCharTok{$}\NormalTok{norm\_par[}\StringTok{"sd"}\NormalTok{],}
  \AttributeTok{WT\_log      =}\NormalTok{ mellon\_wt}\SpecialCharTok{$}\NormalTok{lnorm\_par[}\StringTok{"sdlog"}\NormalTok{], }
  \AttributeTok{AGE =}\NormalTok{ mellon\_age}\SpecialCharTok{$}\NormalTok{norm\_par[}\StringTok{"sd"}\NormalTok{],}
  \AttributeTok{BMI\_log     =}\NormalTok{ mellon\_bmi}\SpecialCharTok{$}\NormalTok{lnorm\_par[}\StringTok{"sdlog"}\NormalTok{]}
\NormalTok{)}

\NormalTok{covariates }\OtherTok{\textless{}{-}} \FunctionTok{c}\NormalTok{(}\StringTok{"CRCL\_MDRD\_norm"}\NormalTok{, }\StringTok{"WT\_log"}\NormalTok{, }\StringTok{"AGE"}\NormalTok{, }\StringTok{"BMI\_log"}\NormalTok{)}

\FunctionTok{names}\NormalTok{(means) }\OtherTok{\textless{}{-}}\NormalTok{ covariates}

\CommentTok{\# Correlation and Covariance Matrix}
\NormalTok{cor\_mellon\_M }\OtherTok{\textless{}{-}}\NormalTok{ cor\_matrix\_Mellon\_M[covariates, covariates]}
\NormalTok{cov\_matrix\_M }\OtherTok{\textless{}{-}} \FunctionTok{diag}\NormalTok{(sds) }\SpecialCharTok{\%*\%}\NormalTok{ cor\_mellon\_M }\SpecialCharTok{\%*\%} \FunctionTok{diag}\NormalTok{(sds)}
\FunctionTok{eigen}\NormalTok{(cov\_matrix\_M)}\SpecialCharTok{$}\NormalTok{values}
\end{Highlighting}
\end{Shaded}

\begin{verbatim}
[1] 2.050769e+02 2.197051e+01 1.024041e-02 2.320348e-03
\end{verbatim}

\begin{Shaded}
\begin{Highlighting}[]
\NormalTok{cor\_mellon\_F }\OtherTok{\textless{}{-}}\NormalTok{ cor\_matrix\_Mellon\_F[covariates, covariates]}
\NormalTok{cov\_matrix\_F }\OtherTok{\textless{}{-}} \FunctionTok{diag}\NormalTok{(sds) }\SpecialCharTok{\%*\%}\NormalTok{ cor\_mellon\_F }\SpecialCharTok{\%*\%} \FunctionTok{diag}\NormalTok{(sds)}
\FunctionTok{eigen}\NormalTok{(cov\_matrix\_F)}\SpecialCharTok{$}\NormalTok{values}
\end{Highlighting}
\end{Shaded}

\begin{verbatim}
[1] 2.064991e+02 2.054837e+01 1.132511e-02 1.194553e-03
\end{verbatim}

\begin{Shaded}
\begin{Highlighting}[]
\CommentTok{\# Simulate}
\NormalTok{sim\_data\_M }\OtherTok{\textless{}{-}} \FunctionTok{correlated\_simulation}\NormalTok{(}\AttributeTok{n =} \DecValTok{93}\NormalTok{, }\AttributeTok{means =}\NormalTok{ means, }\AttributeTok{cov\_matrix =}\NormalTok{ cov\_matrix\_M)}
\NormalTok{sim\_data\_M }\OtherTok{\textless{}{-}} \FunctionTok{as\_tibble}\NormalTok{(sim\_data\_M) }\SpecialCharTok{\%\textgreater{}\%}
  \FunctionTok{mutate}\NormalTok{(}\AttributeTok{SEX =} \DecValTok{0}\NormalTok{)}

\NormalTok{sim\_data\_F }\OtherTok{\textless{}{-}} \FunctionTok{correlated\_simulation}\NormalTok{(}\AttributeTok{n =} \DecValTok{532}\NormalTok{, }\AttributeTok{means =}\NormalTok{ means, }\AttributeTok{cov\_matrix =}\NormalTok{ cov\_matrix\_F)}
\NormalTok{sim\_data\_F }\OtherTok{\textless{}{-}} \FunctionTok{as\_tibble}\NormalTok{(sim\_data\_F) }\SpecialCharTok{\%\textgreater{}\%}
  \FunctionTok{mutate}\NormalTok{(}\AttributeTok{SEX =} \DecValTok{1}\NormalTok{)}

\NormalTok{sim\_data }\OtherTok{\textless{}{-}} \FunctionTok{rbind}\NormalTok{(sim\_data\_M, sim\_data\_F)}

\CommentTok{\# Put the covariates together}
\NormalTok{sim\_cov\_mellon }\OtherTok{\textless{}{-}} \FunctionTok{tibble}\NormalTok{(}
  \AttributeTok{Paper =} \StringTok{"Mellon\_2020"}\NormalTok{,}
  \AttributeTok{ID\_within\_paper =} \DecValTok{1}\SpecialCharTok{:}\NormalTok{n\_patient,}
  \AttributeTok{ICU =} \DecValTok{0}\NormalTok{,}
  \AttributeTok{BURN =} \DecValTok{0}\NormalTok{,}
  \AttributeTok{OBESE =} \DecValTok{1}\NormalTok{,}
  \AttributeTok{CRCL =}\NormalTok{ sim\_data}\SpecialCharTok{$}\NormalTok{CRCL\_MDRD\_norm,}
  \AttributeTok{WT\_log   =}\NormalTok{ sim\_data}\SpecialCharTok{$}\NormalTok{WT\_log,}
  \AttributeTok{AGE  =}\NormalTok{ sim\_data}\SpecialCharTok{$}\NormalTok{AGE,}
  \AttributeTok{BMI\_log  =}\NormalTok{ sim\_data}\SpecialCharTok{$}\NormalTok{BMI\_log,}
  \AttributeTok{SEX =}\NormalTok{ sim\_data}\SpecialCharTok{$}\NormalTok{SEX)}

\NormalTok{sim\_cov\_mellon }\OtherTok{\textless{}{-}}\NormalTok{ sim\_cov\_mellon }\SpecialCharTok{\%\textgreater{}\%}
  \FunctionTok{mutate}\NormalTok{(}\AttributeTok{WT =} \FunctionTok{exp}\NormalTok{(WT\_log),}
         \AttributeTok{BMI =} \FunctionTok{exp}\NormalTok{(BMI\_log))}

\NormalTok{summary\_sim\_cov\_mellon }\OtherTok{\textless{}{-}}\NormalTok{ sim\_cov\_mellon }\SpecialCharTok{|\textgreater{}} 
  \FunctionTok{pivot\_longer}\NormalTok{(}\AttributeTok{cols =} \FunctionTok{c}\NormalTok{(ICU,BURN,OBESE,CRCL,WT,AGE),}
               \AttributeTok{names\_to =} \StringTok{"Covariate"}\NormalTok{) }\SpecialCharTok{|\textgreater{}} 
  \FunctionTok{summarise}\NormalTok{(}\AttributeTok{.by =} \FunctionTok{c}\NormalTok{(Covariate,Paper),}
            \AttributeTok{Min =} \FunctionTok{min}\NormalTok{(value),}
            \AttributeTok{Q1 =} \FunctionTok{quantile}\NormalTok{(value, }\FloatTok{0.25}\NormalTok{),}
            \AttributeTok{Median =} \FunctionTok{quantile}\NormalTok{(value, }\FloatTok{0.5}\NormalTok{),}
            \AttributeTok{Q3 =} \FunctionTok{quantile}\NormalTok{(value, }\FloatTok{0.75}\NormalTok{),}
            \AttributeTok{Max =} \FunctionTok{max}\NormalTok{(value))}
\end{Highlighting}
\end{Shaded}

\begin{Shaded}
\begin{Highlighting}[]
\NormalTok{sim\_cov\_mellon }\OtherTok{\textless{}{-}}\NormalTok{ sim\_cov\_mellon }\SpecialCharTok{\%\textgreater{}\%}
\NormalTok{  dplyr}\SpecialCharTok{::}\FunctionTok{select}\NormalTok{(Paper, ID\_within\_paper, ICU, BURN, OBESE, CRCL, WT, AGE, BMI, SEX)}

\NormalTok{cov\_mellon\_compare }\OtherTok{\textless{}{-}}\NormalTok{ summary\_sim\_cov\_mellon }\SpecialCharTok{|\textgreater{}}
  \FunctionTok{rename\_with}\NormalTok{(}\SpecialCharTok{\textasciitilde{}} \FunctionTok{paste0}\NormalTok{(.x, }\StringTok{"\_sim"}\NormalTok{), }\SpecialCharTok{{-}}\FunctionTok{one\_of}\NormalTok{(}\StringTok{"Covariate"}\NormalTok{, }\StringTok{"Paper"}\NormalTok{)) }\SpecialCharTok{|\textgreater{}}
  \FunctionTok{left\_join}\NormalTok{(}\FunctionTok{rename\_with}\NormalTok{(}
\NormalTok{    cov\_data\_mellon,}
    \SpecialCharTok{\textasciitilde{}} \FunctionTok{paste0}\NormalTok{(.x, }\StringTok{"\_true"}\NormalTok{),}
    \SpecialCharTok{{-}}\FunctionTok{one\_of}\NormalTok{(}\StringTok{"Covariate"}\NormalTok{, }\StringTok{"Paper"}\NormalTok{, }\StringTok{"Unit"}\NormalTok{)}
\NormalTok{  )) }\SpecialCharTok{|\textgreater{}} 
  \FunctionTok{relocate}\NormalTok{(Covariate, }\AttributeTok{.after =}\NormalTok{ Paper) }\SpecialCharTok{|\textgreater{}} 
  \FunctionTok{relocate}\NormalTok{(Unit, }\AttributeTok{.after =}\NormalTok{ Covariate)}

\NormalTok{cov\_mellon\_compare }\SpecialCharTok{|\textgreater{}}
  \FunctionTok{gt}\NormalTok{() }\SpecialCharTok{|\textgreater{}}
  \FunctionTok{fmt\_scientific}\NormalTok{() }\SpecialCharTok{|\textgreater{}} 
  \FunctionTok{tab\_spanner}\NormalTok{(}\AttributeTok{columns =} \FunctionTok{starts\_with}\NormalTok{(}\StringTok{"Min"}\NormalTok{),}
              \AttributeTok{label =} \StringTok{"Min"}\NormalTok{) }\SpecialCharTok{|\textgreater{}} 
  \FunctionTok{tab\_spanner}\NormalTok{(}\AttributeTok{columns =} \FunctionTok{starts\_with}\NormalTok{(}\StringTok{"Q1"}\NormalTok{),}
              \AttributeTok{label =} \StringTok{"Q1"}\NormalTok{) }\SpecialCharTok{|\textgreater{}} 
  \FunctionTok{tab\_spanner}\NormalTok{(}\AttributeTok{columns =} \FunctionTok{starts\_with}\NormalTok{(}\StringTok{"Median"}\NormalTok{),}
              \AttributeTok{label =} \StringTok{"Median"}\NormalTok{) }\SpecialCharTok{|\textgreater{}} 
  \FunctionTok{tab\_spanner}\NormalTok{(}\AttributeTok{columns =} \FunctionTok{starts\_with}\NormalTok{(}\StringTok{"Q3"}\NormalTok{),}
              \AttributeTok{label =} \StringTok{"Q3"}\NormalTok{) }\SpecialCharTok{|\textgreater{}} 
  \FunctionTok{tab\_spanner}\NormalTok{(}\AttributeTok{columns =} \FunctionTok{starts\_with}\NormalTok{(}\StringTok{"Max"}\NormalTok{),}
              \AttributeTok{label =} \StringTok{"Max"}\NormalTok{)}
\end{Highlighting}
\end{Shaded}

\begin{table}
\fontsize{12.0pt}{14.4pt}\selectfont
\begin{tabular*}{\linewidth}{@{\extracolsep{\fill}}lllrrrrrrrrrr}
\toprule
 &  &  & \multicolumn{2}{c}{Min} & \multicolumn{2}{c}{Q1} & \multicolumn{2}{c}{Median} & \multicolumn{2}{c}{Q3} & \multicolumn{2}{c}{Max} \\ 
\cmidrule(lr){4-5} \cmidrule(lr){6-7} \cmidrule(lr){8-9} \cmidrule(lr){10-11} \cmidrule(lr){12-13}
Paper & Covariate & Unit & Min\_sim & Min\_true & Q1\_sim & Q1\_true & Median\_sim & Median\_true & Q3\_sim & Q3\_true & Max\_sim & Max\_true \\ 
\midrule\addlinespace[2.5pt]
Mellon\_2020 & ICU & Unitless & 0.00 & 0.00 & 0.00 & 0.00 & 0.00 & 0.00 & 0.00 & 0.00 & 0.00 & 0.00 \\ 
Mellon\_2020 & BURN & Unitless & 0.00 & 0.00 & 0.00 & 0.00 & 0.00 & 0.00 & 0.00 & 0.00 & 0.00 & 0.00 \\ 
Mellon\_2020 & OBESE & Unitless & 1.00 & 1.00 & 1.00 & 1.00 & 1.00 & 1.00 & 1.00 & 1.00 & 1.00 & 1.00 \\ 
Mellon\_2020 & CRCL & mL/min/1.73m\textasciicircum{}\{2\} & 5.18 $\times$ 10\textsuperscript{1} & 6.20 $\times$ 10\textsuperscript{1} & 8.24 $\times$ 10\textsuperscript{1} & NA & 9.39 $\times$ 10\textsuperscript{1} & 9.40 $\times$ 10\textsuperscript{1} & 1.05 $\times$ 10\textsuperscript{2} & NA & 1.48 $\times$ 10\textsuperscript{2} & 1.33 $\times$ 10\textsuperscript{2} \\ 
Mellon\_2020 & CRCL & ml/min & 5.18 $\times$ 10\textsuperscript{1} & NA & 8.24 $\times$ 10\textsuperscript{1} & NA & 9.39 $\times$ 10\textsuperscript{1} & NA & 1.05 $\times$ 10\textsuperscript{2} & NA & 1.48 $\times$ 10\textsuperscript{2} & NA \\ 
Mellon\_2020 & WT & kg & 8.10 $\times$ 10\textsuperscript{1} & 8.80 $\times$ 10\textsuperscript{1} & 1.02 $\times$ 10\textsuperscript{2} & NA & 1.09 $\times$ 10\textsuperscript{2} & 1.09 $\times$ 10\textsuperscript{2} & 1.16 $\times$ 10\textsuperscript{2} & NA & 1.41 $\times$ 10\textsuperscript{2} & 1.51 $\times$ 10\textsuperscript{2} \\ 
Mellon\_2020 & AGE & year & 3.51 $\times$ 10\textsuperscript{1} & 2.29 $\times$ 10\textsuperscript{1} & 4.87 $\times$ 10\textsuperscript{1} & NA & 5.22 $\times$ 10\textsuperscript{1} & 5.17 $\times$ 10\textsuperscript{1} & 5.57 $\times$ 10\textsuperscript{1} & NA & 7.07 $\times$ 10\textsuperscript{1} & 6.29 $\times$ 10\textsuperscript{1} \\ 
\bottomrule
\end{tabular*}
\end{table}

Then, based on the BMI and WT, the height (HT) is calculated to be able
to calculate the body surface area (BSA). CRCL is reconverted to serum
creatinine (CREAT) based on the MDRD equation.

\begin{Shaded}
\begin{Highlighting}[]
\NormalTok{reconvert\_MDRD }\OtherTok{\textless{}{-}} \ControlFlowTok{function}\NormalTok{(AGE, CRCL, SEX) \{}
    \CommentTok{\# Mutliply by 0.85 for females}
\NormalTok{    sex\_factor }\OtherTok{\textless{}{-}} \FunctionTok{ifelse}\NormalTok{(SEX }\SpecialCharTok{==} \DecValTok{0}\NormalTok{, }\DecValTok{1}\NormalTok{, }\FloatTok{0.742}\NormalTok{)}
    
    \CommentTok{\# Calculate DFG}
\NormalTok{   CREAT }\OtherTok{\textless{}{-}}\NormalTok{ (CRCL }\SpecialCharTok{/}\NormalTok{ (}\DecValTok{175} \SpecialCharTok{*}\NormalTok{ (AGE}\SpecialCharTok{\^{}}\NormalTok{(}\SpecialCharTok{{-}}\FloatTok{0.203}\NormalTok{)) }\SpecialCharTok{*}\NormalTok{ sex\_factor))}\SpecialCharTok{\^{}}\NormalTok{(}\SpecialCharTok{{-}}\DecValTok{1} \SpecialCharTok{/} \FloatTok{1.154}\NormalTok{)}
  
  \FunctionTok{return}\NormalTok{(CREAT)}
\NormalTok{\}}

\NormalTok{sim\_cov\_mellon}\SpecialCharTok{$}\NormalTok{CREAT }\OtherTok{\textless{}{-}} \FunctionTok{mapply}\NormalTok{(reconvert\_MDRD, sim\_cov\_mellon}\SpecialCharTok{$}\NormalTok{AGE, sim\_cov\_mellon}\SpecialCharTok{$}\NormalTok{CRCL, sim\_cov\_mellon}\SpecialCharTok{$}\NormalTok{SEX)}
\NormalTok{sim\_cov\_mellon }\OtherTok{\textless{}{-}}\NormalTok{ sim\_cov\_mellon }\SpecialCharTok{\%\textgreater{}\%}
  \FunctionTok{mutate}\NormalTok{(}
   \AttributeTok{HT =} \FunctionTok{sqrt}\NormalTok{(WT}\SpecialCharTok{/}\NormalTok{BMI) }\SpecialCharTok{*} \DecValTok{100}\NormalTok{, }\CommentTok{\# mutliplied by 100 to convert m to cm}
    \AttributeTok{BSA =}\NormalTok{ (WT}\SpecialCharTok{\^{}}\FloatTok{0.425} \SpecialCharTok{*}\NormalTok{ (HT}\SpecialCharTok{\^{}}\FloatTok{0.725}\NormalTok{) }\SpecialCharTok{*} \FloatTok{0.007184}\NormalTok{),}
    \AttributeTok{OBESE =} \FunctionTok{ifelse}\NormalTok{(BMI }\SpecialCharTok{\textgreater{}} \DecValTok{30}\NormalTok{, }\DecValTok{1}\NormalTok{, }\DecValTok{0}\NormalTok{)}
\NormalTok{  ) }\SpecialCharTok{\%\textgreater{}\%}
\NormalTok{  dplyr}\SpecialCharTok{::}\FunctionTok{select}\NormalTok{(Paper, ID\_within\_paper, ICU, BURN, OBESE, CREAT, WT, BSA, BMI, AGE, SEX, HT)}

\CommentTok{\# Calculate the simulated correlation matrices separately for the two sexes to compare with the original}
\NormalTok{sim\_cov\_mellon\_M }\OtherTok{\textless{}{-}}\NormalTok{ sim\_cov\_mellon }\SpecialCharTok{\%\textgreater{}\%} \FunctionTok{filter}\NormalTok{(SEX }\SpecialCharTok{==} \DecValTok{0}\NormalTok{)}
\NormalTok{sim\_cov\_mellon\_F }\OtherTok{\textless{}{-}}\NormalTok{ sim\_cov\_mellon }\SpecialCharTok{\%\textgreater{}\%} \FunctionTok{filter}\NormalTok{(SEX }\SpecialCharTok{==} \DecValTok{1}\NormalTok{)}

\NormalTok{cor\_sim\_M }\OtherTok{\textless{}{-}} \FunctionTok{cor}\NormalTok{(sim\_cov\_mellon\_M }\SpecialCharTok{\%\textgreater{}\%}\NormalTok{ dplyr}\SpecialCharTok{::}\FunctionTok{select}\NormalTok{(WT, CREAT, AGE, HT, BMI), }\AttributeTok{use =} \StringTok{"complete.obs"}\NormalTok{)}
\NormalTok{cor\_sim\_F }\OtherTok{\textless{}{-}} \FunctionTok{cor}\NormalTok{(sim\_cov\_mellon\_F }\SpecialCharTok{\%\textgreater{}\%}\NormalTok{ dplyr}\SpecialCharTok{::}\FunctionTok{select}\NormalTok{(WT, CREAT, AGE, HT, BMI), }\AttributeTok{use =} \StringTok{"complete.obs"}\NormalTok{)}

\FunctionTok{ggcorrplot}\NormalTok{(cor\_matrix\_Mellon\_F, }\AttributeTok{lab =} \ConstantTok{TRUE}\NormalTok{, }\AttributeTok{title =} \StringTok{"Mellon original (females)"}\NormalTok{)}
\end{Highlighting}
\end{Shaded}

\pandocbounded{\includegraphics[keepaspectratio]{Simulations/covariate_simulation_files/figure-pdf/reconvert_creat_mellon-1.pdf}}

\begin{Shaded}
\begin{Highlighting}[]
\NormalTok{Mellon\_F }\OtherTok{\textless{}{-}} \FunctionTok{ggcorrplot}\NormalTok{(cor\_sim\_F, }\AttributeTok{lab =} \ConstantTok{TRUE}\NormalTok{, }\AttributeTok{title =} \StringTok{"Mellon simulated (females)"}\NormalTok{)}
\NormalTok{Mellon\_F}
\end{Highlighting}
\end{Shaded}

\pandocbounded{\includegraphics[keepaspectratio]{Simulations/covariate_simulation_files/figure-pdf/reconvert_creat_mellon-2.pdf}}

\begin{Shaded}
\begin{Highlighting}[]
\FunctionTok{ggcorrplot}\NormalTok{(cor\_matrix\_Mellon\_M, }\AttributeTok{lab =} \ConstantTok{TRUE}\NormalTok{, }\AttributeTok{title =} \StringTok{"Mellon original (males)"}\NormalTok{)}
\end{Highlighting}
\end{Shaded}

\pandocbounded{\includegraphics[keepaspectratio]{Simulations/covariate_simulation_files/figure-pdf/reconvert_creat_mellon-3.pdf}}

\begin{Shaded}
\begin{Highlighting}[]
\NormalTok{Mellon\_M }\OtherTok{\textless{}{-}} \FunctionTok{ggcorrplot}\NormalTok{(cor\_sim\_M, }\AttributeTok{lab =} \ConstantTok{TRUE}\NormalTok{, }\AttributeTok{title =} \StringTok{"Mellon simulated (males)"}\NormalTok{)}
\NormalTok{Mellon\_M}
\end{Highlighting}
\end{Shaded}

\pandocbounded{\includegraphics[keepaspectratio]{Simulations/covariate_simulation_files/figure-pdf/reconvert_creat_mellon-4.pdf}}

\section{Rambaud et al. (2020)}\label{rambaud2020}

\subsection{Covariate distribution}\label{covariate-distribution-3}

All patients are ICU patients with endocarditis. They are not considered
to be burn victims or obese. Creatinine clearance was calculated using
the CKD-EPI equation. Thus for all patients ICU = 1, BURN = 0 and
OBESE=0. In this paper, the distribution of serum creatinine is given
(in micromol/L), so it can be directly simulated instead of reconverting
CRCL. The proportion of males is 0.794.

For WT and CRCL here are the data from the paper :

\begin{Shaded}
\begin{Highlighting}[]
\NormalTok{cov\_data\_rambaud }\OtherTok{\textless{}{-}}\NormalTok{ cov\_data }\SpecialCharTok{|\textgreater{}} 
  \FunctionTok{filter}\NormalTok{(Paper }\SpecialCharTok{==} \StringTok{"Rambaud\_2020"}\NormalTok{)}

\NormalTok{cov\_data\_rambaud }\SpecialCharTok{|\textgreater{}} 
  \FunctionTok{filter}\NormalTok{(Covariate }\SpecialCharTok{\%in\%} \FunctionTok{c}\NormalTok{(}\StringTok{"WT"}\NormalTok{,}\StringTok{"CREAT"}\NormalTok{,}\StringTok{"AGE"}\NormalTok{,}\StringTok{"HT"}\NormalTok{)) }\SpecialCharTok{|\textgreater{}} 
  \FunctionTok{kable}\NormalTok{()}
\end{Highlighting}
\end{Shaded}

\begin{longtable}[]{@{}lllrrrrr@{}}
\toprule\noalign{}
Paper & Covariate & Unit & Median & Q1 & Q3 & Min & Max \\
\midrule\noalign{}
\endhead
\bottomrule\noalign{}
\endlastfoot
Rambaud\_2020 & WT & kg & 76 & 69 & 87.0 & NA & NA \\
Rambaud\_2020 & HT & cm & 170 & 166 & 175.0 & NA & NA \\
Rambaud\_2020 & AGE & year & 72 & 62 & 80.5 & NA & NA \\
Rambaud\_2020 & CREAT & micromol/L & 87 & 73 & 115.5 & NA & NA \\
\end{longtable}

\begin{Shaded}
\begin{Highlighting}[]
\NormalTok{cor\_matrix\_Rambaud\_F }\OtherTok{\textless{}{-}} \FunctionTok{readRDS}\NormalTok{(}\FunctionTok{here}\NormalTok{(}\StringTok{"a\_priori/For\_publication/Simulations/cor\_matrix\_Rambaud\_F.rds"}\NormalTok{))}
\FunctionTok{print}\NormalTok{(cor\_matrix\_Rambaud\_F)}
\end{Highlighting}
\end{Shaded}

\begin{verbatim}
                     CREAT   CREAT_log        AGE         WT     WT_log
CREAT           1.00000000  0.96241769  0.2320510  0.1329905  0.1360852
CREAT_log       0.96241769  1.00000000  0.2429180  0.1521950  0.1565402
AGE             0.23205099  0.24291801  1.0000000 -0.1726236 -0.1683232
WT              0.13299046  0.15219495 -0.1726236  1.0000000  0.9894565
WT_log          0.13608520  0.15654017 -0.1683232  0.9894565  1.0000000
HT             -0.04892509 -0.04044581 -0.1889338  0.2517176  0.2592944
HT_log         -0.04916482 -0.04082750 -0.1894298  0.2520660  0.2599500
BMI             0.15464517  0.17128168 -0.1162404  0.9435277  0.9336719
CRCL_MDRD_norm -0.78792478 -0.91391033 -0.2872002 -0.1410118 -0.1466372
                        HT      HT_log        BMI CRCL_MDRD_norm
CREAT          -0.04892509 -0.04916482  0.1546452    -0.78792478
CREAT_log      -0.04044581 -0.04082750  0.1712817    -0.91391033
AGE            -0.18893375 -0.18942976 -0.1162404    -0.28720016
WT              0.25171756  0.25206601  0.9435277    -0.14101184
WT_log          0.25929441  0.25994996  0.9336719    -0.14663719
HT              1.00000000  0.99970438 -0.0747791     0.04109155
HT_log          0.99970438  1.00000000 -0.0742787     0.04153856
BMI            -0.07477910 -0.07427870  1.0000000    -0.15925725
CRCL_MDRD_norm  0.04109155  0.04153856 -0.1592572     1.00000000
\end{verbatim}

\begin{Shaded}
\begin{Highlighting}[]
\FunctionTok{ggcorrplot}\NormalTok{(cor\_matrix\_Rambaud\_F, }\AttributeTok{lab =} \ConstantTok{TRUE}\NormalTok{, }\AttributeTok{title =} \StringTok{"Sex: female"}\NormalTok{)}
\end{Highlighting}
\end{Shaded}

\pandocbounded{\includegraphics[keepaspectratio]{Simulations/covariate_simulation_files/figure-pdf/Rambaud-cor-matrix-1.pdf}}

\begin{Shaded}
\begin{Highlighting}[]
\NormalTok{cor\_matrix\_Rambaud\_M }\OtherTok{\textless{}{-}} \FunctionTok{readRDS}\NormalTok{(}\FunctionTok{here}\NormalTok{(}\StringTok{"a\_priori/For\_publication/Simulations/cor\_matrix\_Rambaud\_M.rds"}\NormalTok{))}
\FunctionTok{print}\NormalTok{(cor\_matrix\_Rambaud\_M)}
\end{Highlighting}
\end{Shaded}

\begin{verbatim}
                      CREAT     CREAT_log        AGE          WT      WT_log
CREAT           1.000000000  0.9686820232  0.1909639  0.09370644  0.09181062
CREAT_log       0.968682023  1.0000000000  0.1982311  0.12356685  0.12378068
AGE             0.190963871  0.1982311079  1.0000000 -0.18051719 -0.17233815
WT              0.093706437  0.1235668470 -0.1805172  1.00000000  0.99225399
WT_log          0.091810623  0.1237806818 -0.1723382  0.99225399  1.00000000
HT             -0.005710820  0.0007327162 -0.1229902  0.37551772  0.37978066
HT_log         -0.005537361  0.0010029509 -0.1228308  0.37500256  0.37967684
BMI             0.103837382  0.1331126943 -0.1347419  0.89436959  0.88815355
CRCL_MDRD_norm -0.769174641 -0.8857730397 -0.2449519 -0.12508733 -0.12881176
                          HT       HT_log         BMI CRCL_MDRD_norm
CREAT          -0.0057108200 -0.005537361  0.10383738   -0.769174641
CREAT_log       0.0007327162  0.001002951  0.13311269   -0.885773040
AGE            -0.1229902101 -0.122830828 -0.13474194   -0.244951935
WT              0.3755177211  0.375002564  0.89436959   -0.125087326
WT_log          0.3797806561  0.379676845  0.88815355   -0.128811758
HT              1.0000000000  0.999598212 -0.07118055   -0.003915605
HT_log          0.9995982117  1.000000000 -0.07162513   -0.004244813
BMI            -0.0711805452 -0.071625134  1.00000000   -0.133475436
CRCL_MDRD_norm -0.0039156049 -0.004244813 -0.13347544    1.000000000
\end{verbatim}

\begin{Shaded}
\begin{Highlighting}[]
\FunctionTok{ggcorrplot}\NormalTok{(cor\_matrix\_Rambaud\_M, }\AttributeTok{lab =} \ConstantTok{TRUE}\NormalTok{, }\AttributeTok{title =} \StringTok{"Sex: male"}\NormalTok{)}
\end{Highlighting}
\end{Shaded}

\pandocbounded{\includegraphics[keepaspectratio]{Simulations/covariate_simulation_files/figure-pdf/Rambaud-cor-matrix-2.pdf}}

\begin{Shaded}
\begin{Highlighting}[]
\NormalTok{rambaud\_creat }\OtherTok{\textless{}{-}} \FunctionTok{fit\_cov\_wrapper}\NormalTok{(}
  \AttributeTok{cov\_data =}\NormalTok{ cov\_data\_rambaud,}
  \AttributeTok{cov\_name =} \StringTok{"CREAT"}\NormalTok{,}
  \AttributeTok{min\_plot =} \DecValTok{0}\NormalTok{,}
  \AttributeTok{max\_plot =} \DecValTok{250}
\NormalTok{)}

\NormalTok{rambaud\_creat}\SpecialCharTok{$}\NormalTok{plot}
\end{Highlighting}
\end{Shaded}

\pandocbounded{\includegraphics[keepaspectratio]{Simulations/covariate_simulation_files/figure-pdf/fit-rambaud-creat-1.pdf}}

Log-normal distribution fits the best the data, so it is selected.

\begin{Shaded}
\begin{Highlighting}[]
\NormalTok{rambaud\_wt }\OtherTok{\textless{}{-}} \FunctionTok{fit\_cov\_wrapper}\NormalTok{(}
  \AttributeTok{cov\_data =}\NormalTok{ cov\_data\_rambaud,}
  \AttributeTok{cov\_name =} \StringTok{"WT"}\NormalTok{,}
  \AttributeTok{min\_plot =} \DecValTok{40}\NormalTok{,}
  \AttributeTok{max\_plot =} \DecValTok{120}
\NormalTok{)}

\NormalTok{rambaud\_wt}\SpecialCharTok{$}\NormalTok{plot}
\end{Highlighting}
\end{Shaded}

\pandocbounded{\includegraphics[keepaspectratio]{Simulations/covariate_simulation_files/figure-pdf/fit-rambaud-wt-1.pdf}}

No discernable difference between the two fits. The log-normal
distribution is selected.

\begin{Shaded}
\begin{Highlighting}[]
\NormalTok{rambaud\_age }\OtherTok{\textless{}{-}} \FunctionTok{fit\_cov\_wrapper}\NormalTok{(}
  \AttributeTok{cov\_data =}\NormalTok{ cov\_data\_rambaud,}
  \AttributeTok{cov\_name =} \StringTok{"AGE"}\NormalTok{,}
  \AttributeTok{min\_plot =} \DecValTok{0}\NormalTok{,}
  \AttributeTok{max\_plot =} \DecValTok{120}
\NormalTok{)}

\NormalTok{rambaud\_age}\SpecialCharTok{$}\NormalTok{plot}
\end{Highlighting}
\end{Shaded}

\pandocbounded{\includegraphics[keepaspectratio]{Simulations/covariate_simulation_files/figure-pdf/fit-rambaud-age-1.pdf}}

No discernable difference, so normal distribution is selected, as age
usually follows a normal distribution.

\begin{Shaded}
\begin{Highlighting}[]
\NormalTok{rambaud\_ht }\OtherTok{\textless{}{-}} \FunctionTok{fit\_cov\_wrapper}\NormalTok{(}
  \AttributeTok{cov\_data =}\NormalTok{ cov\_data\_rambaud,}
  \AttributeTok{cov\_name =} \StringTok{"HT"}\NormalTok{,}
  \AttributeTok{min\_plot =} \DecValTok{140}\NormalTok{,}
  \AttributeTok{max\_plot =} \DecValTok{200}
\NormalTok{)}

\NormalTok{rambaud\_ht}\SpecialCharTok{$}\NormalTok{plot}
\end{Highlighting}
\end{Shaded}

\pandocbounded{\includegraphics[keepaspectratio]{Simulations/covariate_simulation_files/figure-pdf/fit-rambaud-ht-1.pdf}}

No discernable difference. Log-normal distribution is selected.

\subsection{Simulation of covariates}\label{simulation-of-covariates-3}

Covariates will be sampled from the following distributions :

\begin{itemize}
\tightlist
\item
  CREAT : Log-normal distribution with a minimum 10 micromol/L with mean
  89.7 micromol/L and standard deviation 0.346 micromol/L
\item
  WT : Normal distribution with resampling of negative values with mean
  76.8 kg, and standard deviation 17.3 \%
\item
  AGE : Normal distribution with resampling of negative values with mean
  \ensuremath{1.22\times 10^{31}} years, and coefficient of variation
  1370 \%
\item
  HT : Normal distribution with resampling of negative values with mean
  170 cm, and standard deviation 3.93 \%
\end{itemize}

The proportion of males is 0.794 in the original article, so the number
of males is 496 and the number of females is 129 to add up to 625.

\begin{Shaded}
\begin{Highlighting}[]
\NormalTok{correlated\_simulation }\OtherTok{\textless{}{-}} \ControlFlowTok{function}\NormalTok{(n, means, cov\_matrix) \{}
  \FunctionTok{set.seed}\NormalTok{(}\DecValTok{1991}\NormalTok{)}
\NormalTok{  collected }\OtherTok{\textless{}{-}} \FunctionTok{matrix}\NormalTok{(}\ConstantTok{NA}\NormalTok{, }\DecValTok{0}\NormalTok{, }\FunctionTok{length}\NormalTok{(means))}
  \FunctionTok{colnames}\NormalTok{(collected) }\OtherTok{\textless{}{-}} \FunctionTok{names}\NormalTok{(means)}

  \ControlFlowTok{while}\NormalTok{ (}\FunctionTok{nrow}\NormalTok{(collected) }\SpecialCharTok{\textless{}}\NormalTok{ n) \{}
\NormalTok{    batch }\OtherTok{\textless{}{-}}\NormalTok{ MASS}\SpecialCharTok{::}\FunctionTok{mvrnorm}\NormalTok{(}\AttributeTok{n =}\NormalTok{ n, }\AttributeTok{mu =}\NormalTok{ means, }\AttributeTok{Sigma =}\NormalTok{ cov\_matrix)}
    \FunctionTok{colnames}\NormalTok{(batch) }\OtherTok{\textless{}{-}} \FunctionTok{names}\NormalTok{(means)}

\NormalTok{    valid }\OtherTok{\textless{}{-}}\NormalTok{ batch[, }\StringTok{"AGE"}\NormalTok{] }\SpecialCharTok{\textgreater{}=} \DecValTok{18} \SpecialCharTok{\&}
\NormalTok{             batch[, }\StringTok{"CREAT\_log"}\NormalTok{] }\SpecialCharTok{\textless{}} \FloatTok{5.991465} \CommentTok{\# log(400) corresponding to CRCL of 10 mL/min for a patient of 24 kg and 116 years}

\NormalTok{    batch\_valid }\OtherTok{\textless{}{-}}\NormalTok{ batch[valid, , drop }\OtherTok{=} \ConstantTok{FALSE}\NormalTok{]}
\NormalTok{    collected }\OtherTok{\textless{}{-}} \FunctionTok{rbind}\NormalTok{(collected, batch\_valid)}
\NormalTok{  \}}

\NormalTok{  collected[}\DecValTok{1}\SpecialCharTok{:}\NormalTok{n, , drop }\OtherTok{=} \ConstantTok{FALSE}\NormalTok{]}
\NormalTok{\}}

\NormalTok{means }\OtherTok{\textless{}{-}} \FunctionTok{c}\NormalTok{(}
  \AttributeTok{CREAT\_log =}\NormalTok{ rambaud\_creat}\SpecialCharTok{$}\NormalTok{lnorm\_par[}\StringTok{"meanlog"}\NormalTok{],}
  \AttributeTok{WT\_log      =}\NormalTok{ rambaud\_wt}\SpecialCharTok{$}\NormalTok{lnorm\_par[}\StringTok{"meanlog"}\NormalTok{], }
  \AttributeTok{AGE     =}\NormalTok{ rambaud\_age}\SpecialCharTok{$}\NormalTok{norm\_par[}\StringTok{"mean"}\NormalTok{],}
  \AttributeTok{HT\_log      =}\NormalTok{ rambaud\_ht}\SpecialCharTok{$}\NormalTok{lnorm\_par[}\StringTok{"meanlog"}\NormalTok{]}
\NormalTok{)}

\NormalTok{sds }\OtherTok{\textless{}{-}} \FunctionTok{c}\NormalTok{(}
  \AttributeTok{CREAT\_log =}\NormalTok{ rambaud\_creat}\SpecialCharTok{$}\NormalTok{lnorm\_par[}\StringTok{"sdlog"}\NormalTok{],}
  \AttributeTok{WT\_log     =}\NormalTok{ rambaud\_wt}\SpecialCharTok{$}\NormalTok{lnorm\_par[}\StringTok{"sdlog"}\NormalTok{],}
  \AttributeTok{AGE =}\NormalTok{ rambaud\_age}\SpecialCharTok{$}\NormalTok{norm\_par[}\StringTok{"sd"}\NormalTok{],}
  \AttributeTok{HT\_log     =}\NormalTok{ rambaud\_ht}\SpecialCharTok{$}\NormalTok{lnorm\_par[}\StringTok{"sdlog"}\NormalTok{]}
\NormalTok{)}

\NormalTok{covariates }\OtherTok{\textless{}{-}} \FunctionTok{c}\NormalTok{(}\StringTok{"CREAT\_log"}\NormalTok{, }\StringTok{"WT\_log"}\NormalTok{, }\StringTok{"AGE"}\NormalTok{, }\StringTok{"HT\_log"}\NormalTok{)}

\FunctionTok{names}\NormalTok{(means) }\OtherTok{\textless{}{-}}\NormalTok{ covariates}

\CommentTok{\# Correlation and Covariance Matrix}
\NormalTok{cor\_rambaud\_M }\OtherTok{\textless{}{-}}\NormalTok{ cor\_matrix\_Rambaud\_M[covariates, covariates]}
\NormalTok{cov\_matrix\_M }\OtherTok{\textless{}{-}} \FunctionTok{diag}\NormalTok{(sds) }\SpecialCharTok{\%*\%}\NormalTok{ cor\_rambaud\_M }\SpecialCharTok{\%*\%} \FunctionTok{diag}\NormalTok{(sds)}
\FunctionTok{eigen}\NormalTok{(cov\_matrix\_M)}\SpecialCharTok{$}\NormalTok{values}
\end{Highlighting}
\end{Shaded}

\begin{verbatim}
[1] 1.888441e+02 1.160604e-01 2.831560e-02 1.302147e-03
\end{verbatim}

\begin{Shaded}
\begin{Highlighting}[]
\NormalTok{cor\_rambaud\_F }\OtherTok{\textless{}{-}}\NormalTok{ cor\_matrix\_Rambaud\_F[covariates, covariates]}
\NormalTok{cov\_matrix\_F }\OtherTok{\textless{}{-}} \FunctionTok{diag}\NormalTok{(sds) }\SpecialCharTok{\%*\%}\NormalTok{ cor\_rambaud\_F }\SpecialCharTok{\%*\%} \FunctionTok{diag}\NormalTok{(sds)}
\FunctionTok{eigen}\NormalTok{(cov\_matrix\_F)}\SpecialCharTok{$}\NormalTok{values}
\end{Highlighting}
\end{Shaded}

\begin{verbatim}
[1] 1.888465e+02 1.143105e-01 2.761902e-02 1.396003e-03
\end{verbatim}

\begin{Shaded}
\begin{Highlighting}[]
\CommentTok{\# Simulate}
\NormalTok{sim\_data\_M }\OtherTok{\textless{}{-}} \FunctionTok{correlated\_simulation}\NormalTok{(}\AttributeTok{n =} \DecValTok{496}\NormalTok{, }\AttributeTok{means =}\NormalTok{ means, }\AttributeTok{cov\_matrix =}\NormalTok{ cov\_matrix\_M)}
\NormalTok{sim\_data\_M }\OtherTok{\textless{}{-}} \FunctionTok{as\_tibble}\NormalTok{(sim\_data\_M) }\SpecialCharTok{\%\textgreater{}\%}
  \FunctionTok{mutate}\NormalTok{(}\AttributeTok{SEX =} \DecValTok{0}\NormalTok{)}

\NormalTok{sim\_data\_F }\OtherTok{\textless{}{-}} \FunctionTok{correlated\_simulation}\NormalTok{(}\AttributeTok{n =} \DecValTok{129}\NormalTok{, }\AttributeTok{means =}\NormalTok{ means, }\AttributeTok{cov\_matrix =}\NormalTok{ cov\_matrix\_F)}
\NormalTok{sim\_data\_F }\OtherTok{\textless{}{-}} \FunctionTok{as\_tibble}\NormalTok{(sim\_data\_F) }\SpecialCharTok{\%\textgreater{}\%}
  \FunctionTok{mutate}\NormalTok{(}\AttributeTok{SEX =} \DecValTok{1}\NormalTok{)}

\NormalTok{sim\_data }\OtherTok{\textless{}{-}} \FunctionTok{rbind}\NormalTok{(sim\_data\_M, sim\_data\_F)}

\CommentTok{\# Put the covariates together}
\NormalTok{sim\_cov\_rambaud }\OtherTok{\textless{}{-}} \FunctionTok{tibble}\NormalTok{(}
  \AttributeTok{Paper =} \StringTok{"Rambaud\_2020"}\NormalTok{,}
  \AttributeTok{ID\_within\_paper =} \DecValTok{1}\SpecialCharTok{:}\NormalTok{n\_patient,}
  \AttributeTok{ICU =} \DecValTok{1}\NormalTok{,}
  \AttributeTok{BURN =} \DecValTok{0}\NormalTok{,}
  \AttributeTok{OBESE =} \DecValTok{0}\NormalTok{,}
  \AttributeTok{CREAT\_log =}\NormalTok{ sim\_data}\SpecialCharTok{$}\NormalTok{CREAT\_log,}
  \AttributeTok{WT\_log   =}\NormalTok{ sim\_data}\SpecialCharTok{$}\NormalTok{WT\_log,}
  \AttributeTok{AGE  =}\NormalTok{ sim\_data}\SpecialCharTok{$}\NormalTok{AGE,}
  \AttributeTok{HT\_log  =}\NormalTok{ sim\_data}\SpecialCharTok{$}\NormalTok{HT\_log,}
  \AttributeTok{SEX =}\NormalTok{ sim\_data}\SpecialCharTok{$}\NormalTok{SEX)}

\NormalTok{sim\_cov\_rambaud }\OtherTok{\textless{}{-}}\NormalTok{ sim\_cov\_rambaud }\SpecialCharTok{\%\textgreater{}\%}
  \FunctionTok{mutate}\NormalTok{(}\AttributeTok{CREAT =} \FunctionTok{exp}\NormalTok{(CREAT\_log),}
         \AttributeTok{HT =} \FunctionTok{exp}\NormalTok{(HT\_log),}
         \AttributeTok{WT =} \FunctionTok{exp}\NormalTok{(WT\_log))}

\NormalTok{summary\_sim\_cov\_rambaud }\OtherTok{\textless{}{-}}\NormalTok{ sim\_cov\_rambaud }\SpecialCharTok{|\textgreater{}} 
  \FunctionTok{pivot\_longer}\NormalTok{(}\AttributeTok{cols =} \FunctionTok{c}\NormalTok{(ICU,BURN,OBESE,CREAT,WT,AGE,HT),}
               \AttributeTok{names\_to =} \StringTok{"Covariate"}\NormalTok{) }\SpecialCharTok{|\textgreater{}} 
  \FunctionTok{summarise}\NormalTok{(}\AttributeTok{.by =} \FunctionTok{c}\NormalTok{(Covariate,Paper),}
            \AttributeTok{Min =} \FunctionTok{min}\NormalTok{(value),}
            \AttributeTok{Q1 =} \FunctionTok{quantile}\NormalTok{(value, }\FloatTok{0.25}\NormalTok{),}
            \AttributeTok{Median =} \FunctionTok{quantile}\NormalTok{(value, }\FloatTok{0.5}\NormalTok{),}
            \AttributeTok{Q3 =} \FunctionTok{quantile}\NormalTok{(value, }\FloatTok{0.75}\NormalTok{),}
            \AttributeTok{Max =} \FunctionTok{max}\NormalTok{(value))}
\end{Highlighting}
\end{Shaded}

\begin{Shaded}
\begin{Highlighting}[]
\NormalTok{cov\_rambaud\_compare }\OtherTok{\textless{}{-}}\NormalTok{ summary\_sim\_cov\_rambaud }\SpecialCharTok{|\textgreater{}}
  \FunctionTok{rename\_with}\NormalTok{(}\SpecialCharTok{\textasciitilde{}} \FunctionTok{paste0}\NormalTok{(.x, }\StringTok{"\_sim"}\NormalTok{), }\SpecialCharTok{{-}}\FunctionTok{one\_of}\NormalTok{(}\StringTok{"Covariate"}\NormalTok{, }\StringTok{"Paper"}\NormalTok{)) }\SpecialCharTok{|\textgreater{}}
  \FunctionTok{left\_join}\NormalTok{(}\FunctionTok{rename\_with}\NormalTok{(}
\NormalTok{    cov\_data\_rambaud,}
    \SpecialCharTok{\textasciitilde{}} \FunctionTok{paste0}\NormalTok{(.x, }\StringTok{"\_true"}\NormalTok{),}
    \SpecialCharTok{{-}}\FunctionTok{one\_of}\NormalTok{(}\StringTok{"Covariate"}\NormalTok{, }\StringTok{"Paper"}\NormalTok{, }\StringTok{"Unit"}\NormalTok{)}
\NormalTok{  )) }\SpecialCharTok{|\textgreater{}} 
  \FunctionTok{relocate}\NormalTok{(Covariate, }\AttributeTok{.after =}\NormalTok{ Paper) }\SpecialCharTok{|\textgreater{}} 
  \FunctionTok{relocate}\NormalTok{(Unit, }\AttributeTok{.after =}\NormalTok{ Covariate)}

\NormalTok{cov\_rambaud\_compare }\SpecialCharTok{|\textgreater{}}
  \FunctionTok{gt}\NormalTok{() }\SpecialCharTok{|\textgreater{}}
  \FunctionTok{fmt\_scientific}\NormalTok{() }\SpecialCharTok{|\textgreater{}} 
    \FunctionTok{tab\_spanner}\NormalTok{(}\AttributeTok{columns =} \FunctionTok{starts\_with}\NormalTok{(}\StringTok{"Min"}\NormalTok{),}
              \AttributeTok{label =} \StringTok{"Min"}\NormalTok{) }\SpecialCharTok{|\textgreater{}} 
      \FunctionTok{tab\_spanner}\NormalTok{(}\AttributeTok{columns =} \FunctionTok{starts\_with}\NormalTok{(}\StringTok{"Q1"}\NormalTok{),}
              \AttributeTok{label =} \StringTok{"Q1"}\NormalTok{) }\SpecialCharTok{|\textgreater{}} 
      \FunctionTok{tab\_spanner}\NormalTok{(}\AttributeTok{columns =} \FunctionTok{starts\_with}\NormalTok{(}\StringTok{"Median"}\NormalTok{),}
              \AttributeTok{label =} \StringTok{"Median"}\NormalTok{) }\SpecialCharTok{|\textgreater{}} 
      \FunctionTok{tab\_spanner}\NormalTok{(}\AttributeTok{columns =} \FunctionTok{starts\_with}\NormalTok{(}\StringTok{"Q3"}\NormalTok{),}
              \AttributeTok{label =} \StringTok{"Q3"}\NormalTok{) }\SpecialCharTok{|\textgreater{}} 
  \FunctionTok{tab\_spanner}\NormalTok{(}\AttributeTok{columns =} \FunctionTok{starts\_with}\NormalTok{(}\StringTok{"Max"}\NormalTok{),}
              \AttributeTok{label =} \StringTok{"Max"}\NormalTok{)}
\end{Highlighting}
\end{Shaded}

\begin{table}
\fontsize{12.0pt}{14.4pt}\selectfont
\begin{tabular*}{\linewidth}{@{\extracolsep{\fill}}lllrrrrrrrrrr}
\toprule
 &  &  & \multicolumn{2}{c}{Min} & \multicolumn{2}{c}{Q1} & \multicolumn{2}{c}{Median} & \multicolumn{2}{c}{Q3} & \multicolumn{2}{c}{Max} \\ 
\cmidrule(lr){4-5} \cmidrule(lr){6-7} \cmidrule(lr){8-9} \cmidrule(lr){10-11} \cmidrule(lr){12-13}
Paper & Covariate & Unit & Min\_sim & Min\_true & Q1\_sim & Q1\_true & Median\_sim & Median\_true & Q3\_sim & Q3\_true & Max\_sim & Max\_true \\ 
\midrule\addlinespace[2.5pt]
Rambaud\_2020 & ICU & Unitless & 1.00 & 1.00 & 1.00 & 1.00 & 1.00 & 1.00 & 1.00 & 1.00 & 1.00 & 1.00 \\ 
Rambaud\_2020 & BURN & Unitless & 0.00 & 0.00 & 0.00 & 0.00 & 0.00 & 0.00 & 0.00 & 0.00 & 0.00 & 0.00 \\ 
Rambaud\_2020 & OBESE & Unitless & 0.00 & 0.00 & 0.00 & 0.00 & 0.00 & 0.00 & 0.00 & 0.00 & 0.00 & 0.00 \\ 
Rambaud\_2020 & CREAT & micromol/L & 2.08 $\times$ 10\textsuperscript{1} & NA & 6.79 $\times$ 10\textsuperscript{1} & 7.30 $\times$ 10\textsuperscript{1} & 8.67 $\times$ 10\textsuperscript{1} & 8.70 $\times$ 10\textsuperscript{1} & 1.12 $\times$ 10\textsuperscript{2} & 1.15 $\times$ 10\textsuperscript{2} & 3.08 $\times$ 10\textsuperscript{2} & NA \\ 
Rambaud\_2020 & WT & kg & 4.13 $\times$ 10\textsuperscript{1} & NA & 6.75 $\times$ 10\textsuperscript{1} & 6.90 $\times$ 10\textsuperscript{1} & 7.65 $\times$ 10\textsuperscript{1} & 7.60 $\times$ 10\textsuperscript{1} & 8.64 $\times$ 10\textsuperscript{1} & 8.70 $\times$ 10\textsuperscript{1} & 1.21 $\times$ 10\textsuperscript{2} & NA \\ 
Rambaud\_2020 & AGE & year & 1.97 $\times$ 10\textsuperscript{1} & NA & 6.04 $\times$ 10\textsuperscript{1} & 6.20 $\times$ 10\textsuperscript{1} & 7.08 $\times$ 10\textsuperscript{1} & 7.20 $\times$ 10\textsuperscript{1} & 8.20 $\times$ 10\textsuperscript{1} & 8.05 $\times$ 10\textsuperscript{1} & 1.12 $\times$ 10\textsuperscript{2} & NA \\ 
Rambaud\_2020 & HT & cm & 1.50 $\times$ 10\textsuperscript{2} & NA & 1.64 $\times$ 10\textsuperscript{2} & 1.66 $\times$ 10\textsuperscript{2} & 1.69 $\times$ 10\textsuperscript{2} & 1.70 $\times$ 10\textsuperscript{2} & 1.75 $\times$ 10\textsuperscript{2} & 1.75 $\times$ 10\textsuperscript{2} & 1.96 $\times$ 10\textsuperscript{2} & NA \\ 
\bottomrule
\end{tabular*}
\end{table}

\begin{Shaded}
\begin{Highlighting}[]
\FunctionTok{plot\_sim\_cov\_wrapper}\NormalTok{(cov\_data\_rambaud,}\StringTok{"CREAT"}\NormalTok{,sim\_cov\_rambaud)}
\end{Highlighting}
\end{Shaded}

\pandocbounded{\includegraphics[keepaspectratio]{Simulations/covariate_simulation_files/figure-pdf/plot-cov-rambaud-1.pdf}}

\begin{Shaded}
\begin{Highlighting}[]
\FunctionTok{plot\_sim\_cov\_wrapper}\NormalTok{(cov\_data\_rambaud,}\StringTok{"WT"}\NormalTok{,sim\_cov\_rambaud)}
\end{Highlighting}
\end{Shaded}

\pandocbounded{\includegraphics[keepaspectratio]{Simulations/covariate_simulation_files/figure-pdf/plot-cov-rambaud-2.pdf}}

\begin{Shaded}
\begin{Highlighting}[]
\FunctionTok{plot\_sim\_cov\_wrapper}\NormalTok{(cov\_data\_rambaud,}\StringTok{"AGE"}\NormalTok{,sim\_cov\_rambaud)}
\end{Highlighting}
\end{Shaded}

\pandocbounded{\includegraphics[keepaspectratio]{Simulations/covariate_simulation_files/figure-pdf/plot-cov-rambaud-3.pdf}}

\begin{Shaded}
\begin{Highlighting}[]
\FunctionTok{plot\_sim\_cov\_wrapper}\NormalTok{(cov\_data\_rambaud,}\StringTok{"HT"}\NormalTok{,sim\_cov\_rambaud)}
\end{Highlighting}
\end{Shaded}

\pandocbounded{\includegraphics[keepaspectratio]{Simulations/covariate_simulation_files/figure-pdf/plot-cov-rambaud-4.pdf}}

CREAT is converted from micromol/L to mg/dL.

\begin{Shaded}
\begin{Highlighting}[]
\NormalTok{sim\_cov\_rambaud }\OtherTok{\textless{}{-}}\NormalTok{ sim\_cov\_rambaud }\SpecialCharTok{\%\textgreater{}\%}
  \FunctionTok{mutate}\NormalTok{(}
    \AttributeTok{BSA =}\NormalTok{ (WT}\SpecialCharTok{\^{}}\FloatTok{0.425} \SpecialCharTok{*}\NormalTok{ (HT}\SpecialCharTok{\^{}}\FloatTok{0.725}\NormalTok{) }\SpecialCharTok{*} \FloatTok{0.007184}\NormalTok{),}
    \AttributeTok{CREAT =}\NormalTok{ CREAT}\SpecialCharTok{/}\FloatTok{88.4}\NormalTok{,}
    \AttributeTok{BMI =}\NormalTok{ WT }\SpecialCharTok{/}\NormalTok{ (HT}\SpecialCharTok{/}\DecValTok{100}\NormalTok{)}\SpecialCharTok{\^{}}\DecValTok{2}\NormalTok{,}
    \AttributeTok{OBESE =} \FunctionTok{ifelse}\NormalTok{(BMI }\SpecialCharTok{\textgreater{}} \DecValTok{30}\NormalTok{, }\DecValTok{1}\NormalTok{, }\DecValTok{0}\NormalTok{)}
\NormalTok{  ) }\SpecialCharTok{\%\textgreater{}\%}
\NormalTok{  dplyr}\SpecialCharTok{::}\FunctionTok{select}\NormalTok{(Paper, ID\_within\_paper, ICU, BURN, OBESE, CREAT, WT, BSA, BMI, AGE, SEX, HT)}

\CommentTok{\# Calculate the simulated correlation matrices separately for the two sexes to compare with the original}
\NormalTok{sim\_cov\_rambaud\_M }\OtherTok{\textless{}{-}}\NormalTok{ sim\_cov\_rambaud }\SpecialCharTok{\%\textgreater{}\%} \FunctionTok{filter}\NormalTok{(SEX }\SpecialCharTok{==} \DecValTok{0}\NormalTok{)}
\NormalTok{sim\_cov\_rambaud\_F }\OtherTok{\textless{}{-}}\NormalTok{ sim\_cov\_rambaud }\SpecialCharTok{\%\textgreater{}\%} \FunctionTok{filter}\NormalTok{(SEX }\SpecialCharTok{==} \DecValTok{1}\NormalTok{)}

\NormalTok{cor\_sim\_M }\OtherTok{\textless{}{-}} \FunctionTok{cor}\NormalTok{(sim\_cov\_rambaud\_M }\SpecialCharTok{\%\textgreater{}\%}\NormalTok{ dplyr}\SpecialCharTok{::}\FunctionTok{select}\NormalTok{(WT, CREAT, AGE, HT, BMI), }\AttributeTok{use =} \StringTok{"complete.obs"}\NormalTok{)}
\NormalTok{cor\_sim\_F }\OtherTok{\textless{}{-}} \FunctionTok{cor}\NormalTok{(sim\_cov\_rambaud\_F }\SpecialCharTok{\%\textgreater{}\%}\NormalTok{ dplyr}\SpecialCharTok{::}\FunctionTok{select}\NormalTok{(WT, CREAT, AGE, HT, BMI), }\AttributeTok{use =} \StringTok{"complete.obs"}\NormalTok{)}

\FunctionTok{ggcorrplot}\NormalTok{(cor\_matrix\_Rambaud\_F, }\AttributeTok{lab =} \ConstantTok{TRUE}\NormalTok{, }\AttributeTok{title =} \StringTok{"Rambaud original (females)"}\NormalTok{)}
\end{Highlighting}
\end{Shaded}

\pandocbounded{\includegraphics[keepaspectratio]{Simulations/covariate_simulation_files/figure-pdf/reconvert_creat_rambaud-1.pdf}}

\begin{Shaded}
\begin{Highlighting}[]
\NormalTok{Rambaud\_F }\OtherTok{\textless{}{-}} \FunctionTok{ggcorrplot}\NormalTok{(cor\_sim\_F, }\AttributeTok{lab =} \ConstantTok{TRUE}\NormalTok{, }\AttributeTok{title =} \StringTok{"Rambaud simulated (females)"}\NormalTok{)}
\NormalTok{Rambaud\_F}
\end{Highlighting}
\end{Shaded}

\pandocbounded{\includegraphics[keepaspectratio]{Simulations/covariate_simulation_files/figure-pdf/reconvert_creat_rambaud-2.pdf}}

\begin{Shaded}
\begin{Highlighting}[]
\FunctionTok{ggcorrplot}\NormalTok{(cor\_matrix\_Rambaud\_M, }\AttributeTok{lab =} \ConstantTok{TRUE}\NormalTok{, }\AttributeTok{title =} \StringTok{"Rambaud original (males)"}\NormalTok{)}
\end{Highlighting}
\end{Shaded}

\pandocbounded{\includegraphics[keepaspectratio]{Simulations/covariate_simulation_files/figure-pdf/reconvert_creat_rambaud-3.pdf}}

\begin{Shaded}
\begin{Highlighting}[]
\NormalTok{Rambaud\_M }\OtherTok{\textless{}{-}} \FunctionTok{ggcorrplot}\NormalTok{(cor\_sim\_M, }\AttributeTok{lab =} \ConstantTok{TRUE}\NormalTok{, }\AttributeTok{title =} \StringTok{"Rambaud simulated (males)"}\NormalTok{)}
\NormalTok{Rambaud\_M}
\end{Highlighting}
\end{Shaded}

\pandocbounded{\includegraphics[keepaspectratio]{Simulations/covariate_simulation_files/figure-pdf/reconvert_creat_rambaud-4.pdf}}

\section{Final covariate table}\label{final-covariate-table}

Make the final covariate dataset with 2500 sets of covariates in a way
that in each 100 sets, we have 25-25 from each covariate's cohort, as
later, each dosing regimen will be simulated for 100 sets of covariates,
and it is important that the 4 cohorts are equally represented.

Also, for the concentration simulation part, we have to know before
applying the models, if a patient needs a different regiment due to his
renal failure or not. That is why an additional IR column is added where
1 indicates a renal failure patient who needs a dose adjustement. For
Carlier, the MDRD equation is used, and for Fournier, the Cockroft and
Gault. For Mellon and Rambaud, all patients have 0 in the IR column, as
it is not explicited in the articles that a dose adjustement was applied
to renal failure patients.

\begin{Shaded}
\begin{Highlighting}[]
\CommentTok{\# Patchwork simulated correlation plots}
\NormalTok{corr\_simulated }\OtherTok{\textless{}{-}}\NormalTok{ (Carlier\_F }\SpecialCharTok{+}\NormalTok{ Fournier\_F }\SpecialCharTok{+}\NormalTok{ Mellon\_F }\SpecialCharTok{+}\NormalTok{ Rambaud\_F }\SpecialCharTok{+}
\NormalTok{  Carlier\_M }\SpecialCharTok{+}\NormalTok{ Fournier\_M }\SpecialCharTok{+}\NormalTok{ Mellon\_M }\SpecialCharTok{+}\NormalTok{ Rambaud\_M) }\SpecialCharTok{+} 
  \FunctionTok{plot\_layout}\NormalTok{(}\AttributeTok{ncol =} \DecValTok{4}\NormalTok{, }\AttributeTok{nrow =} \DecValTok{2}\NormalTok{, }\AttributeTok{byrow =} \ConstantTok{TRUE}\NormalTok{)}
\FunctionTok{ggsave}\NormalTok{(}\AttributeTok{filename =} \FunctionTok{here}\NormalTok{(}\StringTok{"a\_priori/For\_publication/Figures/S1.jpg"}\NormalTok{),}
       \AttributeTok{plot =}\NormalTok{ corr\_simulated,}
       \AttributeTok{width =} \DecValTok{14}\NormalTok{, }\AttributeTok{height =} \DecValTok{8}\NormalTok{, }\AttributeTok{dpi =} \DecValTok{300}\NormalTok{)}

\CommentTok{\# Row{-}bind the datasets to have 25 subjects from each in a sample of 100}
\NormalTok{datasets }\OtherTok{\textless{}{-}} \FunctionTok{list}\NormalTok{(sim\_cov\_carlier, sim\_cov\_fournier, sim\_cov\_mellon, sim\_cov\_rambaud)}

\NormalTok{interleave\_rows }\OtherTok{\textless{}{-}} \ControlFlowTok{function}\NormalTok{(datasets, chunk\_size) \{}
\NormalTok{  n\_chunks }\OtherTok{\textless{}{-}} \FunctionTok{nrow}\NormalTok{(datasets[[}\DecValTok{1}\NormalTok{]]) }\SpecialCharTok{/}\NormalTok{ chunk\_size}
\NormalTok{  interleaved }\OtherTok{\textless{}{-}} \FunctionTok{do.call}\NormalTok{(rbind, }\FunctionTok{lapply}\NormalTok{(}\DecValTok{1}\SpecialCharTok{:}\NormalTok{n\_chunks, }\ControlFlowTok{function}\NormalTok{(i) \{}
    \FunctionTok{do.call}\NormalTok{(rbind, }\FunctionTok{lapply}\NormalTok{(datasets, }\ControlFlowTok{function}\NormalTok{(dataset) \{}
\NormalTok{      dataset[((i }\SpecialCharTok{{-}} \DecValTok{1}\NormalTok{) }\SpecialCharTok{*}\NormalTok{ chunk\_size }\SpecialCharTok{+} \DecValTok{1}\NormalTok{)}\SpecialCharTok{:}\NormalTok{(i }\SpecialCharTok{*}\NormalTok{ chunk\_size), ]}
\NormalTok{    \}))}
\NormalTok{  \}))}
  \FunctionTok{return}\NormalTok{(interleaved)}
\NormalTok{\}}

\NormalTok{COV }\OtherTok{\textless{}{-}} \FunctionTok{interleave\_rows}\NormalTok{(datasets, }\AttributeTok{chunk\_size =} \DecValTok{25}\NormalTok{)}

\CommentTok{\# Convert Paper column to MODEL\_COHORT column which indicates the model that is used to simulated the covariates}
\NormalTok{COV }\OtherTok{\textless{}{-}}\NormalTok{ COV }\SpecialCharTok{\%\textgreater{}\%}
  \FunctionTok{mutate}\NormalTok{(}\AttributeTok{MODEL\_COHORT =} \FunctionTok{case\_when}\NormalTok{(}
\NormalTok{    Paper }\SpecialCharTok{==} \StringTok{"Carlier\_2013"} \SpecialCharTok{\textasciitilde{}} \StringTok{"CARLIER"}\NormalTok{,}
\NormalTok{    Paper }\SpecialCharTok{==} \StringTok{"Fournier\_2018"} \SpecialCharTok{\textasciitilde{}} \StringTok{"FOURNIER"}\NormalTok{,}
\NormalTok{    Paper }\SpecialCharTok{==} \StringTok{"Mellon\_2020"} \SpecialCharTok{\textasciitilde{}} \StringTok{"MELLON"}\NormalTok{,}
\NormalTok{    Paper }\SpecialCharTok{==} \StringTok{"Rambaud\_2020"} \SpecialCharTok{\textasciitilde{}} \StringTok{"RAMBAUD"}\NormalTok{)}
\NormalTok{  )}

\CommentTok{\# Adding ID}
\NormalTok{COV}\SpecialCharTok{$}\NormalTok{ID }\OtherTok{\textless{}{-}} \FunctionTok{seq}\NormalTok{(}\DecValTok{1}\SpecialCharTok{:}\DecValTok{2500}\NormalTok{)}

\CommentTok{\# CRCL calculation functions}
\NormalTok{calculate\_MDRD }\OtherTok{\textless{}{-}} \ControlFlowTok{function}\NormalTok{(AGE, CREAT, SEX) \{}
  \CommentTok{\# Mutliply by 0.742 for females}
\NormalTok{    sex\_factor }\OtherTok{\textless{}{-}} \FunctionTok{ifelse}\NormalTok{(SEX }\SpecialCharTok{==} \DecValTok{0}\NormalTok{, }\DecValTok{1}\NormalTok{, }\FloatTok{0.742}\NormalTok{)}
\NormalTok{    eGFR }\OtherTok{\textless{}{-}} \DecValTok{175} \SpecialCharTok{*}\NormalTok{ (CREAT}\SpecialCharTok{\^{}}\NormalTok{(}\SpecialCharTok{{-}}\FloatTok{1.154}\NormalTok{)) }\SpecialCharTok{*}\NormalTok{ AGE}\SpecialCharTok{\^{}}\NormalTok{(}\SpecialCharTok{{-}}\FloatTok{0.203}\NormalTok{) }\SpecialCharTok{*}\NormalTok{ sex\_factor}
    \FunctionTok{return}\NormalTok{(eGFR)}
\NormalTok{\}}

\NormalTok{calculate\_CG }\OtherTok{\textless{}{-}} \ControlFlowTok{function}\NormalTok{(AGE, CREAT, SEX, WT) \{}
\NormalTok{  sex\_factor }\OtherTok{\textless{}{-}} \FunctionTok{ifelse}\NormalTok{(SEX }\SpecialCharTok{==} \DecValTok{0}\NormalTok{, }\DecValTok{1}\NormalTok{, }\FloatTok{0.85}\NormalTok{)}
\NormalTok{  eGFR }\OtherTok{\textless{}{-}}\NormalTok{ ((}\DecValTok{140} \SpecialCharTok{{-}}\NormalTok{ AGE) }\SpecialCharTok{*}\NormalTok{ WT }\SpecialCharTok{*}\NormalTok{ sex\_factor) }\SpecialCharTok{/}\NormalTok{ (CREAT }\SpecialCharTok{*} \DecValTok{72}\NormalTok{)}
  \FunctionTok{return}\NormalTok{(eGFR)}
\NormalTok{\}}

\CommentTok{\# Add 1 for dose adjustement for CRCL \textless{} 30 mL/min for Carlier and Fournier}
\NormalTok{COV }\OtherTok{\textless{}{-}}\NormalTok{ COV }\SpecialCharTok{\%\textgreater{}\%}
  \FunctionTok{rowwise}\NormalTok{() }\SpecialCharTok{\%\textgreater{}\%}
  \FunctionTok{mutate}\NormalTok{(}
    \AttributeTok{CRCL =} \FunctionTok{case\_when}\NormalTok{(}
\NormalTok{      MODEL\_COHORT }\SpecialCharTok{==} \StringTok{"CARLIER"} \SpecialCharTok{\textasciitilde{}} \FunctionTok{calculate\_MDRD}\NormalTok{(AGE, CREAT, SEX) }\SpecialCharTok{*}\NormalTok{ (BSA }\SpecialCharTok{/} \FloatTok{1.73}\NormalTok{),}
\NormalTok{      MODEL\_COHORT }\SpecialCharTok{==} \StringTok{"FOURNIER"} \SpecialCharTok{\textasciitilde{}} \FunctionTok{calculate\_CG}\NormalTok{(AGE, CREAT, SEX, WT),}
      \ConstantTok{TRUE} \SpecialCharTok{\textasciitilde{}} \ConstantTok{NA\_real\_}
\NormalTok{    ),}
    \AttributeTok{IR =} \FunctionTok{case\_when}\NormalTok{(}
\NormalTok{      MODEL\_COHORT }\SpecialCharTok{==} \StringTok{"CARLIER"} \SpecialCharTok{\&}\NormalTok{ CRCL }\SpecialCharTok{\textless{}} \DecValTok{30} \SpecialCharTok{\textasciitilde{}} \DecValTok{1}\NormalTok{,}
\NormalTok{      MODEL\_COHORT }\SpecialCharTok{==} \StringTok{"FOURNIER"} \SpecialCharTok{\&}\NormalTok{ CRCL }\SpecialCharTok{\textless{}} \DecValTok{30} \SpecialCharTok{\textasciitilde{}} \DecValTok{1}\NormalTok{,}
\NormalTok{      MODEL\_COHORT }\SpecialCharTok{\%in\%} \FunctionTok{c}\NormalTok{(}\StringTok{"MELLON"}\NormalTok{, }\StringTok{"RAMBAUD"}\NormalTok{) }\SpecialCharTok{\textasciitilde{}} \DecValTok{0}\NormalTok{,}
      \ConstantTok{TRUE} \SpecialCharTok{\textasciitilde{}} \DecValTok{0}
\NormalTok{    )}
\NormalTok{  ) }\SpecialCharTok{\%\textgreater{}\%}
  \FunctionTok{ungroup}\NormalTok{()}

\NormalTok{COV }\OtherTok{\textless{}{-}}\NormalTok{ COV }\SpecialCharTok{\%\textgreater{}\%}
\NormalTok{  dplyr}\SpecialCharTok{::}\FunctionTok{select}\NormalTok{(ID, MODEL\_COHORT, WT, CREAT, BURN, ICU, OBESE, BSA, AGE, SEX, HT, IR)}

\CommentTok{\# Export to csv}
\FunctionTok{write.csv}\NormalTok{(COV, }\FunctionTok{here}\NormalTok{(}\StringTok{"a\_priori/For\_publication/Data/COV.csv"}\NormalTok{), }\AttributeTok{quote =}\NormalTok{ F, }\AttributeTok{row.names =}\NormalTok{ F)}

\CommentTok{\# Check correlation (Pearson)}
\NormalTok{cor\_sim }\OtherTok{\textless{}{-}} \FunctionTok{cor}\NormalTok{(COV }\SpecialCharTok{\%\textgreater{}\%}\NormalTok{ dplyr}\SpecialCharTok{::}\FunctionTok{select}\NormalTok{(WT, CREAT, BURN, ICU, OBESE, AGE, SEX, HT), }\AttributeTok{use =} \StringTok{"complete.obs"}\NormalTok{)}
\FunctionTok{ggcorrplot}\NormalTok{(cor\_sim, }\AttributeTok{lab =} \ConstantTok{TRUE}\NormalTok{)}
\end{Highlighting}
\end{Shaded}

\pandocbounded{\includegraphics[keepaspectratio]{Simulations/covariate_simulation_files/figure-pdf/final_table-1.pdf}}

\chapter{Simulation of concentrations and secondary PK
parameters}\label{simulation-of-concentrations-and-secondary-pk-parameters}

\begin{Shaded}
\begin{Highlighting}[]
\FunctionTok{library}\NormalTok{(mrgsolve)}
\FunctionTok{library}\NormalTok{(here)}
\FunctionTok{library}\NormalTok{(tidyverse)}
\FunctionTok{library}\NormalTok{(mapbayr)}
\FunctionTok{library}\NormalTok{(MESS)}
\FunctionTok{library}\NormalTok{(ggplot2)}
\FunctionTok{library}\NormalTok{(dplyr)}
\FunctionTok{library}\NormalTok{(patchwork) }\CommentTok{\# To put different plots on the same grid}
\FunctionTok{library}\NormalTok{(tableone)}
\FunctionTok{library}\NormalTok{(knitr)}
\NormalTok{conflicted}\SpecialCharTok{::}\FunctionTok{conflicts\_prefer}\NormalTok{(dplyr}\SpecialCharTok{::}\NormalTok{filter)}
\NormalTok{conflicted}\SpecialCharTok{::}\FunctionTok{conflicts\_prefer}\NormalTok{(dplyr}\SpecialCharTok{::}\NormalTok{select)}
\end{Highlighting}
\end{Shaded}

The objective is to simulate concentrations and secondary PK parameters
(Cmax, Cmin \& AUC) using 4 validated PopPK models for amoxicillin. The
4 papers are:

\begin{itemize}
\item
  Carlier et al. (2013)
\item
  Fournier et al. (2018)
\item
  Mellon et al. (2020)
\item
  Rambaud et al. (2020)
\end{itemize}

The original Rambaud model was non-parametric which was approximated
using a parametric model (details in ``Rambaud\_validation.html'').

The only route of administration is intraveinous infusion.

Three types of infusion are simulated:

\begin{itemize}
\item
  intermittent (duration = 0.5 h),
\item
  extended (1-2 hours),
\item
  continuous .
\end{itemize}

There are 8 different dosing schemes. For each model cohort the dosing
scheme that was in the original article is used for simulation. For
patients with renal failure (IR = CRCL \textless{} 30 mL/min), a
different dosing scheme is applied if it is explicited in the articles
(\emph{i. e.} Carlier, Fournier). For Mellon, there was a single
administration in the original model. To have a dosing scheme compatible
with the clinical context in which we want to apply our methods,
administration frequency is set to every 6 h for intermittent
administrations. ID means a set of covariates, not a subject (to make
data manipulation easier later).

144 observations are simulated for each subject to have a sufficient
number of observations after the first dose, as well as during a
steady-state interdose interval. Observations are simulated for the
period between the first and second doses and for the same duration for
the first interdose interval that is entirely at steady-state. The
number of observations has been determined to have at least three
observations during the first interdose interval for all dosing schemes.
For example, in the case of an intermittent infusion of 0.5 g q6h, there
is an observation in each 5 minutes between 0 and 6 h \& between 24 h
and 30 h (if steady-state is reached at 22 h). This means, that not only
the dosing grid changes for each dosing scheme, but the observation grid
as well.

\begin{Shaded}
\begin{Highlighting}[]
\NormalTok{here}\SpecialCharTok{::}\FunctionTok{i\_am}\NormalTok{(}\StringTok{"a\_priori/For\_publication/Simulations/concentrations\_simulation.qmd"}\NormalTok{)}
\CommentTok{\# Import or generate the file with the covariates (COV)}
\NormalTok{COV }\OtherTok{\textless{}{-}} \FunctionTok{read.csv}\NormalTok{(}\FunctionTok{here}\NormalTok{(}\StringTok{"a\_priori/For\_publication/Data/COV.csv"}\NormalTok{), }\AttributeTok{quote =} \StringTok{""}\NormalTok{)}
\FunctionTok{set.seed}\NormalTok{(}\DecValTok{1991}\NormalTok{)}
\end{Highlighting}
\end{Shaded}

First, the dataset needs to be created with all the information that
mrgsolve needs for the simulation. In the function create\_dosing\_data,
the input data is created for all the lines that contain an administered
dose.

\begin{Shaded}
\begin{Highlighting}[]
\CommentTok{\# Function to create dosing grid (lines where there is an administered dose)}
\NormalTok{create\_dosing\_data }\OtherTok{\textless{}{-}} \ControlFlowTok{function}\NormalTok{(id\_range, amt, ii, rate, cmt, time\_seq) \{}
\NormalTok{  COV1 }\OtherTok{\textless{}{-}}\NormalTok{ COV }\SpecialCharTok{\%\textgreater{}\%}\NormalTok{ dplyr}\SpecialCharTok{::}\FunctionTok{filter}\NormalTok{(ID }\SpecialCharTok{\%in\%}\NormalTok{ id\_range)}
  
\NormalTok{  ID }\OtherTok{\textless{}{-}}\NormalTok{ id\_range            }\CommentTok{\# ID range for the regimen}
  
  \CommentTok{\# Create base dosing data}
\NormalTok{  dosing\_data }\OtherTok{\textless{}{-}} \FunctionTok{data.frame}\NormalTok{(}
    \AttributeTok{ID =}\NormalTok{ ID,}
    \AttributeTok{amt =} \FunctionTok{rep}\NormalTok{(amt, }\FunctionTok{length}\NormalTok{(ID)),   }\CommentTok{\# Dose amount in mg}
    \AttributeTok{ii =} \FunctionTok{rep}\NormalTok{(ii, }\FunctionTok{length}\NormalTok{(ID)),     }\CommentTok{\# Interdose interval in h}
    \AttributeTok{rate =} \FunctionTok{rep}\NormalTok{(rate, }\FunctionTok{length}\NormalTok{(ID)), }\CommentTok{\# Infusion rate}
    \AttributeTok{evid =} \FunctionTok{rep}\NormalTok{(}\DecValTok{1}\NormalTok{, }\FunctionTok{length}\NormalTok{(ID)),    }\CommentTok{\# EVID}
    \AttributeTok{cmt =}\NormalTok{ cmt                     }\CommentTok{\# Compartment}
\NormalTok{  )}
  
  \CommentTok{\# Add covariates from COV.csv}
\NormalTok{  dosing\_data }\OtherTok{\textless{}{-}} \FunctionTok{cbind}\NormalTok{(dosing\_data, COV1[, }\FunctionTok{setdiff}\NormalTok{(}\FunctionTok{names}\NormalTok{(COV1), }\StringTok{"ID"}\NormalTok{)])}
  
  \CommentTok{\# Repeat dosing data for all time points}
\NormalTok{  dosing\_repeated }\OtherTok{\textless{}{-}}\NormalTok{ dosing\_data[}\FunctionTok{rep}\NormalTok{(}\DecValTok{1}\SpecialCharTok{:}\FunctionTok{nrow}\NormalTok{(dosing\_data), }\AttributeTok{each =} \FunctionTok{length}\NormalTok{(time\_seq)), ]}
\NormalTok{  time }\OtherTok{\textless{}{-}} \FunctionTok{rep}\NormalTok{(time\_seq, }\AttributeTok{times =} \FunctionTok{length}\NormalTok{(ID))}
  
  \CommentTok{\# Combine with time}
\NormalTok{  final\_dosing\_data }\OtherTok{\textless{}{-}} \FunctionTok{cbind}\NormalTok{(dosing\_repeated, time)}
  \FunctionTok{return}\NormalTok{(final\_dosing\_data)}
\NormalTok{\}}
\end{Highlighting}
\end{Shaded}

Then, the information is created for all the lines, that do not contain
a dose, but an observation. The latest possible time is 120 h, as
amoxicillin has a short half life and all subjects have had a complete
steady-state dosing interval by then.

\begin{Shaded}
\begin{Highlighting}[]
\CommentTok{\# Function to create observation grid (lines with concentration measurement)}
\NormalTok{create\_observation\_data }\OtherTok{\textless{}{-}} \ControlFlowTok{function}\NormalTok{(id\_range, obs\_length) \{}
\NormalTok{  COV1 }\OtherTok{\textless{}{-}}\NormalTok{ COV }\SpecialCharTok{\%\textgreater{}\%}\NormalTok{ dplyr}\SpecialCharTok{::}\FunctionTok{filter}\NormalTok{(ID }\SpecialCharTok{\%in\%}\NormalTok{ id\_range)}
 \CommentTok{\# Extract covariates for the specified range}
\NormalTok{  ID }\OtherTok{\textless{}{-}}\NormalTok{ id\_range            }\CommentTok{\# ID range for the regimen}
  
  \CommentTok{\# Create base observation data (as these are lines with concentrations, AMT, II, RATE and EVID are always 0)}
\NormalTok{  obs\_data }\OtherTok{\textless{}{-}} \FunctionTok{data.frame}\NormalTok{(}
    \AttributeTok{ID =}\NormalTok{ ID,                     }\CommentTok{\# ID meaning a set of covariates}
    \AttributeTok{amt =} \FunctionTok{rep}\NormalTok{(}\DecValTok{0}\NormalTok{, }\FunctionTok{length}\NormalTok{(ID)),    }\CommentTok{\# Dose amount in mg}
    \AttributeTok{ii =} \FunctionTok{rep}\NormalTok{(}\DecValTok{0}\NormalTok{, }\FunctionTok{length}\NormalTok{(ID)),     }\CommentTok{\# Interdose interval in h}
    \AttributeTok{rate =} \FunctionTok{rep}\NormalTok{(}\DecValTok{0}\NormalTok{, }\FunctionTok{length}\NormalTok{(ID)),   }\CommentTok{\# Infusion rate}
    \AttributeTok{evid =} \FunctionTok{rep}\NormalTok{(}\DecValTok{0}\NormalTok{, }\FunctionTok{length}\NormalTok{(ID)),   }\CommentTok{\# EVID}
    \AttributeTok{cmt =} \StringTok{"CENTRAL"}              \CommentTok{\# Central compartment}
\NormalTok{  )}
  
  \CommentTok{\# Add covariates}
\NormalTok{  obs\_data }\OtherTok{\textless{}{-}} \FunctionTok{cbind}\NormalTok{(obs\_data, COV1[, }\FunctionTok{setdiff}\NormalTok{(}\FunctionTok{names}\NormalTok{(COV1), }\StringTok{"ID"}\NormalTok{)])}
  
  \CommentTok{\# Repeat observation data for all observation times}
\NormalTok{  obs\_repeated }\OtherTok{\textless{}{-}}\NormalTok{ obs\_data[}\FunctionTok{rep}\NormalTok{(}\DecValTok{1}\SpecialCharTok{:}\FunctionTok{nrow}\NormalTok{(obs\_data), }\AttributeTok{each =}\NormalTok{ obs\_length), ]}
\NormalTok{  time }\OtherTok{\textless{}{-}} \FunctionTok{rep}\NormalTok{(}\FunctionTok{seq}\NormalTok{(}\AttributeTok{from =} \DecValTok{0}\NormalTok{, }\AttributeTok{to =} \DecValTok{120}\NormalTok{, }\AttributeTok{length.out =}\NormalTok{ obs\_length), }\AttributeTok{times =} \FunctionTok{length}\NormalTok{(ID))   }\CommentTok{\# Latest possible time is 120 h as amoxicillin has a short half life}
  
  \CommentTok{\# Combine with time}
\NormalTok{  final\_obs\_data }\OtherTok{\textless{}{-}} \FunctionTok{cbind}\NormalTok{(obs\_repeated, time)}
  \FunctionTok{return}\NormalTok{(final\_obs\_data)}
\NormalTok{\}}
\end{Highlighting}
\end{Shaded}

To remove outliers, the non-physiological PK parameter values are
resimulated (at most 100 times per value) using the \emph{simeta}
function of mrgsolve. A central volume of distribution smaller than 3 L
(the volume of plasma) and a clearance higher than 180 L/h (normal blood
flow) is considered non-physiological. (However, the maximum simulated
clearance is around 100 L/h, so its resimulation is unnecessary.)\\

Simulations are done using the mrgsolve models in .cpp format. To find
the interdose interval which is entirely in steady-state, first, the
time to reach steady state (TSS) has to be calculated. \[
K_{10} = \frac{CL}{V_c}
\]

\[
K_{12} = \frac{Q}{V_c}
\]

\[
K_{21} = \frac{Q}{V_p}
\]

\[
L_2 = \frac{ \left( K_{10} + K_{12} + K_{21} \right) - 
\sqrt{ \left( K_{10} + K_{12} + K_{21} \right)^2 - 4 K_{10} K_{21} } }{2}
\] \[
t_{1/2} = \frac{ln(2)}{L_2}
\] We can consider that steady state is reached when the concentration
is 90 \% of the steady state concentration, which corresponds to a time
of 3.3 times the half life ( as \(1 - e^{(-k_e \cdot t)} = 0.9\) ).

First, concentrations are simulated for all data points between 0 and
120 h, then the results are filtered to choose the first interdose
interval and the first interdose interval entirely in steady-state.

\begin{Shaded}
\begin{Highlighting}[]
\CommentTok{\# Function to simulate based on mrgsolve models}
\NormalTok{simulate\_mrgsim }\OtherTok{\textless{}{-}} \ControlFlowTok{function}\NormalTok{(sim\_final, model, interval) \{}
\NormalTok{  sim\_results }\OtherTok{\textless{}{-}} \FunctionTok{mrgsim}\NormalTok{(model, sim\_final) }\SpecialCharTok{\%\textgreater{}\%} \FunctionTok{as.data.frame}\NormalTok{()}
  
  \FunctionTok{return}\NormalTok{(sim\_results)}
\NormalTok{\}}
\end{Highlighting}
\end{Shaded}

Then, the filtered dataset, and the covariates will be put together, and
the observation and dosing grids will be simulated based on the
functions defined earlier. The dosing scheme is added as three columns
(DOSE, FREQuency, DURation). Each model control stream is used two times
for each simulation. Once, to simulate IPRED and Y (and
\(Cmax_{\text{ind}}\), \(Cmin_{\text{ind}}\) and \(AUC_{\text{ind}}\)
which are based on IPRED), then a second time, the omega matrix is set
to 0 to simulate PRED (and CMAX\_PRED and AUC\_PRED based on PRED). PRED
are typical concentrations which do not contain inter-individual
variability. IPRED has inter-individual variability incorporated, and Y
has both inter-and intra-individual variability.

\begin{Shaded}
\begin{Highlighting}[]
\CommentTok{\# Function to merge covariates and finalize}
\NormalTok{finalize\_results }\OtherTok{\textless{}{-}} \ControlFlowTok{function}\NormalTok{(sim\_results, id\_range, model, dose, freq, dur) \{}
\NormalTok{  COV1 }\OtherTok{\textless{}{-}}\NormalTok{ COV }\SpecialCharTok{\%\textgreater{}\%}\NormalTok{ dplyr}\SpecialCharTok{::}\FunctionTok{filter}\NormalTok{(ID }\SpecialCharTok{\%in\%}\NormalTok{ id\_range)}
  
  \CommentTok{\# Merge with simulation results}
\NormalTok{  result }\OtherTok{\textless{}{-}} \FunctionTok{merge}\NormalTok{(sim\_results, COV1, }\AttributeTok{by =} \StringTok{"ID"}\NormalTok{, }\AttributeTok{all.x =} \ConstantTok{TRUE}\NormalTok{)}
  
\NormalTok{  result }\OtherTok{\textless{}{-}}\NormalTok{ result }\SpecialCharTok{\%\textgreater{}\%}
    \FunctionTok{mutate}\NormalTok{(}\AttributeTok{MODEL =}\NormalTok{ model, }\AttributeTok{DOSE\_ADM =}\NormalTok{ dose, }\AttributeTok{FREQ =}\NormalTok{ freq, }\AttributeTok{DUR =}\NormalTok{ dur) }\SpecialCharTok{\%\textgreater{}\%}
    \FunctionTok{distinct}\NormalTok{(ID, }\AttributeTok{TIME =}\NormalTok{ time, }\AttributeTok{.keep\_all =} \ConstantTok{TRUE}\NormalTok{)}
  
  \FunctionTok{return}\NormalTok{(result)}
\NormalTok{\}}

\NormalTok{generate\_model\_regimen }\OtherTok{\textless{}{-}} \ControlFlowTok{function}\NormalTok{(model\_name, cpp\_file, regimens, sim\_dur,cohort\_name) \{}
\NormalTok{  model }\OtherTok{\textless{}{-}} \FunctionTok{mread}\NormalTok{(}\AttributeTok{model =}\NormalTok{ cpp\_file)}
\NormalTok{  model\_zero }\OtherTok{\textless{}{-}} \FunctionTok{zero\_re}\NormalTok{(model)}
\NormalTok{  COV }\OtherTok{\textless{}{-}} \FunctionTok{read.csv}\NormalTok{(}\FunctionTok{here}\NormalTok{(}\StringTok{"a\_priori/For\_publication/Data/COV.csv"}\NormalTok{), }\AttributeTok{quote =} \StringTok{""}\NormalTok{)}

\NormalTok{  regimen\_assignments }\OtherTok{\textless{}{-}} \FunctionTok{list}\NormalTok{()}

\NormalTok{    cohort\_cov }\OtherTok{\textless{}{-}}\NormalTok{ COV }\SpecialCharTok{\%\textgreater{}\%} \FunctionTok{filter}\NormalTok{(MODEL\_COHORT }\SpecialCharTok{==}\NormalTok{ cohort\_name)}
\NormalTok{    cohort\_regimens }\OtherTok{\textless{}{-}}\NormalTok{ regimens[}\FunctionTok{sapply}\NormalTok{(regimens, }\ControlFlowTok{function}\NormalTok{(x) x}\SpecialCharTok{$}\NormalTok{model\_cohort }\SpecialCharTok{==}\NormalTok{ cohort\_name)]}

    \ControlFlowTok{for}\NormalTok{ (ir\_value }\ControlFlowTok{in} \FunctionTok{unique}\NormalTok{(cohort\_cov}\SpecialCharTok{$}\NormalTok{IR)) \{}
\NormalTok{      eligible\_ids }\OtherTok{\textless{}{-}}\NormalTok{ cohort\_cov }\SpecialCharTok{\%\textgreater{}\%} \FunctionTok{filter}\NormalTok{(IR }\SpecialCharTok{==}\NormalTok{ ir\_value) }\SpecialCharTok{\%\textgreater{}\%} \FunctionTok{pull}\NormalTok{(ID)}

\NormalTok{      matched\_regimens }\OtherTok{\textless{}{-}}\NormalTok{ cohort\_regimens[}\FunctionTok{seq\_along}\NormalTok{(cohort\_regimens) }\SpecialCharTok{\%\%} \FunctionTok{length}\NormalTok{(}\FunctionTok{unique}\NormalTok{(cohort\_cov}\SpecialCharTok{$}\NormalTok{IR)) }\SpecialCharTok{==}\NormalTok{ ir\_value]}

      \CommentTok{\# Distribute eligible subjects evenly across relevant dosing regimens}
\NormalTok{      id\_splits }\OtherTok{\textless{}{-}} \FunctionTok{split}\NormalTok{(eligible\_ids, }\FunctionTok{rep}\NormalTok{(}\DecValTok{1}\SpecialCharTok{:}\FunctionTok{length}\NormalTok{(matched\_regimens), }\AttributeTok{length.out =} \FunctionTok{length}\NormalTok{(eligible\_ids)))}

      \ControlFlowTok{for}\NormalTok{ (i }\ControlFlowTok{in} \FunctionTok{seq\_along}\NormalTok{(matched\_regimens)) \{}
\NormalTok{        matched\_regimens[[i]]}\SpecialCharTok{$}\NormalTok{assigned\_ids }\OtherTok{\textless{}{-}}\NormalTok{ id\_splits[[i]]}
\NormalTok{        regimen\_assignments }\OtherTok{\textless{}{-}} \FunctionTok{append}\NormalTok{(regimen\_assignments, }\FunctionTok{list}\NormalTok{(matched\_regimens[[i]]))}
\NormalTok{      \}}
\NormalTok{    \}}

  \CommentTok{\# Simulate for the assigned regimens}
\NormalTok{  regimen\_results }\OtherTok{\textless{}{-}} \FunctionTok{list}\NormalTok{()}
  \ControlFlowTok{for}\NormalTok{ (regimen }\ControlFlowTok{in}\NormalTok{ regimen\_assignments) \{}
    \ControlFlowTok{if}\NormalTok{ (}\FunctionTok{length}\NormalTok{(regimen}\SpecialCharTok{$}\NormalTok{assigned\_ids) }\SpecialCharTok{==} \DecValTok{0}\NormalTok{) }\ControlFlowTok{next}

\NormalTok{    dose }\OtherTok{\textless{}{-}}\NormalTok{ regimen}\SpecialCharTok{$}\NormalTok{dose}
\NormalTok{    interval }\OtherTok{\textless{}{-}}\NormalTok{ regimen}\SpecialCharTok{$}\NormalTok{interval}
\NormalTok{    duration }\OtherTok{\textless{}{-}}\NormalTok{ regimen}\SpecialCharTok{$}\NormalTok{duration}
\NormalTok{    id\_range }\OtherTok{\textless{}{-}}\NormalTok{ regimen}\SpecialCharTok{$}\NormalTok{assigned\_ids}

\NormalTok{    dosing\_times }\OtherTok{\textless{}{-}} \FunctionTok{seq}\NormalTok{(}\DecValTok{0}\NormalTok{, sim\_dur }\SpecialCharTok{{-}}\NormalTok{ interval, }\AttributeTok{by =}\NormalTok{ interval)}
\NormalTok{    obs\_length }\OtherTok{\textless{}{-}}\NormalTok{ (sim\_dur }\SpecialCharTok{/}\NormalTok{ interval) }\SpecialCharTok{*}\NormalTok{ sim\_dur }\SpecialCharTok{+} \DecValTok{1}

\NormalTok{    dose\_data }\OtherTok{\textless{}{-}} \FunctionTok{create\_dosing\_data}\NormalTok{(}
      \AttributeTok{id\_range =}\NormalTok{ id\_range,}
      \AttributeTok{amt =}\NormalTok{ dose,}
      \AttributeTok{ii =}\NormalTok{ interval,}
      \AttributeTok{rate =}\NormalTok{ dose }\SpecialCharTok{/}\NormalTok{ duration,}
      \AttributeTok{cmt =} \StringTok{"CENTRAL"}\NormalTok{,}
      \AttributeTok{time\_seq =}\NormalTok{ dosing\_times}
\NormalTok{    )}

\NormalTok{    obs\_data }\OtherTok{\textless{}{-}} \FunctionTok{create\_observation\_data}\NormalTok{(}\AttributeTok{id\_range =}\NormalTok{ id\_range, }\AttributeTok{obs\_length =}\NormalTok{ obs\_length)}
\NormalTok{    sim\_data }\OtherTok{\textless{}{-}} \FunctionTok{rbind}\NormalTok{(dose\_data, obs\_data) }\SpecialCharTok{\%\textgreater{}\%} \FunctionTok{arrange}\NormalTok{(ID, time, }\FunctionTok{desc}\NormalTok{(evid))}

\NormalTok{processed\_data }\OtherTok{\textless{}{-}} \FunctionTok{simulate\_mrgsim}\NormalTok{(sim\_data, model, interval)}

\NormalTok{finalized\_data }\OtherTok{\textless{}{-}} \FunctionTok{finalize\_results}\NormalTok{(}
      \AttributeTok{sim\_results =}\NormalTok{ processed\_data,}
      \AttributeTok{id\_range =}\NormalTok{ id\_range,}
      \AttributeTok{model =}\NormalTok{ model\_name,}
      \AttributeTok{dose =}\NormalTok{ dose,}
      \AttributeTok{freq =}\NormalTok{ interval,}
      \AttributeTok{dur =}\NormalTok{ duration}
\NormalTok{    )}

\NormalTok{    merged\_data }\OtherTok{\textless{}{-}}\NormalTok{ finalized\_data}
    
\NormalTok{    regimen\_results }\OtherTok{\textless{}{-}} \FunctionTok{append}\NormalTok{(regimen\_results, }\FunctionTok{list}\NormalTok{(merged\_data))}
\NormalTok{  \}}

\NormalTok{  combined\_results }\OtherTok{\textless{}{-}} \FunctionTok{do.call}\NormalTok{(rbind, regimen\_results)}
  \FunctionTok{return}\NormalTok{(combined\_results)}
\NormalTok{\}}
\end{Highlighting}
\end{Shaded}

8 different dosing schemes are used depending on the model cohort.

\begin{Shaded}
\begin{Highlighting}[]
\CommentTok{\# Define regimen parameters for all models. For each regimen, we take a different batch of 100 sets of covariates (defined by starting ID)}
\NormalTok{regimens }\OtherTok{\textless{}{-}} \FunctionTok{list}\NormalTok{(}
  \FunctionTok{list}\NormalTok{(}\AttributeTok{model\_cohort =} \StringTok{"CARLIER"}\NormalTok{, }\AttributeTok{ir\_group =} \DecValTok{0}\NormalTok{, }\AttributeTok{dose =} \DecValTok{1000}\NormalTok{, }\AttributeTok{interval =} \DecValTok{6}\NormalTok{, }\AttributeTok{duration =} \FloatTok{0.5}\NormalTok{),}
  \FunctionTok{list}\NormalTok{(}\AttributeTok{model\_cohort =} \StringTok{"CARLIER"}\NormalTok{, }\AttributeTok{ir\_group =} \DecValTok{1}\NormalTok{, }\AttributeTok{dose =} \DecValTok{1000}\NormalTok{, }\AttributeTok{interval =} \DecValTok{8}\NormalTok{, }\AttributeTok{duration =} \FloatTok{0.5}\NormalTok{),}
  \FunctionTok{list}\NormalTok{(}\AttributeTok{model\_cohort =} \StringTok{"FOURNIER"}\NormalTok{, }\AttributeTok{ir\_group =} \DecValTok{0}\NormalTok{, }\AttributeTok{dose =} \DecValTok{1000}\NormalTok{, }\AttributeTok{interval =} \DecValTok{6}\NormalTok{, }\AttributeTok{duration =} \DecValTok{2}\NormalTok{),}
  \FunctionTok{list}\NormalTok{(}\AttributeTok{model\_cohort =} \StringTok{"FOURNIER"}\NormalTok{, }\AttributeTok{ir\_group =} \DecValTok{0}\NormalTok{, }\AttributeTok{dose =} \DecValTok{1500}\NormalTok{, }\AttributeTok{interval =} \DecValTok{6}\NormalTok{, }\AttributeTok{duration =} \DecValTok{1}\NormalTok{),}
  \FunctionTok{list}\NormalTok{(}\AttributeTok{model\_cohort =} \StringTok{"FOURNIER"}\NormalTok{, }\AttributeTok{ir\_group =} \DecValTok{0}\NormalTok{, }\AttributeTok{dose =} \DecValTok{2000}\NormalTok{, }\AttributeTok{interval =} \DecValTok{6}\NormalTok{, }\AttributeTok{duration =} \DecValTok{2}\NormalTok{),}
  \FunctionTok{list}\NormalTok{(}\AttributeTok{model\_cohort =} \StringTok{"FOURNIER"}\NormalTok{, }\AttributeTok{ir\_group =} \DecValTok{0}\NormalTok{, }\AttributeTok{dose =} \DecValTok{2000}\NormalTok{, }\AttributeTok{interval =} \DecValTok{8}\NormalTok{, }\AttributeTok{duration =} \DecValTok{1}\NormalTok{),}
  \FunctionTok{list}\NormalTok{(}\AttributeTok{model\_cohort =} \StringTok{"FOURNIER"}\NormalTok{, }\AttributeTok{ir\_group =} \DecValTok{1}\NormalTok{, }\AttributeTok{dose =} \DecValTok{500}\NormalTok{, }\AttributeTok{interval =} \DecValTok{8}\NormalTok{, }\AttributeTok{duration =} \DecValTok{2}\NormalTok{),}
  \FunctionTok{list}\NormalTok{(}\AttributeTok{model\_cohort =} \StringTok{"MELLON"}\NormalTok{, }\AttributeTok{ir\_group =} \DecValTok{0}\NormalTok{, }\AttributeTok{dose =} \DecValTok{1000}\NormalTok{, }\AttributeTok{interval =} \DecValTok{6}\NormalTok{, }\AttributeTok{duration =} \FloatTok{0.5}\NormalTok{),}
  \FunctionTok{list}\NormalTok{(}\AttributeTok{model\_cohort =} \StringTok{"RAMBAUD"}\NormalTok{, }\AttributeTok{ir\_group =} \DecValTok{0}\NormalTok{, }\AttributeTok{dose =} \DecValTok{12000}\NormalTok{, }\AttributeTok{interval =} \DecValTok{24}\NormalTok{, }\AttributeTok{duration =} \DecValTok{24}\NormalTok{),}
  \FunctionTok{list}\NormalTok{(}\AttributeTok{model\_cohort =} \StringTok{"RAMBAUD"}\NormalTok{, }\AttributeTok{ir\_group =} \DecValTok{0}\NormalTok{, }\AttributeTok{dose =} \DecValTok{14000}\NormalTok{, }\AttributeTok{interval =} \DecValTok{24}\NormalTok{, }\AttributeTok{duration =} \DecValTok{24}\NormalTok{)}
\NormalTok{)}

\CommentTok{\# Total simulation time}
\NormalTok{sim\_dur }\OtherTok{\textless{}{-}} \DecValTok{120}
\end{Highlighting}
\end{Shaded}

Finally, the simulations are done with each respective model, the
results are merged and a MODEL column is added to identify the model
used for the simulation of each concentration.

\begin{Shaded}
\begin{Highlighting}[]
\NormalTok{results\_Carlier }\OtherTok{\textless{}{-}} \FunctionTok{generate\_model\_regimen}\NormalTok{(}\StringTok{"amox\_Carlier"}\NormalTok{, }\FunctionTok{here}\NormalTok{(}\StringTok{"a\_priori/For\_publication/Simulations/amox\_Carlier"}\NormalTok{), regimens, sim\_dur,}\AttributeTok{cohort\_name =} \StringTok{"CARLIER"}\NormalTok{)}
\NormalTok{results\_Fournier }\OtherTok{\textless{}{-}} \FunctionTok{generate\_model\_regimen}\NormalTok{(}\StringTok{"amox\_Fournier"}\NormalTok{, }\FunctionTok{here}\NormalTok{(}\StringTok{"a\_priori/For\_publication/Simulations/amox\_Fournier"}\NormalTok{), regimens, sim\_dur,}\AttributeTok{cohort\_name =} \StringTok{"FOURNIER"}\NormalTok{)}
\NormalTok{results\_Mellon }\OtherTok{\textless{}{-}} \FunctionTok{generate\_model\_regimen}\NormalTok{(}\StringTok{"amox\_Mellon"}\NormalTok{, }\FunctionTok{here}\NormalTok{(}\StringTok{"a\_priori/For\_publication/Simulations/amox\_Mellon"}\NormalTok{), regimens, sim\_dur,}\AttributeTok{cohort\_name =} \StringTok{"MELLON"}\NormalTok{)}
\NormalTok{results\_Rambaud }\OtherTok{\textless{}{-}} \FunctionTok{generate\_model\_regimen}\NormalTok{(}\StringTok{"amox\_Rambaud"}\NormalTok{, }\FunctionTok{here}\NormalTok{(}\StringTok{"a\_priori/For\_publication/Simulations/amox\_Rambaud"}\NormalTok{), regimens, sim\_dur, }\AttributeTok{cohort\_name =} \StringTok{"RAMBAUD"}\NormalTok{)}

\CommentTok{\# Combine results from the simulation}
\NormalTok{all\_results\_raw }\OtherTok{\textless{}{-}}\NormalTok{ dplyr}\SpecialCharTok{::}\FunctionTok{bind\_rows}\NormalTok{(}
\NormalTok{  results\_Carlier,}
\NormalTok{  results\_Fournier,}
\NormalTok{  results\_Mellon,}
\NormalTok{  results\_Rambaud}
\NormalTok{)}

\CommentTok{\# Add MODEL column indicating the model which is used for the simulation}
\NormalTok{all\_results\_raw }\OtherTok{\textless{}{-}}\NormalTok{ all\_results\_raw }\SpecialCharTok{\%\textgreater{}\%}
\NormalTok{  dplyr}\SpecialCharTok{::}\FunctionTok{mutate}\NormalTok{(}\AttributeTok{MODEL =} \FunctionTok{case\_when}\NormalTok{(}
\NormalTok{    MODEL }\SpecialCharTok{==} \StringTok{"amox\_Carlier"} \SpecialCharTok{\textasciitilde{}} \StringTok{"CARLIER"}\NormalTok{,}
\NormalTok{    MODEL }\SpecialCharTok{==} \StringTok{"amox\_Fournier"} \SpecialCharTok{\textasciitilde{}} \StringTok{"FOURNIER"}\NormalTok{,}
\NormalTok{     MODEL }\SpecialCharTok{==} \StringTok{"amox\_Mellon"} \SpecialCharTok{\textasciitilde{}} \StringTok{"MELLON"}\NormalTok{,}
\NormalTok{    MODEL }\SpecialCharTok{==} \StringTok{"amox\_Rambaud"} \SpecialCharTok{\textasciitilde{}} \StringTok{"RAMBAUD"}\NormalTok{)}
\NormalTok{  ) }\SpecialCharTok{\%\textgreater{}\%}
  \FunctionTok{distinct}\NormalTok{()}

\CommentTok{\# Filter based on steady{-}state based on the reference (true) clearance and volume}
\NormalTok{ref\_values }\OtherTok{\textless{}{-}}\NormalTok{ all\_results\_raw }\SpecialCharTok{\%\textgreater{}\%}
  \FunctionTok{filter}\NormalTok{(MODEL }\SpecialCharTok{==}\NormalTok{ MODEL\_COHORT) }\SpecialCharTok{\%\textgreater{}\%}
  \FunctionTok{group\_by}\NormalTok{(ID) }\SpecialCharTok{\%\textgreater{}\%}
  \FunctionTok{slice}\NormalTok{(}\DecValTok{1}\NormalTok{) }\SpecialCharTok{\%\textgreater{}\%}
  \FunctionTok{select}\NormalTok{(ID, Vc, CL, Q, Vp)}

\NormalTok{all\_results }\OtherTok{\textless{}{-}}\NormalTok{ all\_results\_raw }\SpecialCharTok{\%\textgreater{}\%}
  \FunctionTok{left\_join}\NormalTok{(ref\_values, }\AttributeTok{by =} \StringTok{"ID"}\NormalTok{, }\AttributeTok{suffix =} \FunctionTok{c}\NormalTok{(}\StringTok{""}\NormalTok{, }\StringTok{"\_ref"}\NormalTok{)) }\SpecialCharTok{\%\textgreater{}\%}
  \FunctionTok{mutate}\NormalTok{(}\AttributeTok{K10 =}\NormalTok{ CL\_ref}\SpecialCharTok{/}\NormalTok{Vc\_ref,}
  \AttributeTok{K12 =}\NormalTok{ Q\_ref}\SpecialCharTok{/}\NormalTok{Vc\_ref,}
  \AttributeTok{K21 =}\NormalTok{ Q\_ref}\SpecialCharTok{/}\NormalTok{Vp\_ref,}
  \AttributeTok{L2 =}\NormalTok{ ((K10}\SpecialCharTok{+}\NormalTok{K12}\SpecialCharTok{+}\NormalTok{K21)}\SpecialCharTok{{-}}\NormalTok{((K10}\SpecialCharTok{+}\NormalTok{K12}\SpecialCharTok{+}\NormalTok{K21)}\SpecialCharTok{**}\DecValTok{2{-}4}\SpecialCharTok{*}\NormalTok{K10}\SpecialCharTok{*}\NormalTok{K21)}\SpecialCharTok{**}\FloatTok{0.5}\NormalTok{)}\SpecialCharTok{/}\DecValTok{2}\NormalTok{,}
  \AttributeTok{TSS =} \FunctionTok{ceiling}\NormalTok{((}\FloatTok{3.3} \SpecialCharTok{*}\NormalTok{ (}\FloatTok{0.693} \SpecialCharTok{/}\NormalTok{ L2)) }\SpecialCharTok{/}\NormalTok{ FREQ) }\SpecialCharTok{*}\NormalTok{ FREQ) }\SpecialCharTok{\%\textgreater{}\%}
\NormalTok{  dplyr}\SpecialCharTok{::}\FunctionTok{filter}\NormalTok{((TIME }\SpecialCharTok{\textgreater{}=} \DecValTok{0} \SpecialCharTok{\&}\NormalTok{ TIME }\SpecialCharTok{\textless{}=}\NormalTok{ (}\DecValTok{0} \SpecialCharTok{+}\NormalTok{ FREQ)) }\SpecialCharTok{|}\NormalTok{ (TIME }\SpecialCharTok{\textgreater{}=}\NormalTok{ TSS }\SpecialCharTok{\&}\NormalTok{ TIME }\SpecialCharTok{\textless{}=}\NormalTok{ (TSS }\SpecialCharTok{+}\NormalTok{ FREQ)))}
\end{Highlighting}
\end{Shaded}

\section{Concentrations (PRED, IPRED,
Y)}\label{concentrations-pred-ipred-y}

Below quantification limit observations are removed (M1 method). LLOQ
concentrations for each model can be found in ``mrgsolve/LLOQ.txt''.

\begin{Shaded}
\begin{Highlighting}[]
\NormalTok{AMOX\_SIM }\OtherTok{\textless{}{-}}\NormalTok{ all\_results }\SpecialCharTok{\%\textgreater{}\%}
  \FunctionTok{select}\NormalTok{(ID, TIME, IPRED, Y, WT, CREAT, BURN, ICU, OBESE, AGE, SEX, HT, BSA, MODEL, MODEL\_COHORT, DOSE\_ADM, FREQ, DUR, Vc, CL, TSS) }\SpecialCharTok{\%\textgreater{}\%}
  \FunctionTok{distinct}\NormalTok{()}

\CommentTok{\# Dataset to plot CL}
\NormalTok{summarized\_SIM }\OtherTok{\textless{}{-}}\NormalTok{ AMOX\_SIM }\SpecialCharTok{\%\textgreater{}\%}
  \FunctionTok{distinct}\NormalTok{(ID, MODEL, CL, Vc)}

\CommentTok{\# Box plot for CL}
\NormalTok{plot\_CL }\OtherTok{\textless{}{-}} \FunctionTok{ggplot}\NormalTok{(summarized\_SIM, }\FunctionTok{aes}\NormalTok{(}\AttributeTok{x =}\NormalTok{ MODEL, }\AttributeTok{y =}\NormalTok{ CL, }\AttributeTok{fill =}\NormalTok{ MODEL)) }\SpecialCharTok{+}
  \FunctionTok{geom\_boxplot}\NormalTok{() }\SpecialCharTok{+}
  \FunctionTok{theme\_minimal}\NormalTok{() }\SpecialCharTok{+}
  \FunctionTok{labs}\NormalTok{(}\AttributeTok{title =} \StringTok{"Clearance distribution"}\NormalTok{, }\AttributeTok{x =} \StringTok{"Model"}\NormalTok{, }\AttributeTok{y =} \StringTok{"CL (L/h)"}\NormalTok{) }\SpecialCharTok{+}
  \FunctionTok{theme}\NormalTok{(}
    \AttributeTok{axis.text.x =} \FunctionTok{element\_text}\NormalTok{(}\AttributeTok{angle =} \DecValTok{30}\NormalTok{, }\AttributeTok{hjust =} \DecValTok{1}\NormalTok{),}
    \AttributeTok{axis.text =} \FunctionTok{element\_text}\NormalTok{(}\AttributeTok{size =} \DecValTok{16}\NormalTok{),    }
    \AttributeTok{axis.title =} \FunctionTok{element\_text}\NormalTok{(}\AttributeTok{size =} \DecValTok{20}\NormalTok{), }
    \AttributeTok{plot.title =} \FunctionTok{element\_text}\NormalTok{(}\AttributeTok{size=}\DecValTok{24}\NormalTok{),}
\NormalTok{  )}

\NormalTok{plot\_CL}
\end{Highlighting}
\end{Shaded}

\pandocbounded{\includegraphics[keepaspectratio]{Simulations/concentrations_simulation_files/figure-pdf/concentrations-1.pdf}}

\begin{Shaded}
\begin{Highlighting}[]
\CommentTok{\# Box plot for Vc}
\NormalTok{plot\_V }\OtherTok{\textless{}{-}} \FunctionTok{ggplot}\NormalTok{(summarized\_SIM, }\FunctionTok{aes}\NormalTok{(}\AttributeTok{x =}\NormalTok{ MODEL, }\AttributeTok{y =}\NormalTok{ Vc, }\AttributeTok{fill =}\NormalTok{ MODEL)) }\SpecialCharTok{+}
  \FunctionTok{geom\_boxplot}\NormalTok{() }\SpecialCharTok{+}
  \FunctionTok{theme\_minimal}\NormalTok{() }\SpecialCharTok{+}
  \FunctionTok{scale\_y\_log10}\NormalTok{() }\SpecialCharTok{+} 
  \FunctionTok{labs}\NormalTok{(}\AttributeTok{title =} \StringTok{"Central volume distribution"}\NormalTok{, }\AttributeTok{x =} \StringTok{"Model"}\NormalTok{, }\AttributeTok{y =} \StringTok{"Vc (L)"}\NormalTok{) }\SpecialCharTok{+}
  \FunctionTok{theme}\NormalTok{(}
    \AttributeTok{axis.text.x =} \FunctionTok{element\_text}\NormalTok{(}\AttributeTok{angle =} \DecValTok{30}\NormalTok{, }\AttributeTok{hjust =} \DecValTok{1}\NormalTok{),}
    \AttributeTok{axis.text =} \FunctionTok{element\_text}\NormalTok{(}\AttributeTok{size =} \DecValTok{16}\NormalTok{),    }
    \AttributeTok{axis.title =} \FunctionTok{element\_text}\NormalTok{(}\AttributeTok{size =} \DecValTok{20}\NormalTok{), }
    \AttributeTok{plot.title =} \FunctionTok{element\_text}\NormalTok{(}\AttributeTok{size=}\DecValTok{24}\NormalTok{),}
\NormalTok{  )}

\NormalTok{plot\_V}
\end{Highlighting}
\end{Shaded}

\pandocbounded{\includegraphics[keepaspectratio]{Simulations/concentrations_simulation_files/figure-pdf/concentrations-2.pdf}}

\section{Trough concentration (Cmin)}\label{trough-concentration-cmin}

The trough concentration is the last concentration for the steady-state
interdose interval

\begin{Shaded}
\begin{Highlighting}[]
\CommentTok{\# Add CMIN column by taking the last steady{-}state concentration}
\NormalTok{AMOX\_CMIN1 }\OtherTok{\textless{}{-}}\NormalTok{ AMOX\_SIM }\SpecialCharTok{\%\textgreater{}\%}
  \FunctionTok{group\_by}\NormalTok{(ID, MODEL) }\SpecialCharTok{\%\textgreater{}\%}
  \FunctionTok{arrange}\NormalTok{(TIME) }\SpecialCharTok{\%\textgreater{}\%}
  \FunctionTok{mutate}\NormalTok{(}\AttributeTok{CMIN\_IND =} \FunctionTok{last}\NormalTok{(IPRED[TIME }\SpecialCharTok{==} \FunctionTok{max}\NormalTok{(TIME)])) }\SpecialCharTok{\%\textgreater{}\%}
  \FunctionTok{slice\_max}\NormalTok{(TIME, }\AttributeTok{n =} \DecValTok{1}\NormalTok{, }\AttributeTok{with\_ties =} \ConstantTok{FALSE}\NormalTok{) }\SpecialCharTok{\%\textgreater{}\%}
  \FunctionTok{ungroup}\NormalTok{() }\SpecialCharTok{\%\textgreater{}\%}
  \FunctionTok{distinct}\NormalTok{(ID, MODEL, CMIN\_IND, }\AttributeTok{.keep\_all =} \ConstantTok{TRUE}\NormalTok{) }\SpecialCharTok{\%\textgreater{}\%}
  \FunctionTok{mutate}\NormalTok{(}\AttributeTok{CON =} \FunctionTok{ifelse}\NormalTok{(DUR }\SpecialCharTok{==} \DecValTok{24}\NormalTok{, }\DecValTok{1}\NormalTok{, }\DecValTok{0}\NormalTok{))}

  
\CommentTok{\# Box plot for CL before removing the outliers}
\NormalTok{plot\_CMIN }\OtherTok{\textless{}{-}} \FunctionTok{ggplot}\NormalTok{(AMOX\_CMIN1, }\FunctionTok{aes}\NormalTok{(}\AttributeTok{x =}\NormalTok{ MODEL, }\AttributeTok{y =}\NormalTok{ CMIN\_IND, }\AttributeTok{fill =}\NormalTok{ MODEL)) }\SpecialCharTok{+}
  \FunctionTok{geom\_boxplot}\NormalTok{() }\SpecialCharTok{+}
  \FunctionTok{scale\_y\_log10}\NormalTok{() }\SpecialCharTok{+}
  \FunctionTok{theme\_minimal}\NormalTok{() }\SpecialCharTok{+}
  \FunctionTok{labs}\NormalTok{(}\AttributeTok{title =} \StringTok{"Cmin distribution"}\NormalTok{, }\AttributeTok{x =} \StringTok{"Model"}\NormalTok{, }\AttributeTok{y =} \StringTok{"Cmin (mg/L)"}\NormalTok{) }\SpecialCharTok{+}
  \FunctionTok{theme}\NormalTok{(}
    \AttributeTok{axis.text.x =} \FunctionTok{element\_text}\NormalTok{(}\AttributeTok{angle =} \DecValTok{30}\NormalTok{, }\AttributeTok{hjust =} \DecValTok{1}\NormalTok{),}
    \AttributeTok{axis.text =} \FunctionTok{element\_text}\NormalTok{(}\AttributeTok{size =} \DecValTok{16}\NormalTok{),    }
    \AttributeTok{axis.title =} \FunctionTok{element\_text}\NormalTok{(}\AttributeTok{size =} \DecValTok{20}\NormalTok{), }
    \AttributeTok{plot.title =} \FunctionTok{element\_text}\NormalTok{(}\AttributeTok{size=}\DecValTok{24}\NormalTok{),}
\NormalTok{  )}

\NormalTok{plot\_CMIN}
\end{Highlighting}
\end{Shaded}

\pandocbounded{\includegraphics[keepaspectratio]{Simulations/concentrations_simulation_files/figure-pdf/CMIN-1.pdf}}

\begin{Shaded}
\begin{Highlighting}[]
\CommentTok{\# Create test data by taking 2400 subjects (24 from each dosing scheme and 6{-}6 for each model covariate cohorte per dosing scheme)}
\NormalTok{CMIN\_test }\OtherTok{\textless{}{-}}\NormalTok{ AMOX\_CMIN1 }\SpecialCharTok{\%\textgreater{}\%}
\NormalTok{  dplyr}\SpecialCharTok{::}\FunctionTok{filter}\NormalTok{(}
\NormalTok{    ID }\SpecialCharTok{\%\%} \DecValTok{100} \SpecialCharTok{\textgreater{}=} \DecValTok{1} \SpecialCharTok{\&}\NormalTok{ ID }\SpecialCharTok{\%\%} \DecValTok{100} \SpecialCharTok{\textless{}=} \DecValTok{6} \SpecialCharTok{|}
\NormalTok{      ID }\SpecialCharTok{\%\%} \DecValTok{100} \SpecialCharTok{\textgreater{}=} \DecValTok{26} \SpecialCharTok{\&}\NormalTok{ ID }\SpecialCharTok{\%\%} \DecValTok{100} \SpecialCharTok{\textless{}=} \DecValTok{31} \SpecialCharTok{|}
\NormalTok{      ID }\SpecialCharTok{\%\%} \DecValTok{100} \SpecialCharTok{\textgreater{}=} \DecValTok{51} \SpecialCharTok{\&}\NormalTok{ ID }\SpecialCharTok{\%\%} \DecValTok{100} \SpecialCharTok{\textless{}=} \DecValTok{56} \SpecialCharTok{|}
\NormalTok{      ID }\SpecialCharTok{\%\%} \DecValTok{100} \SpecialCharTok{\textgreater{}=} \DecValTok{76} \SpecialCharTok{\&}\NormalTok{ ID }\SpecialCharTok{\%\%} \DecValTok{100} \SpecialCharTok{\textless{}=} \DecValTok{81}
\NormalTok{  )}

\CommentTok{\# Create training data by taking 7600 subjects (76 from each dosing scheme and 19{-}19 for each model covariate cohorte per dosing scheme)}
\NormalTok{CMIN\_train }\OtherTok{\textless{}{-}}\NormalTok{ AMOX\_CMIN1 }\SpecialCharTok{\%\textgreater{}\%}
\NormalTok{  dplyr}\SpecialCharTok{::}\FunctionTok{filter}\NormalTok{(}
\NormalTok{    ID }\SpecialCharTok{\%\%} \DecValTok{100} \SpecialCharTok{\textgreater{}=} \DecValTok{7} \SpecialCharTok{\&}\NormalTok{ ID }\SpecialCharTok{\%\%} \DecValTok{100} \SpecialCharTok{\textless{}=} \DecValTok{25} \SpecialCharTok{|}
\NormalTok{      ID }\SpecialCharTok{\%\%} \DecValTok{100} \SpecialCharTok{\textgreater{}=} \DecValTok{32} \SpecialCharTok{\&}\NormalTok{ ID }\SpecialCharTok{\%\%} \DecValTok{100} \SpecialCharTok{\textless{}=} \DecValTok{50} \SpecialCharTok{|}
\NormalTok{      ID }\SpecialCharTok{\%\%} \DecValTok{100} \SpecialCharTok{\textgreater{}=} \DecValTok{57} \SpecialCharTok{\&}\NormalTok{ ID }\SpecialCharTok{\%\%} \DecValTok{100} \SpecialCharTok{\textless{}=} \DecValTok{75} \SpecialCharTok{|}
\NormalTok{      ID }\SpecialCharTok{\%\%} \DecValTok{100} \SpecialCharTok{\textgreater{}=} \DecValTok{82} \SpecialCharTok{\&}\NormalTok{ ID }\SpecialCharTok{\%\%} \DecValTok{100} \SpecialCharTok{\textless{}=} \DecValTok{100}
\NormalTok{  )}

\CommentTok{\# Dataset identifyer}
\NormalTok{CMIN\_train\_compare }\OtherTok{\textless{}{-}}\NormalTok{ CMIN\_train }\SpecialCharTok{\%\textgreater{}\%} \FunctionTok{mutate}\NormalTok{(}\AttributeTok{Dataset =} \StringTok{"Train"}\NormalTok{)}
\NormalTok{CMIN\_test\_compare }\OtherTok{\textless{}{-}}\NormalTok{ CMIN\_test }\SpecialCharTok{\%\textgreater{}\%} \FunctionTok{mutate}\NormalTok{(}\AttributeTok{Dataset =} \StringTok{"Test"}\NormalTok{)}

\CommentTok{\# Combine the three datasets}
\NormalTok{combined\_CMIN }\OtherTok{\textless{}{-}} \FunctionTok{bind\_rows}\NormalTok{(CMIN\_train\_compare, CMIN\_test\_compare)}

\CommentTok{\# Compare the train and test sets}
\NormalTok{vars }\OtherTok{\textless{}{-}} \FunctionTok{c}\NormalTok{(}\StringTok{"CREAT"}\NormalTok{, }\StringTok{"WT"}\NormalTok{, }\StringTok{"CMIN\_IND"}\NormalTok{, }\StringTok{"CL"}\NormalTok{, }\StringTok{"Vc"}\NormalTok{)}
\NormalTok{tableOne5 }\OtherTok{\textless{}{-}} \FunctionTok{CreateTableOne}\NormalTok{(}\AttributeTok{vars =}\NormalTok{ vars, }\AttributeTok{strata =} \StringTok{"Dataset"}\NormalTok{, }\AttributeTok{data =}\NormalTok{ combined\_CMIN)}
\NormalTok{tableOne6 }\OtherTok{\textless{}{-}} \FunctionTok{print}\NormalTok{(tableOne5, }\AttributeTok{nonnormal =} \FunctionTok{c}\NormalTok{(}\StringTok{"CREAT"}\NormalTok{, }\StringTok{"WT"}\NormalTok{, }\StringTok{"CMIN\_IND"}\NormalTok{, }\StringTok{"CL"}\NormalTok{, }\StringTok{"Vc"}\NormalTok{), }\AttributeTok{printToggle=}\NormalTok{F, }\AttributeTok{minMax=}\NormalTok{T)}

\FunctionTok{kableone}\NormalTok{(tableOne6)}
\end{Highlighting}
\end{Shaded}

\begin{longtable}[]{@{}
  >{\raggedright\arraybackslash}p{(\linewidth - 8\tabcolsep) * \real{0.3095}}
  >{\raggedright\arraybackslash}p{(\linewidth - 8\tabcolsep) * \real{0.2619}}
  >{\raggedright\arraybackslash}p{(\linewidth - 8\tabcolsep) * \real{0.2619}}
  >{\raggedright\arraybackslash}p{(\linewidth - 8\tabcolsep) * \real{0.0714}}
  >{\raggedright\arraybackslash}p{(\linewidth - 8\tabcolsep) * \real{0.0952}}@{}}
\toprule\noalign{}
\begin{minipage}[b]{\linewidth}\raggedright
\end{minipage} & \begin{minipage}[b]{\linewidth}\raggedright
Test
\end{minipage} & \begin{minipage}[b]{\linewidth}\raggedright
Train
\end{minipage} & \begin{minipage}[b]{\linewidth}\raggedright
p
\end{minipage} & \begin{minipage}[b]{\linewidth}\raggedright
test
\end{minipage} \\
\midrule\noalign{}
\endhead
\bottomrule\noalign{}
\endlastfoot
n & 600 & 1875 & & \\
CREAT (median {[}range{]}) & 0.76 {[}0.24, 5.32{]} & 0.77 {[}0.13,
8.23{]} & 0.410 & nonnorm \\
WT (median {[}range{]}) & 80.03 {[}41.33, 139.60{]} & 79.99 {[}47.36,
141.17{]} & 0.884 & nonnorm \\
CMIN\_IND (median {[}range{]}) & 2.74 {[}0.00, 215.39{]} & 2.73 {[}0.01,
197.56{]} & 0.368 & nonnorm \\
CL (median {[}range{]}) & 12.78 {[}0.68, 114.45{]} & 12.80 {[}1.08,
76.53{]} & 0.624 & nonnorm \\
Vc (median {[}range{]}) & 9.73 {[}3.07, 65.36{]} & 9.71 {[}3.06,
73.77{]} & 0.554 & nonnorm \\
\end{longtable}

\begin{Shaded}
\begin{Highlighting}[]
\CommentTok{\# Export to csv}
\FunctionTok{write.csv}\NormalTok{(CMIN\_train, }\FunctionTok{here}\NormalTok{(}\StringTok{"a\_priori/For\_publication/Data/AMOX\_CMIN\_OBS\_TRAIN.csv"}\NormalTok{), }\AttributeTok{row.names =} \ConstantTok{FALSE}\NormalTok{, }\AttributeTok{quote =} \ConstantTok{FALSE}\NormalTok{)}
\FunctionTok{write.csv}\NormalTok{(CMIN\_test, }\FunctionTok{here}\NormalTok{(}\StringTok{"a\_priori/For\_publication/Data/AMOX\_CMIN\_OBS\_TEST.csv"}\NormalTok{), }\AttributeTok{row.names =} \ConstantTok{FALSE}\NormalTok{, }\AttributeTok{quote =} \ConstantTok{FALSE}\NormalTok{)}
\end{Highlighting}
\end{Shaded}

\section{Data visualization stratified by dosing
schemes}\label{data-visualization-stratified-by-dosing-schemes}

The following graphs show the individual observed concentration-time
curves for the data stratified by model. One graph shows the
concentration-time curves for the first interdose interval and the other
graph for the interdose interval after the dose which permits to reach
steady state.

\begin{Shaded}
\begin{Highlighting}[]
\CommentTok{\# Add dosing scheme as a categorical variable (to separate dosing schemes)}
\NormalTok{data\_ref }\OtherTok{\textless{}{-}}\NormalTok{ AMOX\_SIM }\SpecialCharTok{\%\textgreater{}\%}
  \FunctionTok{mutate}\NormalTok{(}\AttributeTok{DOSING\_SCHEME =} \FunctionTok{paste}\NormalTok{(}\StringTok{"FREQ:"}\NormalTok{, FREQ, }\StringTok{"DOSE:"}\NormalTok{, DOSE\_ADM, }\StringTok{"DUR:"}\NormalTok{, DUR, }\AttributeTok{sep =} \StringTok{" "}\NormalTok{))}

\NormalTok{dosing\_schemes }\OtherTok{\textless{}{-}} \FunctionTok{unique}\NormalTok{(data\_ref}\SpecialCharTok{$}\NormalTok{DOSING\_SCHEME)}

\CommentTok{\# Loop through each dosing scheme}
\NormalTok{plot\_dosing\_schemes }\OtherTok{\textless{}{-}} \ControlFlowTok{function}\NormalTok{(data\_ref, dosing\_schemes) \{}
\NormalTok{  plot\_list\_before }\OtherTok{\textless{}{-}} \FunctionTok{list}\NormalTok{()}
\NormalTok{  plot\_list\_after }\OtherTok{\textless{}{-}} \FunctionTok{list}\NormalTok{()}
  
  \ControlFlowTok{for}\NormalTok{ (i }\ControlFlowTok{in} \FunctionTok{seq\_along}\NormalTok{(dosing\_schemes)) \{}
\NormalTok{    scheme }\OtherTok{\textless{}{-}}\NormalTok{ dosing\_schemes[i]}
    
    \CommentTok{\# Extract FREQ (=interdose interval)}
\NormalTok{    scheme\_data }\OtherTok{\textless{}{-}} \FunctionTok{subset}\NormalTok{(data\_ref, DOSING\_SCHEME }\SpecialCharTok{==}\NormalTok{ scheme)}
\NormalTok{    freq\_value }\OtherTok{\textless{}{-}} \FunctionTok{unique}\NormalTok{(scheme\_data}\SpecialCharTok{$}\NormalTok{FREQ)}
    
    
    \CommentTok{\# Split data before and after SS}
\NormalTok{    data\_before }\OtherTok{\textless{}{-}} \FunctionTok{subset}\NormalTok{(scheme\_data, TIME }\SpecialCharTok{\textless{}=}\NormalTok{ freq\_value)}
\NormalTok{    data\_after }\OtherTok{\textless{}{-}} \FunctionTok{subset}\NormalTok{(scheme\_data, TIME }\SpecialCharTok{\textgreater{}}\NormalTok{ TSS) }\SpecialCharTok{\%\textgreater{}\%}
      \FunctionTok{mutate}\NormalTok{(}\AttributeTok{TIME =}\NormalTok{ TIME }\SpecialCharTok{{-}}\NormalTok{ TSS)  }\CommentTok{\# Adjust time so that it means time after SS dose}
    
    \CommentTok{\# Plot for before SS}
\NormalTok{    plot\_list\_before[[i]] }\OtherTok{\textless{}{-}} \FunctionTok{ggplot}\NormalTok{(data\_before, }\FunctionTok{aes}\NormalTok{(}\AttributeTok{x =}\NormalTok{ TIME, }\AttributeTok{y =}\NormalTok{ IPRED, }\AttributeTok{color =}\NormalTok{ MODEL, }\AttributeTok{group =}\NormalTok{ ID)) }\SpecialCharTok{+}
      \FunctionTok{geom\_line}\NormalTok{() }\SpecialCharTok{+} 
      \FunctionTok{facet\_wrap}\NormalTok{(}\SpecialCharTok{\textasciitilde{}}\NormalTok{MODEL) }\SpecialCharTok{+}
      \FunctionTok{theme\_minimal}\NormalTok{() }\SpecialCharTok{+} 
      \FunctionTok{labs}\NormalTok{(}
        \AttributeTok{title =} \FunctionTok{paste}\NormalTok{(}\StringTok{"First interdose interval"}\NormalTok{, freq\_value, scheme),}
        \AttributeTok{x =} \StringTok{"Time (h)"}\NormalTok{,}
        \AttributeTok{y =} \StringTok{"IPRED (mg/L)"}\NormalTok{,}
        \AttributeTok{color =} \StringTok{"MODEL"}
\NormalTok{      ) }\SpecialCharTok{+}
      \FunctionTok{theme}\NormalTok{(}
        \AttributeTok{plot.title =} \FunctionTok{element\_text}\NormalTok{(}\AttributeTok{size=}\DecValTok{10}\NormalTok{),}
        \AttributeTok{strip.text =} \FunctionTok{element\_text}\NormalTok{(}\AttributeTok{size =} \DecValTok{8}\NormalTok{), }
        \AttributeTok{axis.text =} \FunctionTok{element\_text}\NormalTok{(}\AttributeTok{size =} \DecValTok{8}\NormalTok{),}
        \AttributeTok{legend.position =} \StringTok{"none"}
\NormalTok{      )}
    
    \CommentTok{\# Plot for after SS is reached}
\NormalTok{    plot\_list\_after[[i]] }\OtherTok{\textless{}{-}} \FunctionTok{ggplot}\NormalTok{(data\_after, }\FunctionTok{aes}\NormalTok{(}\AttributeTok{x =}\NormalTok{ TIME, }\AttributeTok{y =}\NormalTok{ IPRED, }\AttributeTok{color =}\NormalTok{ MODEL, }\AttributeTok{group =}\NormalTok{ ID)) }\SpecialCharTok{+}
      \FunctionTok{geom\_line}\NormalTok{() }\SpecialCharTok{+} 
      \FunctionTok{facet\_wrap}\NormalTok{(}\SpecialCharTok{\textasciitilde{}}\NormalTok{MODEL) }\SpecialCharTok{+}
      \FunctionTok{theme\_minimal}\NormalTok{() }\SpecialCharTok{+} 
      \FunctionTok{labs}\NormalTok{(}
        \AttributeTok{title =} \FunctionTok{paste}\NormalTok{(}\StringTok{"Steady{-}state"}\NormalTok{, freq\_value, scheme),}
        \AttributeTok{x =} \StringTok{"Time after steady{-}state dose (h)"}\NormalTok{,}
        \AttributeTok{y =} \StringTok{"IPRED (mg/L)"}\NormalTok{,}
        \AttributeTok{color =} \StringTok{"MODEL"}
\NormalTok{      ) }\SpecialCharTok{+}
      \FunctionTok{theme}\NormalTok{(}
        \AttributeTok{plot.title =} \FunctionTok{element\_text}\NormalTok{(}\AttributeTok{size=}\DecValTok{10}\NormalTok{),}
        \AttributeTok{strip.text =} \FunctionTok{element\_text}\NormalTok{(}\AttributeTok{size =} \DecValTok{8}\NormalTok{), }
        \AttributeTok{axis.text =} \FunctionTok{element\_text}\NormalTok{(}\AttributeTok{size =} \DecValTok{8}\NormalTok{),}
        \AttributeTok{legend.position =} \StringTok{"none"}
\NormalTok{      )}
\NormalTok{  \}}
  
  \FunctionTok{list}\NormalTok{(}\AttributeTok{before =}\NormalTok{ plot\_list\_before, }\AttributeTok{after =}\NormalTok{ plot\_list\_after)}
\NormalTok{\}}

\NormalTok{conc\_profile\_plots }\OtherTok{\textless{}{-}} \FunctionTok{plot\_dosing\_schemes}\NormalTok{(data\_ref, dosing\_schemes)}
\NormalTok{conc\_profile\_plots}
\end{Highlighting}
\end{Shaded}

\begin{verbatim}
$before
$before[[1]]
\end{verbatim}

\pandocbounded{\includegraphics[keepaspectratio]{Simulations/concentrations_simulation_files/figure-pdf/Data-visualization-1.pdf}}

\begin{verbatim}

$before[[2]]
\end{verbatim}

\pandocbounded{\includegraphics[keepaspectratio]{Simulations/concentrations_simulation_files/figure-pdf/Data-visualization-2.pdf}}

\begin{verbatim}

$before[[3]]
\end{verbatim}

\pandocbounded{\includegraphics[keepaspectratio]{Simulations/concentrations_simulation_files/figure-pdf/Data-visualization-3.pdf}}

\begin{verbatim}

$before[[4]]
\end{verbatim}

\pandocbounded{\includegraphics[keepaspectratio]{Simulations/concentrations_simulation_files/figure-pdf/Data-visualization-4.pdf}}

\begin{verbatim}

$before[[5]]
\end{verbatim}

\pandocbounded{\includegraphics[keepaspectratio]{Simulations/concentrations_simulation_files/figure-pdf/Data-visualization-5.pdf}}

\begin{verbatim}

$before[[6]]
\end{verbatim}

\pandocbounded{\includegraphics[keepaspectratio]{Simulations/concentrations_simulation_files/figure-pdf/Data-visualization-6.pdf}}

\begin{verbatim}

$before[[7]]
\end{verbatim}

\pandocbounded{\includegraphics[keepaspectratio]{Simulations/concentrations_simulation_files/figure-pdf/Data-visualization-7.pdf}}

\begin{verbatim}

$before[[8]]
\end{verbatim}

\pandocbounded{\includegraphics[keepaspectratio]{Simulations/concentrations_simulation_files/figure-pdf/Data-visualization-8.pdf}}

\begin{verbatim}

$before[[9]]
\end{verbatim}

\pandocbounded{\includegraphics[keepaspectratio]{Simulations/concentrations_simulation_files/figure-pdf/Data-visualization-9.pdf}}

\begin{verbatim}


$after
$after[[1]]
\end{verbatim}

\pandocbounded{\includegraphics[keepaspectratio]{Simulations/concentrations_simulation_files/figure-pdf/Data-visualization-10.pdf}}

\begin{verbatim}

$after[[2]]
\end{verbatim}

\pandocbounded{\includegraphics[keepaspectratio]{Simulations/concentrations_simulation_files/figure-pdf/Data-visualization-11.pdf}}

\begin{verbatim}

$after[[3]]
\end{verbatim}

\pandocbounded{\includegraphics[keepaspectratio]{Simulations/concentrations_simulation_files/figure-pdf/Data-visualization-12.pdf}}

\begin{verbatim}

$after[[4]]
\end{verbatim}

\pandocbounded{\includegraphics[keepaspectratio]{Simulations/concentrations_simulation_files/figure-pdf/Data-visualization-13.pdf}}

\begin{verbatim}

$after[[5]]
\end{verbatim}

\pandocbounded{\includegraphics[keepaspectratio]{Simulations/concentrations_simulation_files/figure-pdf/Data-visualization-14.pdf}}

\begin{verbatim}

$after[[6]]
\end{verbatim}

\pandocbounded{\includegraphics[keepaspectratio]{Simulations/concentrations_simulation_files/figure-pdf/Data-visualization-15.pdf}}

\begin{verbatim}

$after[[7]]
\end{verbatim}

\pandocbounded{\includegraphics[keepaspectratio]{Simulations/concentrations_simulation_files/figure-pdf/Data-visualization-16.pdf}}

\begin{verbatim}

$after[[8]]
\end{verbatim}

\pandocbounded{\includegraphics[keepaspectratio]{Simulations/concentrations_simulation_files/figure-pdf/Data-visualization-17.pdf}}

\begin{verbatim}

$after[[9]]
\end{verbatim}

\pandocbounded{\includegraphics[keepaspectratio]{Simulations/concentrations_simulation_files/figure-pdf/Data-visualization-18.pdf}}

Visualization of 5th, 50th and 95th percentiles (5th and 95th
percentiles in dashed)

\begin{Shaded}
\begin{Highlighting}[]
\CommentTok{\# The data is binned to 40 bins to make it smoother}
\NormalTok{data\_percentiles\_ref }\OtherTok{\textless{}{-}}\NormalTok{ data\_ref }\SpecialCharTok{\%\textgreater{}\%} \FunctionTok{filter}\NormalTok{(TIME }\SpecialCharTok{\textgreater{}}\NormalTok{ TSS) }\SpecialCharTok{\%\textgreater{}\%}
      \FunctionTok{mutate}\NormalTok{(}\AttributeTok{TIME =}\NormalTok{ TIME }\SpecialCharTok{{-}}\NormalTok{ TSS) }\SpecialCharTok{|\textgreater{}} 
  \FunctionTok{mutate}\NormalTok{(}\AttributeTok{TIME\_bin =} \FunctionTok{ntile}\NormalTok{(TIME, }\DecValTok{40}\NormalTok{)) }\SpecialCharTok{\%\textgreater{}\%}
  \FunctionTok{group\_by}\NormalTok{(DOSING\_SCHEME, MODEL, TIME\_bin) }\SpecialCharTok{\%\textgreater{}\%}
  \FunctionTok{summarise}\NormalTok{(}
    \AttributeTok{P5 =} \FunctionTok{quantile}\NormalTok{(IPRED, }\FloatTok{0.05}\NormalTok{, }\AttributeTok{na.rm =} \ConstantTok{TRUE}\NormalTok{),}
    \AttributeTok{P50 =} \FunctionTok{quantile}\NormalTok{(IPRED, }\FloatTok{0.50}\NormalTok{, }\AttributeTok{na.rm =} \ConstantTok{TRUE}\NormalTok{),}
    \AttributeTok{P95 =} \FunctionTok{quantile}\NormalTok{(IPRED, }\FloatTok{0.95}\NormalTok{, }\AttributeTok{na.rm =} \ConstantTok{TRUE}\NormalTok{),}
    \AttributeTok{TIME =} \FunctionTok{median}\NormalTok{(TIME, }\AttributeTok{na.rm =} \ConstantTok{TRUE}\NormalTok{),}
    \AttributeTok{.groups =} \StringTok{\textquotesingle{}drop\textquotesingle{}}
\NormalTok{  )}

\CommentTok{\# Loop through each dosing scheme}
\NormalTok{plot\_percentiles }\OtherTok{\textless{}{-}} \ControlFlowTok{function}\NormalTok{(data\_percentiles\_ref, dosing\_schemes) \{}
  \FunctionTok{library}\NormalTok{(ggplot2)}
  \FunctionTok{library}\NormalTok{(patchwork)}
  
\NormalTok{  plot\_list\_ref }\OtherTok{\textless{}{-}} \FunctionTok{list}\NormalTok{()}
  
  \ControlFlowTok{for}\NormalTok{ (i }\ControlFlowTok{in} \FunctionTok{seq\_along}\NormalTok{(dosing\_schemes)) \{}
\NormalTok{    plot\_list\_ref[[i]] }\OtherTok{\textless{}{-}} \FunctionTok{ggplot}\NormalTok{(}
      \FunctionTok{subset}\NormalTok{(data\_percentiles\_ref, DOSING\_SCHEME }\SpecialCharTok{==}\NormalTok{ dosing\_schemes[i]),}
      \FunctionTok{aes}\NormalTok{(}\AttributeTok{x =}\NormalTok{ TIME)}
\NormalTok{    ) }\SpecialCharTok{+}
      \FunctionTok{geom\_line}\NormalTok{(}\FunctionTok{aes}\NormalTok{(}\AttributeTok{y =}\NormalTok{ P5, }\AttributeTok{color =}\NormalTok{ MODEL), }\AttributeTok{linetype =} \StringTok{"dashed"}\NormalTok{) }\SpecialCharTok{+}
      \FunctionTok{geom\_line}\NormalTok{(}\FunctionTok{aes}\NormalTok{(}\AttributeTok{y =}\NormalTok{ P50, }\AttributeTok{color =}\NormalTok{ MODEL), }\AttributeTok{size =} \DecValTok{1}\NormalTok{) }\SpecialCharTok{+}
      \FunctionTok{geom\_line}\NormalTok{(}\FunctionTok{aes}\NormalTok{(}\AttributeTok{y =}\NormalTok{ P95, }\AttributeTok{color =}\NormalTok{ MODEL), }\AttributeTok{linetype =} \StringTok{"dashed"}\NormalTok{) }\SpecialCharTok{+}
      \FunctionTok{theme\_minimal}\NormalTok{() }\SpecialCharTok{+}
      \FunctionTok{labs}\NormalTok{(}
        \AttributeTok{title =} \FunctionTok{paste}\NormalTok{(dosing\_schemes[i]),}
        \AttributeTok{x =} \StringTok{"Time post steady{-}state (binned)"}\NormalTok{,}
        \AttributeTok{y =} \StringTok{"IPRED (mg/L)"}\NormalTok{,}
        \AttributeTok{color =} \StringTok{"MODEL"}
\NormalTok{      ) }\SpecialCharTok{+}
      \FunctionTok{theme}\NormalTok{(}
        \AttributeTok{plot.title =} \FunctionTok{element\_text}\NormalTok{(}\AttributeTok{size =} \DecValTok{10}\NormalTok{),}
        \AttributeTok{axis.text =} \FunctionTok{element\_text}\NormalTok{(}\AttributeTok{size =} \DecValTok{8}\NormalTok{),}
        \AttributeTok{strip.text =} \FunctionTok{element\_text}\NormalTok{(}\AttributeTok{size =} \DecValTok{8}\NormalTok{),}
        \AttributeTok{legend.position =} \StringTok{"bottom"}
\NormalTok{      )}
\NormalTok{  \}}
  
  \FunctionTok{return}\NormalTok{(plot\_list\_ref)}
\NormalTok{\}}


\NormalTok{percentile\_plots }\OtherTok{\textless{}{-}} \FunctionTok{plot\_percentiles}\NormalTok{(data\_percentiles\_ref, dosing\_schemes)}
\end{Highlighting}
\end{Shaded}

\section{Appendix : What if you were interested in something else than
Cmin
?}\label{appendix-what-if-you-were-interested-in-something-else-than-cmin}

\subsection{Exposure (AUC)}\label{exposure-auc}

AUC is calculated based on the sum of the integrals of IPRED values and
cumulated for both the steady-state and non stead-state inter-dose
interval.

\begin{Shaded}
\begin{Highlighting}[]
\NormalTok{AMOX\_AUC1 }\OtherTok{\textless{}{-}}\NormalTok{ all\_results }\SpecialCharTok{\%\textgreater{}\%}
  \FunctionTok{group\_by}\NormalTok{(ID, MODEL) }\SpecialCharTok{\%\textgreater{}\%} 
  \FunctionTok{summarize}\NormalTok{(}
    \AttributeTok{ID =} \FunctionTok{first}\NormalTok{(ID),}
    \AttributeTok{MODEL =} \FunctionTok{first}\NormalTok{(MODEL),}
    \AttributeTok{MODEL\_COHORT =} \FunctionTok{first}\NormalTok{(MODEL\_COHORT),}
    \AttributeTok{AUC\_IND =} \FunctionTok{sum}\NormalTok{(AUC, }\AttributeTok{na.rm =} \ConstantTok{TRUE}\NormalTok{), }\CommentTok{\#total AUC (based on IPRED)}
    \AttributeTok{WT =} \FunctionTok{first}\NormalTok{(WT),    }
    \AttributeTok{CREAT =} \FunctionTok{first}\NormalTok{(CREAT), }
    \AttributeTok{BURN =} \FunctionTok{first}\NormalTok{(BURN),}
    \AttributeTok{ICU =} \FunctionTok{first}\NormalTok{(ICU),}
    \AttributeTok{OBESE =} \FunctionTok{first}\NormalTok{(OBESE),}
    \AttributeTok{AGE =} \FunctionTok{first}\NormalTok{(AGE),}
    \AttributeTok{SEX =} \FunctionTok{first}\NormalTok{(SEX),}
    \AttributeTok{HT =} \FunctionTok{first}\NormalTok{(HT),}
    \AttributeTok{DUR =} \FunctionTok{first}\NormalTok{(DUR),}
    \AttributeTok{FREQ =} \FunctionTok{first}\NormalTok{(FREQ),}
    \AttributeTok{DOSE\_ADM =} \FunctionTok{first}\NormalTok{(DOSE\_ADM),}
    \AttributeTok{TSS =} \FunctionTok{first}\NormalTok{(TSS),}
    \AttributeTok{Vc =} \FunctionTok{first}\NormalTok{(Vc),}
    \AttributeTok{CL =} \FunctionTok{first}\NormalTok{(CL)}
\NormalTok{  )}

\NormalTok{summarized\_AUC }\OtherTok{\textless{}{-}}\NormalTok{ AMOX\_AUC1 }\SpecialCharTok{\%\textgreater{}\%}
  \FunctionTok{distinct}\NormalTok{(ID, MODEL, AUC\_IND) }

\CommentTok{\# Box plot for AUC}
\NormalTok{plot\_AUC }\OtherTok{\textless{}{-}} \FunctionTok{ggplot}\NormalTok{(summarized\_AUC, }\FunctionTok{aes}\NormalTok{(}\AttributeTok{x =}\NormalTok{ MODEL, }\AttributeTok{y =}\NormalTok{ AUC\_IND, }\AttributeTok{fill =}\NormalTok{ MODEL)) }\SpecialCharTok{+}
  \FunctionTok{geom\_boxplot}\NormalTok{() }\SpecialCharTok{+}
  \FunctionTok{scale\_y\_log10}\NormalTok{() }\SpecialCharTok{+}
  \FunctionTok{theme\_minimal}\NormalTok{() }\SpecialCharTok{+}
  \FunctionTok{labs}\NormalTok{(}\AttributeTok{title =} \StringTok{"AUC distribution"}\NormalTok{, }\AttributeTok{x =} \StringTok{"Model"}\NormalTok{, }\AttributeTok{y =} \StringTok{"AUC (mg.h/L)"}\NormalTok{) }\SpecialCharTok{+}
  \FunctionTok{theme}\NormalTok{(}
    \AttributeTok{axis.text.x =} \FunctionTok{element\_text}\NormalTok{(}\AttributeTok{angle =} \DecValTok{30}\NormalTok{, }\AttributeTok{hjust =} \DecValTok{1}\NormalTok{),}
    \AttributeTok{axis.text =} \FunctionTok{element\_text}\NormalTok{(}\AttributeTok{size =} \DecValTok{16}\NormalTok{),    }
    \AttributeTok{axis.title =} \FunctionTok{element\_text}\NormalTok{(}\AttributeTok{size =} \DecValTok{20}\NormalTok{), }
    \AttributeTok{plot.title =} \FunctionTok{element\_text}\NormalTok{(}\AttributeTok{size=}\DecValTok{24}\NormalTok{),}
\NormalTok{  )}

\NormalTok{plot\_AUC}
\end{Highlighting}
\end{Shaded}

\pandocbounded{\includegraphics[keepaspectratio]{Simulations/concentrations_simulation_files/figure-pdf/AUC-1.pdf}}

\begin{Shaded}
\begin{Highlighting}[]
\CommentTok{\# Create test data by taking 2400 subjects (24 from each dosing scheme and 6{-}6 for each model covariate cohorte per dosing scheme)}
\NormalTok{AUC\_test }\OtherTok{\textless{}{-}}\NormalTok{ AMOX\_AUC1 }\SpecialCharTok{\%\textgreater{}\%}
\NormalTok{  dplyr}\SpecialCharTok{::}\FunctionTok{filter}\NormalTok{(}
\NormalTok{    ID }\SpecialCharTok{\%\%} \DecValTok{100} \SpecialCharTok{\textgreater{}=} \DecValTok{1} \SpecialCharTok{\&}\NormalTok{ ID }\SpecialCharTok{\%\%} \DecValTok{100} \SpecialCharTok{\textless{}=} \DecValTok{6} \SpecialCharTok{|}
\NormalTok{      ID }\SpecialCharTok{\%\%} \DecValTok{100} \SpecialCharTok{\textgreater{}=} \DecValTok{26} \SpecialCharTok{\&}\NormalTok{ ID }\SpecialCharTok{\%\%} \DecValTok{100} \SpecialCharTok{\textless{}=} \DecValTok{31} \SpecialCharTok{|}
\NormalTok{      ID }\SpecialCharTok{\%\%} \DecValTok{100} \SpecialCharTok{\textgreater{}=} \DecValTok{51} \SpecialCharTok{\&}\NormalTok{ ID }\SpecialCharTok{\%\%} \DecValTok{100} \SpecialCharTok{\textless{}=} \DecValTok{56} \SpecialCharTok{|}
\NormalTok{      ID }\SpecialCharTok{\%\%} \DecValTok{100} \SpecialCharTok{\textgreater{}=} \DecValTok{76} \SpecialCharTok{\&}\NormalTok{ ID }\SpecialCharTok{\%\%} \DecValTok{100} \SpecialCharTok{\textless{}=} \DecValTok{81}
\NormalTok{  )}

\CommentTok{\# Create training data by taking 7600 subjects (76 from each dosing scheme and 19{-}19 for each model covariate cohorte per dosing scheme)}
\NormalTok{AUC\_train }\OtherTok{\textless{}{-}}\NormalTok{ AMOX\_AUC1 }\SpecialCharTok{\%\textgreater{}\%}
\NormalTok{  dplyr}\SpecialCharTok{::}\FunctionTok{filter}\NormalTok{(}
\NormalTok{    ID }\SpecialCharTok{\%\%} \DecValTok{100} \SpecialCharTok{\textgreater{}=} \DecValTok{7} \SpecialCharTok{\&}\NormalTok{ ID }\SpecialCharTok{\%\%} \DecValTok{100} \SpecialCharTok{\textless{}=} \DecValTok{25} \SpecialCharTok{|}
\NormalTok{      ID }\SpecialCharTok{\%\%} \DecValTok{100} \SpecialCharTok{\textgreater{}=} \DecValTok{32} \SpecialCharTok{\&}\NormalTok{ ID }\SpecialCharTok{\%\%} \DecValTok{100} \SpecialCharTok{\textless{}=} \DecValTok{50} \SpecialCharTok{|}
\NormalTok{      ID }\SpecialCharTok{\%\%} \DecValTok{100} \SpecialCharTok{\textgreater{}=} \DecValTok{57} \SpecialCharTok{\&}\NormalTok{ ID }\SpecialCharTok{\%\%} \DecValTok{100} \SpecialCharTok{\textless{}=} \DecValTok{75} \SpecialCharTok{|}
\NormalTok{      ID }\SpecialCharTok{\%\%} \DecValTok{100} \SpecialCharTok{\textgreater{}=} \DecValTok{82} \SpecialCharTok{\&}\NormalTok{ ID }\SpecialCharTok{\%\%} \DecValTok{100} \SpecialCharTok{\textless{}=} \DecValTok{100}
\NormalTok{  )}

\CommentTok{\# Dataset identifyer}
\NormalTok{AUC\_train\_compare }\OtherTok{\textless{}{-}}\NormalTok{ AUC\_train }\SpecialCharTok{\%\textgreater{}\%} \FunctionTok{mutate}\NormalTok{(}\AttributeTok{Dataset =} \StringTok{"Train"}\NormalTok{)}
\NormalTok{AUC\_test\_compare }\OtherTok{\textless{}{-}}\NormalTok{ AUC\_test }\SpecialCharTok{\%\textgreater{}\%} \FunctionTok{mutate}\NormalTok{(}\AttributeTok{Dataset =} \StringTok{"Test"}\NormalTok{)}

\CommentTok{\# Combine the three datasets}
\NormalTok{combined\_AUC }\OtherTok{\textless{}{-}} \FunctionTok{bind\_rows}\NormalTok{(AUC\_train\_compare, AUC\_test\_compare)}

\CommentTok{\# Compare the train and test sets}
\NormalTok{vars }\OtherTok{\textless{}{-}} \FunctionTok{c}\NormalTok{(}\StringTok{"CREAT"}\NormalTok{, }\StringTok{"WT"}\NormalTok{, }\StringTok{"AUC\_IND"}\NormalTok{, }\StringTok{"CL"}\NormalTok{, }\StringTok{"Vc"}\NormalTok{)}
\NormalTok{tableOne }\OtherTok{\textless{}{-}} \FunctionTok{CreateTableOne}\NormalTok{(}\AttributeTok{vars =}\NormalTok{ vars, }\AttributeTok{strata =} \StringTok{"Dataset"}\NormalTok{, }\AttributeTok{data =}\NormalTok{ combined\_AUC)}
\NormalTok{tableOne2 }\OtherTok{\textless{}{-}} \FunctionTok{print}\NormalTok{(tableOne, }\AttributeTok{nonnormal =} \FunctionTok{c}\NormalTok{(}\StringTok{"CREAT"}\NormalTok{, }\StringTok{"WT"}\NormalTok{, }\StringTok{"AUC\_IND"}\NormalTok{, }\StringTok{"CL"}\NormalTok{, }\StringTok{"Vc"}\NormalTok{), }\AttributeTok{printToggle=}\NormalTok{F, }\AttributeTok{minMax=}\NormalTok{T)}

\FunctionTok{kableone}\NormalTok{(tableOne2)}
\end{Highlighting}
\end{Shaded}

\begin{longtable}[]{@{}
  >{\raggedright\arraybackslash}p{(\linewidth - 8\tabcolsep) * \real{0.2475}}
  >{\raggedright\arraybackslash}p{(\linewidth - 8\tabcolsep) * \real{0.3069}}
  >{\raggedright\arraybackslash}p{(\linewidth - 8\tabcolsep) * \real{0.3069}}
  >{\raggedright\arraybackslash}p{(\linewidth - 8\tabcolsep) * \real{0.0594}}
  >{\raggedright\arraybackslash}p{(\linewidth - 8\tabcolsep) * \real{0.0792}}@{}}
\toprule\noalign{}
\begin{minipage}[b]{\linewidth}\raggedright
\end{minipage} & \begin{minipage}[b]{\linewidth}\raggedright
Test
\end{minipage} & \begin{minipage}[b]{\linewidth}\raggedright
Train
\end{minipage} & \begin{minipage}[b]{\linewidth}\raggedright
p
\end{minipage} & \begin{minipage}[b]{\linewidth}\raggedright
test
\end{minipage} \\
\midrule\noalign{}
\endhead
\bottomrule\noalign{}
\endlastfoot
n & 600 & 1875 & & \\
CREAT (median {[}range{]}) & 0.76 {[}0.24, 5.32{]} & 0.77 {[}0.13,
8.23{]} & 0.410 & nonnorm \\
WT (median {[}range{]}) & 80.03 {[}41.33, 139.60{]} & 79.99 {[}47.36,
141.17{]} & 0.884 & nonnorm \\
AUC\_IND (median {[}range{]}) & 37200.96 {[}5907.92, 1936666.60{]} &
37276.99 {[}6352.83, 1086156.54{]} & 0.410 & nonnorm \\
CL (median {[}range{]}) & 12.78 {[}0.68, 114.45{]} & 12.80 {[}1.08,
76.53{]} & 0.619 & nonnorm \\
Vc (median {[}range{]}) & 9.73 {[}3.07, 65.36{]} & 9.71 {[}3.06,
73.77{]} & 0.554 & nonnorm \\
\end{longtable}

\subsection{Maximal concentration
(Cmax)}\label{maximal-concentration-cmax}

Cmax is the true Cmax, not the highest observed concentration.

\begin{Shaded}
\begin{Highlighting}[]
\NormalTok{AMOX\_CMAX1 }\OtherTok{\textless{}{-}}\NormalTok{ all\_results }\SpecialCharTok{\%\textgreater{}\%}
  \FunctionTok{group\_by}\NormalTok{(ID, MODEL) }\SpecialCharTok{\%\textgreater{}\%} 
  \FunctionTok{summarize}\NormalTok{(}
    \AttributeTok{ID =} \FunctionTok{first}\NormalTok{(ID),}
    \AttributeTok{MODEL =} \FunctionTok{first}\NormalTok{(MODEL),}
    \AttributeTok{MODEL\_COHORT =} \FunctionTok{first}\NormalTok{(MODEL\_COHORT),}
    \AttributeTok{CMAX\_IND =} \FunctionTok{max}\NormalTok{(Cmax, }\AttributeTok{na.rm =} \ConstantTok{TRUE}\NormalTok{),  }\CommentTok{\# Cmax based on IPRED}
    \AttributeTok{WT =} \FunctionTok{first}\NormalTok{(WT), }
    \AttributeTok{CREAT =} \FunctionTok{first}\NormalTok{(CREAT), }
    \AttributeTok{BURN =} \FunctionTok{first}\NormalTok{(BURN),}
    \AttributeTok{ICU =} \FunctionTok{first}\NormalTok{(ICU),}
    \AttributeTok{OBESE =} \FunctionTok{first}\NormalTok{(OBESE),}
    \AttributeTok{AGE =} \FunctionTok{first}\NormalTok{(AGE),}
    \AttributeTok{SEX =} \FunctionTok{first}\NormalTok{(SEX),}
    \AttributeTok{HT =} \FunctionTok{first}\NormalTok{(HT),}
    \AttributeTok{DUR =} \FunctionTok{first}\NormalTok{(DUR),}
    \AttributeTok{FREQ =} \FunctionTok{first}\NormalTok{(FREQ),}
    \AttributeTok{DOSE\_ADM =} \FunctionTok{first}\NormalTok{(DOSE\_ADM),}
    \AttributeTok{TSS =} \FunctionTok{first}\NormalTok{(TSS),}
    \AttributeTok{Vc =} \FunctionTok{first}\NormalTok{(Vc),}
    \AttributeTok{CL =} \FunctionTok{first}\NormalTok{(CL)}
\NormalTok{  ) }\SpecialCharTok{\%\textgreater{}\%}
  \FunctionTok{mutate}\NormalTok{(}\AttributeTok{CON =} \FunctionTok{ifelse}\NormalTok{(DUR }\SpecialCharTok{==} \DecValTok{24}\NormalTok{, }\DecValTok{1}\NormalTok{, }\DecValTok{0}\NormalTok{))}

\NormalTok{summarized\_CMAX }\OtherTok{\textless{}{-}}\NormalTok{ AMOX\_CMAX1 }\SpecialCharTok{\%\textgreater{}\%}
  \FunctionTok{distinct}\NormalTok{(ID, MODEL, CMAX\_IND)}

\CommentTok{\# Box plot for Cmax}
\NormalTok{plot\_CMAX }\OtherTok{\textless{}{-}} \FunctionTok{ggplot}\NormalTok{(summarized\_CMAX, }\FunctionTok{aes}\NormalTok{(}\AttributeTok{x =}\NormalTok{ MODEL, }\AttributeTok{y =}\NormalTok{ CMAX\_IND, }\AttributeTok{fill =}\NormalTok{ MODEL)) }\SpecialCharTok{+}
  \FunctionTok{geom\_boxplot}\NormalTok{() }\SpecialCharTok{+}
  \FunctionTok{theme\_minimal}\NormalTok{() }\SpecialCharTok{+}
  \FunctionTok{labs}\NormalTok{(}\AttributeTok{title =} \StringTok{"Cmax distribution"}\NormalTok{, }\AttributeTok{x =} \StringTok{"Model"}\NormalTok{, }\AttributeTok{y =} \StringTok{"Cmax (mg/L)"}\NormalTok{) }\SpecialCharTok{+}
  \FunctionTok{theme}\NormalTok{(}
    \AttributeTok{axis.text.x =} \FunctionTok{element\_text}\NormalTok{(}\AttributeTok{angle =} \DecValTok{30}\NormalTok{, }\AttributeTok{hjust =} \DecValTok{1}\NormalTok{),}
    \AttributeTok{axis.text =} \FunctionTok{element\_text}\NormalTok{(}\AttributeTok{size =} \DecValTok{16}\NormalTok{),    }
    \AttributeTok{axis.title =} \FunctionTok{element\_text}\NormalTok{(}\AttributeTok{size =} \DecValTok{20}\NormalTok{), }
    \AttributeTok{plot.title =} \FunctionTok{element\_text}\NormalTok{(}\AttributeTok{size=}\DecValTok{24}\NormalTok{),}
\NormalTok{  )}

\NormalTok{plot\_CMAX}
\end{Highlighting}
\end{Shaded}

\pandocbounded{\includegraphics[keepaspectratio]{Simulations/concentrations_simulation_files/figure-pdf/CMAX-1.pdf}}

\begin{Shaded}
\begin{Highlighting}[]
\CommentTok{\# Create test data by taking 2400 subjects (24 from each dosing scheme and 6{-}6 for each model covariate cohorte per dosing scheme)}
\NormalTok{CMAX\_test }\OtherTok{\textless{}{-}}\NormalTok{ AMOX\_CMAX1 }\SpecialCharTok{\%\textgreater{}\%}
\NormalTok{  dplyr}\SpecialCharTok{::}\FunctionTok{filter}\NormalTok{(}
\NormalTok{    ID }\SpecialCharTok{\%\%} \DecValTok{100} \SpecialCharTok{\textgreater{}=} \DecValTok{1} \SpecialCharTok{\&}\NormalTok{ ID }\SpecialCharTok{\%\%} \DecValTok{100} \SpecialCharTok{\textless{}=} \DecValTok{6} \SpecialCharTok{|}
\NormalTok{      ID }\SpecialCharTok{\%\%} \DecValTok{100} \SpecialCharTok{\textgreater{}=} \DecValTok{26} \SpecialCharTok{\&}\NormalTok{ ID }\SpecialCharTok{\%\%} \DecValTok{100} \SpecialCharTok{\textless{}=} \DecValTok{31} \SpecialCharTok{|}
\NormalTok{      ID }\SpecialCharTok{\%\%} \DecValTok{100} \SpecialCharTok{\textgreater{}=} \DecValTok{51} \SpecialCharTok{\&}\NormalTok{ ID }\SpecialCharTok{\%\%} \DecValTok{100} \SpecialCharTok{\textless{}=} \DecValTok{56} \SpecialCharTok{|}
\NormalTok{      ID }\SpecialCharTok{\%\%} \DecValTok{100} \SpecialCharTok{\textgreater{}=} \DecValTok{76} \SpecialCharTok{\&}\NormalTok{ ID }\SpecialCharTok{\%\%} \DecValTok{100} \SpecialCharTok{\textless{}=} \DecValTok{81}
\NormalTok{  )}

\CommentTok{\# Create training data by taking 7600 subjects (76 from each dosing scheme and 19{-}19 for each model covariate cohorte per dosing scheme)}
\NormalTok{CMAX\_train }\OtherTok{\textless{}{-}}\NormalTok{ AMOX\_CMAX1 }\SpecialCharTok{\%\textgreater{}\%}
\NormalTok{  dplyr}\SpecialCharTok{::}\FunctionTok{filter}\NormalTok{(}
\NormalTok{    ID }\SpecialCharTok{\%\%} \DecValTok{100} \SpecialCharTok{\textgreater{}=} \DecValTok{7} \SpecialCharTok{\&}\NormalTok{ ID }\SpecialCharTok{\%\%} \DecValTok{100} \SpecialCharTok{\textless{}=} \DecValTok{25} \SpecialCharTok{|}
\NormalTok{      ID }\SpecialCharTok{\%\%} \DecValTok{100} \SpecialCharTok{\textgreater{}=} \DecValTok{32} \SpecialCharTok{\&}\NormalTok{ ID }\SpecialCharTok{\%\%} \DecValTok{100} \SpecialCharTok{\textless{}=} \DecValTok{50} \SpecialCharTok{|}
\NormalTok{      ID }\SpecialCharTok{\%\%} \DecValTok{100} \SpecialCharTok{\textgreater{}=} \DecValTok{57} \SpecialCharTok{\&}\NormalTok{ ID }\SpecialCharTok{\%\%} \DecValTok{100} \SpecialCharTok{\textless{}=} \DecValTok{75} \SpecialCharTok{|}
\NormalTok{      ID }\SpecialCharTok{\%\%} \DecValTok{100} \SpecialCharTok{\textgreater{}=} \DecValTok{82} \SpecialCharTok{\&}\NormalTok{ ID }\SpecialCharTok{\%\%} \DecValTok{100} \SpecialCharTok{\textless{}=} \DecValTok{100}
\NormalTok{  )}

\CommentTok{\# Dataset identifier}
\NormalTok{CMAX\_train\_compare }\OtherTok{\textless{}{-}}\NormalTok{ CMAX\_train }\SpecialCharTok{\%\textgreater{}\%} \FunctionTok{mutate}\NormalTok{(}\AttributeTok{Dataset =} \StringTok{"Train"}\NormalTok{)}
\NormalTok{CMAX\_test\_compare }\OtherTok{\textless{}{-}}\NormalTok{ CMAX\_test }\SpecialCharTok{\%\textgreater{}\%} \FunctionTok{mutate}\NormalTok{(}\AttributeTok{Dataset =} \StringTok{"Test"}\NormalTok{)}

\CommentTok{\# Combine the three datasets}
\NormalTok{combined\_CMAX }\OtherTok{\textless{}{-}} \FunctionTok{bind\_rows}\NormalTok{(CMAX\_train\_compare, CMAX\_test\_compare)}

\CommentTok{\# Compare the train and test sets}
\NormalTok{vars }\OtherTok{\textless{}{-}} \FunctionTok{c}\NormalTok{(}\StringTok{"CREAT"}\NormalTok{, }\StringTok{"WT"}\NormalTok{, }\StringTok{"CMAX\_IND"}\NormalTok{, }\StringTok{"CL"}\NormalTok{, }\StringTok{"Vc"}\NormalTok{)}
\NormalTok{tableOne3 }\OtherTok{\textless{}{-}} \FunctionTok{CreateTableOne}\NormalTok{(}\AttributeTok{vars =}\NormalTok{ vars, }\AttributeTok{strata =} \StringTok{"Dataset"}\NormalTok{, }\AttributeTok{data =}\NormalTok{ combined\_CMAX)}
\NormalTok{tableOne4}\OtherTok{\textless{}{-}}\FunctionTok{print}\NormalTok{(tableOne3, }\AttributeTok{nonnormal =} \FunctionTok{c}\NormalTok{(}\StringTok{"CREAT"}\NormalTok{, }\StringTok{"WT"}\NormalTok{, }\StringTok{"CMAX\_IND"}\NormalTok{, }\StringTok{"CL"}\NormalTok{, }\StringTok{"Vc"}\NormalTok{), }\AttributeTok{printToggle=}\NormalTok{F, }\AttributeTok{minMax=}\NormalTok{T)}

\FunctionTok{kableone}\NormalTok{(tableOne4)}
\end{Highlighting}
\end{Shaded}

\begin{longtable}[]{@{}
  >{\raggedright\arraybackslash}p{(\linewidth - 8\tabcolsep) * \real{0.3095}}
  >{\raggedright\arraybackslash}p{(\linewidth - 8\tabcolsep) * \real{0.2619}}
  >{\raggedright\arraybackslash}p{(\linewidth - 8\tabcolsep) * \real{0.2619}}
  >{\raggedright\arraybackslash}p{(\linewidth - 8\tabcolsep) * \real{0.0714}}
  >{\raggedright\arraybackslash}p{(\linewidth - 8\tabcolsep) * \real{0.0952}}@{}}
\toprule\noalign{}
\begin{minipage}[b]{\linewidth}\raggedright
\end{minipage} & \begin{minipage}[b]{\linewidth}\raggedright
Test
\end{minipage} & \begin{minipage}[b]{\linewidth}\raggedright
Train
\end{minipage} & \begin{minipage}[b]{\linewidth}\raggedright
p
\end{minipage} & \begin{minipage}[b]{\linewidth}\raggedright
test
\end{minipage} \\
\midrule\noalign{}
\endhead
\bottomrule\noalign{}
\endlastfoot
n & 600 & 1875 & & \\
CREAT (median {[}range{]}) & 0.76 {[}0.24, 5.32{]} & 0.77 {[}0.13,
8.23{]} & 0.410 & nonnorm \\
WT (median {[}range{]}) & 80.03 {[}41.33, 139.60{]} & 79.99 {[}47.36,
141.17{]} & 0.884 & nonnorm \\
CMAX\_IND (median {[}range{]}) & 56.32 {[}15.30, 289.10{]} & 56.43
{[}8.85, 220.68{]} & 0.720 & nonnorm \\
CL (median {[}range{]}) & 12.78 {[}0.68, 114.45{]} & 12.80 {[}1.08,
76.53{]} & 0.619 & nonnorm \\
Vc (median {[}range{]}) & 9.73 {[}3.07, 65.36{]} & 9.71 {[}3.06,
73.77{]} & 0.554 & nonnorm \\
\end{longtable}

\chapter*{References}\label{references}
\addcontentsline{toc}{chapter}{References}

\markboth{References}{References}

\phantomsection\label{refs}
\begin{CSLReferences}{1}{0}
\bibitem[\citeproctext]{ref-Agema2024-cf}
Agema, Bram C, Tolra Kocher, Ayşenur B Öztürk, Eline L Giraud, Nielka P
van Erp, Brenda C M de Winter, Ron H J Mathijssen, Stijn L W Koolen,
Birgit C P Koch, and Sebastiaan D T Sassen. 2024. {``Selecting the Best
Pharmacokinetic Models for a Priori Model-Informed Precision Dosing with
Model Ensembling.''} \emph{Clin. Pharmacokinet.} 63 (10): 1449--61.

\bibitem[\citeproctext]{ref-Carlier2013-bq}
Carlier, Mieke, Michaël Noë, Jan J De Waele, Veronique Stove, Alain G
Verstraete, Jeffrey Lipman, and Jason A Roberts. 2013. {``Population
Pharmacokinetics and Dosing Simulations of Amoxicillin/Clavulanic Acid
in Critically Ill Patients.''} \emph{J. Antimicrob. Chemother.} 68 (11):
2600--2608.

\bibitem[\citeproctext]{ref-fournier2018}
Fournier, Anne, Sylvain Goutelle, Yok-Ai Que, Philippe Eggimann, Olivier
Pantet, Farshid Sadeghipour, Pierre Voirol, and Chantal Csajka. 2018.
{``Population Pharmacokinetic Study of Amoxicillin-Treated Burn Patients
Hospitalized at a Swiss Tertiary-Care Center.''} \emph{Antimicrobial
Agents and Chemotherapy} 62 (9): e00505--18.
\url{https://doi.org/10.1128/AAC.00505-18}.

\bibitem[\citeproctext]{ref-Guilhaumou2019-eh}
Guilhaumou, Romain, Sihem Benaboud, Youssef Bennis, Claire
Dahyot-Fizelier, Eric Dailly, Peggy Gandia, Sylvain Goutelle, et al.
2019. {``Optimization of the Treatment with Beta-Lactam Antibiotics in
Critically Ill Patients-Guidelines from the French Society of
Pharmacology and Therapeutics (Soci{é}t{é} Fran{ç}aise de Pharmacologie
Et {Th{é}rapeutique-SFPT}) and the French Society of Anaesthesia and
Intensive Care Medicine (Soci{é}t{é} Fran{ç}aise d'anesth{é}sie Et
{R{é}animation-SFAR}).''} \emph{Crit. Care} 23 (1): 104.

\bibitem[\citeproctext]{ref-Johnson2023-zo}
Johnson, Alistair E W, Lucas Bulgarelli, Lu Shen, Alvin Gayles, Ayad
Shammout, Steven Horng, Tom J Pollard, et al. 2023. {``{MIMIC-IV}, a
Freely Accessible Electronic Health Record Dataset.''} \emph{Sci. Data}
10 (1): 1.

\bibitem[\citeproctext]{ref-mellon2020}
Mellon, G, K Hammas, C Burdet, X Duval, C Carette, N El-Helali, L
Massias, F Mentré, S Czernichow, and A -C Crémieux. 2020. {``Population
Pharmacokinetics and Dosing Simulations of Amoxicillin in Obese Adults
Receiving Co-Amoxiclav.''} \emph{Journal of Antimicrobial Chemotherapy}
75 (12): 3611--18. \url{https://doi.org/10.1093/jac/dkaa368}.

\bibitem[\citeproctext]{ref-rambaud2020}
Rambaud, Antoine, Benjamin Jean Gaborit, Colin Deschanvres, Paul Le
Turnier, Raphaël Lecomte, Nathalie Asseray-Madani, Anne-Gaëlle Leroy, et
al. 2020. {``Development and Validation of a Dosing Nomogram for
Amoxicillin in Infective Endocarditis.''} \emph{The Journal of
Antimicrobial Chemotherapy} 75 (10): 2941--50.
\url{https://doi.org/10.1093/jac/dkaa232}.

\end{CSLReferences}

\part{MIPD}

\chapter{Objectives}\label{objectives}

To compare the capacity of different MIPD approaches to predict (first)
dose to obtain target concentrations of 40-80 mg.\(L^{-1}\) based on the
patient's covariates.

\chapter{Judgement criteria}\label{judgement-criteria}

The proportion of subjects whose predicted dose allows for obtaining
concentrations in the target interval of 40-80 mg.\(L^{-1}\) (target
interval is based on the guidelines of SFPT - the French Society of
Pharmacology and Therapeutics Guilhaumou et al. (2019)). For patients,
whose extrapolated dose to reach 40 mg/L was higher than 20 g, the
ability to predict the fact that the dose is above 20 g was evaluated.

\chapter{Context}\label{context}

Numerous MIPD approaches have already been developed to predict \emph{a
posteriori} concentrations. Application of the \emph{a priori}
approaches developed in this project could help certain patients to
reach target concentrations interval faster this way improving clinical
results and to reduce the number of necessary blood samples.

\chapter{Precision dosing approaches}\label{precision-dosing-approaches}

\begin{Shaded}
\begin{Highlighting}[]
\FunctionTok{library}\NormalTok{(here)}
\NormalTok{here}\SpecialCharTok{::}\FunctionTok{i\_am}\NormalTok{(}\StringTok{"Amoxicillin/a\_priori/For\_publication/MIPD/Precision\_dosing\_methods.qmd"}\NormalTok{)}
\CommentTok{\# Load openMIPD package from tar.gz file}
\FunctionTok{install.packages}\NormalTok{(}\StringTok{"stp{-}brioche{-}R{-}package/openMIPD\_0.0.6.tar.gz"}\NormalTok{, }\AttributeTok{repos =} \ConstantTok{NULL}\NormalTok{, }\AttributeTok{type =} \StringTok{"source"}\NormalTok{)}

\FunctionTok{library}\NormalTok{(openMIPD)}
\FunctionTok{library}\NormalTok{(dplyr)}
\FunctionTok{library}\NormalTok{(ggplot2)}
\FunctionTok{library}\NormalTok{(readr)}
\FunctionTok{library}\NormalTok{(tidyr)}
\FunctionTok{library}\NormalTok{(rpart.plot)}
\FunctionTok{library}\NormalTok{(ggplot2)}
\FunctionTok{library}\NormalTok{(patchwork)}
\FunctionTok{library}\NormalTok{(mrgsolve)}
\FunctionTok{library}\NormalTok{(kernlab)}

\FunctionTok{set.seed}\NormalTok{(}\DecValTok{1991}\NormalTok{)}
\end{Highlighting}
\end{Shaded}

\chapter{Generate predictions with each
model}\label{generate-predictions-with-each-model}

\begin{Shaded}
\begin{Highlighting}[]
\NormalTok{AMOX\_CMIN\_OBS\_TRAIN }\OtherTok{\textless{}{-}} \FunctionTok{read.csv}\NormalTok{(}\FunctionTok{here}\NormalTok{(}\StringTok{"Amoxicillin/a\_priori/For\_publication/Data/AMOX\_CMIN\_OBS\_TRAIN.csv"}\NormalTok{), }\AttributeTok{quote =} \StringTok{""}\NormalTok{) }\SpecialCharTok{|\textgreater{}} 
  \FunctionTok{mutate}\NormalTok{(}
    \AttributeTok{SEX\_CAT =} \FunctionTok{case\_when}\NormalTok{(}
\NormalTok{      SEX }\SpecialCharTok{==} \DecValTok{0} \SpecialCharTok{\textasciitilde{}} \StringTok{"M"}\NormalTok{,}
\NormalTok{      SEX }\SpecialCharTok{==} \DecValTok{1} \SpecialCharTok{\textasciitilde{}} \StringTok{"F"}
\NormalTok{    )}
\NormalTok{  ) }\SpecialCharTok{|\textgreater{}}
  \FunctionTok{rowwise}\NormalTok{() }\SpecialCharTok{|\textgreater{}} 
  \FunctionTok{mutate}\NormalTok{(}\AttributeTok{CRCL\_CKD\_EPI\_BSA\_NORM =} \FunctionTok{estimate\_GFR}\NormalTok{(AGE,CREAT,SEX\_CAT)) }\SpecialCharTok{|\textgreater{}} 
  \FunctionTok{ungroup}\NormalTok{() }\SpecialCharTok{|\textgreater{}} 
  \FunctionTok{mutate}\NormalTok{(}\AttributeTok{CRCL\_CKD\_EPI\_ABSOLUTE =}\NormalTok{ CRCL\_CKD\_EPI\_BSA\_NORM }\SpecialCharTok{*}\NormalTok{ (BSA}\SpecialCharTok{/}\FloatTok{1.73}\NormalTok{))}

\NormalTok{AMOX\_CMIN\_OBS\_TEST }\OtherTok{\textless{}{-}} \FunctionTok{read.csv}\NormalTok{(}\FunctionTok{here}\NormalTok{(}\StringTok{"Amoxicillin/a\_priori/For\_publication/Data/AMOX\_CMIN\_OBS\_TEST.csv"}\NormalTok{), }\AttributeTok{quote =} \StringTok{""}\NormalTok{) }\SpecialCharTok{|\textgreater{}} 
  \FunctionTok{mutate}\NormalTok{(}
    \AttributeTok{SEX\_CAT =} \FunctionTok{case\_when}\NormalTok{(}
\NormalTok{      SEX }\SpecialCharTok{==} \DecValTok{0} \SpecialCharTok{\textasciitilde{}} \StringTok{"M"}\NormalTok{,}
\NormalTok{      SEX }\SpecialCharTok{==} \DecValTok{1} \SpecialCharTok{\textasciitilde{}} \StringTok{"F"}
\NormalTok{    )}
\NormalTok{  ) }\SpecialCharTok{|\textgreater{}}
  \FunctionTok{rowwise}\NormalTok{() }\SpecialCharTok{|\textgreater{}} 
  \FunctionTok{mutate}\NormalTok{(}\AttributeTok{CRCL\_CKD\_EPI\_BSA\_NORM =} \FunctionTok{estimate\_GFR}\NormalTok{(AGE,CREAT,SEX\_CAT)) }\SpecialCharTok{|\textgreater{}} 
  \FunctionTok{ungroup}\NormalTok{() }\SpecialCharTok{|\textgreater{}} 
  \FunctionTok{mutate}\NormalTok{(}\AttributeTok{CRCL\_CKD\_EPI\_ABSOLUTE =}\NormalTok{ CRCL\_CKD\_EPI\_BSA\_NORM }\SpecialCharTok{*}\NormalTok{ (BSA}\SpecialCharTok{/}\FloatTok{1.73}\NormalTok{))}
\end{Highlighting}
\end{Shaded}

\begin{Shaded}
\begin{Highlighting}[]
\NormalTok{perform\_mrgsolve\_prediction }\OtherTok{\textless{}{-}} \ControlFlowTok{function}\NormalTok{(input\_data,}
\NormalTok{                                        mrgsolve\_model\_path,}
\NormalTok{                                        model\_name,}
\NormalTok{                                        cmin\_target) \{}
  
\NormalTok{  full\_data }\OtherTok{\textless{}{-}} \FunctionTok{select}\NormalTok{(input\_data,}\SpecialCharTok{{-}}\NormalTok{MODEL) }\CommentTok{\#MODEL column identifies the model with which the prediction was made thus we have to drop it to be able to join later on}

\NormalTok{data\_for\_mrgsolve }\OtherTok{\textless{}{-}}\NormalTok{ input\_data }\SpecialCharTok{|\textgreater{}}
  \FunctionTok{select}\NormalTok{(}
\NormalTok{    ID,}
\NormalTok{    TIME,}
\NormalTok{    CREAT,}
\NormalTok{    BURN,}
\NormalTok{    ICU,}
\NormalTok{    OBESE,}
\NormalTok{    AGE,}
\NormalTok{    SEX,}
\NormalTok{    HT,}
\NormalTok{    BSA,}
\NormalTok{    DOSE\_ADM,}
\NormalTok{    FREQ,}
\NormalTok{    DUR,}
\NormalTok{    CMIN\_IND,}
\NormalTok{    MODEL\_COHORT}
\NormalTok{  ) }\SpecialCharTok{|\textgreater{}} \FunctionTok{distinct}\NormalTok{()}

\NormalTok{mrgsolve\_model }\OtherTok{\textless{}{-}} \FunctionTok{mread\_cache}\NormalTok{(mrgsolve\_model\_path)}

\NormalTok{mrgsolve\_dose\_data }\OtherTok{\textless{}{-}} 
      \FunctionTok{rename}\NormalTok{(data\_for\_mrgsolve,}
           \AttributeTok{AMT =}\NormalTok{ DOSE\_ADM,}
           \AttributeTok{II =}\NormalTok{ FREQ}
\NormalTok{           ) }\SpecialCharTok{|\textgreater{}} 
  \FunctionTok{mutate}\NormalTok{(}\AttributeTok{RATE =}\NormalTok{ AMT}\SpecialCharTok{/}\NormalTok{DUR,}
         \AttributeTok{ADDL =}\NormalTok{ TIME}\SpecialCharTok{/}\NormalTok{II }\SpecialCharTok{{-}} \DecValTok{1}\NormalTok{,}
         \AttributeTok{TIME =} \DecValTok{0}\NormalTok{,}
         \AttributeTok{EVID =} \DecValTok{1}\NormalTok{,}
         \AttributeTok{CMT =} \DecValTok{1}\NormalTok{)}

\NormalTok{mrgsolve\_obs\_data }\OtherTok{\textless{}{-}} \FunctionTok{rename}\NormalTok{(data\_for\_mrgsolve,}
           \AttributeTok{AMT =}\NormalTok{ DOSE\_ADM,}
           \AttributeTok{II =}\NormalTok{ FREQ}
\NormalTok{           ) }\SpecialCharTok{|\textgreater{}} 
  \FunctionTok{mutate}\NormalTok{(}\AttributeTok{RATE =} \ConstantTok{NA}\NormalTok{,}
         \AttributeTok{ADDL =} \ConstantTok{NA}\NormalTok{,}
         \AttributeTok{EVID =} \DecValTok{0}\NormalTok{,}
         \AttributeTok{CMT =} \DecValTok{1}\NormalTok{)}
  
\NormalTok{mrgsolve\_input\_data }\OtherTok{\textless{}{-}} \FunctionTok{bind\_rows}\NormalTok{(mrgsolve\_dose\_data,}
\NormalTok{                                        mrgsolve\_obs\_data) }\SpecialCharTok{|\textgreater{}} 
  \FunctionTok{arrange}\NormalTok{(ID,}\SpecialCharTok{{-}}\NormalTok{EVID)}

\NormalTok{mrgsolve\_model }\SpecialCharTok{|\textgreater{}} 
  \FunctionTok{data\_set}\NormalTok{(mrgsolve\_input\_data) }\SpecialCharTok{|\textgreater{}} 
  \FunctionTok{zero\_re}\NormalTok{() }\SpecialCharTok{|\textgreater{}}  
  \FunctionTok{mrgsim}\NormalTok{(}\AttributeTok{obsonly =} \ConstantTok{TRUE}\NormalTok{,}
         \AttributeTok{recover =} \FunctionTok{colnames}\NormalTok{(mrgsolve\_input\_data)) }\SpecialCharTok{|\textgreater{}}
  \FunctionTok{as\_tibble}\NormalTok{() }\SpecialCharTok{|\textgreater{}} 
  \FunctionTok{rename}\NormalTok{(}\AttributeTok{CMIN\_PRED =}\NormalTok{ IPRED) }\SpecialCharTok{|\textgreater{}} 
  \FunctionTok{mutate}\NormalTok{(}\AttributeTok{MODEL =}\NormalTok{ model\_name) }\SpecialCharTok{|\textgreater{}} 
  \FunctionTok{select}\NormalTok{(ID,MODEL,CMIN\_PRED) }\SpecialCharTok{|\textgreater{}} 
  \FunctionTok{right\_join}\NormalTok{(full\_data) }\SpecialCharTok{|\textgreater{}} 
  \FunctionTok{mutate}\NormalTok{(}\AttributeTok{DOSE\_PRED =}\NormalTok{ cmin\_target}\SpecialCharTok{/}\NormalTok{CMIN\_PRED }\SpecialCharTok{*}\NormalTok{ DOSE\_ADM)}
  
\NormalTok{\}}
\end{Highlighting}
\end{Shaded}

\section{CARLIER}\label{carlier}

\begin{Shaded}
\begin{Highlighting}[]
\NormalTok{AMOX\_CMIN\_PRED\_TRAIN\_CARLIER }\OtherTok{\textless{}{-}} \FunctionTok{perform\_mrgsolve\_prediction}\NormalTok{(}
  \AttributeTok{input\_data =}\NormalTok{ AMOX\_CMIN\_OBS\_TRAIN ,}
  \AttributeTok{mrgsolve\_model\_path =} \FunctionTok{here}\NormalTok{(}
    \StringTok{"Amoxicillin/a\_priori/For\_publication/Simulations/amox\_Carlier"}
\NormalTok{  ),}
  \AttributeTok{model\_name =} \StringTok{"CARLIER"}\NormalTok{,}
  \AttributeTok{cmin\_target =} \DecValTok{60} \CommentTok{\#mg/L}
\NormalTok{) }

\NormalTok{AMOX\_CMIN\_PRED\_TEST\_CARLIER }\OtherTok{\textless{}{-}} \FunctionTok{perform\_mrgsolve\_prediction}\NormalTok{(}
  \AttributeTok{input\_data =}\NormalTok{ AMOX\_CMIN\_OBS\_TEST ,}
  \AttributeTok{mrgsolve\_model\_path =} \FunctionTok{here}\NormalTok{(}
    \StringTok{"Amoxicillin/a\_priori/For\_publication/Simulations/amox\_Carlier"}
\NormalTok{  ),}
  \AttributeTok{model\_name =} \StringTok{"CARLIER"}\NormalTok{,}
  \AttributeTok{cmin\_target =} \DecValTok{60} \CommentTok{\#mg/L}
\NormalTok{) }
\end{Highlighting}
\end{Shaded}

\section{FOURNIER}\label{fournier}

\begin{Shaded}
\begin{Highlighting}[]
\NormalTok{AMOX\_CMIN\_PRED\_TRAIN\_FOURNIER }\OtherTok{\textless{}{-}} \FunctionTok{perform\_mrgsolve\_prediction}\NormalTok{(}
  \AttributeTok{input\_data =}\NormalTok{ AMOX\_CMIN\_OBS\_TRAIN ,}
  \AttributeTok{mrgsolve\_model\_path =} \FunctionTok{here}\NormalTok{(}
    \StringTok{"Amoxicillin/a\_priori/For\_publication/Simulations/amox\_Fournier"}
\NormalTok{  ),}
  \AttributeTok{model\_name =} \StringTok{"FOURNIER"}\NormalTok{,}
  \AttributeTok{cmin\_target =} \DecValTok{60} \CommentTok{\#mg/L}
\NormalTok{) }

\NormalTok{AMOX\_CMIN\_PRED\_TEST\_FOURNIER }\OtherTok{\textless{}{-}} \FunctionTok{perform\_mrgsolve\_prediction}\NormalTok{(}
  \AttributeTok{input\_data =}\NormalTok{ AMOX\_CMIN\_OBS\_TEST ,}
  \AttributeTok{mrgsolve\_model\_path =} \FunctionTok{here}\NormalTok{(}
    \StringTok{"Amoxicillin/a\_priori/For\_publication/Simulations/amox\_Fournier"}
\NormalTok{  ),}
  \AttributeTok{model\_name =} \StringTok{"FOURNIER"}\NormalTok{,}
  \AttributeTok{cmin\_target =} \DecValTok{60} \CommentTok{\#mg/L}
\NormalTok{) }
\end{Highlighting}
\end{Shaded}

\section{MELLON}\label{mellon}

\begin{Shaded}
\begin{Highlighting}[]
\NormalTok{AMOX\_CMIN\_PRED\_TRAIN\_MELLON }\OtherTok{\textless{}{-}} \FunctionTok{perform\_mrgsolve\_prediction}\NormalTok{(}
  \AttributeTok{input\_data =}\NormalTok{ AMOX\_CMIN\_OBS\_TRAIN ,}
  \AttributeTok{mrgsolve\_model\_path =} \FunctionTok{here}\NormalTok{(}
    \StringTok{"Amoxicillin/a\_priori/For\_publication/Simulations/amox\_Mellon"}
\NormalTok{  ),}
  \AttributeTok{model\_name =} \StringTok{"MELLON"}\NormalTok{,}
  \AttributeTok{cmin\_target =} \DecValTok{60} \CommentTok{\#mg/L}
\NormalTok{) }

\NormalTok{AMOX\_CMIN\_PRED\_TEST\_MELLON }\OtherTok{\textless{}{-}} \FunctionTok{perform\_mrgsolve\_prediction}\NormalTok{(}
  \AttributeTok{input\_data =}\NormalTok{ AMOX\_CMIN\_OBS\_TEST ,}
  \AttributeTok{mrgsolve\_model\_path =} \FunctionTok{here}\NormalTok{(}
    \StringTok{"Amoxicillin/a\_priori/For\_publication/Simulations/amox\_Mellon"}
\NormalTok{  ),}
  \AttributeTok{model\_name =} \StringTok{"MELLON"}\NormalTok{,}
  \AttributeTok{cmin\_target =} \DecValTok{60} \CommentTok{\#mg/L}
\NormalTok{) }
\end{Highlighting}
\end{Shaded}

\section{RAMBAUD}\label{rambaud}

\begin{Shaded}
\begin{Highlighting}[]
\NormalTok{AMOX\_CMIN\_PRED\_TRAIN\_RAMBAUD }\OtherTok{\textless{}{-}} \FunctionTok{perform\_mrgsolve\_prediction}\NormalTok{(}
  \AttributeTok{input\_data =}\NormalTok{ AMOX\_CMIN\_OBS\_TRAIN ,}
  \AttributeTok{mrgsolve\_model\_path =} \FunctionTok{here}\NormalTok{(}
    \StringTok{"Amoxicillin/a\_priori/For\_publication/Simulations/amox\_Rambaud"}
\NormalTok{  ),}
  \AttributeTok{model\_name =} \StringTok{"RAMBAUD"}\NormalTok{,}
  \AttributeTok{cmin\_target =} \DecValTok{60} \CommentTok{\#mg/L}
\NormalTok{) }

\NormalTok{AMOX\_CMIN\_PRED\_TEST\_RAMBAUD }\OtherTok{\textless{}{-}} \FunctionTok{perform\_mrgsolve\_prediction}\NormalTok{(}
  \AttributeTok{input\_data =}\NormalTok{ AMOX\_CMIN\_OBS\_TEST ,}
  \AttributeTok{mrgsolve\_model\_path =} \FunctionTok{here}\NormalTok{(}
    \StringTok{"Amoxicillin/a\_priori/For\_publication/Simulations/amox\_Rambaud"}
\NormalTok{  ),}
  \AttributeTok{model\_name =} \StringTok{"RAMBAUD"}\NormalTok{,}
  \AttributeTok{cmin\_target =} \DecValTok{60} \CommentTok{\#mg/L}
\NormalTok{) }
\end{Highlighting}
\end{Shaded}

\section{Pool all predictions and save
them}\label{pool-all-predictions-and-save-them}

\begin{Shaded}
\begin{Highlighting}[]
\NormalTok{AMOX\_CMIN\_TRAIN }\OtherTok{\textless{}{-}} \FunctionTok{bind\_rows}\NormalTok{(}
\NormalTok{  AMOX\_CMIN\_PRED\_TRAIN\_CARLIER,}
\NormalTok{  AMOX\_CMIN\_PRED\_TRAIN\_FOURNIER,}
\NormalTok{  AMOX\_CMIN\_PRED\_TRAIN\_MELLON,}
\NormalTok{  AMOX\_CMIN\_PRED\_TRAIN\_RAMBAUD}
\NormalTok{  ) }\SpecialCharTok{|\textgreater{}} 
  \FunctionTok{arrange}\NormalTok{(ID,MODEL) }\SpecialCharTok{|\textgreater{}} 
  \FunctionTok{mutate}\NormalTok{(}\AttributeTok{REFERENCE =} \FunctionTok{if\_else}\NormalTok{(MODEL }\SpecialCharTok{==}\NormalTok{ MODEL\_COHORT, }\DecValTok{1}\NormalTok{, }\DecValTok{0}\NormalTok{)) }\CommentTok{\# Add a variable called REFERENCE which is 1 for the the line where the model used for simulation is the same as the model cohort (otherwise 0) and take only observed values}

\FunctionTok{write.csv}\NormalTok{(AMOX\_CMIN\_TRAIN, }\FunctionTok{here}\NormalTok{(}\StringTok{"Amoxicillin/a\_priori/For\_publication/Data/AMOX\_CMIN\_TRAIN.csv"}\NormalTok{), }\AttributeTok{row.names =} \ConstantTok{FALSE}\NormalTok{, }\AttributeTok{quote =} \ConstantTok{FALSE}\NormalTok{)}

\NormalTok{AMOX\_CMIN\_TEST }\OtherTok{\textless{}{-}} \FunctionTok{bind\_rows}\NormalTok{(}
\NormalTok{  AMOX\_CMIN\_PRED\_TEST\_CARLIER,}
\NormalTok{  AMOX\_CMIN\_PRED\_TEST\_FOURNIER,}
\NormalTok{  AMOX\_CMIN\_PRED\_TEST\_MELLON,}
\NormalTok{  AMOX\_CMIN\_PRED\_TEST\_RAMBAUD}
\NormalTok{  ) }\SpecialCharTok{|\textgreater{}} 
  \FunctionTok{arrange}\NormalTok{(ID,MODEL) }\SpecialCharTok{|\textgreater{}} 
  \FunctionTok{mutate}\NormalTok{(}\AttributeTok{REFERENCE =} \FunctionTok{if\_else}\NormalTok{(MODEL }\SpecialCharTok{==}\NormalTok{ MODEL\_COHORT, }\DecValTok{1}\NormalTok{, }\DecValTok{0}\NormalTok{)) }\CommentTok{\# Add a variable called REFERENCE which is 1 for the the line where the model used for simulation is the same as the model cohort (otherwise 0) and take only observed values}

\FunctionTok{write.csv}\NormalTok{(AMOX\_CMIN\_TEST, }\FunctionTok{here}\NormalTok{(}\StringTok{"Amoxicillin/a\_priori/For\_publication/Data/AMOX\_CMIN\_TEST.csv"}\NormalTok{), }\AttributeTok{row.names =} \ConstantTok{FALSE}\NormalTok{, }\AttributeTok{quote =} \ConstantTok{FALSE}\NormalTok{)}
\end{Highlighting}
\end{Shaded}

\section{Global inspection of model
predictions}\label{global-inspection-of-model-predictions}

\subsection{Within-cohort performance (inter-individual
variability)}\label{within-cohort-performance-inter-individual-variability}

Visualization of the inter-individual variability of models by dividing
PRED (predicted) values by IPRED (observed) both generated by the same
model.

\begin{Shaded}
\begin{Highlighting}[]
\CommentTok{\# CMIN}
\CommentTok{\# Divide each PRED by the IPRED reference }
\NormalTok{AMOX\_CMIN2 }\OtherTok{\textless{}{-}} \FunctionTok{bind\_rows}\NormalTok{(AMOX\_CMIN\_TRAIN,AMOX\_CMIN\_TEST) }\SpecialCharTok{\%\textgreater{}\%}
  \FunctionTok{filter}\NormalTok{(REFERENCE }\SpecialCharTok{==} \DecValTok{1}\NormalTok{) }\SpecialCharTok{|\textgreater{}} 
  \FunctionTok{group\_by}\NormalTok{(ID, MODEL) }\SpecialCharTok{\%\textgreater{}\%}
  \FunctionTok{mutate}\NormalTok{(}
    \AttributeTok{ratio =}\NormalTok{ (CMIN\_PRED }\SpecialCharTok{/}\NormalTok{ CMIN\_IND) }\SpecialCharTok{*} \DecValTok{100}
\NormalTok{  ) }\SpecialCharTok{\%\textgreater{}\%}
  \FunctionTok{ungroup}\NormalTok{()}

\CommentTok{\# Boxplot of ratios stratified by model}
\NormalTok{stratified\_ratios\_CMIN }\OtherTok{\textless{}{-}} \FunctionTok{ggplot}\NormalTok{(AMOX\_CMIN2, }\FunctionTok{aes}\NormalTok{(}\AttributeTok{x =}\NormalTok{ MODEL, }\AttributeTok{y =}\NormalTok{ ratio, }\AttributeTok{fill =}\NormalTok{ MODEL)) }\SpecialCharTok{+}
  \FunctionTok{geom\_boxplot}\NormalTok{() }\SpecialCharTok{+}
  \FunctionTok{geom\_hline}\NormalTok{(}\AttributeTok{yintercept =} \DecValTok{80}\NormalTok{, }\AttributeTok{linetype =} \StringTok{"dashed"}\NormalTok{, }\AttributeTok{color =} \StringTok{"red"}\NormalTok{) }\SpecialCharTok{+} 
  \FunctionTok{geom\_hline}\NormalTok{(}\AttributeTok{yintercept =} \DecValTok{125}\NormalTok{, }\AttributeTok{linetype =} \StringTok{"dashed"}\NormalTok{, }\AttributeTok{color =} \StringTok{"red"}\NormalTok{) }\SpecialCharTok{+}
  \FunctionTok{labs}\NormalTok{(}
    \AttributeTok{title =} \StringTok{"Predicted/observed ratios for the same model (Cmax)"}\NormalTok{,}
    \AttributeTok{x =} \StringTok{"MODEL"}\NormalTok{,}
    \AttributeTok{y =} \StringTok{"Ratio (\%)"}
\NormalTok{  ) }\SpecialCharTok{+}
  \FunctionTok{theme\_minimal}\NormalTok{() }\SpecialCharTok{+}
  \FunctionTok{theme}\NormalTok{(}\AttributeTok{legend.position =} \StringTok{"none"}\NormalTok{) }\SpecialCharTok{+}
  \FunctionTok{theme}\NormalTok{(}
    \AttributeTok{axis.text.x =} \FunctionTok{element\_text}\NormalTok{(}\AttributeTok{angle =} \DecValTok{20}\NormalTok{, }\AttributeTok{hjust =} \DecValTok{1}\NormalTok{, }\AttributeTok{size =} \DecValTok{16}\NormalTok{),}
    \AttributeTok{axis.text =} \FunctionTok{element\_text}\NormalTok{(}\AttributeTok{size =} \DecValTok{16}\NormalTok{),    }
    \AttributeTok{axis.title =} \FunctionTok{element\_text}\NormalTok{(}\AttributeTok{size =} \DecValTok{20}\NormalTok{), }
    \AttributeTok{plot.title =} \FunctionTok{element\_text}\NormalTok{(}\AttributeTok{size=}\DecValTok{20}\NormalTok{),}
\NormalTok{  )}

\NormalTok{stratified\_ratios\_CMIN }\SpecialCharTok{+} \FunctionTok{scale\_y\_log10}\NormalTok{()}
\end{Highlighting}
\end{Shaded}

\pandocbounded{\includegraphics[keepaspectratio]{MIPD/Precision_dosing_methods_files/figure-pdf/Ratios-var-1.pdf}}

\subsection{Full dataset performance (model
performance)}\label{full-dataset-performance-model-performance}

In this section, the focus is no longer solely on observed
concentrations (IPRED), but the performance of the models is explored by
illustrating how the predicted concentrations (PRED) for \textbf{all}
the cohorts (not just for covariates on which the model was developed)
compare the the observed concentration. Visualization of PRED/IPRED
ratios. In all cases we compare PRED (predicted) values to a reference
IPRED (observed), which is the IPRED of the model whose cohort was used
to simulate a particular set of covariates. Red lines indicate the
bioequivalence range of 80-125 \%.

\begin{Shaded}
\begin{Highlighting}[]
\CommentTok{\# CMIN}
\CommentTok{\# Divide each PRED by the IPRED reference }
\NormalTok{AMOX\_CMIN3 }\OtherTok{\textless{}{-}} \FunctionTok{bind\_rows}\NormalTok{(AMOX\_CMIN\_TRAIN,AMOX\_CMIN\_TEST) }\SpecialCharTok{\%\textgreater{}\%}
  \FunctionTok{group\_by}\NormalTok{(ID) }\SpecialCharTok{\%\textgreater{}\%}
  \FunctionTok{mutate}\NormalTok{(}
    \AttributeTok{ref\_CMIN =} \FunctionTok{first}\NormalTok{(CMIN\_IND[REFERENCE }\SpecialCharTok{==} \DecValTok{1}\NormalTok{], }\AttributeTok{default =} \ConstantTok{NA}\NormalTok{),}
    \AttributeTok{ratio =}\NormalTok{ (CMIN\_PRED }\SpecialCharTok{/}\NormalTok{ ref\_CMIN) }\SpecialCharTok{*} \DecValTok{100}
\NormalTok{  ) }\SpecialCharTok{\%\textgreater{}\%}
  \FunctionTok{ungroup}\NormalTok{()}

\CommentTok{\# Boxplot of ratios stratified by model}
\NormalTok{perf\_CMIN }\OtherTok{\textless{}{-}} \FunctionTok{ggplot}\NormalTok{(AMOX\_CMIN3, }\FunctionTok{aes}\NormalTok{(}\AttributeTok{x =}\NormalTok{ MODEL, }\AttributeTok{y =}\NormalTok{ ratio, }\AttributeTok{fill =}\NormalTok{ MODEL)) }\SpecialCharTok{+}
  \FunctionTok{geom\_boxplot}\NormalTok{() }\SpecialCharTok{+}
  \FunctionTok{scale\_y\_log10}\NormalTok{() }\SpecialCharTok{+}
  \FunctionTok{geom\_hline}\NormalTok{(}\AttributeTok{yintercept =} \DecValTok{80}\NormalTok{, }\AttributeTok{linetype =} \StringTok{"dashed"}\NormalTok{, }\AttributeTok{color =} \StringTok{"red"}\NormalTok{) }\SpecialCharTok{+} 
  \FunctionTok{geom\_hline}\NormalTok{(}\AttributeTok{yintercept =} \DecValTok{125}\NormalTok{, }\AttributeTok{linetype =} \StringTok{"dashed"}\NormalTok{, }\AttributeTok{color =} \StringTok{"red"}\NormalTok{) }\SpecialCharTok{+}
  \FunctionTok{labs}\NormalTok{(}
    \AttributeTok{title =} \StringTok{"Boxplots of predicted/observed ratios (CMIN)"}\NormalTok{,}
    \AttributeTok{x =} \StringTok{"MODEL"}\NormalTok{,}
    \AttributeTok{y =} \StringTok{"Ratio (\%)"}
\NormalTok{  ) }\SpecialCharTok{+}
  \FunctionTok{theme\_minimal}\NormalTok{() }\SpecialCharTok{+}
  \FunctionTok{theme}\NormalTok{(}\AttributeTok{legend.position =} \StringTok{"none"}\NormalTok{) }\SpecialCharTok{+}
  \FunctionTok{theme}\NormalTok{(}
    \AttributeTok{axis.text.x =} \FunctionTok{element\_text}\NormalTok{(}\AttributeTok{angle =} \DecValTok{20}\NormalTok{, }\AttributeTok{hjust =} \DecValTok{1}\NormalTok{, }\AttributeTok{size =} \DecValTok{16}\NormalTok{),}
    \AttributeTok{axis.text =} \FunctionTok{element\_text}\NormalTok{(}\AttributeTok{size =} \DecValTok{16}\NormalTok{),    }
    \AttributeTok{axis.title =} \FunctionTok{element\_text}\NormalTok{(}\AttributeTok{size =} \DecValTok{20}\NormalTok{), }
    \AttributeTok{plot.title =} \FunctionTok{element\_text}\NormalTok{(}\AttributeTok{size=}\DecValTok{20}\NormalTok{),}
\NormalTok{  )}

\NormalTok{perf\_CMIN}
\end{Highlighting}
\end{Shaded}

\pandocbounded{\includegraphics[keepaspectratio]{MIPD/Precision_dosing_methods_files/figure-pdf/Ratios-1.pdf}}

\chapter{Single model approach}\label{single-model-approach}

\begin{Shaded}
\begin{Highlighting}[]
\NormalTok{explore\_predictions }\OtherTok{\textless{}{-}} \ControlFlowTok{function}\NormalTok{(data, }\AttributeTok{conc\_inf =} \DecValTok{40}\NormalTok{, }\AttributeTok{conc\_sup =} \DecValTok{80}\NormalTok{, }\AttributeTok{freq\_column =} \StringTok{"FREQ"}\NormalTok{) \{}
  \CommentTok{\# Calculate true dose range and prediction correctness}
\NormalTok{  data }\OtherTok{\textless{}{-}}\NormalTok{ data }\SpecialCharTok{\%\textgreater{}\%}
    \FunctionTok{mutate}\NormalTok{(}
      \AttributeTok{DOSE\_inf =}\NormalTok{ (conc\_inf }\SpecialCharTok{/}\NormalTok{ CMIN\_IND) }\SpecialCharTok{*}\NormalTok{ DOSE\_ADM,}
      \AttributeTok{DOSE\_sup =}\NormalTok{ (conc\_sup }\SpecialCharTok{/}\NormalTok{ CMIN\_IND) }\SpecialCharTok{*}\NormalTok{ DOSE\_ADM}
\NormalTok{    ) }\SpecialCharTok{\%\textgreater{}\%}
    \FunctionTok{mutate}\NormalTok{(}
      \AttributeTok{Prediction\_correctness =} \FunctionTok{ifelse}\NormalTok{(}
\NormalTok{        (DOSE\_PRED }\SpecialCharTok{\textgreater{}=}\NormalTok{ DOSE\_inf }\SpecialCharTok{\&}\NormalTok{ DOSE\_PRED }\SpecialCharTok{\textless{}=}\NormalTok{ DOSE\_sup),}
        \StringTok{"Correct"}\NormalTok{, }\StringTok{"Incorrect"}
\NormalTok{      )}
\NormalTok{    ) }\SpecialCharTok{\%\textgreater{}\%}
    \FunctionTok{drop\_na}\NormalTok{(Prediction\_correctness) }\SpecialCharTok{\%\textgreater{}\%}
    \FunctionTok{mutate}\NormalTok{(}
      \AttributeTok{Dosing =} \FunctionTok{case\_when}\NormalTok{(}
\NormalTok{        Prediction\_correctness }\SpecialCharTok{==} \StringTok{"Correct"} \SpecialCharTok{\textasciitilde{}} \StringTok{"On target"}\NormalTok{,}
\NormalTok{        DOSE\_PRED }\SpecialCharTok{\textless{}}\NormalTok{ DOSE\_inf }\SpecialCharTok{\textasciitilde{}} \StringTok{"Underdosed"}\NormalTok{,}
\NormalTok{        DOSE\_PRED }\SpecialCharTok{\textgreater{}}\NormalTok{ DOSE\_sup }\SpecialCharTok{\textasciitilde{}} \StringTok{"Overdosed"}
\NormalTok{      )}
\NormalTok{    )}

  \CommentTok{\# Proportions of under{-} and overdosing}
\NormalTok{  dosing }\OtherTok{\textless{}{-}}\NormalTok{ data }\SpecialCharTok{\%\textgreater{}\%}
    \FunctionTok{count}\NormalTok{(Dosing) }\SpecialCharTok{\%\textgreater{}\%}
    \FunctionTok{mutate}\NormalTok{(}
      \AttributeTok{Proportion =}\NormalTok{ n }\SpecialCharTok{/} \FunctionTok{sum}\NormalTok{(n) }\SpecialCharTok{*} \DecValTok{100}\NormalTok{,}
      \AttributeTok{Dosing =} \FunctionTok{factor}\NormalTok{(Dosing, }\AttributeTok{levels =} \FunctionTok{c}\NormalTok{(}\StringTok{"Overdosed"}\NormalTok{, }\StringTok{"On target"}\NormalTok{, }\StringTok{"Underdosed"}\NormalTok{)),}
      \AttributeTok{Label =} \FunctionTok{paste0}\NormalTok{(Dosing, }\StringTok{"}\SpecialCharTok{\textbackslash{}n}\StringTok{"}\NormalTok{, }\FunctionTok{round}\NormalTok{(Proportion), }\StringTok{"\%"}\NormalTok{)}
\NormalTok{    )}

  \CommentTok{\# Over/underdosed graph}
\NormalTok{  p1 }\OtherTok{\textless{}{-}} \FunctionTok{ggplot}\NormalTok{(dosing, }\FunctionTok{aes}\NormalTok{(}\AttributeTok{x =} \StringTok{""}\NormalTok{, }\AttributeTok{y =}\NormalTok{ Proportion, }\AttributeTok{fill =}\NormalTok{ Dosing)) }\SpecialCharTok{+}
    \FunctionTok{geom\_bar}\NormalTok{(}\AttributeTok{stat =} \StringTok{"identity"}\NormalTok{, }\AttributeTok{width =} \FloatTok{0.5}\NormalTok{) }\SpecialCharTok{+}
    \FunctionTok{geom\_text}\NormalTok{(}\FunctionTok{aes}\NormalTok{(}\AttributeTok{label =}\NormalTok{ Label), }\AttributeTok{position =} \FunctionTok{position\_stack}\NormalTok{(}\AttributeTok{vjust =} \FloatTok{0.5}\NormalTok{), }\AttributeTok{color =} \StringTok{"white"}\NormalTok{, }\AttributeTok{size =} \DecValTok{5}\NormalTok{) }\SpecialCharTok{+}
    \FunctionTok{scale\_fill\_manual}\NormalTok{(}\AttributeTok{values =} \FunctionTok{c}\NormalTok{(}\StringTok{"Underdosed"} \OtherTok{=} \StringTok{"darkorange"}\NormalTok{, }\StringTok{"On target"} \OtherTok{=} \StringTok{"chartreuse4"}\NormalTok{, }\StringTok{"Overdosed"} \OtherTok{=} \StringTok{"\#A91A27"}\NormalTok{)) }\SpecialCharTok{+}
    \FunctionTok{labs}\NormalTok{(}\AttributeTok{y =} \StringTok{"\%"}\NormalTok{, }\AttributeTok{x =} \ConstantTok{NULL}\NormalTok{, }\AttributeTok{title =} \StringTok{"Target attainment"}\NormalTok{, }\AttributeTok{fill =} \StringTok{"Dosing Category"}\NormalTok{) }\SpecialCharTok{+}
    \FunctionTok{theme\_minimal}\NormalTok{() }\SpecialCharTok{+}
    \FunctionTok{theme}\NormalTok{(}
      \AttributeTok{axis.text.x =} \FunctionTok{element\_blank}\NormalTok{(),}
      \AttributeTok{axis.ticks.x =} \FunctionTok{element\_blank}\NormalTok{(),}
      \AttributeTok{plot.title =} \FunctionTok{element\_text}\NormalTok{(}\AttributeTok{size =} \DecValTok{20}\NormalTok{),}
      \AttributeTok{axis.text =} \FunctionTok{element\_text}\NormalTok{(}\AttributeTok{size =} \DecValTok{16}\NormalTok{),}
      \AttributeTok{axis.title =} \FunctionTok{element\_text}\NormalTok{(}\AttributeTok{size =} \DecValTok{20}\NormalTok{),}
      \AttributeTok{legend.position =} \StringTok{"none"}
\NormalTok{    ) }\SpecialCharTok{+}
    \FunctionTok{scale\_y\_continuous}\NormalTok{(}\AttributeTok{breaks =} \FunctionTok{seq}\NormalTok{(}\DecValTok{0}\NormalTok{, }\DecValTok{100}\NormalTok{, }\AttributeTok{by =} \DecValTok{10}\NormalTok{)) }\SpecialCharTok{+} 
        \FunctionTok{coord\_cartesian}\NormalTok{(}\AttributeTok{ylim =} \FunctionTok{c}\NormalTok{(}\DecValTok{0}\NormalTok{, }\DecValTok{100}\NormalTok{))}

  \CommentTok{\# CREAT binning (based on the American Kidney Fund\textquotesingle{}s categories)}
\NormalTok{data }\OtherTok{\textless{}{-}}\NormalTok{ data }\SpecialCharTok{\%\textgreater{}\%}
  \FunctionTok{mutate}\NormalTok{(}
    \AttributeTok{CREAT\_bin =} \FunctionTok{case\_when}\NormalTok{(}\CommentTok{\# males}
\NormalTok{      SEX }\SpecialCharTok{==} \DecValTok{0} \SpecialCharTok{\textasciitilde{}} \FunctionTok{cut}\NormalTok{(}
\NormalTok{        CREAT,}
        \AttributeTok{breaks =} \FunctionTok{c}\NormalTok{(}\DecValTok{0}\NormalTok{, }\FloatTok{0.7}\NormalTok{, }\FloatTok{1.3}\NormalTok{, }\ConstantTok{Inf}\NormalTok{),}
        \AttributeTok{labels =} \FunctionTok{c}\NormalTok{(}\StringTok{"Below normal"}\NormalTok{, }\StringTok{"Normal"}\NormalTok{, }\StringTok{"Above normal"}\NormalTok{),}
        \AttributeTok{right =} \ConstantTok{FALSE}
\NormalTok{      ),}
\NormalTok{      SEX }\SpecialCharTok{==} \DecValTok{1} \SpecialCharTok{\textasciitilde{}} \FunctionTok{cut}\NormalTok{( }\CommentTok{\# females}
\NormalTok{        CREAT,}
        \AttributeTok{breaks =} \FunctionTok{c}\NormalTok{(}\DecValTok{0}\NormalTok{, }\FloatTok{0.6}\NormalTok{, }\FloatTok{1.1}\NormalTok{, }\ConstantTok{Inf}\NormalTok{),}
        \AttributeTok{labels =} \FunctionTok{c}\NormalTok{(}\StringTok{"Below normal"}\NormalTok{, }\StringTok{"Normal"}\NormalTok{, }\StringTok{"Above normal"}\NormalTok{),}
        \AttributeTok{right =} \ConstantTok{FALSE}
\NormalTok{      )}
\NormalTok{    )}
\NormalTok{  )}

  \CommentTok{\# Binning stats}
\NormalTok{  bin\_summary }\OtherTok{\textless{}{-}}\NormalTok{ data }\SpecialCharTok{\%\textgreater{}\%}
    \FunctionTok{group\_by}\NormalTok{(CREAT\_bin, Prediction\_correctness) }\SpecialCharTok{\%\textgreater{}\%}
    \FunctionTok{summarise}\NormalTok{(}\AttributeTok{Count =} \FunctionTok{n}\NormalTok{(), }\AttributeTok{.groups =} \StringTok{"drop"}\NormalTok{) }\SpecialCharTok{\%\textgreater{}\%}
    \FunctionTok{pivot\_wider}\NormalTok{(}\AttributeTok{names\_from =}\NormalTok{ Prediction\_correctness, }\AttributeTok{values\_from =}\NormalTok{ Count, }\AttributeTok{values\_fill =} \DecValTok{0}\NormalTok{) }\SpecialCharTok{\%\textgreater{}\%}
    \FunctionTok{mutate}\NormalTok{(}\AttributeTok{Proportion\_Correct =}\NormalTok{ (Correct }\SpecialCharTok{/}\NormalTok{ (Correct }\SpecialCharTok{+}\NormalTok{ Incorrect)) }\SpecialCharTok{*} \DecValTok{100}\NormalTok{)}

  \CommentTok{\# Binning graph}
\NormalTok{  p2 }\OtherTok{\textless{}{-}} \FunctionTok{ggplot}\NormalTok{(bin\_summary, }\FunctionTok{aes}\NormalTok{(}\AttributeTok{x =}\NormalTok{ CREAT\_bin, }\AttributeTok{y =}\NormalTok{ Proportion\_Correct)) }\SpecialCharTok{+}
    \FunctionTok{geom\_bar}\NormalTok{(}\AttributeTok{stat =} \StringTok{"identity"}\NormalTok{, }\AttributeTok{fill =} \StringTok{"chartreuse4"}\NormalTok{, }\AttributeTok{color =} \StringTok{"chartreuse4"}\NormalTok{, }\AttributeTok{size =} \DecValTok{1}\NormalTok{) }\SpecialCharTok{+}
    \FunctionTok{geom\_text}\NormalTok{(}\FunctionTok{aes}\NormalTok{(}\AttributeTok{label =} \FunctionTok{round}\NormalTok{(Proportion\_Correct)), }\AttributeTok{vjust =} \SpecialCharTok{{-}}\FloatTok{0.5}\NormalTok{, }\AttributeTok{size =} \DecValTok{5}\NormalTok{, }\AttributeTok{color =} \StringTok{"black"}\NormalTok{) }\SpecialCharTok{+}
    \FunctionTok{labs}\NormalTok{(}\AttributeTok{title =} \StringTok{"Target attainement by serum creatinine level"}\NormalTok{, }\AttributeTok{x =} \StringTok{"Serum creatinine"}\NormalTok{, }\AttributeTok{y =} \StringTok{"Correct predictions (\%)"}\NormalTok{) }\SpecialCharTok{+}
    \FunctionTok{theme\_minimal}\NormalTok{() }\SpecialCharTok{+}
    \FunctionTok{theme}\NormalTok{(}
      \AttributeTok{plot.title =} \FunctionTok{element\_text}\NormalTok{(}\AttributeTok{size =} \DecValTok{24}\NormalTok{),}
      \AttributeTok{axis.text =} \FunctionTok{element\_text}\NormalTok{(}\AttributeTok{size =} \DecValTok{14}\NormalTok{),}
      \AttributeTok{axis.title =} \FunctionTok{element\_text}\NormalTok{(}\AttributeTok{size =} \DecValTok{16}\NormalTok{)}
\NormalTok{    ) }\SpecialCharTok{+}
    \FunctionTok{scale\_y\_continuous}\NormalTok{(}\AttributeTok{limits =} \FunctionTok{c}\NormalTok{(}\DecValTok{0}\NormalTok{, }\DecValTok{100}\NormalTok{), }\AttributeTok{breaks =} \FunctionTok{seq}\NormalTok{(}\DecValTok{0}\NormalTok{, }\DecValTok{100}\NormalTok{, }\AttributeTok{by =} \DecValTok{20}\NormalTok{))}
  
    \CommentTok{\# CRCL binning for chronic kidney disease categories}
\NormalTok{data }\OtherTok{\textless{}{-}}\NormalTok{ data }\SpecialCharTok{\%\textgreater{}\%}
  \FunctionTok{mutate}\NormalTok{(}
    \AttributeTok{CRCL\_bin =} \FunctionTok{cut}\NormalTok{(}
\NormalTok{        CRCL\_CKD\_EPI\_ABSOLUTE,}
        \AttributeTok{breaks =} \FunctionTok{c}\NormalTok{(}\DecValTok{0}\NormalTok{, }\DecValTok{30}\NormalTok{, }\DecValTok{60}\NormalTok{, }\DecValTok{90}\NormalTok{, }\DecValTok{130}\NormalTok{, }\ConstantTok{Inf}\NormalTok{),}
        \AttributeTok{labels =} \FunctionTok{c}\NormalTok{(}\StringTok{"0{-}30"}\NormalTok{, }\StringTok{"31{-}60"}\NormalTok{, }\StringTok{"61{-}90"}\NormalTok{,  }\StringTok{"90{-}130"}\NormalTok{,  }\StringTok{"\textgreater{} 130"}\NormalTok{),}
        \AttributeTok{right =} \ConstantTok{FALSE}
\NormalTok{      )}
\NormalTok{    )}

  \CommentTok{\# Binning stats}
\NormalTok{  bin\_summary }\OtherTok{\textless{}{-}}\NormalTok{ data }\SpecialCharTok{\%\textgreater{}\%}
    \FunctionTok{group\_by}\NormalTok{(CRCL\_bin, Prediction\_correctness) }\SpecialCharTok{\%\textgreater{}\%}
    \FunctionTok{summarise}\NormalTok{(}\AttributeTok{Count =} \FunctionTok{n}\NormalTok{(), }\AttributeTok{.groups =} \StringTok{"drop"}\NormalTok{) }\SpecialCharTok{\%\textgreater{}\%}
    \FunctionTok{pivot\_wider}\NormalTok{(}\AttributeTok{names\_from =}\NormalTok{ Prediction\_correctness, }\AttributeTok{values\_from =}\NormalTok{ Count, }\AttributeTok{values\_fill =} \DecValTok{0}\NormalTok{) }\SpecialCharTok{\%\textgreater{}\%}
    \FunctionTok{mutate}\NormalTok{(}\AttributeTok{Proportion\_Correct =}\NormalTok{ (Correct }\SpecialCharTok{/}\NormalTok{ (Correct }\SpecialCharTok{+}\NormalTok{ Incorrect)) }\SpecialCharTok{*} \DecValTok{100}\NormalTok{)}

  \CommentTok{\# Binning graph}
\NormalTok{  p3 }\OtherTok{\textless{}{-}} \FunctionTok{ggplot}\NormalTok{(bin\_summary, }\FunctionTok{aes}\NormalTok{(}\AttributeTok{x =}\NormalTok{ CRCL\_bin, }\AttributeTok{y =}\NormalTok{ Proportion\_Correct)) }\SpecialCharTok{+}
    \FunctionTok{geom\_bar}\NormalTok{(}\AttributeTok{stat =} \StringTok{"identity"}\NormalTok{, }\AttributeTok{fill =} \StringTok{"chartreuse4"}\NormalTok{, }\AttributeTok{color =} \StringTok{"chartreuse4"}\NormalTok{, }\AttributeTok{size =} \DecValTok{1}\NormalTok{) }\SpecialCharTok{+}
    \FunctionTok{geom\_text}\NormalTok{(}\FunctionTok{aes}\NormalTok{(}\AttributeTok{label =} \FunctionTok{round}\NormalTok{(Proportion\_Correct)), }\AttributeTok{vjust =} \SpecialCharTok{{-}}\FloatTok{0.5}\NormalTok{, }\AttributeTok{size =} \DecValTok{5}\NormalTok{, }\AttributeTok{color =} \StringTok{"black"}\NormalTok{) }\SpecialCharTok{+}
    \FunctionTok{labs}\NormalTok{(}\AttributeTok{title =} \StringTok{"Target attainement by creatinine clearance category"}\NormalTok{, }\AttributeTok{x =} \StringTok{"Creatinine clearance (mL/min)"}\NormalTok{, }\AttributeTok{y =} \StringTok{"Correct predictions (\%)"}\NormalTok{) }\SpecialCharTok{+}
    \FunctionTok{theme\_minimal}\NormalTok{() }\SpecialCharTok{+}
    \FunctionTok{theme}\NormalTok{(}
      \AttributeTok{plot.title =} \FunctionTok{element\_text}\NormalTok{(}\AttributeTok{size =} \DecValTok{24}\NormalTok{),}
      \AttributeTok{axis.text =} \FunctionTok{element\_text}\NormalTok{(}\AttributeTok{size =} \DecValTok{14}\NormalTok{),}
      \AttributeTok{axis.title =} \FunctionTok{element\_text}\NormalTok{(}\AttributeTok{size =} \DecValTok{16}\NormalTok{)}
\NormalTok{    ) }\SpecialCharTok{+}
    \FunctionTok{scale\_y\_continuous}\NormalTok{(}\AttributeTok{limits =} \FunctionTok{c}\NormalTok{(}\DecValTok{0}\NormalTok{, }\DecValTok{100}\NormalTok{), }\AttributeTok{breaks =} \FunctionTok{seq}\NormalTok{(}\DecValTok{0}\NormalTok{, }\DecValTok{100}\NormalTok{, }\AttributeTok{by =} \DecValTok{20}\NormalTok{))}

  \CommentTok{\# Summary statistics}
\NormalTok{  summary\_stats }\OtherTok{\textless{}{-}}\NormalTok{ data }\SpecialCharTok{\%\textgreater{}\%}
    \FunctionTok{group\_by}\NormalTok{(Prediction\_correctness) }\SpecialCharTok{\%\textgreater{}\%}
    \FunctionTok{summarise}\NormalTok{(}
      \AttributeTok{mean\_CREAT =} \FunctionTok{mean}\NormalTok{(CREAT, }\AttributeTok{na.rm =} \ConstantTok{TRUE}\NormalTok{),}
      \AttributeTok{sd\_CREAT =} \FunctionTok{sd}\NormalTok{(CREAT, }\AttributeTok{na.rm =} \ConstantTok{TRUE}\NormalTok{),}
      \AttributeTok{mean\_WT =} \FunctionTok{mean}\NormalTok{(WT, }\AttributeTok{na.rm =} \ConstantTok{TRUE}\NormalTok{),}
      \AttributeTok{sd\_WT =} \FunctionTok{sd}\NormalTok{(WT, }\AttributeTok{na.rm =} \ConstantTok{TRUE}\NormalTok{),}
      \AttributeTok{mean\_AGE =} \FunctionTok{mean}\NormalTok{(AGE, }\AttributeTok{na.rm =} \ConstantTok{TRUE}\NormalTok{),}
      \AttributeTok{sd\_AGE =} \FunctionTok{sd}\NormalTok{(AGE, }\AttributeTok{na.rm =} \ConstantTok{TRUE}\NormalTok{),}
      \AttributeTok{Count =} \FunctionTok{n}\NormalTok{()}
\NormalTok{    ) }\SpecialCharTok{\%\textgreater{}\%}
    \FunctionTok{mutate}\NormalTok{(}\AttributeTok{Proportion =}\NormalTok{ Count }\SpecialCharTok{/} \FunctionTok{sum}\NormalTok{(Count))}

\NormalTok{  correct\_proportion }\OtherTok{\textless{}{-}}\NormalTok{ summary\_stats }\SpecialCharTok{\%\textgreater{}\%}
    \FunctionTok{filter}\NormalTok{(Prediction\_correctness }\SpecialCharTok{==} \StringTok{"Correct"}\NormalTok{) }\SpecialCharTok{\%\textgreater{}\%}
    \FunctionTok{pull}\NormalTok{(Proportion)}

  \FunctionTok{message}\NormalTok{(}\FunctionTok{sprintf}\NormalTok{(}\StringTok{"Proportion of \textquotesingle{}correct\textquotesingle{} predictions: \%.2f\%\%"}\NormalTok{, correct\_proportion }\SpecialCharTok{*} \DecValTok{100}\NormalTok{))}

  \FunctionTok{return}\NormalTok{(}\FunctionTok{list}\NormalTok{(}
    \AttributeTok{target\_attainment =}\NormalTok{ p1,}
    \AttributeTok{creat\_plot =}\NormalTok{ p2,}
    \AttributeTok{crcl\_plot =}\NormalTok{ p3,}
    \AttributeTok{summary\_stats =}\NormalTok{ summary\_stats}
\NormalTok{  ))}
\NormalTok{\}}
\end{Highlighting}
\end{Shaded}

\section{CARLIER}\label{carlier-1}

\begin{Shaded}
\begin{Highlighting}[]
\CommentTok{\# In this specific case predictions were already}
\CommentTok{\# created before, thus "making predictions"}
\CommentTok{\# simply involve loading them and filtering for the correct model}
\NormalTok{AMOX\_CMIN\_PRED\_TRAIN\_CARLIER }\OtherTok{\textless{}{-}} \FunctionTok{read\_csv}\NormalTok{(}
  \FunctionTok{here}\NormalTok{(}\StringTok{"Amoxicillin/a\_priori/For\_publication/Data/AMOX\_CMIN\_TRAIN.csv"}\NormalTok{)}
\NormalTok{  ) }\SpecialCharTok{|\textgreater{}} 
  \FunctionTok{filter}\NormalTok{(MODEL }\SpecialCharTok{==} \StringTok{"CARLIER"}\NormalTok{)}

\NormalTok{AMOX\_CMIN\_PRED\_TEST\_CARLIER }\OtherTok{\textless{}{-}} \FunctionTok{read\_csv}\NormalTok{(}
  \FunctionTok{here}\NormalTok{(}\StringTok{"Amoxicillin/a\_priori/For\_publication/Data/AMOX\_CMIN\_TEST.csv"}\NormalTok{)}
\NormalTok{  ) }\SpecialCharTok{|\textgreater{}} 
  \FunctionTok{filter}\NormalTok{(MODEL }\SpecialCharTok{==} \StringTok{"CARLIER"}\NormalTok{)}
\end{Highlighting}
\end{Shaded}

\begin{Shaded}
\begin{Highlighting}[]
\NormalTok{results\_carlier\_train }\OtherTok{\textless{}{-}} \FunctionTok{explore\_predictions}\NormalTok{(AMOX\_CMIN\_PRED\_TRAIN\_CARLIER)}
\NormalTok{results\_carlier\_train}\SpecialCharTok{$}\NormalTok{target\_attainment }
\end{Highlighting}
\end{Shaded}

\pandocbounded{\includegraphics[keepaspectratio]{MIPD/Precision_dosing_methods_files/figure-pdf/evaluate-predictions-carlier-train-1.pdf}}

\begin{Shaded}
\begin{Highlighting}[]
\NormalTok{creat\_plot\_carlier\_train }\OtherTok{\textless{}{-}}\NormalTok{ results\_carlier\_train}\SpecialCharTok{$}\NormalTok{creat\_plot}
\NormalTok{creat\_plot\_carlier\_train}
\end{Highlighting}
\end{Shaded}

\pandocbounded{\includegraphics[keepaspectratio]{MIPD/Precision_dosing_methods_files/figure-pdf/evaluate-predictions-carlier-train-2.pdf}}

\begin{Shaded}
\begin{Highlighting}[]
\NormalTok{crcl\_plot\_carlier\_train }\OtherTok{\textless{}{-}}\NormalTok{ results\_carlier\_train}\SpecialCharTok{$}\NormalTok{crcl\_plot}
\NormalTok{crcl\_plot\_carlier\_train}
\end{Highlighting}
\end{Shaded}

\pandocbounded{\includegraphics[keepaspectratio]{MIPD/Precision_dosing_methods_files/figure-pdf/evaluate-predictions-carlier-train-3.pdf}}

\begin{Shaded}
\begin{Highlighting}[]
\NormalTok{results\_carlier\_train}\SpecialCharTok{$}\NormalTok{summary\_stats }
\end{Highlighting}
\end{Shaded}

\begin{verbatim}
# A tibble: 2 x 9
  Prediction_correctness mean_CREAT sd_CREAT mean_WT sd_WT mean_AGE sd_AGE Count
  <chr>                       <dbl>    <dbl>   <dbl> <dbl>    <dbl>  <dbl> <int>
1 Correct                     1.02     0.560    78.7  13.8     64.2   17.2   511
2 Incorrect                   0.899    0.602    86.5  19.0     57.3   15.6  1364
# i 1 more variable: Proportion <dbl>
\end{verbatim}

\begin{Shaded}
\begin{Highlighting}[]
\NormalTok{results\_carlier\_test }\OtherTok{\textless{}{-}} \FunctionTok{explore\_predictions}\NormalTok{(AMOX\_CMIN\_PRED\_TEST\_CARLIER)}
\NormalTok{results\_carlier\_test}\SpecialCharTok{$}\NormalTok{target\_attainment }
\end{Highlighting}
\end{Shaded}

\pandocbounded{\includegraphics[keepaspectratio]{MIPD/Precision_dosing_methods_files/figure-pdf/evaluate-predictions-carlier-test-1.pdf}}

\begin{Shaded}
\begin{Highlighting}[]
\NormalTok{creat\_plot\_carlier\_test }\OtherTok{\textless{}{-}}\NormalTok{ results\_carlier\_test}\SpecialCharTok{$}\NormalTok{creat\_plot}
\NormalTok{creat\_plot\_carlier\_test}
\end{Highlighting}
\end{Shaded}

\pandocbounded{\includegraphics[keepaspectratio]{MIPD/Precision_dosing_methods_files/figure-pdf/evaluate-predictions-carlier-test-2.pdf}}

\begin{Shaded}
\begin{Highlighting}[]
\NormalTok{crcl\_plot\_carlier\_test }\OtherTok{\textless{}{-}}\NormalTok{ results\_carlier\_test}\SpecialCharTok{$}\NormalTok{crcl\_plot}
\NormalTok{crcl\_plot\_carlier\_test}
\end{Highlighting}
\end{Shaded}

\pandocbounded{\includegraphics[keepaspectratio]{MIPD/Precision_dosing_methods_files/figure-pdf/evaluate-predictions-carlier-test-3.pdf}}

\begin{Shaded}
\begin{Highlighting}[]
\NormalTok{results\_carlier\_test}\SpecialCharTok{$}\NormalTok{summary\_stats }
\end{Highlighting}
\end{Shaded}

\begin{verbatim}
# A tibble: 2 x 9
  Prediction_correctness mean_CREAT sd_CREAT mean_WT sd_WT mean_AGE sd_AGE Count
  <chr>                       <dbl>    <dbl>   <dbl> <dbl>    <dbl>  <dbl> <int>
1 Correct                     1.06     0.591    79.2  14.8     66.5   17.1   175
2 Incorrect                   0.903    0.568    86.5  19.7     57.2   15.3   425
# i 1 more variable: Proportion <dbl>
\end{verbatim}

\section{FOURNIER}\label{fournier-1}

\begin{Shaded}
\begin{Highlighting}[]
\CommentTok{\# In this specific case predictions were already}
\CommentTok{\# created before, thus "making predictions"}
\CommentTok{\# simply involve loading them and filtering for the correct model}
\NormalTok{AMOX\_CMIN\_PRED\_TRAIN\_FOURNIER }\OtherTok{\textless{}{-}} \FunctionTok{read\_csv}\NormalTok{(}
  \FunctionTok{here}\NormalTok{(}\StringTok{"Amoxicillin/a\_priori/For\_publication/Data/AMOX\_CMIN\_TRAIN.csv"}\NormalTok{)}
\NormalTok{  ) }\SpecialCharTok{|\textgreater{}} 
  \FunctionTok{filter}\NormalTok{(MODEL }\SpecialCharTok{==} \StringTok{"FOURNIER"}\NormalTok{)}

\NormalTok{AMOX\_CMIN\_PRED\_TEST\_FOURNIER }\OtherTok{\textless{}{-}} \FunctionTok{read\_csv}\NormalTok{(}
  \FunctionTok{here}\NormalTok{(}\StringTok{"Amoxicillin/a\_priori/For\_publication/Data/AMOX\_CMIN\_TEST.csv"}\NormalTok{)}
\NormalTok{  ) }\SpecialCharTok{|\textgreater{}} 
  \FunctionTok{filter}\NormalTok{(MODEL }\SpecialCharTok{==} \StringTok{"FOURNIER"}\NormalTok{)}
\end{Highlighting}
\end{Shaded}

\begin{Shaded}
\begin{Highlighting}[]
\NormalTok{results\_fournier\_train }\OtherTok{\textless{}{-}} \FunctionTok{explore\_predictions}\NormalTok{(AMOX\_CMIN\_PRED\_TRAIN\_FOURNIER)}
\NormalTok{results\_fournier\_train}\SpecialCharTok{$}\NormalTok{target\_attainment }
\end{Highlighting}
\end{Shaded}

\pandocbounded{\includegraphics[keepaspectratio]{MIPD/Precision_dosing_methods_files/figure-pdf/evaluate-predictions-fournier-train-1.pdf}}

\begin{Shaded}
\begin{Highlighting}[]
\NormalTok{creat\_plot\_fournier\_train }\OtherTok{\textless{}{-}}\NormalTok{ results\_fournier\_train}\SpecialCharTok{$}\NormalTok{creat\_plot}
\NormalTok{creat\_plot\_fournier\_train}
\end{Highlighting}
\end{Shaded}

\pandocbounded{\includegraphics[keepaspectratio]{MIPD/Precision_dosing_methods_files/figure-pdf/evaluate-predictions-fournier-train-2.pdf}}

\begin{Shaded}
\begin{Highlighting}[]
\NormalTok{crcl\_plot\_fournier\_train }\OtherTok{\textless{}{-}}\NormalTok{ results\_fournier\_train}\SpecialCharTok{$}\NormalTok{crcl\_plot}
\NormalTok{crcl\_plot\_fournier\_train}
\end{Highlighting}
\end{Shaded}

\pandocbounded{\includegraphics[keepaspectratio]{MIPD/Precision_dosing_methods_files/figure-pdf/evaluate-predictions-fournier-train-3.pdf}}

\begin{Shaded}
\begin{Highlighting}[]
\NormalTok{results\_fournier\_train}\SpecialCharTok{$}\NormalTok{summary\_stats }
\end{Highlighting}
\end{Shaded}

\begin{verbatim}
# A tibble: 2 x 9
  Prediction_correctness mean_CREAT sd_CREAT mean_WT sd_WT mean_AGE sd_AGE Count
  <chr>                       <dbl>    <dbl>   <dbl> <dbl>    <dbl>  <dbl> <int>
1 Correct                     0.945    0.537    80.0  16.0     64.4   17.1   539
2 Incorrect                   0.925    0.615    86.1  18.6     57.1   15.5  1336
# i 1 more variable: Proportion <dbl>
\end{verbatim}

\begin{Shaded}
\begin{Highlighting}[]
\NormalTok{results\_fournier\_test }\OtherTok{\textless{}{-}} \FunctionTok{explore\_predictions}\NormalTok{(AMOX\_CMIN\_PRED\_TEST\_FOURNIER)}
\NormalTok{results\_fournier\_test}\SpecialCharTok{$}\NormalTok{target\_attainment }
\end{Highlighting}
\end{Shaded}

\pandocbounded{\includegraphics[keepaspectratio]{MIPD/Precision_dosing_methods_files/figure-pdf/evaluate-predictions-fournier-test-1.pdf}}

\begin{Shaded}
\begin{Highlighting}[]
\NormalTok{creat\_plot\_fournier\_test }\OtherTok{\textless{}{-}}\NormalTok{ results\_fournier\_test}\SpecialCharTok{$}\NormalTok{creat\_plot}
\NormalTok{creat\_plot\_fournier\_test}
\end{Highlighting}
\end{Shaded}

\pandocbounded{\includegraphics[keepaspectratio]{MIPD/Precision_dosing_methods_files/figure-pdf/evaluate-predictions-fournier-test-2.pdf}}

\begin{Shaded}
\begin{Highlighting}[]
\NormalTok{crcl\_plot\_fournier\_test }\OtherTok{\textless{}{-}}\NormalTok{ results\_fournier\_test}\SpecialCharTok{$}\NormalTok{crcl\_plot}
\NormalTok{crcl\_plot\_fournier\_test}
\end{Highlighting}
\end{Shaded}

\pandocbounded{\includegraphics[keepaspectratio]{MIPD/Precision_dosing_methods_files/figure-pdf/evaluate-predictions-fournier-test-3.pdf}}

\begin{Shaded}
\begin{Highlighting}[]
\NormalTok{results\_fournier\_test}\SpecialCharTok{$}\NormalTok{summary\_stats }
\end{Highlighting}
\end{Shaded}

\begin{verbatim}
# A tibble: 2 x 9
  Prediction_correctness mean_CREAT sd_CREAT mean_WT sd_WT mean_AGE sd_AGE Count
  <chr>                       <dbl>    <dbl>   <dbl> <dbl>    <dbl>  <dbl> <int>
1 Correct                     0.904    0.368    79.2  16.1     65.2   16.3   177
2 Incorrect                   0.970    0.646    86.6  19.3     57.7   15.9   423
# i 1 more variable: Proportion <dbl>
\end{verbatim}

\section{MELLON}\label{mellon-1}

\begin{Shaded}
\begin{Highlighting}[]
\CommentTok{\# In this specific case predictions were already}
\CommentTok{\# created before, thus "making predictions"}
\CommentTok{\# simply involve loading them and filtering for the correct model}
\NormalTok{AMOX\_CMIN\_PRED\_TRAIN\_MELLON }\OtherTok{\textless{}{-}} \FunctionTok{read\_csv}\NormalTok{(}
  \FunctionTok{here}\NormalTok{(}\StringTok{"Amoxicillin/a\_priori/For\_publication/Data/AMOX\_CMIN\_TRAIN.csv"}\NormalTok{)}
\NormalTok{  ) }\SpecialCharTok{|\textgreater{}} 
  \FunctionTok{filter}\NormalTok{(MODEL }\SpecialCharTok{==} \StringTok{"MELLON"}\NormalTok{)}

\NormalTok{AMOX\_CMIN\_PRED\_TEST\_MELLON }\OtherTok{\textless{}{-}} \FunctionTok{read\_csv}\NormalTok{(}
  \FunctionTok{here}\NormalTok{(}\StringTok{"Amoxicillin/a\_priori/For\_publication/Data/AMOX\_CMIN\_TEST.csv"}\NormalTok{)}
\NormalTok{  ) }\SpecialCharTok{|\textgreater{}} 
  \FunctionTok{filter}\NormalTok{(MODEL }\SpecialCharTok{==} \StringTok{"MELLON"}\NormalTok{)}
\end{Highlighting}
\end{Shaded}

\begin{Shaded}
\begin{Highlighting}[]
\NormalTok{results\_mellon\_train }\OtherTok{\textless{}{-}} \FunctionTok{explore\_predictions}\NormalTok{(AMOX\_CMIN\_PRED\_TRAIN\_MELLON)}
\NormalTok{results\_mellon\_train}\SpecialCharTok{$}\NormalTok{target\_attainment }
\end{Highlighting}
\end{Shaded}

\pandocbounded{\includegraphics[keepaspectratio]{MIPD/Precision_dosing_methods_files/figure-pdf/evaluate-predictions-mellon-train-1.pdf}}

\begin{Shaded}
\begin{Highlighting}[]
\NormalTok{creat\_plot\_mellon\_train }\OtherTok{\textless{}{-}}\NormalTok{ results\_mellon\_train}\SpecialCharTok{$}\NormalTok{creat\_plot}
\NormalTok{creat\_plot\_mellon\_train}
\end{Highlighting}
\end{Shaded}

\pandocbounded{\includegraphics[keepaspectratio]{MIPD/Precision_dosing_methods_files/figure-pdf/evaluate-predictions-mellon-train-2.pdf}}

\begin{Shaded}
\begin{Highlighting}[]
\NormalTok{crcl\_plot\_mellon\_train }\OtherTok{\textless{}{-}}\NormalTok{ results\_mellon\_train}\SpecialCharTok{$}\NormalTok{crcl\_plot}
\NormalTok{crcl\_plot\_mellon\_train}
\end{Highlighting}
\end{Shaded}

\pandocbounded{\includegraphics[keepaspectratio]{MIPD/Precision_dosing_methods_files/figure-pdf/evaluate-predictions-mellon-train-3.pdf}}

\begin{Shaded}
\begin{Highlighting}[]
\NormalTok{results\_mellon\_train}\SpecialCharTok{$}\NormalTok{summary\_stats }
\end{Highlighting}
\end{Shaded}

\begin{verbatim}
# A tibble: 2 x 9
  Prediction_correctness mean_CREAT sd_CREAT mean_WT sd_WT mean_AGE sd_AGE Count
  <chr>                       <dbl>    <dbl>   <dbl> <dbl>    <dbl>  <dbl> <int>
1 Correct                     0.744    0.211    89.9  19.5     59.0   14.3   473
2 Incorrect                   0.994    0.664    82.5  17.2     59.3   16.9  1402
# i 1 more variable: Proportion <dbl>
\end{verbatim}

\begin{Shaded}
\begin{Highlighting}[]
\NormalTok{results\_mellon\_test }\OtherTok{\textless{}{-}} \FunctionTok{explore\_predictions}\NormalTok{(AMOX\_CMIN\_PRED\_TEST\_MELLON)}
\NormalTok{results\_mellon\_test}\SpecialCharTok{$}\NormalTok{target\_attainment }
\end{Highlighting}
\end{Shaded}

\pandocbounded{\includegraphics[keepaspectratio]{MIPD/Precision_dosing_methods_files/figure-pdf/evaluate-predictions-mellon-test-1.pdf}}

\begin{Shaded}
\begin{Highlighting}[]
\NormalTok{creat\_plot\_mellon\_test }\OtherTok{\textless{}{-}}\NormalTok{ results\_mellon\_test}\SpecialCharTok{$}\NormalTok{creat\_plot}
\NormalTok{creat\_plot\_mellon\_test}
\end{Highlighting}
\end{Shaded}

\pandocbounded{\includegraphics[keepaspectratio]{MIPD/Precision_dosing_methods_files/figure-pdf/evaluate-predictions-mellon-test-2.pdf}}

\begin{Shaded}
\begin{Highlighting}[]
\NormalTok{crcl\_plot\_mellon\_test }\OtherTok{\textless{}{-}}\NormalTok{ results\_mellon\_test}\SpecialCharTok{$}\NormalTok{crcl\_plot}
\NormalTok{crcl\_plot\_mellon\_test}
\end{Highlighting}
\end{Shaded}

\pandocbounded{\includegraphics[keepaspectratio]{MIPD/Precision_dosing_methods_files/figure-pdf/evaluate-predictions-mellon-test-3.pdf}}

\begin{Shaded}
\begin{Highlighting}[]
\NormalTok{results\_mellon\_test}\SpecialCharTok{$}\NormalTok{summary\_stats }
\end{Highlighting}
\end{Shaded}

\begin{verbatim}
# A tibble: 2 x 9
  Prediction_correctness mean_CREAT sd_CREAT mean_WT sd_WT mean_AGE sd_AGE Count
  <chr>                       <dbl>    <dbl>   <dbl> <dbl>    <dbl>  <dbl> <int>
1 Correct                     0.755    0.232    88.7  20.8     57.6   15.0   161
2 Incorrect                   1.02     0.648    82.8  17.6     60.7   16.8   439
# i 1 more variable: Proportion <dbl>
\end{verbatim}

\section{RAMBAUD}\label{rambaud-1}

\begin{Shaded}
\begin{Highlighting}[]
\CommentTok{\# In this specific case predictions were already}
\CommentTok{\# created before, thus "making predictions"}
\CommentTok{\# simply involve loading them and filtering for the correct model}
\NormalTok{AMOX\_CMIN\_PRED\_TRAIN\_RAMBAUD }\OtherTok{\textless{}{-}} \FunctionTok{read\_csv}\NormalTok{(}
  \FunctionTok{here}\NormalTok{(}\StringTok{"Amoxicillin/a\_priori/For\_publication/Data/AMOX\_CMIN\_TRAIN.csv"}\NormalTok{)}
\NormalTok{  ) }\SpecialCharTok{|\textgreater{}} 
  \FunctionTok{filter}\NormalTok{(MODEL }\SpecialCharTok{==} \StringTok{"RAMBAUD"}\NormalTok{)}

\NormalTok{AMOX\_CMIN\_PRED\_TEST\_RAMBAUD }\OtherTok{\textless{}{-}} \FunctionTok{read\_csv}\NormalTok{(}
  \FunctionTok{here}\NormalTok{(}\StringTok{"Amoxicillin/a\_priori/For\_publication/Data/AMOX\_CMIN\_TEST.csv"}\NormalTok{)}
\NormalTok{  ) }\SpecialCharTok{|\textgreater{}} 
  \FunctionTok{filter}\NormalTok{(MODEL }\SpecialCharTok{==} \StringTok{"RAMBAUD"}\NormalTok{)}
\end{Highlighting}
\end{Shaded}

\begin{Shaded}
\begin{Highlighting}[]
\NormalTok{results\_rambaud\_train }\OtherTok{\textless{}{-}} \FunctionTok{explore\_predictions}\NormalTok{(AMOX\_CMIN\_PRED\_TRAIN\_RAMBAUD)}
\NormalTok{results\_rambaud\_train}\SpecialCharTok{$}\NormalTok{target\_attainment }
\end{Highlighting}
\end{Shaded}

\pandocbounded{\includegraphics[keepaspectratio]{MIPD/Precision_dosing_methods_files/figure-pdf/evaluate-predictions-rambaud-train-1.pdf}}

\begin{Shaded}
\begin{Highlighting}[]
\NormalTok{creat\_plot\_rambaud\_train }\OtherTok{\textless{}{-}}\NormalTok{ results\_rambaud\_train}\SpecialCharTok{$}\NormalTok{creat\_plot}
\NormalTok{creat\_plot\_rambaud\_train}
\end{Highlighting}
\end{Shaded}

\pandocbounded{\includegraphics[keepaspectratio]{MIPD/Precision_dosing_methods_files/figure-pdf/evaluate-predictions-rambaud-train-2.pdf}}

\begin{Shaded}
\begin{Highlighting}[]
\NormalTok{crcl\_plot\_rambaud\_train }\OtherTok{\textless{}{-}}\NormalTok{ results\_rambaud\_train}\SpecialCharTok{$}\NormalTok{crcl\_plot}
\NormalTok{crcl\_plot\_rambaud\_train}
\end{Highlighting}
\end{Shaded}

\pandocbounded{\includegraphics[keepaspectratio]{MIPD/Precision_dosing_methods_files/figure-pdf/evaluate-predictions-rambaud-train-3.pdf}}

\begin{Shaded}
\begin{Highlighting}[]
\NormalTok{results\_rambaud\_train}\SpecialCharTok{$}\NormalTok{summary\_stats }
\end{Highlighting}
\end{Shaded}

\begin{verbatim}
# A tibble: 2 x 9
  Prediction_correctness mean_CREAT sd_CREAT mean_WT sd_WT mean_AGE sd_AGE Count
  <chr>                       <dbl>    <dbl>   <dbl> <dbl>    <dbl>  <dbl> <int>
1 Correct                     1.07     0.467    76.9  13.7     70.7   14.9   329
2 Incorrect                   0.902    0.613    86.0  18.5     56.7   15.5  1546
# i 1 more variable: Proportion <dbl>
\end{verbatim}

\begin{Shaded}
\begin{Highlighting}[]
\NormalTok{results\_rambaud\_test }\OtherTok{\textless{}{-}} \FunctionTok{explore\_predictions}\NormalTok{(AMOX\_CMIN\_PRED\_TEST\_RAMBAUD)}
\NormalTok{results\_rambaud\_test}\SpecialCharTok{$}\NormalTok{target\_attainment }
\end{Highlighting}
\end{Shaded}

\pandocbounded{\includegraphics[keepaspectratio]{MIPD/Precision_dosing_methods_files/figure-pdf/evaluate-predictions-rambaud-test-1.pdf}}

\begin{Shaded}
\begin{Highlighting}[]
\NormalTok{creat\_plot\_rambaud\_test }\OtherTok{\textless{}{-}}\NormalTok{ results\_rambaud\_test}\SpecialCharTok{$}\NormalTok{creat\_plot}
\NormalTok{creat\_plot\_rambaud\_test}
\end{Highlighting}
\end{Shaded}

\pandocbounded{\includegraphics[keepaspectratio]{MIPD/Precision_dosing_methods_files/figure-pdf/evaluate-predictions-rambaud-test-2.pdf}}

\begin{Shaded}
\begin{Highlighting}[]
\NormalTok{crcl\_plot\_rambaud\_test }\OtherTok{\textless{}{-}}\NormalTok{ results\_rambaud\_test}\SpecialCharTok{$}\NormalTok{crcl\_plot}
\NormalTok{crcl\_plot\_rambaud\_test}
\end{Highlighting}
\end{Shaded}

\pandocbounded{\includegraphics[keepaspectratio]{MIPD/Precision_dosing_methods_files/figure-pdf/evaluate-predictions-rambaud-test-3.pdf}}

\begin{Shaded}
\begin{Highlighting}[]
\NormalTok{results\_rambaud\_test}\SpecialCharTok{$}\NormalTok{summary\_stats }
\end{Highlighting}
\end{Shaded}

\begin{verbatim}
# A tibble: 2 x 9
  Prediction_correctness mean_CREAT sd_CREAT mean_WT sd_WT mean_AGE sd_AGE Count
  <chr>                       <dbl>    <dbl>   <dbl> <dbl>    <dbl>  <dbl> <int>
1 Correct                     1.07     0.521    77.2  15.1     73.2   14.7   114
2 Incorrect                   0.922    0.588    86.1  19.1     56.8   15.2   486
# i 1 more variable: Proportion <dbl>
\end{verbatim}

\section{Meta model}\label{meta-model}

This is an ensembling method where a PopPK model is built based on data
simulated with the four models (Carlier, Fournier, Mellon, and Rambaud).

The model is built on the training set used for the ensembling
algorithms and tested on the test set. The data is rich and does not
contain below quantification limit concentrations. The 3 three
compartment structural model was a better fit than a two compartment one
(OFV drop \textless{} 0.1 \%), however as the estimate of the second
peripheral volume was negligeable (\textless{} 1 L), the two-compartment
model was preferred. The pharmacostatistical model is developed using
the automatic statistical model building tool in Monolix 2024R1. The
best model has IIV on all parameters with no covariance between them.
The error model is combined with log-normal distribution.

The covariate model is built using the automatic SCM tool of Monolix
with respective forward and backward p values of 5 and 1 \% (Likelihood
Ratio Test). CREAT and WT were added on the clearance.

\begin{Shaded}
\begin{Highlighting}[]
\NormalTok{AMOX\_CMIN\_PRED\_TEST\_META }\OtherTok{\textless{}{-}} \FunctionTok{perform\_mrgsolve\_prediction}\NormalTok{(}
  \AttributeTok{input\_data =}\NormalTok{ AMOX\_CMIN\_OBS\_TEST ,}
  \AttributeTok{mrgsolve\_model\_path =} \FunctionTok{here}\NormalTok{(}
    \StringTok{"Amoxicillin/a\_priori/For\_publication/MIPD/amox\_Meta"}
\NormalTok{  ),}
  \AttributeTok{model\_name =} \StringTok{"META"}\NormalTok{,}
  \AttributeTok{cmin\_target =} \DecValTok{60} \CommentTok{\#mg/L}
\NormalTok{) }
\end{Highlighting}
\end{Shaded}

\begin{Shaded}
\begin{Highlighting}[]
\NormalTok{results\_meta\_test }\OtherTok{\textless{}{-}} \FunctionTok{explore\_predictions}\NormalTok{(AMOX\_CMIN\_PRED\_TEST\_META)}
\NormalTok{results\_meta\_test}\SpecialCharTok{$}\NormalTok{target\_attainment }
\end{Highlighting}
\end{Shaded}

\pandocbounded{\includegraphics[keepaspectratio]{MIPD/Precision_dosing_methods_files/figure-pdf/evaluate-predictions-meta-1.pdf}}

\begin{Shaded}
\begin{Highlighting}[]
\NormalTok{creat\_plot\_meta\_test }\OtherTok{\textless{}{-}}\NormalTok{ results\_meta\_test}\SpecialCharTok{$}\NormalTok{creat\_plot}
\NormalTok{creat\_plot\_meta\_test}
\end{Highlighting}
\end{Shaded}

\pandocbounded{\includegraphics[keepaspectratio]{MIPD/Precision_dosing_methods_files/figure-pdf/evaluate-predictions-meta-2.pdf}}

\begin{Shaded}
\begin{Highlighting}[]
\NormalTok{crcl\_plot\_meta\_test }\OtherTok{\textless{}{-}}\NormalTok{ results\_meta\_test}\SpecialCharTok{$}\NormalTok{crcl\_plot}
\NormalTok{crcl\_plot\_meta\_test}
\end{Highlighting}
\end{Shaded}

\pandocbounded{\includegraphics[keepaspectratio]{MIPD/Precision_dosing_methods_files/figure-pdf/evaluate-predictions-meta-3.pdf}}

\begin{Shaded}
\begin{Highlighting}[]
\NormalTok{results\_meta\_test}\SpecialCharTok{$}\NormalTok{summary\_stats }
\end{Highlighting}
\end{Shaded}

\begin{verbatim}
# A tibble: 2 x 9
  Prediction_correctness mean_CREAT sd_CREAT mean_WT sd_WT mean_AGE sd_AGE Count
  <chr>                       <dbl>    <dbl>   <dbl> <dbl>    <dbl>  <dbl> <int>
1 Correct                     0.816    0.279    89.0  20.3     59.5   14.2   161
2 Incorrect                   0.999    0.649    82.7  17.8     60.1   17.1   439
# i 1 more variable: Proportion <dbl>
\end{verbatim}

\chapter{Standard dose}\label{standard-dose}

Based on the our standard dose screening, out of five potential dosing
regimens guided by the
\href{https://www.infectiologie.com/UserFiles/File/spilf/recos/doses-spilf-sfpt-casfm-2023.pdf}{SPILF}
recommendation and hospital practices, 200 mg/kg is selected as standard
dose reference as it gave the highest target attainment on clinical
data.

\begin{Shaded}
\begin{Highlighting}[]
\NormalTok{AMOX\_CMIN\_PRED\_TRAIN\_STD }\OtherTok{\textless{}{-}}\NormalTok{ AMOX\_CMIN\_TRAIN }\SpecialCharTok{\%\textgreater{}\%}
  \FunctionTok{mutate}\NormalTok{(}\AttributeTok{DOSE\_PRED =}\NormalTok{ (}\DecValTok{200} \SpecialCharTok{*}\NormalTok{ WT) }\SpecialCharTok{/}\NormalTok{ (}\DecValTok{24}\SpecialCharTok{/}\NormalTok{FREQ)) }\SpecialCharTok{\%\textgreater{}\%}
\NormalTok{  dplyr}\SpecialCharTok{::}\FunctionTok{select}\NormalTok{(ID, CREAT, WT, BURN, SEX, AGE, HT, FREQ, DOSE\_ADM, DOSE\_PRED, CMIN\_IND, SEX, BSA, CRCL\_CKD\_EPI\_ABSOLUTE) }\SpecialCharTok{\%\textgreater{}\%}
  \FunctionTok{distinct}\NormalTok{()}

\NormalTok{AMOX\_CMIN\_PRED\_TEST\_STD }\OtherTok{\textless{}{-}}\NormalTok{ AMOX\_CMIN\_TEST }\SpecialCharTok{\%\textgreater{}\%}
  \FunctionTok{mutate}\NormalTok{(}\AttributeTok{DOSE\_PRED =}\NormalTok{ (}\DecValTok{200} \SpecialCharTok{*}\NormalTok{ WT) }\SpecialCharTok{/}\NormalTok{ (}\DecValTok{24}\SpecialCharTok{/}\NormalTok{FREQ)) }\SpecialCharTok{\%\textgreater{}\%}
\NormalTok{  dplyr}\SpecialCharTok{::}\FunctionTok{select}\NormalTok{(ID, CREAT, WT, BURN, SEX, AGE, HT, FREQ, DOSE\_ADM, DOSE\_PRED, CMIN\_IND, SEX, BSA, CRCL\_CKD\_EPI\_ABSOLUTE) }\SpecialCharTok{\%\textgreater{}\%}
  \FunctionTok{distinct}\NormalTok{()}
\end{Highlighting}
\end{Shaded}

\begin{Shaded}
\begin{Highlighting}[]
\NormalTok{results\_standard\_train }\OtherTok{\textless{}{-}} \FunctionTok{explore\_predictions}\NormalTok{(AMOX\_CMIN\_PRED\_TRAIN\_STD)}
\NormalTok{results\_standard\_train}\SpecialCharTok{$}\NormalTok{target\_attainment }
\end{Highlighting}
\end{Shaded}

\pandocbounded{\includegraphics[keepaspectratio]{MIPD/Precision_dosing_methods_files/figure-pdf/evaluate-prediction-standard-dose-train-1.pdf}}

\begin{Shaded}
\begin{Highlighting}[]
\NormalTok{results\_standard\_train}\SpecialCharTok{$}\NormalTok{creat\_plot}
\end{Highlighting}
\end{Shaded}

\pandocbounded{\includegraphics[keepaspectratio]{MIPD/Precision_dosing_methods_files/figure-pdf/evaluate-prediction-standard-dose-train-2.pdf}}

\begin{Shaded}
\begin{Highlighting}[]
\NormalTok{results\_standard\_train}\SpecialCharTok{$}\NormalTok{summary\_stats }
\end{Highlighting}
\end{Shaded}

\begin{verbatim}
# A tibble: 2 x 9
  Prediction_correctness mean_CREAT sd_CREAT mean_WT sd_WT mean_AGE sd_AGE Count
  <chr>                       <dbl>    <dbl>   <dbl> <dbl>    <dbl>  <dbl> <int>
1 Correct                     1.14     0.655    76.6  12.4     67.7   15.9   312
2 Incorrect                   0.889    0.571    85.9  18.6     57.5   15.9  1563
# i 1 more variable: Proportion <dbl>
\end{verbatim}

\begin{Shaded}
\begin{Highlighting}[]
\NormalTok{results\_standard\_test }\OtherTok{\textless{}{-}} \FunctionTok{explore\_predictions}\NormalTok{(AMOX\_CMIN\_PRED\_TEST\_STD)}
\NormalTok{results\_standard\_test}\SpecialCharTok{$}\NormalTok{target\_attainment }
\end{Highlighting}
\end{Shaded}

\pandocbounded{\includegraphics[keepaspectratio]{MIPD/Precision_dosing_methods_files/figure-pdf/evaluate-prediction-standard-dose-test-1.pdf}}

\begin{Shaded}
\begin{Highlighting}[]
\NormalTok{results\_standard\_test}\SpecialCharTok{$}\NormalTok{creat\_plot}
\end{Highlighting}
\end{Shaded}

\pandocbounded{\includegraphics[keepaspectratio]{MIPD/Precision_dosing_methods_files/figure-pdf/evaluate-prediction-standard-dose-test-2.pdf}}

\begin{Shaded}
\begin{Highlighting}[]
\NormalTok{results\_standard\_test}\SpecialCharTok{$}\NormalTok{summary\_stats }
\end{Highlighting}
\end{Shaded}

\begin{verbatim}
# A tibble: 2 x 9
  Prediction_correctness mean_CREAT sd_CREAT mean_WT sd_WT mean_AGE sd_AGE Count
  <chr>                       <dbl>    <dbl>   <dbl> <dbl>    <dbl>  <dbl> <int>
1 Correct                     1.05     0.497    77.9  13.7     70.2   14.1   107
2 Incorrect                   0.929    0.593    85.8  19.3     57.7   16.0   493
# i 1 more variable: Proportion <dbl>
\end{verbatim}

\chapter{Uninformed model ensembling}\label{uninformed-model-ensembling}

All the models are given the same weight (one quarter in this case). The
performance of the models or the development cohort characteristics are
not taken into account.

\begin{Shaded}
\begin{Highlighting}[]
\NormalTok{AMOX\_CMIN\_PRED\_TRAIN\_UNINF\_MOD\_ENS }\OtherTok{\textless{}{-}}\NormalTok{  AMOX\_CMIN\_TRAIN }\SpecialCharTok{\%\textgreater{}\%}
  \FunctionTok{mutate}\NormalTok{(}\AttributeTok{WEIGHT =} \FloatTok{0.25}\NormalTok{) }\SpecialCharTok{\%\textgreater{}\%} \CommentTok{\# Add uniform weight}
  \FunctionTok{mutate}\NormalTok{(}\AttributeTok{WEIGHTED\_PRED =}\NormalTok{ CMIN\_PRED }\SpecialCharTok{*}\NormalTok{ WEIGHT) }\SpecialCharTok{\%\textgreater{}\%} \CommentTok{\# Weigh prediction}
  \FunctionTok{group\_by}\NormalTok{(ID) }\SpecialCharTok{\%\textgreater{}\%}
  \FunctionTok{mutate}\NormalTok{(}\AttributeTok{WEIGHTED\_PREDICTION =} \FunctionTok{sum}\NormalTok{(WEIGHTED\_PRED)) }\SpecialCharTok{\%\textgreater{}\%} \CommentTok{\# Ensemble weighted predictions}
  \FunctionTok{ungroup}\NormalTok{() }\SpecialCharTok{\%\textgreater{}\%}
\NormalTok{  dplyr}\SpecialCharTok{::}\FunctionTok{select}\NormalTok{(ID, CREAT, WT, BURN, SEX, AGE, HT, FREQ, DOSE\_ADM, WEIGHTED\_PREDICTION, CMIN\_IND, SEX, CRCL\_CKD\_EPI\_ABSOLUTE) }\SpecialCharTok{\%\textgreater{}\%}
  \FunctionTok{distinct}\NormalTok{() }\SpecialCharTok{\%\textgreater{}\%}
\NormalTok{  dplyr}\SpecialCharTok{::}\FunctionTok{mutate}\NormalTok{(}\AttributeTok{DOSE\_PRED =}\NormalTok{ (}\DecValTok{60}\SpecialCharTok{/}\NormalTok{WEIGHTED\_PREDICTION) }\SpecialCharTok{*}\NormalTok{ DOSE\_ADM) }\CommentTok{\# Administered dose extrapolation to reach 60 mg/L}

\NormalTok{AMOX\_CMIN\_PRED\_TEST\_UNINF\_MOD\_ENS }\OtherTok{\textless{}{-}}\NormalTok{ AMOX\_CMIN\_TEST }\SpecialCharTok{\%\textgreater{}\%}
  \FunctionTok{mutate}\NormalTok{(}\AttributeTok{WEIGHT =} \FloatTok{0.25}\NormalTok{) }\SpecialCharTok{\%\textgreater{}\%} \CommentTok{\# Add uniform weight}
  \FunctionTok{mutate}\NormalTok{(}\AttributeTok{WEIGHTED\_PRED =}\NormalTok{ CMIN\_PRED }\SpecialCharTok{*}\NormalTok{ WEIGHT) }\SpecialCharTok{\%\textgreater{}\%} \CommentTok{\# Weigh prediction}
  \FunctionTok{group\_by}\NormalTok{(ID) }\SpecialCharTok{\%\textgreater{}\%}
  \FunctionTok{mutate}\NormalTok{(}\AttributeTok{WEIGHTED\_PREDICTION =} \FunctionTok{sum}\NormalTok{(WEIGHTED\_PRED)) }\SpecialCharTok{\%\textgreater{}\%} \CommentTok{\# Ensemble weighted predictions}
  \FunctionTok{ungroup}\NormalTok{() }\SpecialCharTok{\%\textgreater{}\%}
\NormalTok{  dplyr}\SpecialCharTok{::}\FunctionTok{select}\NormalTok{(ID, CREAT, WT, BURN, SEX, AGE, HT, FREQ, DOSE\_ADM, WEIGHTED\_PREDICTION, CMIN\_IND, SEX, CRCL\_CKD\_EPI\_ABSOLUTE) }\SpecialCharTok{\%\textgreater{}\%}
  \FunctionTok{distinct}\NormalTok{() }\SpecialCharTok{\%\textgreater{}\%}
\NormalTok{  dplyr}\SpecialCharTok{::}\FunctionTok{mutate}\NormalTok{(}\AttributeTok{DOSE\_PRED =}\NormalTok{ (}\DecValTok{60}\SpecialCharTok{/}\NormalTok{WEIGHTED\_PREDICTION) }\SpecialCharTok{*}\NormalTok{ DOSE\_ADM) }\CommentTok{\# Administered dose extrapolation to reach 60 mg/L}
\end{Highlighting}
\end{Shaded}

\begin{Shaded}
\begin{Highlighting}[]
\NormalTok{results\_uninf\_mod\_ens\_train }\OtherTok{\textless{}{-}} \FunctionTok{explore\_predictions}\NormalTok{(AMOX\_CMIN\_PRED\_TRAIN\_UNINF\_MOD\_ENS)}
\NormalTok{results\_uninf\_mod\_ens\_train}\SpecialCharTok{$}\NormalTok{target\_attainment }
\end{Highlighting}
\end{Shaded}

\pandocbounded{\includegraphics[keepaspectratio]{MIPD/Precision_dosing_methods_files/figure-pdf/evaluate-prediction-uninf-mod-ens-train-1.pdf}}

\begin{Shaded}
\begin{Highlighting}[]
\NormalTok{results\_uninf\_mod\_ens\_train}\SpecialCharTok{$}\NormalTok{creat\_plot}
\end{Highlighting}
\end{Shaded}

\pandocbounded{\includegraphics[keepaspectratio]{MIPD/Precision_dosing_methods_files/figure-pdf/evaluate-prediction-uninf-mod-ens-train-2.pdf}}

\begin{Shaded}
\begin{Highlighting}[]
\NormalTok{results\_uninf\_mod\_ens\_train}\SpecialCharTok{$}\NormalTok{summary\_stats }
\end{Highlighting}
\end{Shaded}

\begin{verbatim}
# A tibble: 2 x 9
  Prediction_correctness mean_CREAT sd_CREAT mean_WT sd_WT mean_AGE sd_AGE Count
  <chr>                       <dbl>    <dbl>   <dbl> <dbl>    <dbl>  <dbl> <int>
1 Correct                     0.952    0.528    83.6  17.9     63.0   16.7   611
2 Incorrect                   0.921    0.623    84.7  18.2     57.4   15.8  1264
# i 1 more variable: Proportion <dbl>
\end{verbatim}

\begin{Shaded}
\begin{Highlighting}[]
\NormalTok{results\_uninf\_mod\_ens\_test }\OtherTok{\textless{}{-}} \FunctionTok{explore\_predictions}\NormalTok{(AMOX\_CMIN\_PRED\_TEST\_UNINF\_MOD\_ENS)}
\NormalTok{results\_uninf\_mod\_ens\_test}\SpecialCharTok{$}\NormalTok{target\_attainment }
\end{Highlighting}
\end{Shaded}

\pandocbounded{\includegraphics[keepaspectratio]{MIPD/Precision_dosing_methods_files/figure-pdf/evaluate-prediction-uninf-mod-ens-test-1.pdf}}

\begin{Shaded}
\begin{Highlighting}[]
\NormalTok{results\_uninf\_mod\_ens\_test}\SpecialCharTok{$}\NormalTok{creat\_plot}
\end{Highlighting}
\end{Shaded}

\pandocbounded{\includegraphics[keepaspectratio]{MIPD/Precision_dosing_methods_files/figure-pdf/evaluate-prediction-uninf-mod-ens-test-2.pdf}}

\begin{Shaded}
\begin{Highlighting}[]
\NormalTok{results\_uninf\_mod\_ens\_test}\SpecialCharTok{$}\NormalTok{summary\_stats }
\end{Highlighting}
\end{Shaded}

\begin{verbatim}
# A tibble: 2 x 9
  Prediction_correctness mean_CREAT sd_CREAT mean_WT sd_WT mean_AGE sd_AGE Count
  <chr>                       <dbl>    <dbl>   <dbl> <dbl>    <dbl>  <dbl> <int>
1 Correct                     0.934    0.473    83.8  19.2     63.6   16.1   225
2 Incorrect                   0.960    0.634    84.8  18.4     57.7   16.2   375
# i 1 more variable: Proportion <dbl>
\end{verbatim}

\chapter{Nomogram}\label{nomogram}

The nomogram developed by Rambaud et al. (2020) in Nantes which predicts
the dose in endocarditis patients based solely on absolute glomerular
filtration rate.

The daily dose in g to reach 20 mg/L is predicted by the nomogram based
on CRCL in mL/min (x) estimated using the CKD-EPI equation \[
\text{Dose} = 0.0001x^2 + 0.0613x + 1.157
\] and then extrapolated to attain 60 mg/L.

Although the nomogram has been validated only for CRCL between 30 and
120 mL/min and continuous infusion, in this study, it was also evaluated
in patients with CRCL outside of this range and applied to intermittent
infusion.

\begin{Shaded}
\begin{Highlighting}[]
\NormalTok{AMOX\_CMIN\_PRED\_TRAIN\_NOMOGRAM }\OtherTok{\textless{}{-}}\NormalTok{  AMOX\_CMIN\_TRAIN }\SpecialCharTok{\%\textgreater{}\%}
\NormalTok{    dplyr}\SpecialCharTok{::}\FunctionTok{select}\NormalTok{(ID, CREAT, WT, BURN, SEX, AGE, HT, FREQ, DOSE\_ADM, CMIN\_IND, SEX, CRCL\_CKD\_EPI\_ABSOLUTE) }\SpecialCharTok{\%\textgreater{}\%}
  \FunctionTok{mutate}\NormalTok{(}\AttributeTok{DOSE\_PRED =}\NormalTok{ ((}\FloatTok{0.0001} \SpecialCharTok{*}\NormalTok{ CRCL\_CKD\_EPI\_ABSOLUTE}\SpecialCharTok{\^{}}\DecValTok{2} \SpecialCharTok{+} \FloatTok{0.0613} \SpecialCharTok{*}\NormalTok{ CRCL\_CKD\_EPI\_ABSOLUTE }\SpecialCharTok{+} \FloatTok{1.157}\NormalTok{) }\SpecialCharTok{*} \DecValTok{3000}\NormalTok{) }\SpecialCharTok{/}\NormalTok{ (}\DecValTok{24}\SpecialCharTok{/}\NormalTok{FREQ)) }\SpecialCharTok{|\textgreater{}} 
  \FunctionTok{distinct}\NormalTok{()}

\NormalTok{AMOX\_CMIN\_PRED\_TEST\_NOMOGRAM }\OtherTok{\textless{}{-}}\NormalTok{ AMOX\_CMIN\_TEST }\SpecialCharTok{\%\textgreater{}\%}
\NormalTok{  dplyr}\SpecialCharTok{::}\FunctionTok{select}\NormalTok{(ID, CREAT, WT, BURN, SEX, AGE, HT, FREQ, DOSE\_ADM, CMIN\_IND, SEX, CRCL\_CKD\_EPI\_ABSOLUTE) }\SpecialCharTok{\%\textgreater{}\%}
  \FunctionTok{mutate}\NormalTok{(}\AttributeTok{DOSE\_PRED =}\NormalTok{ ((}\FloatTok{0.0001} \SpecialCharTok{*}\NormalTok{ CRCL\_CKD\_EPI\_ABSOLUTE}\SpecialCharTok{\^{}}\DecValTok{2} \SpecialCharTok{+} \FloatTok{0.0613} \SpecialCharTok{*}\NormalTok{ CRCL\_CKD\_EPI\_ABSOLUTE }\SpecialCharTok{+} \FloatTok{1.157}\NormalTok{) }\SpecialCharTok{*} \DecValTok{3000}\NormalTok{) }\SpecialCharTok{/}\NormalTok{ (}\DecValTok{24}\SpecialCharTok{/}\NormalTok{FREQ)) }\SpecialCharTok{|\textgreater{}} 
  \FunctionTok{distinct}\NormalTok{() }
\end{Highlighting}
\end{Shaded}

\begin{Shaded}
\begin{Highlighting}[]
\NormalTok{results\_nomogram\_train }\OtherTok{\textless{}{-}} \FunctionTok{explore\_predictions}\NormalTok{(AMOX\_CMIN\_PRED\_TRAIN\_NOMOGRAM)}
\NormalTok{results\_nomogram\_train}\SpecialCharTok{$}\NormalTok{target\_attainment }
\end{Highlighting}
\end{Shaded}

\pandocbounded{\includegraphics[keepaspectratio]{MIPD/Precision_dosing_methods_files/figure-pdf/evaluate-prediction-nomogram-train-1.pdf}}

\begin{Shaded}
\begin{Highlighting}[]
\NormalTok{results\_nomogram\_train}\SpecialCharTok{$}\NormalTok{creat\_plot}
\end{Highlighting}
\end{Shaded}

\pandocbounded{\includegraphics[keepaspectratio]{MIPD/Precision_dosing_methods_files/figure-pdf/evaluate-prediction-nomogram-train-2.pdf}}

\begin{Shaded}
\begin{Highlighting}[]
\NormalTok{results\_nomogram\_train}\SpecialCharTok{$}\NormalTok{summary\_stats }
\end{Highlighting}
\end{Shaded}

\begin{verbatim}
# A tibble: 2 x 9
  Prediction_correctness mean_CREAT sd_CREAT mean_WT sd_WT mean_AGE sd_AGE Count
  <chr>                       <dbl>    <dbl>   <dbl> <dbl>    <dbl>  <dbl> <int>
1 Correct                     1.08     0.592    75.3  11.9     67.8   15.9   379
2 Incorrect                   0.893    0.588    86.7  18.7     57.0   15.7  1496
# i 1 more variable: Proportion <dbl>
\end{verbatim}

\begin{Shaded}
\begin{Highlighting}[]
\NormalTok{results\_nomogram\_test }\OtherTok{\textless{}{-}} \FunctionTok{explore\_predictions}\NormalTok{(AMOX\_CMIN\_PRED\_TEST\_NOMOGRAM)}
\NormalTok{results\_nomogram\_test}\SpecialCharTok{$}\NormalTok{target\_attainment }
\end{Highlighting}
\end{Shaded}

\pandocbounded{\includegraphics[keepaspectratio]{MIPD/Precision_dosing_methods_files/figure-pdf/evaluate-prediction-nomogram-test-1.pdf}}

\begin{Shaded}
\begin{Highlighting}[]
\NormalTok{results\_nomogram\_test}\SpecialCharTok{$}\NormalTok{creat\_plot}
\end{Highlighting}
\end{Shaded}

\pandocbounded{\includegraphics[keepaspectratio]{MIPD/Precision_dosing_methods_files/figure-pdf/evaluate-prediction-nomogram-test-2.pdf}}

\begin{Shaded}
\begin{Highlighting}[]
\NormalTok{results\_nomogram\_test}\SpecialCharTok{$}\NormalTok{summary\_stats }
\end{Highlighting}
\end{Shaded}

\begin{verbatim}
# A tibble: 2 x 9
  Prediction_correctness mean_CREAT sd_CREAT mean_WT sd_WT mean_AGE sd_AGE Count
  <chr>                       <dbl>    <dbl>   <dbl> <dbl>    <dbl>  <dbl> <int>
1 Correct                     1.10     0.577    75.6  12.8     67.1   15.8   112
2 Incorrect                   0.915    0.574    86.4  19.2     58.3   16.1   488
# i 1 more variable: Proportion <dbl>
\end{verbatim}

\chapter{Classification tree informed
ensembling}\label{classification-tree-informed-ensembling}

This method is used to attribute weights to models based on their
probability to give correct predictions for a set of covariates.

The method is based on Agema et al. (2024).

Bioequivalence ratios between \(Cmin_{\text{pred}}\) and
\(Cmin_{\text{ind}}\) are calculated and transformed to a binary
variable based on if the ratio is in the bioequivalence range
{[}0.80-1.25{]} or not. This categorical variable is called CORRECT.

First, for each set of covariates, the ratios between the predicted
values of the 4 models are compared to the \emph{true} value. Then, we
add the variable CORRECT which takes the value of YES if the ratio is in
the bioequivalence range, and the value of NO, if not.

Then, decision tree is developed where the predictors are the seven
covariates and a binary variable indicating continuous or intermittent
infusion (CON), and the target variable is the categorical variable
named CORRECT. A decision tree is fitted separately for each MODEL where
the following parameters need to be defined

- \textbf{Complexity} -- a penalty that the information gain has to
overcome to add a split. A higher complexity value leads to a smaller
tree meaning that subjects are divided into smaller number of
categories, so their scores will be less diverse. Comlexity is found by
automatic hyperparameter tuning based on cross-validation error.\\
\textbf{Minsplit} -- minimum number of samples to split a node (smaller
value, deeper tree).\\
\textbf{Minbucket} -- minimum number of samples in the terminal node
(smaller value, deeper tree). Minsplit and minbucket in the range of
2-30 range does not affect the results , so 4 is fixed as a value to
have deeper trees.\\
- \textbf{xval} -- number of cross validations. 10-fold cross validation
is used (the same number as for the ML methods).\\
- \textbf{splitting criterion} - entropy or Gini impurity. Gini
meausures the frequency (f) with which an element is misclassified:

\[ 
  Gini\ impurity = 2 \cdot f \cdot \left(1 - f \right) \
\] Entropy (called information in rpart) measures the disorder of the
predictors as a function of the target:

\[ 
  Entropy = f \cdot log_2(f)
\]

Normally, the two splitting criteria give similar results. Entropy is
more computationally complex as it uses logarithms.

For certain models, the root node is not split any further (same model
score for all subjects).\\
In the obtained leaf\_results table, only the leaf nodes are included
with the following columns

\begin{itemize}
\item
  \textbf{node} - node number
\item
  \textbf{var} - splitting variable (it always says leaf, as only leaves
  are included in the table)
\item
  \textbf{n}- number of subjects per node
\item
  \textbf{wt} - subject weight (all the subject have the same weight)
\item
  \textbf{dev} - Gini index
\item
  \textbf{yval} - the predicted class
\item
  \textbf{ncompete} - Alternative splitting variables that were not
  selected for a node but give slightly worse performance than the
  selected variable. As only leaves are included, there are no competing
  splitting variables.
\item
  \textbf{nsurrogate} - Used for missing values. Not applicable to
  leaves.
\item
  \textbf{yval2{[}, 2{]}} - number of NO for the CORRECT variable
\item
  \textbf{yval2{[}, 3{]}} - number of YES
\item
  \textbf{yval2{[}, 4{]}} - proportion of NO
\item
  \textbf{yval2{[}, 5{]}} - proportion of YES
\item
  \textbf{prob\_yes} - same as yval2{[}, 5{]}, this column is used later
  to attribute model weights
\item
  \textbf{n\_obs} - number of observations
\end{itemize}

To this table, in the path column is added containing all the decisions
that led to that particular leaf. Then, this string is transformed to
lower and upper limits for continuous and categories for categorical
covariates. If there are no upper or lower limits or selected
categories, NA is added.

In the original method ( Agema et al. (2024)), the leaves are selected
with at least a YES proportion of 0.5 and then the selected models are
taken into account with equal weight. Here, all models are used with
their respective YES proportions. This way, no subject will end up
without a model, and the weights will be more nuancedly calculated.
Then, for each subject, the four corresponding nodes are found based on
its covariates (one for each model). The probabilites (proportions) of
having YES (=prob\_yes or score) are normalized by their sum for each
subject, so that the scores of the models for a subject will always add
up to 1. These normalized scores (=weights) are used to ensemble the
model predictions.

Finally, the method is tested by ensembling the predictions of the test
data using the weights calibrated base on the training data. The
administered dose is extrapolated by dividing the weighted concentration
prediction by the target concentration (= 60 mg/L).

Our \emph{openMIPD} package is used to apply this method.

\begin{Shaded}
\begin{Highlighting}[]
\NormalTok{AMOX\_CMIN\_TRAINED\_CT }\OtherTok{\textless{}{-}} \FunctionTok{classification\_tree\_model\_ensembling\_train}\NormalTok{(}
  \AttributeTok{data =}\NormalTok{ AMOX\_CMIN\_TRAIN, }
  \AttributeTok{target\_variable =} \StringTok{"CMIN"}\NormalTok{, }
  \AttributeTok{continuous\_cov =} \FunctionTok{c}\NormalTok{(}\StringTok{"WT"}\NormalTok{, }\StringTok{"CREAT"}\NormalTok{, }\StringTok{"AGE"}\NormalTok{), }
  \AttributeTok{categorical\_cov =} \FunctionTok{c}\NormalTok{(}\StringTok{"OBESE"}\NormalTok{, }\StringTok{"ICU"}\NormalTok{, }\StringTok{"BURN"}\NormalTok{, }\StringTok{"SEX"}\NormalTok{, }\StringTok{"CON"}\NormalTok{))}
\end{Highlighting}
\end{Shaded}

\pandocbounded{\includegraphics[keepaspectratio]{MIPD/Precision_dosing_methods_files/figure-pdf/train-ct-1.pdf}}

\begin{verbatim}
[1] "Confusion matrix for model: CARLIER"

 node number: 2 
   root
   ICU=0

 node number: 6 
   root
   ICU=1
   BURN=1

 node number: 28 
   root
   ICU=1
   BURN=0
   CON=0
   CREAT< 3.217

 node number: 29 
   root
   ICU=1
   BURN=0
   CON=0
   CREAT>=3.217

 node number: 60 
   root
   ICU=1
   BURN=0
   CON=1
   CREAT>=0.7062
   WT< 91.37

 node number: 122 
   root
   ICU=1
   BURN=0
   CON=1
   CREAT>=0.7062
   WT>=91.37
   AGE< 72.47

 node number: 123 
   root
   ICU=1
   BURN=0
   CON=1
   CREAT>=0.7062
   WT>=91.37
   AGE>=72.47

 node number: 62 
   root
   ICU=1
   BURN=0
   CON=1
   CREAT< 0.7062
   AGE< 43.76

 node number: 504 
   root
   ICU=1
   BURN=0
   CON=1
   CREAT< 0.7062
   AGE>=43.76
   WT< 83.02
   WT>=68.9
   CREAT< 0.6348

 node number: 505 
   root
   ICU=1
   BURN=0
   CON=1
   CREAT< 0.7062
   AGE>=43.76
   WT< 83.02
   WT>=68.9
   CREAT>=0.6348

 node number: 253 
   root
   ICU=1
   BURN=0
   CON=1
   CREAT< 0.7062
   AGE>=43.76
   WT< 83.02
   WT< 68.9

 node number: 127 
   root
   ICU=1
   BURN=0
   CON=1
   CREAT< 0.7062
   AGE>=43.76
   WT>=83.02
\end{verbatim}

\pandocbounded{\includegraphics[keepaspectratio]{MIPD/Precision_dosing_methods_files/figure-pdf/train-ct-2.pdf}}

\begin{verbatim}
[1] "Confusion matrix for model: FOURNIER"

 node number: 2 
   root
   CON=0

 node number: 6 
   root
   CON=1
   CREAT>=1.467

 node number: 28 
   root
   CON=1
   CREAT< 1.467
   CREAT>=0.8499
   AGE>=90.15

 node number: 29 
   root
   CON=1
   CREAT< 1.467
   CREAT>=0.8499
   AGE< 90.15

 node number: 15 
   root
   CON=1
   CREAT< 1.467
   CREAT< 0.8499
\end{verbatim}

\pandocbounded{\includegraphics[keepaspectratio]{MIPD/Precision_dosing_methods_files/figure-pdf/train-ct-3.pdf}}

\begin{verbatim}
[1] "Confusion matrix for model: MELLON"

 node number: 1 
   root
\end{verbatim}

\pandocbounded{\includegraphics[keepaspectratio]{MIPD/Precision_dosing_methods_files/figure-pdf/train-ct-4.pdf}}

\begin{verbatim}
[1] "Confusion matrix for model: RAMBAUD"

 node number: 2 
   root
   CON=0

 node number: 6 
   root
   CON=1
   CREAT>=1.942

 node number: 28 
   root
   CON=1
   CREAT< 1.942
   CREAT< 1.645
   CREAT>=1.62

 node number: 232 
   root
   CON=1
   CREAT< 1.942
   CREAT< 1.645
   CREAT< 1.62
   CREAT< 1.512
   CREAT>=0.951
   CREAT< 0.9689

 node number: 932 
   root
   CON=1
   CREAT< 1.942
   CREAT< 1.645
   CREAT< 1.62
   CREAT< 1.512
   CREAT>=0.951
   CREAT>=0.9689
   AGE< 84.53
   CREAT>=1.467

 node number: 3732 
   root
   CON=1
   CREAT< 1.942
   CREAT< 1.645
   CREAT< 1.62
   CREAT< 1.512
   CREAT>=0.951
   CREAT>=0.9689
   AGE< 84.53
   CREAT< 1.467
   WT< 76.02
   WT>=70.93

 node number: 7466 
   root
   CON=1
   CREAT< 1.942
   CREAT< 1.645
   CREAT< 1.62
   CREAT< 1.512
   CREAT>=0.951
   CREAT>=0.9689
   AGE< 84.53
   CREAT< 1.467
   WT< 76.02
   WT< 70.93
   CREAT< 0.9979

 node number: 29868 
   root
   CON=1
   CREAT< 1.942
   CREAT< 1.645
   CREAT< 1.62
   CREAT< 1.512
   CREAT>=0.951
   CREAT>=0.9689
   AGE< 84.53
   CREAT< 1.467
   WT< 76.02
   WT< 70.93
   CREAT>=0.9979
   CREAT>=1.029
   CREAT< 1.147

 node number: 59738 
   root
   CON=1
   CREAT< 1.942
   CREAT< 1.645
   CREAT< 1.62
   CREAT< 1.512
   CREAT>=0.951
   CREAT>=0.9689
   AGE< 84.53
   CREAT< 1.467
   WT< 76.02
   WT< 70.93
   CREAT>=0.9979
   CREAT>=1.029
   CREAT>=1.147
   AGE>=76.56

 node number: 119478 
   root
   CON=1
   CREAT< 1.942
   CREAT< 1.645
   CREAT< 1.62
   CREAT< 1.512
   CREAT>=0.951
   CREAT>=0.9689
   AGE< 84.53
   CREAT< 1.467
   WT< 76.02
   WT< 70.93
   CREAT>=0.9979
   CREAT>=1.029
   CREAT>=1.147
   AGE< 76.56
   WT>=64.87

 node number: 119479 
   root
   CON=1
   CREAT< 1.942
   CREAT< 1.645
   CREAT< 1.62
   CREAT< 1.512
   CREAT>=0.951
   CREAT>=0.9689
   AGE< 84.53
   CREAT< 1.467
   WT< 76.02
   WT< 70.93
   CREAT>=0.9979
   CREAT>=1.029
   CREAT>=1.147
   AGE< 76.56
   WT< 64.87

 node number: 14935 
   root
   CON=1
   CREAT< 1.942
   CREAT< 1.645
   CREAT< 1.62
   CREAT< 1.512
   CREAT>=0.951
   CREAT>=0.9689
   AGE< 84.53
   CREAT< 1.467
   WT< 76.02
   WT< 70.93
   CREAT>=0.9979
   CREAT< 1.029

 node number: 29872 
   root
   CON=1
   CREAT< 1.942
   CREAT< 1.645
   CREAT< 1.62
   CREAT< 1.512
   CREAT>=0.951
   CREAT>=0.9689
   AGE< 84.53
   CREAT< 1.467
   WT>=76.02
   WT>=76.98
   CREAT< 1.25
   CREAT>=1.021
   AGE>=77.47

 node number: 119492 
   root
   CON=1
   CREAT< 1.942
   CREAT< 1.645
   CREAT< 1.62
   CREAT< 1.512
   CREAT>=0.951
   CREAT>=0.9689
   AGE< 84.53
   CREAT< 1.467
   WT>=76.02
   WT>=76.98
   CREAT< 1.25
   CREAT>=1.021
   AGE< 77.47
   WT>=83.65
   CREAT< 1.075

 node number: 238986 
   root
   CON=1
   CREAT< 1.942
   CREAT< 1.645
   CREAT< 1.62
   CREAT< 1.512
   CREAT>=0.951
   CREAT>=0.9689
   AGE< 84.53
   CREAT< 1.467
   WT>=76.02
   WT>=76.98
   CREAT< 1.25
   CREAT>=1.021
   AGE< 77.47
   WT>=83.65
   CREAT>=1.075
   WT< 96.2

 node number: 238987 
   root
   CON=1
   CREAT< 1.942
   CREAT< 1.645
   CREAT< 1.62
   CREAT< 1.512
   CREAT>=0.951
   CREAT>=0.9689
   AGE< 84.53
   CREAT< 1.467
   WT>=76.02
   WT>=76.98
   CREAT< 1.25
   CREAT>=1.021
   AGE< 77.47
   WT>=83.65
   CREAT>=1.075
   WT>=96.2

 node number: 59747 
   root
   CON=1
   CREAT< 1.942
   CREAT< 1.645
   CREAT< 1.62
   CREAT< 1.512
   CREAT>=0.951
   CREAT>=0.9689
   AGE< 84.53
   CREAT< 1.467
   WT>=76.02
   WT>=76.98
   CREAT< 1.25
   CREAT>=1.021
   AGE< 77.47
   WT< 83.65

 node number: 29874 
   root
   CON=1
   CREAT< 1.942
   CREAT< 1.645
   CREAT< 1.62
   CREAT< 1.512
   CREAT>=0.951
   CREAT>=0.9689
   AGE< 84.53
   CREAT< 1.467
   WT>=76.02
   WT>=76.98
   CREAT< 1.25
   CREAT< 1.021
   WT< 88.56

 node number: 29875 
   root
   CON=1
   CREAT< 1.942
   CREAT< 1.645
   CREAT< 1.62
   CREAT< 1.512
   CREAT>=0.951
   CREAT>=0.9689
   AGE< 84.53
   CREAT< 1.467
   WT>=76.02
   WT>=76.98
   CREAT< 1.25
   CREAT< 1.021
   WT>=88.56

 node number: 59752 
   root
   CON=1
   CREAT< 1.942
   CREAT< 1.645
   CREAT< 1.62
   CREAT< 1.512
   CREAT>=0.951
   CREAT>=0.9689
   AGE< 84.53
   CREAT< 1.467
   WT>=76.02
   WT>=76.98
   CREAT>=1.25
   CREAT>=1.272
   WT< 85.56
   CREAT< 1.32

 node number: 59753 
   root
   CON=1
   CREAT< 1.942
   CREAT< 1.645
   CREAT< 1.62
   CREAT< 1.512
   CREAT>=0.951
   CREAT>=0.9689
   AGE< 84.53
   CREAT< 1.467
   WT>=76.02
   WT>=76.98
   CREAT>=1.25
   CREAT>=1.272
   WT< 85.56
   CREAT>=1.32

 node number: 59754 
   root
   CON=1
   CREAT< 1.942
   CREAT< 1.645
   CREAT< 1.62
   CREAT< 1.512
   CREAT>=0.951
   CREAT>=0.9689
   AGE< 84.53
   CREAT< 1.467
   WT>=76.02
   WT>=76.98
   CREAT>=1.25
   CREAT>=1.272
   WT>=85.56
   WT>=97.42

 node number: 59755 
   root
   CON=1
   CREAT< 1.942
   CREAT< 1.645
   CREAT< 1.62
   CREAT< 1.512
   CREAT>=0.951
   CREAT>=0.9689
   AGE< 84.53
   CREAT< 1.467
   WT>=76.02
   WT>=76.98
   CREAT>=1.25
   CREAT>=1.272
   WT>=85.56
   WT< 97.42

 node number: 14939 
   root
   CON=1
   CREAT< 1.942
   CREAT< 1.645
   CREAT< 1.62
   CREAT< 1.512
   CREAT>=0.951
   CREAT>=0.9689
   AGE< 84.53
   CREAT< 1.467
   WT>=76.02
   WT>=76.98
   CREAT>=1.25
   CREAT< 1.272

 node number: 3735 
   root
   CON=1
   CREAT< 1.942
   CREAT< 1.645
   CREAT< 1.62
   CREAT< 1.512
   CREAT>=0.951
   CREAT>=0.9689
   AGE< 84.53
   CREAT< 1.467
   WT>=76.02
   WT< 76.98

 node number: 3736 
   root
   CON=1
   CREAT< 1.942
   CREAT< 1.645
   CREAT< 1.62
   CREAT< 1.512
   CREAT>=0.951
   CREAT>=0.9689
   AGE>=84.53
   WT< 75.38
   CREAT< 1.41
   CREAT>=1.305

 node number: 7474 
   root
   CON=1
   CREAT< 1.942
   CREAT< 1.645
   CREAT< 1.62
   CREAT< 1.512
   CREAT>=0.951
   CREAT>=0.9689
   AGE>=84.53
   WT< 75.38
   CREAT< 1.41
   CREAT< 1.305
   WT>=71.86

 node number: 14950 
   root
   CON=1
   CREAT< 1.942
   CREAT< 1.645
   CREAT< 1.62
   CREAT< 1.512
   CREAT>=0.951
   CREAT>=0.9689
   AGE>=84.53
   WT< 75.38
   CREAT< 1.41
   CREAT< 1.305
   WT< 71.86
   CREAT< 1.07

 node number: 14951 
   root
   CON=1
   CREAT< 1.942
   CREAT< 1.645
   CREAT< 1.62
   CREAT< 1.512
   CREAT>=0.951
   CREAT>=0.9689
   AGE>=84.53
   WT< 75.38
   CREAT< 1.41
   CREAT< 1.305
   WT< 71.86
   CREAT>=1.07

 node number: 1869 
   root
   CON=1
   CREAT< 1.942
   CREAT< 1.645
   CREAT< 1.62
   CREAT< 1.512
   CREAT>=0.951
   CREAT>=0.9689
   AGE>=84.53
   WT< 75.38
   CREAT>=1.41

 node number: 935 
   root
   CON=1
   CREAT< 1.942
   CREAT< 1.645
   CREAT< 1.62
   CREAT< 1.512
   CREAT>=0.951
   CREAT>=0.9689
   AGE>=84.53
   WT>=75.38

 node number: 3744 
   root
   CON=1
   CREAT< 1.942
   CREAT< 1.645
   CREAT< 1.62
   CREAT< 1.512
   CREAT< 0.951
   WT>=68.37
   WT< 81.18
   CREAT< 0.9264
   CREAT>=0.8078
   CREAT< 0.8581

 node number: 7490 
   root
   CON=1
   CREAT< 1.942
   CREAT< 1.645
   CREAT< 1.62
   CREAT< 1.512
   CREAT< 0.951
   WT>=68.37
   WT< 81.18
   CREAT< 0.9264
   CREAT>=0.8078
   CREAT>=0.8581
   WT< 76.48

 node number: 7491 
   root
   CON=1
   CREAT< 1.942
   CREAT< 1.645
   CREAT< 1.62
   CREAT< 1.512
   CREAT< 0.951
   WT>=68.37
   WT< 81.18
   CREAT< 0.9264
   CREAT>=0.8078
   CREAT>=0.8581
   WT>=76.48

 node number: 7492 
   root
   CON=1
   CREAT< 1.942
   CREAT< 1.645
   CREAT< 1.62
   CREAT< 1.512
   CREAT< 0.951
   WT>=68.37
   WT< 81.18
   CREAT< 0.9264
   CREAT< 0.8078
   WT>=75.71
   WT< 77.98

 node number: 14986 
   root
   CON=1
   CREAT< 1.942
   CREAT< 1.645
   CREAT< 1.62
   CREAT< 1.512
   CREAT< 0.951
   WT>=68.37
   WT< 81.18
   CREAT< 0.9264
   CREAT< 0.8078
   WT>=75.71
   WT>=77.98
   CREAT< 0.5177

 node number: 14987 
   root
   CON=1
   CREAT< 1.942
   CREAT< 1.645
   CREAT< 1.62
   CREAT< 1.512
   CREAT< 0.951
   WT>=68.37
   WT< 81.18
   CREAT< 0.9264
   CREAT< 0.8078
   WT>=75.71
   WT>=77.98
   CREAT>=0.5177

 node number: 14988 
   root
   CON=1
   CREAT< 1.942
   CREAT< 1.645
   CREAT< 1.62
   CREAT< 1.512
   CREAT< 0.951
   WT>=68.37
   WT< 81.18
   CREAT< 0.9264
   CREAT< 0.8078
   WT< 75.71
   WT< 71.89
   WT>=70.23

 node number: 14989 
   root
   CON=1
   CREAT< 1.942
   CREAT< 1.645
   CREAT< 1.62
   CREAT< 1.512
   CREAT< 0.951
   WT>=68.37
   WT< 81.18
   CREAT< 0.9264
   CREAT< 0.8078
   WT< 75.71
   WT< 71.89
   WT< 70.23

 node number: 29980 
   root
   CON=1
   CREAT< 1.942
   CREAT< 1.645
   CREAT< 1.62
   CREAT< 1.512
   CREAT< 0.951
   WT>=68.37
   WT< 81.18
   CREAT< 0.9264
   CREAT< 0.8078
   WT< 75.71
   WT>=71.89
   WT>=73.09
   WT< 74.82

 node number: 29981 
   root
   CON=1
   CREAT< 1.942
   CREAT< 1.645
   CREAT< 1.62
   CREAT< 1.512
   CREAT< 0.951
   WT>=68.37
   WT< 81.18
   CREAT< 0.9264
   CREAT< 0.8078
   WT< 75.71
   WT>=71.89
   WT>=73.09
   WT>=74.82

 node number: 14991 
   root
   CON=1
   CREAT< 1.942
   CREAT< 1.645
   CREAT< 1.62
   CREAT< 1.512
   CREAT< 0.951
   WT>=68.37
   WT< 81.18
   CREAT< 0.9264
   CREAT< 0.8078
   WT< 75.71
   WT>=71.89
   WT< 73.09

 node number: 937 
   root
   CON=1
   CREAT< 1.942
   CREAT< 1.645
   CREAT< 1.62
   CREAT< 1.512
   CREAT< 0.951
   WT>=68.37
   WT< 81.18
   CREAT>=0.9264

 node number: 7504 
   root
   CON=1
   CREAT< 1.942
   CREAT< 1.645
   CREAT< 1.62
   CREAT< 1.512
   CREAT< 0.951
   WT>=68.37
   WT>=81.18
   AGE< 85.28
   AGE>=55.96
   OBESE=1
   WT< 97.23

 node number: 15010 
   root
   CON=1
   CREAT< 1.942
   CREAT< 1.645
   CREAT< 1.62
   CREAT< 1.512
   CREAT< 0.951
   WT>=68.37
   WT>=81.18
   AGE< 85.28
   AGE>=55.96
   OBESE=1
   WT>=97.23
   AGE>=69.7

 node number: 15011 
   root
   CON=1
   CREAT< 1.942
   CREAT< 1.645
   CREAT< 1.62
   CREAT< 1.512
   CREAT< 0.951
   WT>=68.37
   WT>=81.18
   AGE< 85.28
   AGE>=55.96
   OBESE=1
   WT>=97.23
   AGE< 69.7

 node number: 15012 
   root
   CON=1
   CREAT< 1.942
   CREAT< 1.645
   CREAT< 1.62
   CREAT< 1.512
   CREAT< 0.951
   WT>=68.37
   WT>=81.18
   AGE< 85.28
   AGE>=55.96
   OBESE=0
   CREAT>=0.788
   WT< 88.59

 node number: 15013 
   root
   CON=1
   CREAT< 1.942
   CREAT< 1.645
   CREAT< 1.62
   CREAT< 1.512
   CREAT< 0.951
   WT>=68.37
   WT>=81.18
   AGE< 85.28
   AGE>=55.96
   OBESE=0
   CREAT>=0.788
   WT>=88.59

 node number: 7507 
   root
   CON=1
   CREAT< 1.942
   CREAT< 1.645
   CREAT< 1.62
   CREAT< 1.512
   CREAT< 0.951
   WT>=68.37
   WT>=81.18
   AGE< 85.28
   AGE>=55.96
   OBESE=0
   CREAT< 0.788

 node number: 1877 
   root
   CON=1
   CREAT< 1.942
   CREAT< 1.645
   CREAT< 1.62
   CREAT< 1.512
   CREAT< 0.951
   WT>=68.37
   WT>=81.18
   AGE< 85.28
   AGE< 55.96

 node number: 939 
   root
   CON=1
   CREAT< 1.942
   CREAT< 1.645
   CREAT< 1.62
   CREAT< 1.512
   CREAT< 0.951
   WT>=68.37
   WT>=81.18
   AGE>=85.28

 node number: 940 
   root
   CON=1
   CREAT< 1.942
   CREAT< 1.645
   CREAT< 1.62
   CREAT< 1.512
   CREAT< 0.951
   WT< 68.37
   AGE< 56.29
   WT< 62.83

 node number: 941 
   root
   CON=1
   CREAT< 1.942
   CREAT< 1.645
   CREAT< 1.62
   CREAT< 1.512
   CREAT< 0.951
   WT< 68.37
   AGE< 56.29
   WT>=62.83

 node number: 1884 
   root
   CON=1
   CREAT< 1.942
   CREAT< 1.645
   CREAT< 1.62
   CREAT< 1.512
   CREAT< 0.951
   WT< 68.37
   AGE>=56.29
   AGE>=90.78
   SEX=1

 node number: 1885 
   root
   CON=1
   CREAT< 1.942
   CREAT< 1.645
   CREAT< 1.62
   CREAT< 1.512
   CREAT< 0.951
   WT< 68.37
   AGE>=56.29
   AGE>=90.78
   SEX=0

 node number: 7544 
   root
   CON=1
   CREAT< 1.942
   CREAT< 1.645
   CREAT< 1.62
   CREAT< 1.512
   CREAT< 0.951
   WT< 68.37
   AGE>=56.29
   AGE< 90.78
   AGE< 82.96
   CREAT< 0.74
   CREAT>=0.6156

 node number: 7545 
   root
   CON=1
   CREAT< 1.942
   CREAT< 1.645
   CREAT< 1.62
   CREAT< 1.512
   CREAT< 0.951
   WT< 68.37
   AGE>=56.29
   AGE< 90.78
   AGE< 82.96
   CREAT< 0.74
   CREAT< 0.6156

 node number: 3773 
   root
   CON=1
   CREAT< 1.942
   CREAT< 1.645
   CREAT< 1.62
   CREAT< 1.512
   CREAT< 0.951
   WT< 68.37
   AGE>=56.29
   AGE< 90.78
   AGE< 82.96
   CREAT>=0.74

 node number: 1887 
   root
   CON=1
   CREAT< 1.942
   CREAT< 1.645
   CREAT< 1.62
   CREAT< 1.512
   CREAT< 0.951
   WT< 68.37
   AGE>=56.29
   AGE< 90.78
   AGE>=82.96

 node number: 59 
   root
   CON=1
   CREAT< 1.942
   CREAT< 1.645
   CREAT< 1.62
   CREAT>=1.512

 node number: 120 
   root
   CON=1
   CREAT< 1.942
   CREAT>=1.645
   WT< 81.53
   SEX=0
   CREAT< 1.757

 node number: 121 
   root
   CON=1
   CREAT< 1.942
   CREAT>=1.645
   WT< 81.53
   SEX=0
   CREAT>=1.757

 node number: 61 
   root
   CON=1
   CREAT< 1.942
   CREAT>=1.645
   WT< 81.53
   SEX=1

 node number: 31 
   root
   CON=1
   CREAT< 1.942
   CREAT>=1.645
   WT>=81.53
\end{verbatim}

To evaluate the method training, the decision trees and their
corresponding confusion matrices are visualized. On the visualized
nodes, the first line indicates the target variable majority class in
that node (YES/NO), the second line the proportion of correct
predictions (= proportion of YES), and the third line the percentage of
subjects in that particular node.

\begin{Shaded}
\begin{Highlighting}[]
\NormalTok{AMOX\_CMIN\_TRAINED\_CT}\SpecialCharTok{$}\NormalTok{plot}
\end{Highlighting}
\end{Shaded}

\begin{verbatim}
$obj
n= 1875 

node), split, n, loss, yval, (yprob)
      * denotes terminal node

     1) root 1875 242 NO (0.870933333 0.129066667)  
       2) CON=0 1425   4 NO (0.997192982 0.002807018) *
       3) CON=1 450 212 YES (0.471111111 0.528888889)  
         6) CREAT>=1.941857 10   1 NO (0.900000000 0.100000000) *
         7) CREAT< 1.941857 440 203 YES (0.461363636 0.538636364)  
          14) CREAT< 1.645314 419 198 YES (0.472553699 0.527446301)  
            28) CREAT>=1.619656 6   0 NO (1.000000000 0.000000000) *
            29) CREAT< 1.619656 413 192 YES (0.464891041 0.535108959)  
              58) CREAT< 1.512278 401 191 YES (0.476309227 0.523690773)  
               116) CREAT>=0.9510073 197  91 NO (0.538071066 0.461928934)  
                 232) CREAT< 0.968943 10   2 NO (0.800000000 0.200000000) *
                 233) CREAT>=0.968943 187  89 NO (0.524064171 0.475935829)  
                   466) AGE< 84.53017 155  69 NO (0.554838710 0.445161290)  
                     932) CREAT>=1.466512 9   1 NO (0.888888889 0.111111111) *
                     933) CREAT< 1.466512 146  68 NO (0.534246575 0.465753425)  
                      1866) WT< 76.02046 51  19 NO (0.627450980 0.372549020)  
                        3732) WT>=70.93219 11   1 NO (0.909090909 0.090909091) *
                        3733) WT< 70.93219 40  18 NO (0.550000000 0.450000000)  
                          7466) CREAT< 0.9978998 4   0 NO (1.000000000 0.000000000) *
                          7467) CREAT>=0.9978998 36  18 NO (0.500000000 0.500000000)  
                           14934) CREAT>=1.029231 30  13 NO (0.566666667 0.433333333)  
                             29868) CREAT< 1.147175 12   3 NO (0.750000000 0.250000000) *
                             29869) CREAT>=1.147175 18   8 YES (0.444444444 0.555555556)  
                               59738) AGE>=76.55758 4   1 NO (0.750000000 0.250000000) *
                               59739) AGE< 76.55758 14   5 YES (0.357142857 0.642857143)  
                                119478) WT>=64.86754 7   3 NO (0.571428571 0.428571429) *
                                119479) WT< 64.86754 7   1 YES (0.142857143 0.857142857) *
                           14935) CREAT< 1.029231 6   1 YES (0.166666667 0.833333333) *
                      1867) WT>=76.02046 95  46 YES (0.484210526 0.515789474)  
                        3734) WT>=76.98115 91  45 NO (0.505494505 0.494505495)  
                          7468) CREAT< 1.250401 55  23 NO (0.581818182 0.418181818)  
                           14936) CREAT>=1.021213 45  16 NO (0.644444444 0.355555556)  
                             29872) AGE>=77.46915 12   1 NO (0.916666667 0.083333333) *
                             29873) AGE< 77.46915 33  15 NO (0.545454545 0.454545455)  
                               59746) WT>=83.64663 23   7 NO (0.695652174 0.304347826)  
                                119492) CREAT< 1.074594 10   2 NO (0.800000000 0.200000000) *
                                119493) CREAT>=1.074594 13   5 NO (0.615384615 0.384615385)  
                                  238986) WT< 96.19688 8   2 NO (0.750000000 0.250000000) *
                                  238987) WT>=96.19688 5   2 YES (0.400000000 0.600000000) *
                               59747) WT< 83.64663 10   2 YES (0.200000000 0.800000000) *
                           14937) CREAT< 1.021213 10   3 YES (0.300000000 0.700000000)  
                             29874) WT< 88.56377 5   2 NO (0.600000000 0.400000000) *
                             29875) WT>=88.56377 5   0 YES (0.000000000 1.000000000) *
                          7469) CREAT>=1.250401 36  14 YES (0.388888889 0.611111111)  
                           14938) CREAT>=1.271543 32  14 YES (0.437500000 0.562500000)  
                             29876) WT< 85.5644 15   6 NO (0.600000000 0.400000000)  
                               59752) CREAT< 1.320108 7   1 NO (0.857142857 0.142857143) *
                               59753) CREAT>=1.320108 8   3 YES (0.375000000 0.625000000) *
                             29877) WT>=85.5644 17   5 YES (0.294117647 0.705882353)  
                               59754) WT>=97.4223 5   2 NO (0.600000000 0.400000000) *
                               59755) WT< 97.4223 12   2 YES (0.166666667 0.833333333) *
                           14939) CREAT< 1.271543 4   0 YES (0.000000000 1.000000000) *
                        3735) WT< 76.98115 4   0 YES (0.000000000 1.000000000) *
                   467) AGE>=84.53017 32  12 YES (0.375000000 0.625000000)  
                     934) WT< 75.37886 21  10 NO (0.523809524 0.476190476)  
                      1868) CREAT< 1.41021 17   6 NO (0.647058824 0.352941176)  
                        3736) CREAT>=1.30464 4   0 NO (1.000000000 0.000000000) *
                        3737) CREAT< 1.30464 13   6 NO (0.538461538 0.461538462)  
                          7474) WT>=71.85974 4   1 NO (0.750000000 0.250000000) *
                          7475) WT< 71.85974 9   4 YES (0.444444444 0.555555556)  
                           14950) CREAT< 1.069633 4   1 NO (0.750000000 0.250000000) *
                           14951) CREAT>=1.069633 5   1 YES (0.200000000 0.800000000) *
                      1869) CREAT>=1.41021 4   0 YES (0.000000000 1.000000000) *
                     935) WT>=75.37886 11   1 YES (0.090909091 0.909090909) *
               117) CREAT< 0.9510073 204  85 YES (0.416666667 0.583333333)  
                 234) WT>=68.36883 134  63 YES (0.470149254 0.529850746)  
                   468) WT< 81.18263 77  32 NO (0.584415584 0.415584416)  
                     936) CREAT< 0.9263891 73  28 NO (0.616438356 0.383561644)  
                      1872) CREAT>=0.807796 19   4 NO (0.789473684 0.210526316)  
                        3744) CREAT< 0.858128 7   0 NO (1.000000000 0.000000000) *
                        3745) CREAT>=0.858128 12   4 NO (0.666666667 0.333333333)  
                          7490) WT< 76.48189 7   1 NO (0.857142857 0.142857143) *
                          7491) WT>=76.48189 5   2 YES (0.400000000 0.600000000) *
                      1873) CREAT< 0.807796 54  24 NO (0.555555556 0.444444444)  
                        3746) WT>=75.71249 19   6 NO (0.684210526 0.315789474)  
                          7492) WT< 77.98096 7   0 NO (1.000000000 0.000000000) *
                          7493) WT>=77.98096 12   6 NO (0.500000000 0.500000000)  
                           14986) CREAT< 0.5176931 5   1 NO (0.800000000 0.200000000) *
                           14987) CREAT>=0.5176931 7   2 YES (0.285714286 0.714285714) *
                        3747) WT< 75.71249 35  17 YES (0.485714286 0.514285714)  
                          7494) WT< 71.88627 19   7 NO (0.631578947 0.368421053)  
                           14988) WT>=70.23495 11   2 NO (0.818181818 0.181818182) *
                           14989) WT< 70.23495 8   3 YES (0.375000000 0.625000000) *
                          7495) WT>=71.88627 16   5 YES (0.312500000 0.687500000)  
                           14990) WT>=73.09432 10   5 NO (0.500000000 0.500000000)  
                             29980) WT< 74.82279 6   2 NO (0.666666667 0.333333333) *
                             29981) WT>=74.82279 4   1 YES (0.250000000 0.750000000) *
                           14991) WT< 73.09432 6   0 YES (0.000000000 1.000000000) *
                     937) CREAT>=0.9263891 4   0 YES (0.000000000 1.000000000) *
                   469) WT>=81.18263 57  18 YES (0.315789474 0.684210526)  
                     938) AGE< 85.28493 52  18 YES (0.346153846 0.653846154)  
                      1876) AGE>=55.96424 37  16 YES (0.432432432 0.567567568)  
                        3752) OBESE=1 14   4 NO (0.714285714 0.285714286)  
                          7504) WT< 97.23022 4   0 NO (1.000000000 0.000000000) *
                          7505) WT>=97.23022 10   4 NO (0.600000000 0.400000000)  
                           15010) AGE>=69.69896 5   1 NO (0.800000000 0.200000000) *
                           15011) AGE< 69.69896 5   2 YES (0.400000000 0.600000000) *
                        3753) OBESE=0 23   6 YES (0.260869565 0.739130435)  
                          7506) CREAT>=0.7879555 8   4 NO (0.500000000 0.500000000)  
                           15012) WT< 88.59148 4   1 NO (0.750000000 0.250000000) *
                           15013) WT>=88.59148 4   1 YES (0.250000000 0.750000000) *
                          7507) CREAT< 0.7879555 15   2 YES (0.133333333 0.866666667) *
                      1877) AGE< 55.96424 15   2 YES (0.133333333 0.866666667) *
                     939) AGE>=85.28493 5   0 YES (0.000000000 1.000000000) *
                 235) WT< 68.36883 70  22 YES (0.314285714 0.685714286)  
                   470) AGE< 56.2855 11   3 NO (0.727272727 0.272727273)  
                     940) WT< 62.83313 6   0 NO (1.000000000 0.000000000) *
                     941) WT>=62.83313 5   2 YES (0.400000000 0.600000000) *
                   471) AGE>=56.2855 59  14 YES (0.237288136 0.762711864)  
                     942) AGE>=90.78492 11   5 NO (0.545454545 0.454545455)  
                      1884) SEX=1 4   0 NO (1.000000000 0.000000000) *
                      1885) SEX=0 7   2 YES (0.285714286 0.714285714) *
                     943) AGE< 90.78492 48   8 YES (0.166666667 0.833333333)  
                      1886) AGE< 82.95539 35   8 YES (0.228571429 0.771428571)  
                        3772) CREAT< 0.7400186 12   6 NO (0.500000000 0.500000000)  
                          7544) CREAT>=0.6155857 5   0 NO (1.000000000 0.000000000) *
                          7545) CREAT< 0.6155857 7   1 YES (0.142857143 0.857142857) *
                        3773) CREAT>=0.7400186 23   2 YES (0.086956522 0.913043478) *
                      1887) AGE>=82.95539 13   0 YES (0.000000000 1.000000000) *
              59) CREAT>=1.512278 12   1 YES (0.083333333 0.916666667) *
          15) CREAT>=1.645314 21   5 YES (0.238095238 0.761904762)  
            30) WT< 81.53417 13   5 YES (0.384615385 0.615384615)  
              60) SEX=0 9   4 NO (0.555555556 0.444444444)  
               120) CREAT< 1.756746 4   1 NO (0.750000000 0.250000000) *
               121) CREAT>=1.756746 5   2 YES (0.400000000 0.600000000) *
              61) SEX=1 4   0 YES (0.000000000 1.000000000) *
            31) WT>=81.53417 8   0 YES (0.000000000 1.000000000) *

$snipped.nodes
NULL

$xlim
[1] 0 1

$ylim
[1] 0 1

$x
  [1] 0.178935914 0.008790483 0.349081345 0.024384436 0.673778254 0.369991699
  [7] 0.039978389 0.700005010 0.471176314 0.178900103 0.055572342 0.302227865
 [13] 0.147826411 0.071166295 0.224486527 0.107958278 0.086760248 0.129156307
 [19] 0.102354201 0.155958414 0.131592862 0.117948154 0.145237571 0.133542107
 [25] 0.156933036 0.149136059 0.164730012 0.180323965 0.341014777 0.298984201
 [31] 0.250009442 0.218334226 0.195917918 0.240750533 0.223207336 0.211511871
 [37] 0.234902800 0.227105824 0.242699777 0.258293730 0.281684659 0.273887683
 [43] 0.289481636 0.347958959 0.328466518 0.312872565 0.305075589 0.320669542
 [49] 0.344060471 0.336263494 0.351857447 0.367451400 0.383045353 0.456629318
 [55] 0.436649566 0.412284015 0.398639306 0.425928724 0.414233259 0.437624188
 [61] 0.429827212 0.445421165 0.461015118 0.476609071 0.763452525 0.685300019
 [67] 0.603553594 0.543370682 0.503898488 0.492203024 0.515593953 0.507796976
 [73] 0.523390929 0.582842875 0.550680347 0.538984882 0.562375812 0.554578835
 [79] 0.570172788 0.615005403 0.593563717 0.585766741 0.601360694 0.636447088
 [85] 0.624751623 0.616954647 0.632548600 0.648142553 0.663736506 0.767046444
 [91] 0.745604758 0.718315341 0.691025923 0.679330458 0.702721388 0.694924411
 [97] 0.710518364 0.745604758 0.733909294 0.726112317 0.741706270 0.757300223
[103] 0.772894176 0.788488129 0.841605031 0.811879058 0.804082082 0.819676035
[109] 0.871331004 0.843066964 0.835269988 0.850863941 0.899595043 0.885950335
[115] 0.874254870 0.866457893 0.882051846 0.897645799 0.913239752 0.928833705
[121] 0.977564808 0.963920099 0.952224634 0.944427658 0.960021611 0.975615564
[127] 0.991209517

$y
  [1] 0.97297037 0.01664697 0.92183008 0.01664697 0.87068979 0.81954950
  [7] 0.01664697 0.76840921 0.71726892 0.66612864 0.01664697 0.61498835
 [13] 0.56384806 0.01664697 0.51270777 0.46156748 0.01664697 0.41042719
 [19] 0.01664697 0.35928690 0.30814661 0.01664697 0.25700633 0.01664697
 [25] 0.20586604 0.01664697 0.01664697 0.01664697 0.46156748 0.41042719
 [31] 0.35928690 0.30814661 0.01664697 0.25700633 0.20586604 0.01664697
 [37] 0.15472575 0.01664697 0.01664697 0.01664697 0.30814661 0.01664697
 [43] 0.01664697 0.35928690 0.30814661 0.25700633 0.01664697 0.01664697
 [49] 0.25700633 0.01664697 0.01664697 0.01664697 0.01664697 0.56384806
 [55] 0.51270777 0.46156748 0.01664697 0.41042719 0.01664697 0.35928690
 [61] 0.01664697 0.01664697 0.01664697 0.01664697 0.66612864 0.61498835
 [67] 0.56384806 0.51270777 0.46156748 0.01664697 0.41042719 0.01664697
 [73] 0.01664697 0.46156748 0.41042719 0.01664697 0.35928690 0.01664697
 [79] 0.01664697 0.41042719 0.35928690 0.01664697 0.01664697 0.35928690
 [85] 0.30814661 0.01664697 0.01664697 0.01664697 0.01664697 0.56384806
 [91] 0.51270777 0.46156748 0.41042719 0.01664697 0.35928690 0.01664697
 [97] 0.01664697 0.41042719 0.35928690 0.01664697 0.01664697 0.01664697
[103] 0.01664697 0.01664697 0.61498835 0.56384806 0.01664697 0.01664697
[109] 0.56384806 0.51270777 0.01664697 0.01664697 0.51270777 0.46156748
[115] 0.41042719 0.01664697 0.01664697 0.01664697 0.01664697 0.01664697
[121] 0.81954950 0.76840921 0.71726892 0.01664697 0.01664697 0.01664697
[127] 0.01664697

$branch.x
       [,1]        [,2]      [,3]       [,4]      [,5]      [,6]       [,7]
x 0.1789359 0.008790483 0.3490813 0.02438444 0.6737783 0.3699917 0.03997839
         NA 0.008790483 0.3490813 0.02438444 0.6737783 0.3699917 0.03997839
         NA 0.178935914 0.1789359 0.34908134 0.3490813 0.6737783 0.36999170
       [,8]      [,9]     [,10]      [,11]     [,12]     [,13]      [,14]
x 0.7000050 0.4711763 0.1789001 0.05557234 0.3022279 0.1478264 0.07116629
  0.7000050 0.4711763 0.1789001 0.05557234 0.3022279 0.1478264 0.07116629
  0.3699917 0.7000050 0.4711763 0.17890010 0.1789001 0.3022279 0.14782641
      [,15]     [,16]      [,17]     [,18]     [,19]     [,20]     [,21]
x 0.2244865 0.1079583 0.08676025 0.1291563 0.1023542 0.1559584 0.1315929
  0.2244865 0.1079583 0.08676025 0.1291563 0.1023542 0.1559584 0.1315929
  0.1478264 0.2244865 0.10795828 0.1079583 0.1291563 0.1291563 0.1559584
      [,22]     [,23]     [,24]     [,25]     [,26]    [,27]     [,28]
x 0.1179482 0.1452376 0.1335421 0.1569330 0.1491361 0.164730 0.1803240
  0.1179482 0.1452376 0.1335421 0.1569330 0.1491361 0.164730 0.1803240
  0.1315929 0.1315929 0.1452376 0.1452376 0.1569330 0.156933 0.1559584
      [,29]     [,30]     [,31]     [,32]     [,33]     [,34]     [,35]
x 0.3410148 0.2989842 0.2500094 0.2183342 0.1959179 0.2407505 0.2232073
  0.3410148 0.2989842 0.2500094 0.2183342 0.1959179 0.2407505 0.2232073
  0.2244865 0.3410148 0.2989842 0.2500094 0.2183342 0.2183342 0.2407505
      [,36]     [,37]     [,38]     [,39]     [,40]     [,41]     [,42]
x 0.2115119 0.2349028 0.2271058 0.2426998 0.2582937 0.2816847 0.2738877
  0.2115119 0.2349028 0.2271058 0.2426998 0.2582937 0.2816847 0.2738877
  0.2232073 0.2232073 0.2349028 0.2349028 0.2407505 0.2500094 0.2816847
      [,43]     [,44]     [,45]     [,46]     [,47]     [,48]     [,49]
x 0.2894816 0.3479590 0.3284665 0.3128726 0.3050756 0.3206695 0.3440605
  0.2894816 0.3479590 0.3284665 0.3128726 0.3050756 0.3206695 0.3440605
  0.2816847 0.2989842 0.3479590 0.3284665 0.3128726 0.3128726 0.3284665
      [,50]     [,51]     [,52]     [,53]     [,54]     [,55]     [,56]
x 0.3362635 0.3518574 0.3674514 0.3830454 0.4566293 0.4366496 0.4122840
  0.3362635 0.3518574 0.3674514 0.3830454 0.4566293 0.4366496 0.4122840
  0.3440605 0.3440605 0.3479590 0.3410148 0.3022279 0.4566293 0.4366496
      [,57]     [,58]     [,59]     [,60]     [,61]     [,62]     [,63]
x 0.3986393 0.4259287 0.4142333 0.4376242 0.4298272 0.4454212 0.4610151
  0.3986393 0.4259287 0.4142333 0.4376242 0.4298272 0.4454212 0.4610151
  0.4122840 0.4122840 0.4259287 0.4259287 0.4376242 0.4376242 0.4366496
      [,64]     [,65]     [,66]     [,67]     [,68]     [,69]     [,70]
x 0.4766091 0.7634525 0.6853000 0.6035536 0.5433707 0.5038985 0.4922030
  0.4766091 0.7634525 0.6853000 0.6035536 0.5433707 0.5038985 0.4922030
  0.4566293 0.4711763 0.7634525 0.6853000 0.6035536 0.5433707 0.5038985
      [,71]    [,72]     [,73]     [,74]     [,75]     [,76]     [,77]
x 0.5155940 0.507797 0.5233909 0.5828429 0.5506803 0.5389849 0.5623758
  0.5155940 0.507797 0.5233909 0.5828429 0.5506803 0.5389849 0.5623758
  0.5038985 0.515594 0.5155940 0.5433707 0.5828429 0.5506803 0.5506803
      [,78]     [,79]     [,80]     [,81]     [,82]     [,83]     [,84]
x 0.5545788 0.5701728 0.6150054 0.5935637 0.5857667 0.6013607 0.6364471
  0.5545788 0.5701728 0.6150054 0.5935637 0.5857667 0.6013607 0.6364471
  0.5623758 0.5623758 0.5828429 0.6150054 0.5935637 0.5935637 0.6150054
      [,85]     [,86]     [,87]     [,88]     [,89]     [,90]     [,91]
x 0.6247516 0.6169546 0.6325486 0.6481426 0.6637365 0.7670464 0.7456048
  0.6247516 0.6169546 0.6325486 0.6481426 0.6637365 0.7670464 0.7456048
  0.6364471 0.6247516 0.6247516 0.6364471 0.6035536 0.6853000 0.7670464
      [,92]     [,93]     [,94]     [,95]     [,96]     [,97]     [,98]
x 0.7183153 0.6910259 0.6793305 0.7027214 0.6949244 0.7105184 0.7456048
  0.7183153 0.6910259 0.6793305 0.7027214 0.6949244 0.7105184 0.7456048
  0.7456048 0.7183153 0.6910259 0.6910259 0.7027214 0.7027214 0.7183153
      [,99]    [,100]    [,101]    [,102]    [,103]    [,104]    [,105]
x 0.7339093 0.7261123 0.7417063 0.7573002 0.7728942 0.7884881 0.8416050
  0.7339093 0.7261123 0.7417063 0.7573002 0.7728942 0.7884881 0.8416050
  0.7456048 0.7339093 0.7339093 0.7456048 0.7456048 0.7670464 0.7634525
     [,106]    [,107]    [,108]   [,109]   [,110]   [,111]    [,112]   [,113]
x 0.8118791 0.8040821 0.8196760 0.871331 0.843067 0.835270 0.8508639 0.899595
  0.8118791 0.8040821 0.8196760 0.871331 0.843067 0.835270 0.8508639 0.899595
  0.8416050 0.8118791 0.8118791 0.841605 0.871331 0.843067 0.8430670 0.871331
     [,114]    [,115]    [,116]    [,117]    [,118]    [,119]    [,120]
x 0.8859503 0.8742549 0.8664579 0.8820518 0.8976458 0.9132398 0.9288337
  0.8859503 0.8742549 0.8664579 0.8820518 0.8976458 0.9132398 0.9288337
  0.8995950 0.8859503 0.8742549 0.8742549 0.8859503 0.8995950 0.7000050
     [,121]    [,122]    [,123]    [,124]    [,125]    [,126]    [,127]
x 0.9775648 0.9639201 0.9522246 0.9444277 0.9600216 0.9756156 0.9912095
  0.9775648 0.9639201 0.9522246 0.9444277 0.9600216 0.9756156 0.9912095
  0.6737783 0.9775648 0.9639201 0.9522246 0.9522246 0.9639201 0.9775648

$branch.y
       [,1]       [,2]      [,3]       [,4]      [,5]      [,6]       [,7]
y 0.9990631 0.04273971 0.9425712 0.04273971 0.8914309 0.8402906 0.04273971
         NA 0.94257117 0.9425712 0.89143089 0.8914309 0.8402906 0.78915031
         NA 0.94257117 0.9425712 0.89143089 0.8914309 0.8402906 0.78915031
       [,8]    [,9]     [,10]      [,11]     [,12]     [,13]      [,14]
y 0.7891503 0.73801 0.6868697 0.04273971 0.6357294 0.5845892 0.04273971
  0.7891503 0.73801 0.6868697 0.63572944 0.6357294 0.5845892 0.53344886
  0.7891503 0.73801 0.6868697 0.63572944 0.6357294 0.5845892 0.53344886
      [,15]     [,16]      [,17]     [,18]      [,19]    [,20]     [,21]
y 0.5334489 0.4823086 0.04273971 0.4311683 0.04273971 0.380028 0.3288877
  0.5334489 0.4823086 0.43116829 0.4311683 0.38002800 0.380028 0.3288877
  0.5334489 0.4823086 0.43116829 0.4311683 0.38002800 0.380028 0.3288877
       [,22]     [,23]      [,24]     [,25]      [,26]      [,27]      [,28]
y 0.04273971 0.2777474 0.04273971 0.2266071 0.04273971 0.04273971 0.04273971
  0.27774742 0.2777474 0.22660713 0.2266071 0.17546684 0.17546684 0.32888771
  0.27774742 0.2777474 0.22660713 0.2266071 0.17546684 0.17546684 0.32888771
      [,29]     [,30]    [,31]     [,32]      [,33]     [,34]     [,35]
y 0.4823086 0.4311683 0.380028 0.3288877 0.04273971 0.2777474 0.2266071
  0.4823086 0.4311683 0.380028 0.3288877 0.27774742 0.2777474 0.2266071
  0.4823086 0.4311683 0.380028 0.3288877 0.27774742 0.2777474 0.2266071
       [,36]     [,37]      [,38]      [,39]      [,40]     [,41]      [,42]
y 0.04273971 0.1754668 0.04273971 0.04273971 0.04273971 0.3288877 0.04273971
  0.17546684 0.1754668 0.12432655 0.12432655 0.22660713 0.3288877 0.27774742
  0.17546684 0.1754668 0.12432655 0.12432655 0.22660713 0.3288877 0.27774742
       [,43]    [,44]     [,45]     [,46]      [,47]      [,48]     [,49]
y 0.04273971 0.380028 0.3288877 0.2777474 0.04273971 0.04273971 0.2777474
  0.27774742 0.380028 0.3288877 0.2777474 0.22660713 0.22660713 0.2777474
  0.27774742 0.380028 0.3288877 0.2777474 0.22660713 0.22660713 0.2777474
       [,50]      [,51]      [,52]      [,53]     [,54]     [,55]     [,56]
y 0.04273971 0.04273971 0.04273971 0.04273971 0.5845892 0.5334489 0.4823086
  0.22660713 0.22660713 0.32888771 0.43116829 0.5845892 0.5334489 0.4823086
  0.22660713 0.22660713 0.32888771 0.43116829 0.5845892 0.5334489 0.4823086
       [,57]     [,58]      [,59]    [,60]      [,61]      [,62]      [,63]
y 0.04273971 0.4311683 0.04273971 0.380028 0.04273971 0.04273971 0.04273971
  0.43116829 0.4311683 0.38002800 0.380028 0.32888771 0.32888771 0.48230858
  0.43116829 0.4311683 0.38002800 0.380028 0.32888771 0.32888771 0.48230858
       [,64]     [,65]     [,66]     [,67]     [,68]     [,69]      [,70]
y 0.04273971 0.6868697 0.6357294 0.5845892 0.5334489 0.4823086 0.04273971
  0.53344886 0.6868697 0.6357294 0.5845892 0.5334489 0.4823086 0.43116829
  0.53344886 0.6868697 0.6357294 0.5845892 0.5334489 0.4823086 0.43116829
      [,71]      [,72]      [,73]     [,74]     [,75]      [,76]    [,77]
y 0.4311683 0.04273971 0.04273971 0.4823086 0.4311683 0.04273971 0.380028
  0.4311683 0.38002800 0.38002800 0.4823086 0.4311683 0.38002800 0.380028
  0.4311683 0.38002800 0.38002800 0.4823086 0.4311683 0.38002800 0.380028
       [,78]      [,79]     [,80]    [,81]      [,82]      [,83]    [,84]
y 0.04273971 0.04273971 0.4311683 0.380028 0.04273971 0.04273971 0.380028
  0.32888771 0.32888771 0.4311683 0.380028 0.32888771 0.32888771 0.380028
  0.32888771 0.32888771 0.4311683 0.380028 0.32888771 0.32888771 0.380028
      [,85]      [,86]      [,87]      [,88]      [,89]     [,90]     [,91]
y 0.3288877 0.04273971 0.04273971 0.04273971 0.04273971 0.5845892 0.5334489
  0.3288877 0.27774742 0.27774742 0.32888771 0.53344886 0.5845892 0.5334489
  0.3288877 0.27774742 0.27774742 0.32888771 0.53344886 0.5845892 0.5334489
      [,92]     [,93]      [,94]    [,95]      [,96]      [,97]     [,98]
y 0.4823086 0.4311683 0.04273971 0.380028 0.04273971 0.04273971 0.4311683
  0.4823086 0.4311683 0.38002800 0.380028 0.32888771 0.32888771 0.4311683
  0.4823086 0.4311683 0.38002800 0.380028 0.32888771 0.32888771 0.4311683
     [,99]     [,100]     [,101]     [,102]     [,103]     [,104]    [,105]
y 0.380028 0.04273971 0.04273971 0.04273971 0.04273971 0.04273971 0.6357294
  0.380028 0.32888771 0.32888771 0.38002800 0.48230858 0.53344886 0.6357294
  0.380028 0.32888771 0.32888771 0.38002800 0.48230858 0.53344886 0.6357294
     [,106]     [,107]     [,108]    [,109]    [,110]     [,111]     [,112]
y 0.5845892 0.04273971 0.04273971 0.5845892 0.5334489 0.04273971 0.04273971
  0.5845892 0.53344886 0.53344886 0.5845892 0.5334489 0.48230858 0.48230858
  0.5845892 0.53344886 0.53344886 0.5845892 0.5334489 0.48230858 0.48230858
     [,113]    [,114]    [,115]     [,116]     [,117]     [,118]     [,119]
y 0.5334489 0.4823086 0.4311683 0.04273971 0.04273971 0.04273971 0.04273971
  0.5334489 0.4823086 0.4311683 0.38002800 0.38002800 0.43116829 0.48230858
  0.5334489 0.4823086 0.4311683 0.38002800 0.38002800 0.43116829 0.48230858
      [,120]    [,121]    [,122]  [,123]     [,124]     [,125]     [,126]
y 0.04273971 0.8402906 0.7891503 0.73801 0.04273971 0.04273971 0.04273971
  0.73801002 0.8402906 0.7891503 0.73801 0.68686973 0.68686973 0.73801002
  0.73801002 0.8402906 0.7891503 0.73801 0.68686973 0.68686973 0.73801002
      [,127]
y 0.04273971
  0.78915031
  0.78915031

$labs
  [1] "NO\n0.13\n100%" "NO\n0.00\n76%"  "YES\n0.53\n24%" "NO\n0.10\n1%"  
  [5] "YES\n0.54\n23%" "YES\n0.53\n22%" "NO\n0.00\n0%"   "YES\n0.54\n22%"
  [9] "YES\n0.52\n21%" "NO\n0.46\n11%"  "NO\n0.20\n1%"   "NO\n0.48\n10%" 
 [13] "NO\n0.45\n8%"   "NO\n0.11\n0%"   "NO\n0.47\n8%"   "NO\n0.37\n3%"  
 [17] "NO\n0.09\n1%"   "NO\n0.45\n2%"   "NO\n0.00\n0%"   "NO\n0.50\n2%"  
 [21] "NO\n0.43\n2%"   "NO\n0.25\n1%"   "YES\n0.56\n1%"  "NO\n0.25\n0%"  
 [25] "YES\n0.64\n1%"  "NO\n0.43\n0%"   "YES\n0.86\n0%"  "YES\n0.83\n0%" 
 [29] "YES\n0.52\n5%"  "NO\n0.49\n5%"   "NO\n0.42\n3%"   "NO\n0.36\n2%"  
 [33] "NO\n0.08\n1%"   "NO\n0.45\n2%"   "NO\n0.30\n1%"   "NO\n0.20\n1%"  
 [37] "NO\n0.38\n1%"   "NO\n0.25\n0%"   "YES\n0.60\n0%"  "YES\n0.80\n1%" 
 [41] "YES\n0.70\n1%"  "NO\n0.40\n0%"   "YES\n1.00\n0%"  "YES\n0.61\n2%" 
 [45] "YES\n0.56\n2%"  "NO\n0.40\n1%"   "NO\n0.14\n0%"   "YES\n0.62\n0%" 
 [49] "YES\n0.71\n1%"  "NO\n0.40\n0%"   "YES\n0.83\n1%"  "YES\n1.00\n0%" 
 [53] "YES\n1.00\n0%"  "YES\n0.62\n2%"  "NO\n0.48\n1%"   "NO\n0.35\n1%"  
 [57] "NO\n0.00\n0%"   "NO\n0.46\n1%"   "NO\n0.25\n0%"   "YES\n0.56\n0%" 
 [61] "NO\n0.25\n0%"   "YES\n0.80\n0%"  "YES\n1.00\n0%"  "YES\n0.91\n1%" 
 [65] "YES\n0.58\n11%" "YES\n0.53\n7%"  "NO\n0.42\n4%"   "NO\n0.38\n4%"  
 [69] "NO\n0.21\n1%"   "NO\n0.00\n0%"   "NO\n0.33\n1%"   "NO\n0.14\n0%"  
 [73] "YES\n0.60\n0%"  "NO\n0.44\n3%"   "NO\n0.32\n1%"   "NO\n0.00\n0%"  
 [77] "NO\n0.50\n1%"   "NO\n0.20\n0%"   "YES\n0.71\n0%"  "YES\n0.51\n2%" 
 [81] "NO\n0.37\n1%"   "NO\n0.18\n1%"   "YES\n0.62\n0%"  "YES\n0.69\n1%" 
 [85] "NO\n0.50\n1%"   "NO\n0.33\n0%"   "YES\n0.75\n0%"  "YES\n1.00\n0%" 
 [89] "YES\n1.00\n0%"  "YES\n0.68\n3%"  "YES\n0.65\n3%"  "YES\n0.57\n2%" 
 [93] "NO\n0.29\n1%"   "NO\n0.00\n0%"   "NO\n0.40\n1%"   "NO\n0.20\n0%"  
 [97] "YES\n0.60\n0%"  "YES\n0.74\n1%"  "NO\n0.50\n0%"   "NO\n0.25\n0%"  
[101] "YES\n0.75\n0%"  "YES\n0.87\n1%"  "YES\n0.87\n1%"  "YES\n1.00\n0%" 
[105] "YES\n0.69\n4%"  "NO\n0.27\n1%"   "NO\n0.00\n0%"   "YES\n0.60\n0%" 
[109] "YES\n0.76\n3%"  "NO\n0.45\n1%"   "NO\n0.00\n0%"   "YES\n0.71\n0%" 
[113] "YES\n0.83\n3%"  "YES\n0.77\n2%"  "NO\n0.50\n1%"   "NO\n0.00\n0%"  
[117] "YES\n0.86\n0%"  "YES\n0.91\n1%"  "YES\n1.00\n1%"  "YES\n0.92\n1%" 
[121] "YES\n0.76\n1%"  "YES\n0.62\n1%"  "NO\n0.44\n0%"   "NO\n0.25\n0%"  
[125] "YES\n0.60\n0%"  "YES\n1.00\n0%"  "YES\n1.00\n0%" 

$cex
[1] 0.15

$boxes
$boxes$x1
  [1] 0.1689653258 0.0004551891 0.3407460508 0.0162109067 0.6654429595
  [6] 0.3616564052 0.0318048597 0.6916697155 0.4628410200 0.1705648093
 [11] 0.0473988126 0.2938925707 0.1396528817 0.0629927655 0.2163129978
 [16] 0.0997847481 0.0785867184 0.1209827779 0.0941806713 0.1477848845
 [21] 0.1234193330 0.1097746242 0.1369022771 0.1253685771 0.1485977418
 [26] 0.1409625301 0.1563947183 0.1719886712 0.3326794828 0.2908106713
 [31] 0.2418359130 0.2101606961 0.1877443888 0.2325770034 0.2150338064
 [36] 0.2033383417 0.2267292711 0.2189322946 0.2343644828 0.2499584357
 [41] 0.2733493651 0.2657141534 0.2811463416 0.3396236650 0.3201312238
 [46] 0.3046990356 0.2969020592 0.3123342474 0.3357251768 0.3280899650
 [51] 0.3435221532 0.3591161061 0.3747100590 0.4482940243 0.4284760369
 [56] 0.4041104855 0.3904657767 0.4177551943 0.4060597296 0.4292888942
 [61] 0.4216536825 0.4370858707 0.4526798236 0.4682737765 0.7551172307
 [66] 0.6769647245 0.5953800641 0.5351971521 0.4957249588 0.4840294941
 [71] 0.5074204235 0.4996234470 0.5150556353 0.5746693454 0.5425068176
 [76] 0.5308113529 0.5542022822 0.5464053058 0.5618374940 0.6066701086
 [81] 0.5853901881 0.5775932116 0.5930253998 0.6281117939 0.6165780939
 [86] 0.6087811174 0.6242133056 0.6398072586 0.6554012115 0.7587111495
 [91] 0.7372694643 0.7099800467 0.6828523938 0.6711569291 0.6945478585
 [96] 0.6867508820 0.7021830702 0.7372694643 0.7257357643 0.7179387878
[101] 0.7333709760 0.7489649289 0.7645588819 0.7801528348 0.8332697369
[106] 0.8037055288 0.7959085524 0.8113407406 0.8629957096 0.8348934347
[111] 0.8270964582 0.8425286464 0.8912597493 0.8776150405 0.8660813405
[116] 0.8582843640 0.8737165522 0.8893105052 0.9049044581 0.9204984110
[121] 0.9692295138 0.9555848050 0.9440511051 0.9362541286 0.9516863168
[126] 0.9672802697 0.9828742226

$boxes$y1
  [1]  0.9555631094 -0.0007602899  0.9044228207 -0.0007602899  0.8532825319
  [6]  0.8021422432 -0.0007602899  0.7510019545  0.6998616658  0.6487213770
 [11] -0.0007602899  0.5975810883  0.5464407996 -0.0007602899  0.4953005108
 [16]  0.4441602221 -0.0007602899  0.3930199334 -0.0007602899  0.3418796446
 [21]  0.2907393559 -0.0007602899  0.2395990672 -0.0007602899  0.1884587784
 [26] -0.0007602899 -0.0007602899 -0.0007602899  0.4441602221  0.3930199334
 [31]  0.3418796446  0.2907393559 -0.0007602899  0.2395990672  0.1884587784
 [36] -0.0007602899  0.1373184897 -0.0007602899 -0.0007602899 -0.0007602899
 [41]  0.2907393559 -0.0007602899 -0.0007602899  0.3418796446  0.2907393559
 [46]  0.2395990672 -0.0007602899 -0.0007602899  0.2395990672 -0.0007602899
 [51] -0.0007602899 -0.0007602899 -0.0007602899  0.5464407996  0.4953005108
 [56]  0.4441602221 -0.0007602899  0.3930199334 -0.0007602899  0.3418796446
 [61] -0.0007602899 -0.0007602899 -0.0007602899 -0.0007602899  0.6487213770
 [66]  0.5975810883  0.5464407996  0.4953005108  0.4441602221 -0.0007602899
 [71]  0.3930199334 -0.0007602899 -0.0007602899  0.4441602221  0.3930199334
 [76] -0.0007602899  0.3418796446 -0.0007602899 -0.0007602899  0.3930199334
 [81]  0.3418796446 -0.0007602899 -0.0007602899  0.3418796446  0.2907393559
 [86] -0.0007602899 -0.0007602899 -0.0007602899 -0.0007602899  0.5464407996
 [91]  0.4953005108  0.4441602221  0.3930199334 -0.0007602899  0.3418796446
 [96] -0.0007602899 -0.0007602899  0.3930199334  0.3418796446 -0.0007602899
[101] -0.0007602899 -0.0007602899 -0.0007602899 -0.0007602899  0.5975810883
[106]  0.5464407996 -0.0007602899 -0.0007602899  0.5464407996  0.4953005108
[111] -0.0007602899 -0.0007602899  0.4953005108  0.4441602221  0.3930199334
[116] -0.0007602899 -0.0007602899 -0.0007602899 -0.0007602899 -0.0007602899
[121]  0.8021422432  0.7510019545  0.6998616658 -0.0007602899 -0.0007602899
[126] -0.0007602899 -0.0007602899

$boxes$x2
  [1] 0.18890650 0.01712578 0.35741664 0.03255797 0.68211355 0.37832699
  [7] 0.04815192 0.70834030 0.47951161 0.18723540 0.06374587 0.31056316
 [13] 0.15599994 0.07933982 0.23266006 0.11613181 0.09493378 0.13732984
 [19] 0.11052773 0.16413194 0.13976639 0.12612168 0.15357287 0.14171564
 [25] 0.16526833 0.15730959 0.17306531 0.18865926 0.34935007 0.30715773
 [31] 0.25818297 0.22650775 0.20409145 0.24892406 0.23138087 0.21968540
 [37] 0.24307633 0.23527935 0.25103507 0.26662902 0.29001995 0.28206121
 [43] 0.29781693 0.35629425 0.33680181 0.32104609 0.31324912 0.32900484
 [49] 0.35239576 0.34443702 0.36019274 0.37578669 0.39138065 0.46496461
 [55] 0.44482310 0.42045754 0.40681284 0.43410225 0.42240679 0.44595948
 [61] 0.43800074 0.45375646 0.46935041 0.48494436 0.77178782 0.69363531
 [67] 0.61172712 0.55154421 0.51207202 0.50037655 0.52376748 0.51597051
 [73] 0.53172622 0.59101640 0.55885388 0.54715841 0.57054934 0.56275236
 [79] 0.57850808 0.62334070 0.60173725 0.59394027 0.60969599 0.64478238
 [85] 0.63292515 0.62512818 0.64088389 0.65647785 0.67207180 0.77538174
 [91] 0.75394005 0.72665063 0.69919945 0.68750399 0.71089492 0.70309794
 [97] 0.71885366 0.75394005 0.74208282 0.73428585 0.75004156 0.76563552
[103] 0.78122947 0.79682342 0.84994033 0.82005259 0.81225561 0.82801133
[109] 0.87966630 0.85124049 0.84344352 0.85919923 0.90793034 0.89428563
[115] 0.88242840 0.87463142 0.89038714 0.90598109 0.92157505 0.93716900
[121] 0.98590010 0.97225539 0.96039816 0.95260119 0.96835691 0.98395086
[127] 0.99954481

$boxes$y2
  [1] 0.99906311 0.04273971 0.94792282 0.04273971 0.89678253 0.84564224
  [7] 0.04273971 0.79450195 0.74336167 0.69222138 0.04273971 0.64108109
 [13] 0.58994080 0.04273971 0.53880051 0.48766022 0.04273971 0.43651993
 [19] 0.04273971 0.38537964 0.33423936 0.04273971 0.28309907 0.04273971
 [25] 0.23195878 0.04273971 0.04273971 0.04273971 0.48766022 0.43651993
 [31] 0.38537964 0.33423936 0.04273971 0.28309907 0.23195878 0.04273971
 [37] 0.18081849 0.04273971 0.04273971 0.04273971 0.33423936 0.04273971
 [43] 0.04273971 0.38537964 0.33423936 0.28309907 0.04273971 0.04273971
 [49] 0.28309907 0.04273971 0.04273971 0.04273971 0.04273971 0.58994080
 [55] 0.53880051 0.48766022 0.04273971 0.43651993 0.04273971 0.38537964
 [61] 0.04273971 0.04273971 0.04273971 0.04273971 0.69222138 0.64108109
 [67] 0.58994080 0.53880051 0.48766022 0.04273971 0.43651993 0.04273971
 [73] 0.04273971 0.48766022 0.43651993 0.04273971 0.38537964 0.04273971
 [79] 0.04273971 0.43651993 0.38537964 0.04273971 0.04273971 0.38537964
 [85] 0.33423936 0.04273971 0.04273971 0.04273971 0.04273971 0.58994080
 [91] 0.53880051 0.48766022 0.43651993 0.04273971 0.38537964 0.04273971
 [97] 0.04273971 0.43651993 0.38537964 0.04273971 0.04273971 0.04273971
[103] 0.04273971 0.04273971 0.64108109 0.58994080 0.04273971 0.04273971
[109] 0.58994080 0.53880051 0.04273971 0.04273971 0.53880051 0.48766022
[115] 0.43651993 0.04273971 0.04273971 0.04273971 0.04273971 0.04273971
[121] 0.84564224 0.79450195 0.74336167 0.04273971 0.04273971 0.04273971
[127] 0.04273971


$split.labs
[1] ""

$split.cex
  [1] 1 1 1 1 1 1 1 1 1 1 1 1 1 1 1 1 1 1 1 1 1 1 1 1 1 1 1 1 1 1 1 1 1 1 1 1 1
 [38] 1 1 1 1 1 1 1 1 1 1 1 1 1 1 1 1 1 1 1 1 1 1 1 1 1 1 1 1 1 1 1 1 1 1 1 1 1
 [75] 1 1 1 1 1 1 1 1 1 1 1 1 1 1 1 1 1 1 1 1 1 1 1 1 1 1 1 1 1 1 1 1 1 1 1 1 1
[112] 1 1 1 1 1 1 1 1 1 1 1 1 1 1 1 1

$split.box
$split.box$x1
  [1] 0.16496238         NA 0.32760782         NA 0.65402237 0.34851817
  [7]         NA 0.68024913 0.44806749 0.15750893         NA 0.28692786
 [13] 0.12635288         NA 0.21083947 0.09259357         NA 0.11185337
 [19]         NA 0.13693783 0.11183698         NA 0.12821992         NA
 [25] 0.14156833         NA         NA         NA 0.32565007 0.27922832
 [31] 0.23098885 0.20131658         NA 0.22538583 0.20345145         NA
 [37] 0.22125574         NA         NA         NA 0.26803760         NA
 [43]         NA 0.32648543 0.31481946 0.29311668         NA         NA
 [49] 0.32869576         NA         NA         NA         NA 0.44298226
 [55] 0.41689368 0.39081049         NA 0.41056402         NA 0.41786831
 [61]         NA         NA         NA         NA 0.74808782 0.67165296
 [67] 0.58216242 0.52026186 0.48250731         NA 0.50194689         NA
 [73]         NA 0.56747817 0.53703329         NA 0.54098464         NA
 [79]         NA 0.60135834 0.57819901         NA         NA 0.62108238
 [85] 0.61110456         NA         NA         NA         NA 0.75174644
 [91] 0.72858711 0.70058005 0.67737886         NA 0.68570374         NA
 [97]         NA 0.72249593 0.72026223         NA         NA         NA
[103]         NA         NA 0.82630503 0.79823200         NA         NA
[109] 0.85431336 0.82974343         NA         NA 0.88429504 0.86455916
[115] 0.85114605         NA         NA         NA         NA         NA
[121] 0.96391775 0.95059657 0.93246875         NA         NA         NA
[127]         NA

$split.box$y1
  [1] 0.9338857        NA 0.8827454        NA 0.8316051 0.7804648        NA
  [8] 0.7293245 0.6781842 0.6270440        NA 0.5759037 0.5247634        NA
 [15] 0.4736231 0.4224828        NA 0.3713425        NA 0.3202022 0.2690619
 [22]        NA 0.2179216        NA 0.1667814        NA        NA        NA
 [29] 0.4224828 0.3713425 0.3202022 0.2690619        NA 0.2179216 0.1667814
 [36]        NA 0.1156411        NA        NA        NA 0.2690619        NA
 [43]        NA 0.3202022 0.2690619 0.2179216        NA        NA 0.2179216
 [50]        NA        NA        NA        NA 0.5247634 0.4736231 0.4224828
 [57]        NA 0.3713425        NA 0.3202022        NA        NA        NA
 [64]        NA 0.6270440 0.5759037 0.5247634 0.4736231 0.4224828        NA
 [71] 0.3713425        NA        NA 0.4224828 0.3713425        NA 0.3202022
 [78]        NA        NA 0.3713425 0.3202022        NA        NA 0.3202022
 [85] 0.2690619        NA        NA        NA        NA 0.5247634 0.4736231
 [92] 0.4224828 0.3713425        NA 0.3202022        NA        NA 0.3713425
 [99] 0.3202022        NA        NA        NA        NA        NA 0.5759037
[106] 0.5247634        NA        NA 0.5247634 0.4736231        NA        NA
[113] 0.4736231 0.4224828 0.3713425        NA        NA        NA        NA
[120]        NA 0.7804648 0.7293245 0.6781842        NA        NA        NA
[127]        NA

$split.box$x2
  [1] 0.1929094        NA 0.3705549        NA 0.6935341 0.3914652        NA
  [8] 0.7197609 0.4942851 0.2002913        NA 0.3175279 0.1692999        NA
 [15] 0.2381336 0.1233230        NA 0.1464592        NA 0.1749790 0.1513487
 [22]        NA 0.1622552        NA 0.1722977        NA        NA        NA
 [29] 0.3563795 0.3187401 0.2690300 0.2353519        NA 0.2561152 0.2429632
 [36]        NA 0.2485499        NA        NA        NA 0.2953317        NA
 [43]        NA 0.3694325 0.3421136 0.3326284        NA        NA 0.3594252
 [50]        NA        NA        NA        NA 0.4702764 0.4564054 0.4337575
 [57]        NA 0.4412934        NA 0.4573801        NA        NA        NA
 [64]        NA 0.7788172 0.6989471 0.6249448 0.5664795 0.5252897        NA
 [71] 0.5292410        NA        NA 0.5982076 0.5643274        NA 0.5837670
 [78]        NA        NA 0.6286525 0.6089284        NA        NA 0.6518118
 [85] 0.6383987        NA        NA        NA        NA 0.7823464 0.7626224
 [92] 0.7360506 0.7046730        NA 0.7197390        NA        NA 0.7687136
 [99] 0.7475564        NA        NA        NA        NA        NA 0.8569050
[106] 0.8255261        NA        NA 0.8883487 0.8563905        NA        NA
[113] 0.9148950 0.9073415 0.8973637        NA        NA        NA        NA
[120]        NA 0.9912119 0.9772436 0.9719805        NA        NA        NA
[127]        NA

$split.box$y2
  [1] 0.9512567        NA 0.9001164        NA 0.8489761 0.7978358        NA
  [8] 0.7466955 0.6955552 0.6444149        NA 0.5932746 0.5421343        NA
 [15] 0.4909941 0.4398538        NA 0.3887135        NA 0.3375732 0.2864329
 [22]        NA 0.2352926        NA 0.1841523        NA        NA        NA
 [29] 0.4398538 0.3887135 0.3375732 0.2864329        NA 0.2352926 0.1841523
 [36]        NA 0.1330120        NA        NA        NA 0.2864329        NA
 [43]        NA 0.3375732 0.2864329 0.2352926        NA        NA 0.2352926
 [50]        NA        NA        NA        NA 0.5421343 0.4909941 0.4398538
 [57]        NA 0.3887135        NA 0.3375732        NA        NA        NA
 [64]        NA 0.6444149 0.5932746 0.5421343 0.4909941 0.4398538        NA
 [71] 0.3887135        NA        NA 0.4398538 0.3887135        NA 0.3375732
 [78]        NA        NA 0.3887135 0.3375732        NA        NA 0.3375732
 [85] 0.2864329        NA        NA        NA        NA 0.5421343 0.4909941
 [92] 0.4398538 0.3887135        NA 0.3375732        NA        NA 0.3887135
 [99] 0.3375732        NA        NA        NA        NA        NA 0.5932746
[106] 0.5421343        NA        NA 0.5421343 0.4909941        NA        NA
[113] 0.4909941 0.4398538 0.3887135        NA        NA        NA        NA
[120]        NA 0.7978358 0.7466955 0.6955552        NA        NA        NA
[127]        NA
\end{verbatim}

\begin{Shaded}
\begin{Highlighting}[]
\NormalTok{AMOX\_CMIN\_TRAINED\_CT\_confusion\_matrix }\OtherTok{\textless{}{-}}\NormalTok{ AMOX\_CMIN\_TRAINED\_CT}\SpecialCharTok{$}\NormalTok{confusion\_matrix}
\FunctionTok{print}\NormalTok{(AMOX\_CMIN\_TRAINED\_CT\_confusion\_matrix)}
\end{Highlighting}
\end{Shaded}

\begin{verbatim}
$CARLIER
\end{verbatim}

\pandocbounded{\includegraphics[keepaspectratio]{MIPD/Precision_dosing_methods_files/figure-pdf/evaluate-prediction-ct-train-1.pdf}}

\begin{verbatim}

$FOURNIER
\end{verbatim}

\pandocbounded{\includegraphics[keepaspectratio]{MIPD/Precision_dosing_methods_files/figure-pdf/evaluate-prediction-ct-train-2.pdf}}

\begin{verbatim}

$MELLON
\end{verbatim}

\pandocbounded{\includegraphics[keepaspectratio]{MIPD/Precision_dosing_methods_files/figure-pdf/evaluate-prediction-ct-train-3.pdf}}

\begin{verbatim}

$RAMBAUD
\end{verbatim}

\pandocbounded{\includegraphics[keepaspectratio]{MIPD/Precision_dosing_methods_files/figure-pdf/evaluate-prediction-ct-train-4.pdf}}

\begin{Shaded}
\begin{Highlighting}[]
\NormalTok{cfit\_list }\OtherTok{\textless{}{-}}\NormalTok{ AMOX\_CMIN\_TRAINED\_CT}\SpecialCharTok{$}\NormalTok{cfit\_list}

\CommentTok{\# Export the plot of the four trees}
\FunctionTok{jpeg}\NormalTok{(}\FunctionTok{here}\NormalTok{(}\StringTok{"Amoxicillin/a\_priori/For\_publication/Figures/S3.jpg"}\NormalTok{), }\AttributeTok{width =} \DecValTok{7}\NormalTok{, }\AttributeTok{height =} \DecValTok{7}\NormalTok{, }\AttributeTok{units =} \StringTok{"in"}\NormalTok{, }\AttributeTok{res =} \DecValTok{300}\NormalTok{)}
\NormalTok{n\_trees }\OtherTok{\textless{}{-}} \FunctionTok{length}\NormalTok{(cfit\_list)}
\FunctionTok{par}\NormalTok{(}\AttributeTok{mfrow =} \FunctionTok{c}\NormalTok{(}\FunctionTok{ceiling}\NormalTok{(n\_trees }\SpecialCharTok{/} \DecValTok{2}\NormalTok{), }\DecValTok{2}\NormalTok{))}

\ControlFlowTok{for}\NormalTok{ (model\_name }\ControlFlowTok{in} \FunctionTok{names}\NormalTok{(cfit\_list)) \{}
  \FunctionTok{rpart.plot}\NormalTok{(cfit\_list[[model\_name]],}
             \AttributeTok{main =}\NormalTok{ model\_name,}
             \AttributeTok{roundint =} \ConstantTok{FALSE}\NormalTok{)}
\NormalTok{\}}
\FunctionTok{dev.off}\NormalTok{()}
\end{Highlighting}
\end{Shaded}

\begin{verbatim}
pdf 
  2 
\end{verbatim}

\begin{Shaded}
\begin{Highlighting}[]
\CommentTok{\# Export the plot of the four confusion matrices}
\NormalTok{confusion\_plot }\OtherTok{\textless{}{-}}
\NormalTok{  (AMOX\_CMIN\_TRAINED\_CT\_confusion\_matrix}\SpecialCharTok{$}\NormalTok{CARLIER  }\SpecialCharTok{+} \FunctionTok{ggtitle}\NormalTok{(}\StringTok{"Carlier"}\NormalTok{))  }\SpecialCharTok{+}
\NormalTok{  (AMOX\_CMIN\_TRAINED\_CT\_confusion\_matrix}\SpecialCharTok{$}\NormalTok{FOURNIER }\SpecialCharTok{+} \FunctionTok{ggtitle}\NormalTok{(}\StringTok{"Fournier"}\NormalTok{)) }\SpecialCharTok{+}
\NormalTok{  (AMOX\_CMIN\_TRAINED\_CT\_confusion\_matrix}\SpecialCharTok{$}\NormalTok{MELLON   }\SpecialCharTok{+} \FunctionTok{ggtitle}\NormalTok{(}\StringTok{"Mellon"}\NormalTok{))   }\SpecialCharTok{+}
\NormalTok{  (AMOX\_CMIN\_TRAINED\_CT\_confusion\_matrix}\SpecialCharTok{$}\NormalTok{RAMBAUD  }\SpecialCharTok{+} \FunctionTok{ggtitle}\NormalTok{(}\StringTok{"Rambaud"}\NormalTok{))  }\SpecialCharTok{+}
  \FunctionTok{plot\_layout}\NormalTok{(}\AttributeTok{ncol =} \DecValTok{2}\NormalTok{)}
\FunctionTok{ggsave}\NormalTok{(}\FunctionTok{here}\NormalTok{(}\StringTok{"Amoxicillin/a\_priori/For\_publication/Figures/S4.jpg"}\NormalTok{),}
       \AttributeTok{plot =}\NormalTok{ confusion\_plot,}
       \AttributeTok{width =} \DecValTok{7}\NormalTok{, }\AttributeTok{height =} \DecValTok{7}\NormalTok{, }\AttributeTok{units =} \StringTok{"in"}\NormalTok{, }\AttributeTok{dpi =} \DecValTok{300}\NormalTok{)}
\end{Highlighting}
\end{Shaded}

\begin{Shaded}
\begin{Highlighting}[]
\NormalTok{AMOX\_CMIN\_PRED\_TEST\_CT }\OtherTok{\textless{}{-}} \FunctionTok{classification\_tree\_model\_ensembling\_test}\NormalTok{(}
  \AttributeTok{test\_data =}\NormalTok{ AMOX\_CMIN\_TEST, }
  \AttributeTok{train\_results =}\NormalTok{ AMOX\_CMIN\_TRAINED\_CT) }
\end{Highlighting}
\end{Shaded}

To evaluate the method, a predictions as a function of observed
concentrations goodness of fit plot is presented as well a boxplot
indicating the predicted/observed ratios in \%. Then, the average model
weights are plotted, stratified by covariate categories/quantiles.

\begin{Shaded}
\begin{Highlighting}[]
\NormalTok{test\_data\_CT }\OtherTok{\textless{}{-}}\NormalTok{ AMOX\_CMIN\_PRED\_TEST\_CT}\SpecialCharTok{$}\NormalTok{test\_results}
\NormalTok{AMOX\_CMIN\_PRED\_TEST\_CT}\SpecialCharTok{$}\NormalTok{GOF\_plot}
\end{Highlighting}
\end{Shaded}

\pandocbounded{\includegraphics[keepaspectratio]{MIPD/Precision_dosing_methods_files/figure-pdf/evaluate-prediction-ct-test-1.pdf}}

\begin{Shaded}
\begin{Highlighting}[]
\NormalTok{AMOX\_CMIN\_PRED\_TEST\_CT}\SpecialCharTok{$}\NormalTok{Boxplot}
\end{Highlighting}
\end{Shaded}

\pandocbounded{\includegraphics[keepaspectratio]{MIPD/Precision_dosing_methods_files/figure-pdf/evaluate-prediction-ct-test-2.pdf}}

\begin{Shaded}
\begin{Highlighting}[]
\CommentTok{\# Plots showing the average attributed weight stratified by covariates}
\NormalTok{weight\_BURN\_CT }\OtherTok{\textless{}{-}} \FunctionTok{generate\_weight\_plot}\NormalTok{(test\_data\_CT, }\StringTok{"BURN"}\NormalTok{, }\AttributeTok{is\_categorical =} \ConstantTok{TRUE}\NormalTok{)}
\NormalTok{weight\_ICU\_CT }\OtherTok{\textless{}{-}} \FunctionTok{generate\_weight\_plot}\NormalTok{(test\_data\_CT, }\StringTok{"ICU"}\NormalTok{, }\AttributeTok{is\_categorical =} \ConstantTok{TRUE}\NormalTok{)}
\NormalTok{weight\_OBESE\_CT }\OtherTok{\textless{}{-}} \FunctionTok{generate\_weight\_plot}\NormalTok{(test\_data\_CT, }\StringTok{"OBESE"}\NormalTok{, }\AttributeTok{is\_categorical =} \ConstantTok{TRUE}\NormalTok{)}
\NormalTok{weight\_CREAT\_CT }\OtherTok{\textless{}{-}} \FunctionTok{generate\_weight\_plot}\NormalTok{(test\_data\_CT, }\StringTok{"CREAT"}\NormalTok{, }\AttributeTok{is\_categorical =} \ConstantTok{FALSE}\NormalTok{)}
\NormalTok{weight\_WT\_CT }\OtherTok{\textless{}{-}} \FunctionTok{generate\_weight\_plot}\NormalTok{(test\_data\_CT, }\StringTok{"WT"}\NormalTok{, }\AttributeTok{is\_categorical =} \ConstantTok{FALSE}\NormalTok{)}
\NormalTok{weight\_AGE\_CT }\OtherTok{\textless{}{-}} \FunctionTok{generate\_weight\_plot}\NormalTok{(test\_data\_CT, }\StringTok{"AGE"}\NormalTok{, }\AttributeTok{is\_categorical =} \ConstantTok{FALSE}\NormalTok{)}
\NormalTok{weight\_SEX\_CT }\OtherTok{\textless{}{-}} \FunctionTok{generate\_weight\_plot}\NormalTok{(test\_data\_CT, }\StringTok{"SEX"}\NormalTok{, }\AttributeTok{is\_categorical =} \ConstantTok{TRUE}\NormalTok{)}
\NormalTok{weight\_CON\_CT }\OtherTok{\textless{}{-}} \FunctionTok{generate\_weight\_plot}\NormalTok{(test\_data\_CT, }\StringTok{"CON"}\NormalTok{, }\AttributeTok{is\_categorical =} \ConstantTok{TRUE}\NormalTok{)}
\NormalTok{weight\_BURN\_CT}
\end{Highlighting}
\end{Shaded}

\pandocbounded{\includegraphics[keepaspectratio]{MIPD/Precision_dosing_methods_files/figure-pdf/evaluate-prediction-ct-test-3.pdf}}

\begin{Shaded}
\begin{Highlighting}[]
\NormalTok{weight\_ICU\_CT}
\end{Highlighting}
\end{Shaded}

\pandocbounded{\includegraphics[keepaspectratio]{MIPD/Precision_dosing_methods_files/figure-pdf/evaluate-prediction-ct-test-4.pdf}}

\begin{Shaded}
\begin{Highlighting}[]
\NormalTok{weight\_OBESE\_CT}
\end{Highlighting}
\end{Shaded}

\pandocbounded{\includegraphics[keepaspectratio]{MIPD/Precision_dosing_methods_files/figure-pdf/evaluate-prediction-ct-test-5.pdf}}

\begin{Shaded}
\begin{Highlighting}[]
\NormalTok{weight\_CREAT\_CT}
\end{Highlighting}
\end{Shaded}

\pandocbounded{\includegraphics[keepaspectratio]{MIPD/Precision_dosing_methods_files/figure-pdf/evaluate-prediction-ct-test-6.pdf}}

\begin{Shaded}
\begin{Highlighting}[]
\NormalTok{weight\_WT\_CT}
\end{Highlighting}
\end{Shaded}

\pandocbounded{\includegraphics[keepaspectratio]{MIPD/Precision_dosing_methods_files/figure-pdf/evaluate-prediction-ct-test-7.pdf}}

\begin{Shaded}
\begin{Highlighting}[]
\NormalTok{weight\_AGE\_CT}
\end{Highlighting}
\end{Shaded}

\pandocbounded{\includegraphics[keepaspectratio]{MIPD/Precision_dosing_methods_files/figure-pdf/evaluate-prediction-ct-test-8.pdf}}

\begin{Shaded}
\begin{Highlighting}[]
\NormalTok{weight\_SEX\_CT}
\end{Highlighting}
\end{Shaded}

\pandocbounded{\includegraphics[keepaspectratio]{MIPD/Precision_dosing_methods_files/figure-pdf/evaluate-prediction-ct-test-9.pdf}}

\begin{Shaded}
\begin{Highlighting}[]
\NormalTok{weight\_CON\_CT}
\end{Highlighting}
\end{Shaded}

\pandocbounded{\includegraphics[keepaspectratio]{MIPD/Precision_dosing_methods_files/figure-pdf/evaluate-prediction-ct-test-10.pdf}}

\begin{Shaded}
\begin{Highlighting}[]
\CommentTok{\# Print overall average weight}
\NormalTok{test\_data\_CT }\SpecialCharTok{\%\textgreater{}\%}
  \FunctionTok{group\_by}\NormalTok{(MODEL) }\SpecialCharTok{\%\textgreater{}\%}
  \FunctionTok{summarise}\NormalTok{(}\AttributeTok{avg\_weight =} \FunctionTok{mean}\NormalTok{(WEIGHT, }\AttributeTok{na.rm =} \ConstantTok{TRUE}\NormalTok{))}
\end{Highlighting}
\end{Shaded}

\begin{verbatim}
# A tibble: 4 x 2
  MODEL    avg_weight
  <chr>         <dbl>
1 CARLIER      0.240 
2 FOURNIER     0.281 
3 MELLON       0.394 
4 RAMBAUD      0.0847
\end{verbatim}

\begin{Shaded}
\begin{Highlighting}[]
\CommentTok{\# Extrapolate dose based on the ensembled concentrations and compare it to the true simulated dose.}
\NormalTok{test\_data\_CT }\OtherTok{\textless{}{-}}\NormalTok{ test\_data\_CT }\SpecialCharTok{\%\textgreater{}\%}
\NormalTok{  dplyr}\SpecialCharTok{::}\FunctionTok{filter}\NormalTok{(REFERENCE }\SpecialCharTok{==} \DecValTok{1}\NormalTok{) }\SpecialCharTok{\%\textgreater{}\%}
\NormalTok{  dplyr}\SpecialCharTok{::}\FunctionTok{mutate}\NormalTok{(}\AttributeTok{DOSE\_PRED =}\NormalTok{ (}\DecValTok{60}\SpecialCharTok{/}\NormalTok{WEIGHTED\_PREDICTION) }\SpecialCharTok{*}\NormalTok{ DOSE\_ADM) }\CommentTok{\# Administered dose extrapolation to reach 60 mg/L}

\NormalTok{final\_results\_CT }\OtherTok{\textless{}{-}} \FunctionTok{explore\_predictions}\NormalTok{(test\_data\_CT)}
\NormalTok{final\_results\_CT}\SpecialCharTok{$}\NormalTok{target\_attainment }
\end{Highlighting}
\end{Shaded}

\pandocbounded{\includegraphics[keepaspectratio]{MIPD/Precision_dosing_methods_files/figure-pdf/evaluate-prediction-ct-test-11.pdf}}

\begin{Shaded}
\begin{Highlighting}[]
\NormalTok{crcl\_plot\_CT }\OtherTok{\textless{}{-}}\NormalTok{ final\_results\_CT}\SpecialCharTok{$}\NormalTok{crcl\_plot}
\NormalTok{crcl\_plot\_CT}
\end{Highlighting}
\end{Shaded}

\pandocbounded{\includegraphics[keepaspectratio]{MIPD/Precision_dosing_methods_files/figure-pdf/evaluate-prediction-ct-test-12.pdf}}

\begin{Shaded}
\begin{Highlighting}[]
\NormalTok{final\_results\_CT}\SpecialCharTok{$}\NormalTok{summary\_stats }
\end{Highlighting}
\end{Shaded}

\begin{verbatim}
# A tibble: 2 x 9
  Prediction_correctness mean_CREAT sd_CREAT mean_WT sd_WT mean_AGE sd_AGE Count
  <chr>                       <dbl>    <dbl>   <dbl> <dbl>    <dbl>  <dbl> <int>
1 Correct                     0.967    0.525    85.0  19.4     63.7   16.1   234
2 Incorrect                   0.939    0.611    84.0  18.3     57.5   16.1   366
# i 1 more variable: Proportion <dbl>
\end{verbatim}

\begin{Shaded}
\begin{Highlighting}[]
\CommentTok{\# Keep data for method ensembling}
\NormalTok{test\_data\_CT\_ens }\OtherTok{\textless{}{-}}\NormalTok{ test\_data\_CT }\SpecialCharTok{\%\textgreater{}\%}
\NormalTok{  dplyr}\SpecialCharTok{::}\FunctionTok{mutate}\NormalTok{(}\AttributeTok{CT =}\NormalTok{ DOSE\_PRED) }\SpecialCharTok{\%\textgreater{}\%} 
\NormalTok{  dplyr}\SpecialCharTok{::}\FunctionTok{select}\NormalTok{(}\StringTok{"ID"}\NormalTok{,}\StringTok{"CT"}\NormalTok{)}
\end{Highlighting}
\end{Shaded}

\chapter{Regression tree informed
ensembling}\label{regression-tree-informed-ensembling}

This method is similar to the previous decision tree-based method, with
the difference that not prediction correctness is used as a target
variable, but the log-transformed prediction/observation ratio:

\[
\text{Ratio} = \left( log (\frac{predicted}{true}) \right)
\]

Therefore these are not classification, but regression trees.

Our \emph{openMIPD} package is used to apply this method.

\begin{Shaded}
\begin{Highlighting}[]
\NormalTok{train\_results\_RT }\OtherTok{\textless{}{-}} \FunctionTok{regression\_tree\_model\_ensembling\_train}\NormalTok{(}
  \AttributeTok{data =}\NormalTok{ AMOX\_CMIN\_TRAIN, }
  \AttributeTok{target\_variable =} \StringTok{"CMIN"}\NormalTok{, }
   \AttributeTok{continuous\_cov =} \FunctionTok{c}\NormalTok{(}\StringTok{"WT"}\NormalTok{, }\StringTok{"CREAT"}\NormalTok{, }\StringTok{"AGE"}\NormalTok{), }
  \AttributeTok{categorical\_cov =} \FunctionTok{c}\NormalTok{(}\StringTok{"OBESE"}\NormalTok{, }\StringTok{"ICU"}\NormalTok{, }\StringTok{"BURN"}\NormalTok{, }\StringTok{"SEX"}\NormalTok{, }\StringTok{"CON"}\NormalTok{))}
\end{Highlighting}
\end{Shaded}

\pandocbounded{\includegraphics[keepaspectratio]{MIPD/Precision_dosing_methods_files/figure-pdf/train-RT-1.pdf}}

\pandocbounded{\includegraphics[keepaspectratio]{MIPD/Precision_dosing_methods_files/figure-pdf/train-RT-2.pdf}}

\pandocbounded{\includegraphics[keepaspectratio]{MIPD/Precision_dosing_methods_files/figure-pdf/train-RT-3.pdf}}

\pandocbounded{\includegraphics[keepaspectratio]{MIPD/Precision_dosing_methods_files/figure-pdf/train-RT-4.pdf}}

To evaluate the method training, the regression trees are visualized. On
the visualized nodes, the first line indicates average log-transformed
prediction/observation ratio and the second line, percentage of subjects
in that particular node.

\begin{Shaded}
\begin{Highlighting}[]
\NormalTok{model\_trees }\OtherTok{\textless{}{-}}\NormalTok{ train\_results\_RT}\SpecialCharTok{$}\NormalTok{model\_trees}
\FunctionTok{print}\NormalTok{(model\_trees)}
\end{Highlighting}
\end{Shaded}

\begin{verbatim}
$CARLIER
n= 1875 

node), split, n, deviance, yval
      * denotes terminal node

 1) root 1875 2692.425000  0.69638490  
   2) ICU=1 1400 1853.970000  0.40045100  
     4) BURN=0 925  574.951600  0.20183820  
       8) CON=0 475  506.481000  0.08100565  
        16) AGE< 40.85028 12    7.321785 -0.68656630 *
        17) AGE>=40.85028 463  491.906000  0.10089950 *
       9) CON=1 450   54.214780  0.32938360  
        18) CREAT< 0.616343 49    6.462869  0.01440493 *
        19) CREAT>=0.616343 401   42.296510  0.36787230 *
     5) BURN=1 475 1171.473000  0.78722330  
      10) CREAT< 0.5087704 76  174.040400 -0.48906900  
        20) AGE>=59.18075 33   49.227520 -1.28384300 *
        21) AGE< 59.18075 43   87.970590  0.12087400 *
      11) CREAT>=0.5087704 399  850.054100  1.03032700  
        22) CREAT< 0.9848673 253  524.074200  0.75196240 *
        23) CREAT>=0.9848673 146  272.404300  1.51269800 *
   3) ICU=0 475  354.477200  1.56861100  
     6) CREAT< 0.6542423 191  120.165400  1.20338200  
      12) AGE>=38.19377 190  108.736200  1.18563500  
        24) WT>=99.06353 155   75.117550  1.08629700 *
        25) WT< 99.06353 35   25.315370  1.62556000 *
      13) AGE< 38.19377 1    0.000000  4.57524000 *
     7) CREAT>=0.6542423 284  191.699000  1.81424100  
      14) AGE< 54.2105 177   90.278490  1.64723700 *
      15) AGE>=54.2105 107   88.317920  2.09049800  
        30) WT>=97.69869 92   53.747770  1.97854400 *
        31) WT< 97.69869 15   26.344670  2.77715100 *

$FOURNIER
n= 1875 

node), split, n, deviance, yval
      * denotes terminal node

 1) root 1875 2628.93300  0.26753090  
   2) ICU=1 1400 1830.01400 -0.03066269  
     4) CREAT>=1.178378 372  377.13490 -0.41182870  
       8) BURN=0 268  147.10200 -0.64752560  
        16) CON=0 113   56.58890 -1.24862500 *
        17) CON=1 155   19.91817 -0.20930510 *
       9) BURN=1 104  176.77890  0.19554420 *
     5) CREAT< 1.178378 1028 1379.27400  0.10726900  
      10) WT< 80.18777 715  927.39920  0.01951742 *
      11) WT>=80.18777 313  433.79230  0.30772380  
        22) BURN=0 182  126.53870  0.08750073 *
        23) BURN=1 131  286.16390  0.61368260 *
   3) ICU=0 475  307.52240  1.14641700  
     6) CREAT< 0.6542423 191  110.69490  0.88276950 *
     7) CREAT>=0.6542423 284  174.62220  1.32373000 *

$MELLON
n= 1875 

node), split, n, deviance, yval
      * denotes terminal node

 1) root 1875 3.868175e+03 -0.70212590  
   2) CREAT>=1.166427 380 8.243601e+02 -1.92133400  
     4) CON=0 223 4.447690e+02 -2.73001500  
       8) BURN=0 117 9.745607e+01 -3.49990300  
        16) ICU=1 115 7.326346e+01 -3.55985900 *
        17) ICU=0 2 9.246126e-03 -0.05244169 *
       9) BURN=1 106 2.014183e+02 -1.88023400  
        18) CREAT>=1.865161 38 3.811241e+01 -2.55213000 *
        19) CREAT< 1.865161 68 1.365644e+02 -1.50476200 *
     5) CON=1 157 2.661634e+01 -0.77269770 *
   3) CREAT< 1.166427 1495 2.335380e+03 -0.39222680  
     6) WT< 89.17191 922 1.730529e+03 -0.65514230  
      12) CREAT>=0.6333707 564 8.670489e+02 -0.95270150  
        24) CON=0 365 6.742422e+02 -1.35980700  
          48) AGE>=52.98266 227 3.125459e+02 -1.67203100 *
          49) AGE< 52.98266 138 3.031671e+02 -0.84622140 *
        25) CON=1 199 2.135837e+01 -0.20600000 *
      13) CREAT< 0.6333707 358 7.348699e+02 -0.18636200  
        26) AGE>=47.39281 266 5.162627e+02 -0.40285310  
          52) BURN=1 78 1.466561e+02 -1.02542600 *
          53) BURN=0 188 3.268308e+02 -0.14455150 *
        27) AGE< 47.39281 92 1.700943e+02  0.43957980 *
     7) WT>=89.17191 573 4.385681e+02  0.03082408 *

$RAMBAUD
n=1873 (2 observations effacées parce que manquantes)

node), split, n, deviance, yval
      * denotes terminal node

 1) root 1873 63975.60000  -8.114526000  
   2) CON=0 1423 24946.16000 -10.680160000  
     4) CREAT< 1.076506 1160 13492.49000 -11.858430000  
       8) ICU=1 692  6463.13500 -13.720700000  
        16) BURN=0 346  1561.07500 -14.941320000  
          32) AGE< 62.58397 181   573.61180 -16.053300000 *
          33) AGE>=62.58397 165   518.14930 -13.721520000 *
        17) BURN=1 346  3871.04000 -12.500080000  
          34) CREAT< 0.6717583 182  1707.51400 -13.750540000 *
          35) CREAT>=0.6717583 164  1563.12200 -11.112380000 *
       9) ICU=0 468  1080.86600  -9.104804000  
        18) CREAT< 0.7492174 331   326.76760  -9.680235000 *
        19) CREAT>=0.7492174 137   379.69480  -7.714528000  
          38) SEX=0 53   100.81630  -8.885622000 *
          39) SEX=1 84   160.32880  -6.975624000 *
     5) CREAT>=1.076506 263  2740.09400  -5.483224000  
      10) CREAT< 1.487287 127   660.62490  -7.787233000  
        20) BURN=0 68   276.66960  -8.604884000  
          40) WT< 93.02454 61   162.29470  -9.011419000 *
          41) WT>=93.02454 7    16.44012  -5.062221000 *
        21) BURN=1 59   286.09730  -6.844855000 *
      11) CREAT>=1.487287 136   775.73540  -3.331686000  
        22) CREAT< 2.391933 95   337.58880  -4.376843000  
          44) BURN=0 50    78.72085  -5.501902000 *
          45) BURN=1 45   125.26020  -3.126778000 *
        23) CREAT>=2.391933 41    93.92215  -0.909980700 *
   3) CON=1 450    42.59539  -0.001435875 *
\end{verbatim}

\begin{Shaded}
\begin{Highlighting}[]
\CommentTok{\# Export the plot of the four trees}
\FunctionTok{jpeg}\NormalTok{(}\FunctionTok{here}\NormalTok{(}\StringTok{"Amoxicillin/a\_priori/For\_publication/Figures/S5.jpg"}\NormalTok{), }\AttributeTok{width =} \DecValTok{7}\NormalTok{, }\AttributeTok{height =} \DecValTok{7}\NormalTok{, }\AttributeTok{units =} \StringTok{"in"}\NormalTok{, }\AttributeTok{res =} \DecValTok{300}\NormalTok{)}
\NormalTok{n\_trees }\OtherTok{\textless{}{-}} \FunctionTok{length}\NormalTok{(model\_trees)}
\FunctionTok{par}\NormalTok{(}\AttributeTok{mfrow =} \FunctionTok{c}\NormalTok{(}\FunctionTok{ceiling}\NormalTok{(n\_trees }\SpecialCharTok{/} \DecValTok{2}\NormalTok{), }\DecValTok{2}\NormalTok{))}

\ControlFlowTok{for}\NormalTok{ (model\_name }\ControlFlowTok{in} \FunctionTok{names}\NormalTok{(model\_trees)) \{}
  \FunctionTok{rpart.plot}\NormalTok{(model\_trees[[model\_name]],}
             \AttributeTok{main =}\NormalTok{ model\_name,}
             \AttributeTok{roundint =} \ConstantTok{FALSE}\NormalTok{)}
\NormalTok{\}}
\FunctionTok{dev.off}\NormalTok{()}
\end{Highlighting}
\end{Shaded}

\begin{verbatim}
pdf 
  2 
\end{verbatim}

\begin{Shaded}
\begin{Highlighting}[]
\NormalTok{test\_results\_RT }\OtherTok{\textless{}{-}} \FunctionTok{regression\_tree\_model\_ensembling\_test}\NormalTok{(}
  \AttributeTok{test\_data =}\NormalTok{ AMOX\_CMIN\_TEST, }
  \AttributeTok{train\_results =}\NormalTok{ train\_results\_RT) }
\end{Highlighting}
\end{Shaded}

To evaluate the method, a predictions as a function of observed
concentrations goodness of fit plot is presented as well a boxplot
indicating the predicted/observed ratios in \%. Then, the average model
weights are plotted, stratified by covariate categories/quantiles.

\begin{Shaded}
\begin{Highlighting}[]
\NormalTok{test\_data\_RT }\OtherTok{\textless{}{-}}\NormalTok{ test\_results\_RT}\SpecialCharTok{$}\NormalTok{test\_results}
\NormalTok{test\_results\_RT}\SpecialCharTok{$}\NormalTok{GOF\_plot}
\end{Highlighting}
\end{Shaded}

\pandocbounded{\includegraphics[keepaspectratio]{MIPD/Precision_dosing_methods_files/figure-pdf/evaluate-prediction-RT-1.pdf}}

\begin{Shaded}
\begin{Highlighting}[]
\NormalTok{test\_results\_RT}\SpecialCharTok{$}\NormalTok{Boxplot}
\end{Highlighting}
\end{Shaded}

\pandocbounded{\includegraphics[keepaspectratio]{MIPD/Precision_dosing_methods_files/figure-pdf/evaluate-prediction-RT-2.pdf}}

\begin{Shaded}
\begin{Highlighting}[]
\CommentTok{\# Plots showing the average attributed weight stratified by covariates}
\NormalTok{weight\_BURN\_RT }\OtherTok{\textless{}{-}} \FunctionTok{generate\_weight\_plot}\NormalTok{(test\_data\_RT, }\StringTok{"BURN"}\NormalTok{, }\AttributeTok{is\_categorical =} \ConstantTok{TRUE}\NormalTok{)}
\NormalTok{weight\_ICU\_RT }\OtherTok{\textless{}{-}} \FunctionTok{generate\_weight\_plot}\NormalTok{(test\_data\_RT, }\StringTok{"ICU"}\NormalTok{, }\AttributeTok{is\_categorical =} \ConstantTok{TRUE}\NormalTok{)}
\NormalTok{weight\_OBESE\_RT }\OtherTok{\textless{}{-}} \FunctionTok{generate\_weight\_plot}\NormalTok{(test\_data\_RT, }\StringTok{"OBESE"}\NormalTok{, }\AttributeTok{is\_categorical =} \ConstantTok{TRUE}\NormalTok{)}
\NormalTok{weight\_CREAT\_RT }\OtherTok{\textless{}{-}} \FunctionTok{generate\_weight\_plot}\NormalTok{(test\_data\_RT, }\StringTok{"CREAT"}\NormalTok{, }\AttributeTok{is\_categorical =} \ConstantTok{FALSE}\NormalTok{)}
\NormalTok{weight\_WT\_RT }\OtherTok{\textless{}{-}} \FunctionTok{generate\_weight\_plot}\NormalTok{(test\_data\_RT, }\StringTok{"WT"}\NormalTok{, }\AttributeTok{is\_categorical =} \ConstantTok{FALSE}\NormalTok{)}
\NormalTok{weight\_AGE\_RT }\OtherTok{\textless{}{-}} \FunctionTok{generate\_weight\_plot}\NormalTok{(test\_data\_RT, }\StringTok{"AGE"}\NormalTok{, }\AttributeTok{is\_categorical =} \ConstantTok{FALSE}\NormalTok{)}
\NormalTok{weight\_SEX\_RT }\OtherTok{\textless{}{-}} \FunctionTok{generate\_weight\_plot}\NormalTok{(test\_data\_RT, }\StringTok{"SEX"}\NormalTok{, }\AttributeTok{is\_categorical =} \ConstantTok{TRUE}\NormalTok{)}
\NormalTok{weight\_CON\_RT }\OtherTok{\textless{}{-}} \FunctionTok{generate\_weight\_plot}\NormalTok{(test\_data\_RT, }\StringTok{"CON"}\NormalTok{, }\AttributeTok{is\_categorical =} \ConstantTok{TRUE}\NormalTok{)}
\NormalTok{weight\_BURN\_RT}
\end{Highlighting}
\end{Shaded}

\pandocbounded{\includegraphics[keepaspectratio]{MIPD/Precision_dosing_methods_files/figure-pdf/evaluate-prediction-RT-3.pdf}}

\begin{Shaded}
\begin{Highlighting}[]
\NormalTok{weight\_ICU\_RT}
\end{Highlighting}
\end{Shaded}

\pandocbounded{\includegraphics[keepaspectratio]{MIPD/Precision_dosing_methods_files/figure-pdf/evaluate-prediction-RT-4.pdf}}

\begin{Shaded}
\begin{Highlighting}[]
\NormalTok{weight\_OBESE\_RT }\OtherTok{\textless{}{-}}\NormalTok{ weight\_OBESE\_RT }\SpecialCharTok{+} 
  \FunctionTok{labs}\NormalTok{(}\AttributeTok{title =} \StringTok{"RT inf ens weight attribution, stratified by OBESE"}\NormalTok{)}
\FunctionTok{ggsave}\NormalTok{(}\AttributeTok{filename =} \FunctionTok{here}\NormalTok{(}\StringTok{"Amoxicillin/a\_priori/For\_publication/Figures/S6b.jpg"}\NormalTok{),}
       \AttributeTok{plot =}\NormalTok{ weight\_OBESE\_RT,}
       \AttributeTok{width =} \DecValTok{8}\NormalTok{, }\AttributeTok{height =} \DecValTok{6}\NormalTok{, }\AttributeTok{dpi =} \DecValTok{300}\NormalTok{)}
\NormalTok{weight\_CREAT\_RT}
\end{Highlighting}
\end{Shaded}

\pandocbounded{\includegraphics[keepaspectratio]{MIPD/Precision_dosing_methods_files/figure-pdf/evaluate-prediction-RT-5.pdf}}

\begin{Shaded}
\begin{Highlighting}[]
\NormalTok{weight\_WT\_RT}
\end{Highlighting}
\end{Shaded}

\pandocbounded{\includegraphics[keepaspectratio]{MIPD/Precision_dosing_methods_files/figure-pdf/evaluate-prediction-RT-6.pdf}}

\begin{Shaded}
\begin{Highlighting}[]
\NormalTok{weight\_AGE\_RT}
\end{Highlighting}
\end{Shaded}

\pandocbounded{\includegraphics[keepaspectratio]{MIPD/Precision_dosing_methods_files/figure-pdf/evaluate-prediction-RT-7.pdf}}

\begin{Shaded}
\begin{Highlighting}[]
\NormalTok{weight\_SEX\_RT}
\end{Highlighting}
\end{Shaded}

\pandocbounded{\includegraphics[keepaspectratio]{MIPD/Precision_dosing_methods_files/figure-pdf/evaluate-prediction-RT-8.pdf}}

\begin{Shaded}
\begin{Highlighting}[]
\NormalTok{weight\_CON\_RT}
\end{Highlighting}
\end{Shaded}

\pandocbounded{\includegraphics[keepaspectratio]{MIPD/Precision_dosing_methods_files/figure-pdf/evaluate-prediction-RT-9.pdf}}

\begin{Shaded}
\begin{Highlighting}[]
\NormalTok{test\_data\_RT }\SpecialCharTok{\%\textgreater{}\%}
  \FunctionTok{group\_by}\NormalTok{(MODEL) }\SpecialCharTok{\%\textgreater{}\%}
  \FunctionTok{summarise}\NormalTok{(}\AttributeTok{avg\_weight =} \FunctionTok{mean}\NormalTok{(WEIGHT, }\AttributeTok{na.rm =} \ConstantTok{TRUE}\NormalTok{))}
\end{Highlighting}
\end{Shaded}

\begin{verbatim}
# A tibble: 4 x 2
  MODEL    avg_weight
  <chr>         <dbl>
1 CARLIER       0.132
2 FOURNIER      0.322
3 MELLON        0.302
4 RAMBAUD       0.243
\end{verbatim}

\begin{Shaded}
\begin{Highlighting}[]
\CommentTok{\# Extrapolate dose based on the ensembled concentrations and compare it to the true simulated dose.}
\NormalTok{test\_data\_RT }\OtherTok{\textless{}{-}}\NormalTok{ test\_data\_RT }\SpecialCharTok{\%\textgreater{}\%}
\NormalTok{  dplyr}\SpecialCharTok{::}\FunctionTok{filter}\NormalTok{(REFERENCE }\SpecialCharTok{==} \DecValTok{1}\NormalTok{) }\SpecialCharTok{\%\textgreater{}\%}
\NormalTok{  dplyr}\SpecialCharTok{::}\FunctionTok{mutate}\NormalTok{(}\AttributeTok{DOSE\_PRED =}\NormalTok{ (}\DecValTok{60}\SpecialCharTok{/}\NormalTok{WEIGHTED\_PREDICTION) }\SpecialCharTok{*}\NormalTok{ DOSE\_ADM) }\CommentTok{\# Administered dose extrapolation to reach 60 mg/L}

\NormalTok{final\_results\_RT }\OtherTok{\textless{}{-}} \FunctionTok{explore\_predictions}\NormalTok{(test\_data\_RT)}
\NormalTok{final\_results\_RT}\SpecialCharTok{$}\NormalTok{target\_attainment }
\end{Highlighting}
\end{Shaded}

\pandocbounded{\includegraphics[keepaspectratio]{MIPD/Precision_dosing_methods_files/figure-pdf/evaluate-prediction-RT-10.pdf}}

\begin{Shaded}
\begin{Highlighting}[]
\NormalTok{crcl\_plot\_RT }\OtherTok{\textless{}{-}}\NormalTok{ final\_results\_RT}\SpecialCharTok{$}\NormalTok{crcl\_plot}
\NormalTok{crcl\_plot\_RT}
\end{Highlighting}
\end{Shaded}

\pandocbounded{\includegraphics[keepaspectratio]{MIPD/Precision_dosing_methods_files/figure-pdf/evaluate-prediction-RT-11.pdf}}

\begin{Shaded}
\begin{Highlighting}[]
\NormalTok{final\_results\_RT}\SpecialCharTok{$}\NormalTok{summary\_stats }
\end{Highlighting}
\end{Shaded}

\begin{verbatim}
# A tibble: 2 x 9
  Prediction_correctness mean_CREAT sd_CREAT mean_WT sd_WT mean_AGE sd_AGE Count
  <chr>                       <dbl>    <dbl>   <dbl> <dbl>    <dbl>  <dbl> <int>
1 Correct                     1.02     0.603    84.4  18.9     63.9   16.2   251
2 Incorrect                   0.897    0.555    84.4  18.6     57.0   15.9   349
# i 1 more variable: Proportion <dbl>
\end{verbatim}

\begin{Shaded}
\begin{Highlighting}[]
\CommentTok{\# Keep data for method ensembling}
\NormalTok{test\_data\_RT\_ens }\OtherTok{\textless{}{-}}\NormalTok{ test\_data\_RT }\SpecialCharTok{\%\textgreater{}\%}
\NormalTok{  dplyr}\SpecialCharTok{::}\FunctionTok{mutate}\NormalTok{(}\AttributeTok{RT =}\NormalTok{ DOSE\_PRED) }\SpecialCharTok{\%\textgreater{}\%} 
\NormalTok{  dplyr}\SpecialCharTok{::}\FunctionTok{select}\NormalTok{(}\StringTok{"ID"}\NormalTok{,}\StringTok{"RT"}\NormalTok{)}
\end{Highlighting}
\end{Shaded}

\chapter{Weighed model ensembling}\label{weighed-model-ensembling}

This method is used to attribute weights to models based on their
performance in different subgroups and on the overall performance of
models in different subgroups.

The method is based on Agema et al. (2024).

\subsection{Model influence}\label{model-influence}

To calculate model influence, first, differences are calculated between
\(Cmin_{\text{pred}}\) and \(Cmin_{\text{ind}}\) for a particular set of
covariates. Then we use this difference to calculate pooled MPE and RMSE
values for each model. MPE and rRMSE will be used to calculate scores as
discussed in a later section.

\pandocbounded{\includegraphics[keepaspectratio]{MIPD/images/clipboard-1093848730.png}}

\subsection{Subgroup influence}\label{subgroup-influence}

The data is divided into subgroups. For categorical covariates (ICU,
BURN, OBESE, SEX, CON (indicating continuous infusion)), a subgroup is a
category, thus we have two-two-two-two-two subgroups for ICU, BURN, SEX,
OBESE, and CON. Continuous covariates (WT, CREAT, AGE) are divided into
four quantiles, and each quantile will represent a subgroup.\\
First, ratios are calculated between\(Cmin_{\text{pred}}\) and the
\(Cmin_{\text{ind}}\) for a particular set of covariates. For each
sugbroup we will calculate the proportion of ratios \textbf{outside} the
bioequivalence range of {[}0.80-1.25{]}. This proportion will represent
subgroup influence. The logic behind giving more influence to subgroups
with more ratios outside the bioequivalence range is that sugbroup
having overall good predictions (many ratios inside the bioequivalence
range), means that we can use any of the models for that subgroup, thus
its influence will be lower in attributing model weights. A model having
a good performance in a sugbroup with bad predictions (many ratios
outside the bioequivalence range) will be preferred to a model having
good performance in a subgroup having overall good predictions.

\pandocbounded{\includegraphics[keepaspectratio]{MIPD/images/clipboard-2578669345.png}}

\subsection{Score calculation}\label{score-calculation}

Scores are calculated with a negative exponential function: \[
\text{Score}_{\text{MPE}} = e^{\text{penalty} \cdot |\text{MPE}|}
\] \[
\text{Score}_{\text{RMSE}} = e^{\text{penalty} \cdot |\text{RMSE}|}
\]Model score is the product of RMSE and MPE scores. \[
\text{Score}_{\text{Carlier}} = \text{Score}_{\text{RMSE}} \cdot \text{Score}_{\text{MPE}}
\] Model score is normalized by the sum of scores. \[
\text{Influence}_{\text{Carlier}} = \frac{\text{Score}_{\text{Carlier}}}{\sum \text{Score}}
\] For this negative exponential calculation, the two penalties have to
be defined

\pandocbounded{\includegraphics[keepaspectratio]{MIPD/images/clipboard-106276185.png}}

To test the algorithm we check to which 8 subgroups (ICU, BURN, WT,
CREAT, OBESE, SEX, AGE, CON (= continuous or intermittant infusion)) a
subject belongs, and we add the obtain the corresponding model and
influence scores from the all\_scores table. We normalize the four
subgroup influences by their sum (so they add up to 1 for that subject).
Then, for each subgroup we multiply the model influences with the
normalized subgroup influence. As a product of this multiplication we
have a model weights for each of the four models for each of the four
subgroups. For each model, we add the four weights (for example Carlier
weight for ICU = 0, Carlier weight for BURN = 1 etc). As a final step,
we normalize the weights of the models by their sum so they add up to 1
for each subject. Then, we ensemble the separate Cmax predictions of the
four models based on their weight. The administered dose is extrapolated
by dividing the weighted concentration prediction by the target
concentration (= 60 mg/L).

The penalties are automatically tuned based on the percentage of correct
dose predictions.

Our \emph{openMIPD} package is used to apply this method.

\begin{Shaded}
\begin{Highlighting}[]
\NormalTok{AMOX\_CMIN\_TRAIN }\OtherTok{\textless{}{-}}\NormalTok{ AMOX\_CMIN\_TRAIN }\SpecialCharTok{\%\textgreater{}\%}
  \FunctionTok{mutate}\NormalTok{(}\AttributeTok{dosing\_reg\_strata =} \FunctionTok{paste0}\NormalTok{(DOSE\_ADM,FREQ,DUR))}

\NormalTok{wme\_tuning }\OtherTok{\textless{}{-}} \FunctionTok{weighed\_model\_ensembling\_tuning}\NormalTok{(}
    \AttributeTok{train\_data =}\NormalTok{ AMOX\_CMIN\_TRAIN,}
    \AttributeTok{target\_variable =} \StringTok{"CMIN"}\NormalTok{,}
    \AttributeTok{continuous\_cov =} \FunctionTok{c}\NormalTok{(}\StringTok{"WT"}\NormalTok{, }\StringTok{"CREAT"}\NormalTok{, }\StringTok{"AGE"}\NormalTok{),}
    \AttributeTok{categorical\_cov =} \FunctionTok{c}\NormalTok{(}\StringTok{"OBESE"}\NormalTok{, }\StringTok{"ICU"}\NormalTok{, }\StringTok{"BURN"}\NormalTok{, }\StringTok{"SEX"}\NormalTok{,}\StringTok{"CON"}\NormalTok{),}
    \AttributeTok{penalties\_grid =} \ConstantTok{NULL}\NormalTok{,}
    \AttributeTok{conc\_inf =} \DecValTok{40}\NormalTok{, }\AttributeTok{conc\_sup =} \DecValTok{80}\NormalTok{, }\AttributeTok{conc\_target =} \DecValTok{60}\NormalTok{,}
    \AttributeTok{freq\_column =} \StringTok{"FREQ"}\NormalTok{,}
    \AttributeTok{cv\_stratification\_col =} \StringTok{"dosing\_reg\_strata"}\NormalTok{,}
    \AttributeTok{target\_log\_dose =} \ConstantTok{TRUE}\NormalTok{)}

\NormalTok{wme\_tuning}\SpecialCharTok{$}\NormalTok{best\_pen\_RMSE}
\end{Highlighting}
\end{Shaded}

\begin{verbatim}
[1] -8
\end{verbatim}

\begin{Shaded}
\begin{Highlighting}[]
\NormalTok{wme\_tuning}\SpecialCharTok{$}\NormalTok{best\_pen\_MPE}
\end{Highlighting}
\end{Shaded}

\begin{verbatim}
[1] -1
\end{verbatim}

To visualize method training, the mean percentage error (MPE) and
relative root mean squared error (RMSE) for each model in each covariate
quantile/category is plotted as well as subgroup influences (the
proportion of incorrect predictions for each covariate
quantile/category, all models included).

\begin{Shaded}
\begin{Highlighting}[]
\NormalTok{training\_results\_WME }\OtherTok{\textless{}{-}} \FunctionTok{weighed\_model\_ensembling\_train}\NormalTok{(}
  \AttributeTok{data =}\NormalTok{ AMOX\_CMIN\_TRAIN, }
  \AttributeTok{target\_variable =} \StringTok{"CMIN"}\NormalTok{, }
  \AttributeTok{pen\_RMSE =}\NormalTok{ wme\_tuning}\SpecialCharTok{$}\NormalTok{best\_pen\_RMSE,}
  \AttributeTok{pen\_MPE =}\NormalTok{ wme\_tuning}\SpecialCharTok{$}\NormalTok{best\_pen\_MPE,}
  \AttributeTok{continuous\_cov =} \FunctionTok{c}\NormalTok{(}\StringTok{"WT"}\NormalTok{, }\StringTok{"CREAT"}\NormalTok{, }\StringTok{"AGE"}\NormalTok{), }
  \AttributeTok{categorical\_cov =} \FunctionTok{c}\NormalTok{(}\StringTok{"OBESE"}\NormalTok{, }\StringTok{"ICU"}\NormalTok{, }\StringTok{"BURN"}\NormalTok{, }\StringTok{"SEX"}\NormalTok{, }\StringTok{"CON"}\NormalTok{))}
\end{Highlighting}
\end{Shaded}

\begin{Shaded}
\begin{Highlighting}[]
\NormalTok{all\_scores }\OtherTok{\textless{}{-}}\NormalTok{ training\_results\_WME}\SpecialCharTok{$}\NormalTok{all\_scores}
\FunctionTok{print}\NormalTok{(all\_scores)}
\end{Highlighting}
\end{Shaded}

\begin{verbatim}
# A tibble: 88 x 10
   COVARIATE QUANTILE SUBGROUP_INFLUENCE MODEL   MPE  RMSE MODEL_INFLUENCE LABEL
   <chr>     <chr>                 <dbl> <chr> <dbl> <dbl>           <dbl> <chr>
 1 AGE       18.3-49~              0.890 CARL~ 7.95  1.88         1.03e-10 AGE_7
 2 AGE       18.3-49~              0.890 FOUR~ 3.85  1.52         1.10e- 7 AGE_7
 3 AGE       18.3-49~              0.890 MELL~ 1.92  1.79         8.80e- 8 AGE_7
 4 AGE       18.3-49~              0.890 RAMB~ 0.942 1.76         3.01e- 7 AGE_7
 5 AGE       49.4-57~              0.875 CARL~ 7.02  1.44         8.75e- 9 AGE_9
 6 AGE       49.4-57~              0.875 FOUR~ 3.86  1.04         5.15e- 6 AGE_9
 7 AGE       49.4-57~              0.875 MELL~ 1.30  1.34         6.27e- 6 AGE_9
 8 AGE       49.4-57~              0.875 RAMB~ 0.918 1.04         9.57e- 5 AGE_9
 9 AGE       57.3-69~              0.812 CARL~ 2.60  0.944        3.92e- 5 AGE_~
10 AGE       57.3-69~              0.812 FOUR~ 1.64  0.852        2.12e- 4 AGE_~
# i 78 more rows
# i 2 more variables: lower <dbl>, upper <dbl>
\end{verbatim}

\begin{Shaded}
\begin{Highlighting}[]
\NormalTok{MPE\_plot }\OtherTok{\textless{}{-}}\NormalTok{ training\_results\_WME}\SpecialCharTok{$}\NormalTok{MPE\_plot }
\NormalTok{MPE\_plot}
\end{Highlighting}
\end{Shaded}

\pandocbounded{\includegraphics[keepaspectratio]{MIPD/Precision_dosing_methods_files/figure-pdf/evaluate-WME-training-1.pdf}}

\begin{Shaded}
\begin{Highlighting}[]
\FunctionTok{ggsave}\NormalTok{(}\AttributeTok{filename =} \FunctionTok{here}\NormalTok{(}\StringTok{"Amoxicillin/a\_priori/For\_publication/Figures/S9.jpg"}\NormalTok{),}
       \AttributeTok{plot =}\NormalTok{ MPE\_plot,}
       \AttributeTok{width =} \DecValTok{8}\NormalTok{, }\AttributeTok{height =} \DecValTok{6}\NormalTok{, }\AttributeTok{dpi =} \DecValTok{300}\NormalTok{)}
\NormalTok{RMSE\_plot }\OtherTok{\textless{}{-}}\NormalTok{ training\_results\_WME}\SpecialCharTok{$}\NormalTok{RMSE\_plot}
\NormalTok{RMSE\_plot}
\end{Highlighting}
\end{Shaded}

\pandocbounded{\includegraphics[keepaspectratio]{MIPD/Precision_dosing_methods_files/figure-pdf/evaluate-WME-training-2.pdf}}

\begin{Shaded}
\begin{Highlighting}[]
\FunctionTok{ggsave}\NormalTok{(}\AttributeTok{filename =} \FunctionTok{here}\NormalTok{(}\StringTok{"Amoxicillin/a\_priori/For\_publication/Figures/S10.jpg"}\NormalTok{),}
       \AttributeTok{plot =}\NormalTok{ RMSE\_plot,}
       \AttributeTok{width =} \DecValTok{8}\NormalTok{, }\AttributeTok{height =} \DecValTok{6}\NormalTok{, }\AttributeTok{dpi =} \DecValTok{300}\NormalTok{)}
\NormalTok{SUBGROUP\_INFLUENCE\_plot }\OtherTok{\textless{}{-}}\NormalTok{ training\_results\_WME}\SpecialCharTok{$}\NormalTok{SUBGROUP\_INFLUENCE\_plot}
\NormalTok{SUBGROUP\_INFLUENCE\_plot}
\end{Highlighting}
\end{Shaded}

\pandocbounded{\includegraphics[keepaspectratio]{MIPD/Precision_dosing_methods_files/figure-pdf/evaluate-WME-training-3.pdf}}

\begin{Shaded}
\begin{Highlighting}[]
\FunctionTok{ggsave}\NormalTok{(}\AttributeTok{filename =} \FunctionTok{here}\NormalTok{(}\StringTok{"Amoxicillin/a\_priori/For\_publication/Figures/S11.jpg"}\NormalTok{),}
       \AttributeTok{plot =}\NormalTok{ SUBGROUP\_INFLUENCE\_plot,}
       \AttributeTok{width =} \DecValTok{8}\NormalTok{, }\AttributeTok{height =} \DecValTok{6}\NormalTok{, }\AttributeTok{dpi =} \DecValTok{300}\NormalTok{)}
\end{Highlighting}
\end{Shaded}

\begin{Shaded}
\begin{Highlighting}[]
\NormalTok{test\_results\_WME }\OtherTok{\textless{}{-}}  \FunctionTok{weighed\_model\_ensembling\_test}\NormalTok{(}
  \AttributeTok{train\_results =}\NormalTok{ training\_results\_WME, }
  \AttributeTok{test\_data =}\NormalTok{ AMOX\_CMIN\_TEST)}
\end{Highlighting}
\end{Shaded}

To evaluate the method, a predictions as a function of observed
concentrations goodness of fit plot is presented as well a boxplot
indicating the predicted/observed ratios in \%. Then, the average model
weights are plotted, stratified by covariate categories/quantiles.

\begin{Shaded}
\begin{Highlighting}[]
\NormalTok{test\_data\_WME }\OtherTok{\textless{}{-}}\NormalTok{ test\_results\_WME}\SpecialCharTok{$}\NormalTok{test\_results}
\NormalTok{test\_results\_WME}\SpecialCharTok{$}\NormalTok{GOF\_plot}
\end{Highlighting}
\end{Shaded}

\pandocbounded{\includegraphics[keepaspectratio]{MIPD/Precision_dosing_methods_files/figure-pdf/evaluate-WME-prediction-1.pdf}}

\begin{Shaded}
\begin{Highlighting}[]
\NormalTok{test\_results\_WME}\SpecialCharTok{$}\NormalTok{Boxplot}
\end{Highlighting}
\end{Shaded}

\pandocbounded{\includegraphics[keepaspectratio]{MIPD/Precision_dosing_methods_files/figure-pdf/evaluate-WME-prediction-2.pdf}}

\begin{Shaded}
\begin{Highlighting}[]
\CommentTok{\# Plots showing the average attributed weight stratified by covariates}
\NormalTok{weight\_BURN\_WME }\OtherTok{\textless{}{-}} \FunctionTok{generate\_weight\_plot}\NormalTok{(test\_data\_WME, }\StringTok{"BURN"}\NormalTok{, }\AttributeTok{is\_categorical =} \ConstantTok{TRUE}\NormalTok{)}
\NormalTok{weight\_ICU\_WME }\OtherTok{\textless{}{-}} \FunctionTok{generate\_weight\_plot}\NormalTok{(test\_data\_WME, }\StringTok{"ICU"}\NormalTok{, }\AttributeTok{is\_categorical =} \ConstantTok{TRUE}\NormalTok{)}
\NormalTok{weight\_OBESE\_WME }\OtherTok{\textless{}{-}} \FunctionTok{generate\_weight\_plot}\NormalTok{(test\_data\_WME, }\StringTok{"OBESE"}\NormalTok{, }\AttributeTok{is\_categorical =} \ConstantTok{TRUE}\NormalTok{)}
\NormalTok{weight\_CREAT\_WME }\OtherTok{\textless{}{-}} \FunctionTok{generate\_weight\_plot}\NormalTok{(test\_data\_WME, }\StringTok{"CREAT"}\NormalTok{, }\AttributeTok{is\_categorical =} \ConstantTok{FALSE}\NormalTok{)}
\NormalTok{weight\_WT\_WME }\OtherTok{\textless{}{-}} \FunctionTok{generate\_weight\_plot}\NormalTok{(test\_data\_WME, }\StringTok{"WT"}\NormalTok{, }\AttributeTok{is\_categorical =} \ConstantTok{FALSE}\NormalTok{)}
\NormalTok{weight\_AGE\_WME }\OtherTok{\textless{}{-}} \FunctionTok{generate\_weight\_plot}\NormalTok{(test\_data\_WME, }\StringTok{"AGE"}\NormalTok{, }\AttributeTok{is\_categorical =} \ConstantTok{FALSE}\NormalTok{)}
\NormalTok{weight\_SEX\_WME }\OtherTok{\textless{}{-}} \FunctionTok{generate\_weight\_plot}\NormalTok{(test\_data\_WME, }\StringTok{"SEX"}\NormalTok{, }\AttributeTok{is\_categorical =} \ConstantTok{TRUE}\NormalTok{)}
\NormalTok{weight\_CON\_WME }\OtherTok{\textless{}{-}} \FunctionTok{generate\_weight\_plot}\NormalTok{(test\_data\_WME, }\StringTok{"CON"}\NormalTok{, }\AttributeTok{is\_categorical =} \ConstantTok{TRUE}\NormalTok{)}
\NormalTok{weight\_BURN\_WME}
\end{Highlighting}
\end{Shaded}

\pandocbounded{\includegraphics[keepaspectratio]{MIPD/Precision_dosing_methods_files/figure-pdf/evaluate-WME-prediction-3.pdf}}

\begin{Shaded}
\begin{Highlighting}[]
\NormalTok{weight\_ICU\_WME}
\end{Highlighting}
\end{Shaded}

\pandocbounded{\includegraphics[keepaspectratio]{MIPD/Precision_dosing_methods_files/figure-pdf/evaluate-WME-prediction-4.pdf}}

\begin{Shaded}
\begin{Highlighting}[]
\NormalTok{weight\_OBESE\_WME}
\end{Highlighting}
\end{Shaded}

\pandocbounded{\includegraphics[keepaspectratio]{MIPD/Precision_dosing_methods_files/figure-pdf/evaluate-WME-prediction-5.pdf}}

\begin{Shaded}
\begin{Highlighting}[]
\NormalTok{weight\_CREAT\_WME}
\end{Highlighting}
\end{Shaded}

\pandocbounded{\includegraphics[keepaspectratio]{MIPD/Precision_dosing_methods_files/figure-pdf/evaluate-WME-prediction-6.pdf}}

\begin{Shaded}
\begin{Highlighting}[]
\NormalTok{weight\_WT\_WME}
\end{Highlighting}
\end{Shaded}

\pandocbounded{\includegraphics[keepaspectratio]{MIPD/Precision_dosing_methods_files/figure-pdf/evaluate-WME-prediction-7.pdf}}

\begin{Shaded}
\begin{Highlighting}[]
\NormalTok{weight\_AGE\_WME}
\end{Highlighting}
\end{Shaded}

\pandocbounded{\includegraphics[keepaspectratio]{MIPD/Precision_dosing_methods_files/figure-pdf/evaluate-WME-prediction-8.pdf}}

\begin{Shaded}
\begin{Highlighting}[]
\NormalTok{weight\_SEX\_WME}
\end{Highlighting}
\end{Shaded}

\pandocbounded{\includegraphics[keepaspectratio]{MIPD/Precision_dosing_methods_files/figure-pdf/evaluate-WME-prediction-9.pdf}}

\begin{Shaded}
\begin{Highlighting}[]
\NormalTok{weight\_CON\_WME}
\end{Highlighting}
\end{Shaded}

\pandocbounded{\includegraphics[keepaspectratio]{MIPD/Precision_dosing_methods_files/figure-pdf/evaluate-WME-prediction-10.pdf}}

\begin{Shaded}
\begin{Highlighting}[]
\NormalTok{weight\_OBESE\_WME}
\end{Highlighting}
\end{Shaded}

\pandocbounded{\includegraphics[keepaspectratio]{MIPD/Precision_dosing_methods_files/figure-pdf/evaluate-WME-prediction-11.pdf}}

\begin{Shaded}
\begin{Highlighting}[]
\NormalTok{test\_data\_WME }\SpecialCharTok{\%\textgreater{}\%}
  \FunctionTok{group\_by}\NormalTok{(MODEL) }\SpecialCharTok{\%\textgreater{}\%}
  \FunctionTok{summarise}\NormalTok{(}\AttributeTok{avg\_weight =} \FunctionTok{mean}\NormalTok{(WEIGHT, }\AttributeTok{na.rm =} \ConstantTok{TRUE}\NormalTok{))}
\end{Highlighting}
\end{Shaded}

\begin{verbatim}
# A tibble: 4 x 2
  MODEL    avg_weight
  <chr>         <dbl>
1 CARLIER      0.0280
2 FOURNIER     0.154 
3 MELLON       0.135 
4 RAMBAUD      0.683 
\end{verbatim}

\begin{Shaded}
\begin{Highlighting}[]
\CommentTok{\# Extrapolate dose based on the ensembled concentrations and compare it to the true simulated dose.}
\NormalTok{test\_data\_WME }\OtherTok{\textless{}{-}}\NormalTok{ test\_data\_WME }\SpecialCharTok{\%\textgreater{}\%}
\NormalTok{  dplyr}\SpecialCharTok{::}\FunctionTok{filter}\NormalTok{(REFERENCE }\SpecialCharTok{==} \DecValTok{1}\NormalTok{) }\SpecialCharTok{\%\textgreater{}\%}
\NormalTok{  dplyr}\SpecialCharTok{::}\FunctionTok{mutate}\NormalTok{(}\AttributeTok{DOSE\_PRED =}\NormalTok{ (}\DecValTok{60}\SpecialCharTok{/}\NormalTok{WEIGHTED\_PREDICTION) }\SpecialCharTok{*}\NormalTok{ DOSE\_ADM) }\CommentTok{\# Administered dose extrapolation to reach 60 mg/L}

\NormalTok{final\_results\_WME }\OtherTok{\textless{}{-}} \FunctionTok{explore\_predictions}\NormalTok{(test\_data\_WME)}
\NormalTok{final\_results\_WME}\SpecialCharTok{$}\NormalTok{target\_attainment }
\end{Highlighting}
\end{Shaded}

\pandocbounded{\includegraphics[keepaspectratio]{MIPD/Precision_dosing_methods_files/figure-pdf/evaluate-WME-prediction-12.pdf}}

\begin{Shaded}
\begin{Highlighting}[]
\NormalTok{crcl\_plot\_WME }\OtherTok{\textless{}{-}}\NormalTok{ final\_results\_WME}\SpecialCharTok{$}\NormalTok{crcl\_plot}
\NormalTok{crcl\_plot\_WME}
\end{Highlighting}
\end{Shaded}

\pandocbounded{\includegraphics[keepaspectratio]{MIPD/Precision_dosing_methods_files/figure-pdf/evaluate-WME-prediction-13.pdf}}

\begin{Shaded}
\begin{Highlighting}[]
\NormalTok{final\_results\_WME}\SpecialCharTok{$}\NormalTok{summary\_stats }
\end{Highlighting}
\end{Shaded}

\begin{verbatim}
# A tibble: 2 x 9
  Prediction_correctness mean_CREAT sd_CREAT mean_WT sd_WT mean_AGE sd_AGE Count
  <chr>                       <dbl>    <dbl>   <dbl> <dbl>    <dbl>  <dbl> <int>
1 Correct                     0.958    0.488    82.7  18.2     66.6   16.1   178
2 Incorrect                   0.945    0.613    85.4  18.7     57.1   15.6   419
# i 1 more variable: Proportion <dbl>
\end{verbatim}

\begin{Shaded}
\begin{Highlighting}[]
\CommentTok{\# Keep data for method ensembling}
\NormalTok{test\_data\_WME\_ens }\OtherTok{\textless{}{-}}\NormalTok{ test\_data\_WME }\SpecialCharTok{\%\textgreater{}\%}
\NormalTok{  dplyr}\SpecialCharTok{::}\FunctionTok{mutate}\NormalTok{(}\AttributeTok{WME =}\NormalTok{ DOSE\_PRED) }\SpecialCharTok{\%\textgreater{}\%} 
\NormalTok{  dplyr}\SpecialCharTok{::}\FunctionTok{select}\NormalTok{(}\StringTok{"ID"}\NormalTok{,}\StringTok{"WME"}\NormalTok{)}
\end{Highlighting}
\end{Shaded}

\chapter{Factor Analysis of Mixed Data
(FAMD)}\label{factor-analysis-of-mixed-data-famd}

In FAMD, Principal Component Analyisis (PCA) is applied to continuous
covariates and multiple correspondence analysis (MCA) to categorical
covariates. FAMD is an unsupervised machine learning algorithm which is
based on dimensionality reduction. Principal components are uncorrelated
linear combinations of initial variables. They are directions in the
variable space, perpendicular to each other that explain a maximum
variance of the data. The first principal component explains most of the
variance, as it has most of the dispersion of the data points along it,
then the subsequent ones explain less and less and if a certain
percentage (in this case, 90 \%) of the data variance is explained, no
more principal components are added. This process simplifies the dataset
while retaining most of the relevant information.

Nearest neighbors is a classification technique that relies on the
proximity of a data point to different classes in a variable space. To
measure proximity, we can use euclidean distance, which is the length of
a straight line connecting to points, or the Mahalanobis distance which
accounts for variance and correlation between the variables as well.
Mahalanobis distances are calculated between the centroids (geometric
means) of model cohorts and a new subject. Model weights were calculated
by taking the normalized reciprocal of these distances.

Our \emph{openMIPD} package is used to apply this method.

\begin{Shaded}
\begin{Highlighting}[]
\NormalTok{train\_result\_FAMD }\OtherTok{\textless{}{-}} \FunctionTok{famd\_train}\NormalTok{(}
  \AttributeTok{train =}\NormalTok{ AMOX\_CMIN\_TRAIN, }
  \AttributeTok{continuous\_cov =} \FunctionTok{c}\NormalTok{(}\StringTok{"WT"}\NormalTok{, }\StringTok{"CREAT"}\NormalTok{, }\StringTok{"AGE"}\NormalTok{), }
  \AttributeTok{categorical\_cov =} \FunctionTok{c}\NormalTok{(}\StringTok{"OBESE"}\NormalTok{, }\StringTok{"ICU"}\NormalTok{, }\StringTok{"BURN"}\NormalTok{, }\StringTok{"SEX"}\NormalTok{), }
  \AttributeTok{target\_variable =} \StringTok{"CMIN"}\NormalTok{)}
\end{Highlighting}
\end{Shaded}

To visualize the model, the contribution of covariates to the first
(left) and second (right) principal component ar visualized for FAMD.
Notably, BURN contributed the least to the first dimension, and the most
to the second. The contribution of continuous covariates to the first
and second principal components in PCA is also plotted. WT contributes
the most to the results, and CREAT the least (out of the continuous
covariates). CREAT and age are positively correlated as the angle
between them is smaller, while between CREAT and WT there is an
important negative correlation.

\begin{Shaded}
\begin{Highlighting}[]
\NormalTok{train\_result\_FAMD}\SpecialCharTok{$}\NormalTok{principal\_components}
\end{Highlighting}
\end{Shaded}

\begin{verbatim}
       eigenvalue percentage of variance cumulative percentage of variance
comp 1  3.2226916              46.038451                          46.03845
comp 2  1.2679382              18.113402                          64.15185
comp 3  0.9498823              13.569747                          77.72160
comp 4  0.6445674               9.208106                          86.92971
comp 5  0.5435385               7.764835                          94.69454
\end{verbatim}

\begin{Shaded}
\begin{Highlighting}[]
\NormalTok{contrib\_dim1 }\OtherTok{\textless{}{-}}\NormalTok{ train\_result\_FAMD}\SpecialCharTok{$}\NormalTok{contrib\_dim1}
\NormalTok{contrib\_dim1 }
\end{Highlighting}
\end{Shaded}

\pandocbounded{\includegraphics[keepaspectratio]{MIPD/Precision_dosing_methods_files/figure-pdf/evaluate-FAMD-training-1.pdf}}

\begin{Shaded}
\begin{Highlighting}[]
\FunctionTok{ggsave}\NormalTok{(}\AttributeTok{filename =} \FunctionTok{here}\NormalTok{(}\StringTok{"Amoxicillin/a\_priori/For\_publication/Figures/S7a.jpg"}\NormalTok{),}
       \AttributeTok{plot =}\NormalTok{ contrib\_dim1,}
       \AttributeTok{width =} \DecValTok{8}\NormalTok{, }\AttributeTok{height =} \DecValTok{6}\NormalTok{, }\AttributeTok{dpi =} \DecValTok{300}\NormalTok{)}
\NormalTok{contrib\_dim2 }\OtherTok{\textless{}{-}}\NormalTok{ train\_result\_FAMD}\SpecialCharTok{$}\NormalTok{contrib\_dim2}
\NormalTok{contrib\_dim2 }
\end{Highlighting}
\end{Shaded}

\pandocbounded{\includegraphics[keepaspectratio]{MIPD/Precision_dosing_methods_files/figure-pdf/evaluate-FAMD-training-2.pdf}}

\begin{Shaded}
\begin{Highlighting}[]
\FunctionTok{ggsave}\NormalTok{(}\AttributeTok{filename =} \FunctionTok{here}\NormalTok{(}\StringTok{"Amoxicillin/a\_priori/For\_publication/Figures/S7b.jpg"}\NormalTok{),}
       \AttributeTok{plot =}\NormalTok{ contrib\_dim2,}
       \AttributeTok{width =} \DecValTok{8}\NormalTok{, }\AttributeTok{height =} \DecValTok{6}\NormalTok{, }\AttributeTok{dpi =} \DecValTok{300}\NormalTok{)}
\NormalTok{contrib\_quant\_var }\OtherTok{\textless{}{-}}\NormalTok{ train\_result\_FAMD}\SpecialCharTok{$}\NormalTok{contrib\_quant\_var}
\NormalTok{contrib\_quant\_var}
\end{Highlighting}
\end{Shaded}

\pandocbounded{\includegraphics[keepaspectratio]{MIPD/Precision_dosing_methods_files/figure-pdf/evaluate-FAMD-training-3.pdf}}

\begin{Shaded}
\begin{Highlighting}[]
\FunctionTok{ggsave}\NormalTok{(}\AttributeTok{filename =} \FunctionTok{here}\NormalTok{(}\StringTok{"Amoxicillin/a\_priori/For\_publication/Figures/S8.jpg"}\NormalTok{),}
       \AttributeTok{plot =}\NormalTok{ contrib\_quant\_var,}
       \AttributeTok{width =} \DecValTok{8}\NormalTok{, }\AttributeTok{height =} \DecValTok{6}\NormalTok{, }\AttributeTok{dpi =} \DecValTok{300}\NormalTok{)}
\end{Highlighting}
\end{Shaded}

\begin{Shaded}
\begin{Highlighting}[]
\NormalTok{test\_result\_FAMD }\OtherTok{\textless{}{-}} \FunctionTok{famd\_test}\NormalTok{(}
  \AttributeTok{test =}\NormalTok{ AMOX\_CMIN\_TEST, }
  \AttributeTok{train\_results =}\NormalTok{ train\_result\_FAMD)}
\end{Highlighting}
\end{Shaded}

To evaluate the method, a predictions as a function of observed
concentrations goodness of fit plot is presented as well as the average
model weights are plotted, stratified by covariate categories/quantiles.

\begin{Shaded}
\begin{Highlighting}[]
\NormalTok{test\_data\_FAMD }\OtherTok{\textless{}{-}}\NormalTok{ test\_result\_FAMD}\SpecialCharTok{$}\NormalTok{test\_results}
\NormalTok{test\_result\_FAMD}\SpecialCharTok{$}\NormalTok{GOF\_plot}
\end{Highlighting}
\end{Shaded}

\pandocbounded{\includegraphics[keepaspectratio]{MIPD/Precision_dosing_methods_files/figure-pdf/evaluate-FAMD-prediction-1.pdf}}

\begin{Shaded}
\begin{Highlighting}[]
\NormalTok{weight\_BURN\_FAMD }\OtherTok{\textless{}{-}} \FunctionTok{generate\_weight\_plot}\NormalTok{(test\_data\_FAMD, }\StringTok{"BURN"}\NormalTok{, }\AttributeTok{is\_categorical =} \ConstantTok{TRUE}\NormalTok{)}
\NormalTok{weight\_ICU\_FAMD }\OtherTok{\textless{}{-}} \FunctionTok{generate\_weight\_plot}\NormalTok{(test\_data\_FAMD, }\StringTok{"ICU"}\NormalTok{, }\AttributeTok{is\_categorical =} \ConstantTok{TRUE}\NormalTok{)}
\NormalTok{weight\_OBESE\_FAMD }\OtherTok{\textless{}{-}} \FunctionTok{generate\_weight\_plot}\NormalTok{(test\_data\_FAMD, }\StringTok{"OBESE"}\NormalTok{, }\AttributeTok{is\_categorical =} \ConstantTok{TRUE}\NormalTok{)}
\NormalTok{weight\_CREAT\_FAMD }\OtherTok{\textless{}{-}} \FunctionTok{generate\_weight\_plot}\NormalTok{(test\_data\_FAMD, }\StringTok{"CREAT"}\NormalTok{, }\AttributeTok{is\_categorical =} \ConstantTok{FALSE}\NormalTok{)}
\NormalTok{weight\_WT\_FAMD }\OtherTok{\textless{}{-}} \FunctionTok{generate\_weight\_plot}\NormalTok{(test\_data\_FAMD, }\StringTok{"WT"}\NormalTok{, }\AttributeTok{is\_categorical =} \ConstantTok{FALSE}\NormalTok{)}
\NormalTok{weight\_AGE\_FAMD }\OtherTok{\textless{}{-}} \FunctionTok{generate\_weight\_plot}\NormalTok{(test\_data\_FAMD, }\StringTok{"AGE"}\NormalTok{, }\AttributeTok{is\_categorical =} \ConstantTok{FALSE}\NormalTok{)}
\NormalTok{weight\_SEX\_FAMD }\OtherTok{\textless{}{-}} \FunctionTok{generate\_weight\_plot}\NormalTok{(test\_data\_FAMD, }\StringTok{"SEX"}\NormalTok{, }\AttributeTok{is\_categorical =} \ConstantTok{TRUE}\NormalTok{)}
\NormalTok{weight\_BURN\_FAMD}
\end{Highlighting}
\end{Shaded}

\pandocbounded{\includegraphics[keepaspectratio]{MIPD/Precision_dosing_methods_files/figure-pdf/evaluate-FAMD-prediction-2.pdf}}

\begin{Shaded}
\begin{Highlighting}[]
\NormalTok{weight\_ICU\_FAMD}
\end{Highlighting}
\end{Shaded}

\pandocbounded{\includegraphics[keepaspectratio]{MIPD/Precision_dosing_methods_files/figure-pdf/evaluate-FAMD-prediction-3.pdf}}

\begin{Shaded}
\begin{Highlighting}[]
\NormalTok{weight\_OBESE\_FAMD}
\end{Highlighting}
\end{Shaded}

\pandocbounded{\includegraphics[keepaspectratio]{MIPD/Precision_dosing_methods_files/figure-pdf/evaluate-FAMD-prediction-4.pdf}}

\begin{Shaded}
\begin{Highlighting}[]
\NormalTok{weight\_CREAT\_FAMD}
\end{Highlighting}
\end{Shaded}

\pandocbounded{\includegraphics[keepaspectratio]{MIPD/Precision_dosing_methods_files/figure-pdf/evaluate-FAMD-prediction-5.pdf}}

\begin{Shaded}
\begin{Highlighting}[]
\NormalTok{weight\_AGE\_FAMD}
\end{Highlighting}
\end{Shaded}

\pandocbounded{\includegraphics[keepaspectratio]{MIPD/Precision_dosing_methods_files/figure-pdf/evaluate-FAMD-prediction-6.pdf}}

\begin{Shaded}
\begin{Highlighting}[]
\NormalTok{weight\_SEX\_FAMD}
\end{Highlighting}
\end{Shaded}

\pandocbounded{\includegraphics[keepaspectratio]{MIPD/Precision_dosing_methods_files/figure-pdf/evaluate-FAMD-prediction-7.pdf}}

\begin{Shaded}
\begin{Highlighting}[]
\NormalTok{weight\_OBESE\_FAMD }\OtherTok{\textless{}{-}}\NormalTok{ weight\_OBESE\_FAMD }\SpecialCharTok{+} 
  \FunctionTok{labs}\NormalTok{(}\AttributeTok{title =} \StringTok{"FAMD weight attribution, stratified by OBESE"}\NormalTok{)}
\FunctionTok{ggsave}\NormalTok{(}\AttributeTok{filename =} \FunctionTok{here}\NormalTok{(}\StringTok{"Amoxicillin/a\_priori/For\_publication/Figures/S6a.jpg"}\NormalTok{),}
       \AttributeTok{plot =}\NormalTok{ weight\_OBESE\_FAMD,}
       \AttributeTok{width =} \DecValTok{8}\NormalTok{, }\AttributeTok{height =} \DecValTok{6}\NormalTok{, }\AttributeTok{dpi =} \DecValTok{300}\NormalTok{)}

\NormalTok{test\_data\_FAMD }\SpecialCharTok{\%\textgreater{}\%}
  \FunctionTok{group\_by}\NormalTok{(MODEL) }\SpecialCharTok{\%\textgreater{}\%}
  \FunctionTok{summarise}\NormalTok{(}\AttributeTok{avg\_weight =} \FunctionTok{mean}\NormalTok{(WEIGHT, }\AttributeTok{na.rm =} \ConstantTok{TRUE}\NormalTok{))}
\end{Highlighting}
\end{Shaded}

\begin{verbatim}
# A tibble: 4 x 2
  MODEL    avg_weight
  <chr>         <dbl>
1 CARLIER       0.248
2 FOURNIER      0.257
3 MELLON        0.189
4 RAMBAUD       0.306
\end{verbatim}

\begin{Shaded}
\begin{Highlighting}[]
\CommentTok{\# Extrapolate dose based on the ensembled concentrations and compare it to the true simulated dose.}
\NormalTok{test\_data\_FAMD }\OtherTok{\textless{}{-}}\NormalTok{ test\_data\_FAMD }\SpecialCharTok{\%\textgreater{}\%}
\NormalTok{  dplyr}\SpecialCharTok{::}\FunctionTok{filter}\NormalTok{(REFERENCE }\SpecialCharTok{==} \DecValTok{1}\NormalTok{) }\SpecialCharTok{\%\textgreater{}\%}
\NormalTok{  dplyr}\SpecialCharTok{::}\FunctionTok{mutate}\NormalTok{(}\AttributeTok{DOSE\_PRED =}\NormalTok{ (}\DecValTok{60}\SpecialCharTok{/}\NormalTok{WEIGHTED\_PREDICTION) }\SpecialCharTok{*}\NormalTok{ DOSE\_ADM) }\CommentTok{\# Administered dose extrapolation to reach 60 mg/L}

\NormalTok{final\_results\_FAMD }\OtherTok{\textless{}{-}} \FunctionTok{explore\_predictions}\NormalTok{(test\_data\_FAMD)}
\NormalTok{final\_results\_FAMD}\SpecialCharTok{$}\NormalTok{target\_attainment }
\end{Highlighting}
\end{Shaded}

\pandocbounded{\includegraphics[keepaspectratio]{MIPD/Precision_dosing_methods_files/figure-pdf/evaluate-FAMD-prediction-8.pdf}}

\begin{Shaded}
\begin{Highlighting}[]
\NormalTok{crcl\_plot\_FAMD }\OtherTok{\textless{}{-}}\NormalTok{ final\_results\_FAMD}\SpecialCharTok{$}\NormalTok{crcl\_plot}
\NormalTok{crcl\_plot\_FAMD}
\end{Highlighting}
\end{Shaded}

\pandocbounded{\includegraphics[keepaspectratio]{MIPD/Precision_dosing_methods_files/figure-pdf/evaluate-FAMD-prediction-9.pdf}}

\begin{Shaded}
\begin{Highlighting}[]
\NormalTok{final\_results\_FAMD}\SpecialCharTok{$}\NormalTok{summary\_stats }
\end{Highlighting}
\end{Shaded}

\begin{verbatim}
# A tibble: 2 x 9
  Prediction_correctness mean_CREAT sd_CREAT mean_WT sd_WT mean_AGE sd_AGE Count
  <chr>                       <dbl>    <dbl>   <dbl> <dbl>    <dbl>  <dbl> <int>
1 Correct                     0.953    0.499    84.4  19.1     64.0   16.2   241
2 Incorrect                   0.948    0.627    84.4  18.4     57.2   16.0   359
# i 1 more variable: Proportion <dbl>
\end{verbatim}

\begin{Shaded}
\begin{Highlighting}[]
\CommentTok{\# Keep data for method ensembling}
\NormalTok{test\_data\_FAMD\_ens }\OtherTok{\textless{}{-}}\NormalTok{ test\_data\_FAMD }\SpecialCharTok{\%\textgreater{}\%}
\NormalTok{  dplyr}\SpecialCharTok{::}\FunctionTok{mutate}\NormalTok{(}\AttributeTok{FAMD =}\NormalTok{ DOSE\_PRED) }\SpecialCharTok{\%\textgreater{}\%} 
\NormalTok{ dplyr}\SpecialCharTok{::}\FunctionTok{select}\NormalTok{(}\StringTok{"ID"}\NormalTok{,}\StringTok{"FAMD"}\NormalTok{)}
\end{Highlighting}
\end{Shaded}

\chapter{Method ensembling}\label{method-ensembling}

Ensembling for machine learning (XGBoost, Random forest, k nearest
neighobrs, support vector machine) and PopPK ensembling (classification
tree informed ensembling, regression tree informed ensembling, weighed
model ensembling, FAMD). A dose prediction to reach 60 mg/L is made for
each subject and the method with the dose prediction closest to the true
dose is selected. A classification tree is trained to predict the most
suited dose for each subject.

\begin{Shaded}
\begin{Highlighting}[]
\CommentTok{\# First prediction for the four single ML algorithms}
\CommentTok{\# For ML, take only one line per patient (the true concentration)}
\NormalTok{train }\OtherTok{\textless{}{-}}\NormalTok{ AMOX\_CMIN\_TRAIN }\SpecialCharTok{\%\textgreater{}\%}
\NormalTok{  dplyr}\SpecialCharTok{::}\FunctionTok{filter}\NormalTok{(REFERENCE }\SpecialCharTok{==} \DecValTok{1}\NormalTok{) }\SpecialCharTok{\%\textgreater{}\%}
  \FunctionTok{mutate}\NormalTok{(}\AttributeTok{II =}\NormalTok{ FREQ)}

\NormalTok{test }\OtherTok{\textless{}{-}}\NormalTok{ AMOX\_CMIN\_TEST }\SpecialCharTok{\%\textgreater{}\%}
\NormalTok{  dplyr}\SpecialCharTok{::}\FunctionTok{filter}\NormalTok{(REFERENCE }\SpecialCharTok{==} \DecValTok{1}\NormalTok{) }\SpecialCharTok{\%\textgreater{}\%}
  \FunctionTok{mutate}\NormalTok{(}\AttributeTok{II =}\NormalTok{ FREQ)}

\NormalTok{test\_data\_ML }\OtherTok{\textless{}{-}} \FunctionTok{machine\_learning}\NormalTok{(}\AttributeTok{train =}\NormalTok{ train, }\AttributeTok{test =}\NormalTok{ test,  }\AttributeTok{continuous\_cov =} \FunctionTok{c}\NormalTok{(}\StringTok{"WT"}\NormalTok{, }\StringTok{"CREAT"}\NormalTok{, }\StringTok{"AGE"}\NormalTok{), }
                                 \AttributeTok{categorical\_cov =} \FunctionTok{c}\NormalTok{(}\StringTok{"OBESE"}\NormalTok{, }\StringTok{"ICU"}\NormalTok{, }\StringTok{"BURN"}\NormalTok{, }\StringTok{"SEX"}\NormalTok{, }\StringTok{"CON"}\NormalTok{), }\AttributeTok{target\_variable =} \StringTok{"CMIN"}\NormalTok{, }\AttributeTok{target\_concentration =} \DecValTok{60}\NormalTok{)}
\end{Highlighting}
\end{Shaded}

\begin{Shaded}
\begin{Highlighting}[]
\FunctionTok{set.seed}\NormalTok{(}\DecValTok{1991}\NormalTok{)}
\NormalTok{conflicted}\SpecialCharTok{::}\FunctionTok{conflicts\_prefer}\NormalTok{(dplyr}\SpecialCharTok{::}\NormalTok{filter)}

\CommentTok{\# Put together the predictions of the 8 methods}
\NormalTok{datasets }\OtherTok{\textless{}{-}} \FunctionTok{list}\NormalTok{(test\_data\_ML, test\_data\_WME\_ens, test\_data\_FAMD\_ens, test\_data\_RT\_ens, test\_data\_CT\_ens)}
\CommentTok{\# Merge the datasets}
\NormalTok{data }\OtherTok{\textless{}{-}} \FunctionTok{reduce}\NormalTok{(datasets, full\_join, }\AttributeTok{by =} \StringTok{"ID"}\NormalTok{) }\SpecialCharTok{\%\textgreater{}\%} \FunctionTok{arrange}\NormalTok{(ID)}

\CommentTok{\# Divide the data to training and test data (50{-}50)}
\NormalTok{idx }\OtherTok{\textless{}{-}} \FunctionTok{seq\_len}\NormalTok{(}\FunctionTok{nrow}\NormalTok{(data)) }\SpecialCharTok{\%\%} \DecValTok{2} \SpecialCharTok{==} \DecValTok{1}

\NormalTok{train\_data  }\OtherTok{\textless{}{-}}\NormalTok{ data[idx, ]}
\NormalTok{test\_data }\OtherTok{\textless{}{-}}\NormalTok{ data[}\SpecialCharTok{!}\NormalTok{idx, ]}

\CommentTok{\# Add the name of the best method (whose dose prediction is the closest to the target dose) in a new column called Method}
\NormalTok{methods }\OtherTok{\textless{}{-}} \FunctionTok{c}\NormalTok{(}\StringTok{"RF"}\NormalTok{, }\StringTok{"XGB"}\NormalTok{, }\StringTok{"KNN"}\NormalTok{, }\StringTok{"SVM"}\NormalTok{, }\StringTok{"WME"}\NormalTok{, }\StringTok{"CT"}\NormalTok{, }\StringTok{"RT"}\NormalTok{, }\StringTok{"FAMD"}\NormalTok{)}

\NormalTok{train\_data }\OtherTok{\textless{}{-}}\NormalTok{ train\_data }\SpecialCharTok{\%\textgreater{}\%}
  \FunctionTok{rowwise}\NormalTok{() }\SpecialCharTok{\%\textgreater{}\%}
\NormalTok{  dplyr}\SpecialCharTok{::}\FunctionTok{mutate}\NormalTok{(}
    \AttributeTok{Method =}\NormalTok{ methods[}\FunctionTok{which.min}\NormalTok{(}\FunctionTok{abs}\NormalTok{(}\FunctionTok{c\_across}\NormalTok{(}\FunctionTok{all\_of}\NormalTok{(methods)) }\SpecialCharTok{{-}}\NormalTok{ DOSE\_TARGET))]}
\NormalTok{  ) }\SpecialCharTok{\%\textgreater{}\%}
  \FunctionTok{ungroup}\NormalTok{()}

\CommentTok{\# Range for complexity parameter to test (based on cross{-}validation)}
\NormalTok{cp\_values }\OtherTok{\textless{}{-}} \FunctionTok{seq}\NormalTok{(}\FloatTok{0.001}\NormalTok{, }\FloatTok{0.04}\NormalTok{, }\AttributeTok{by =} \FloatTok{0.001}\NormalTok{) }

\CommentTok{\# Convert categorical variables to factors}
\NormalTok{train\_data }\OtherTok{\textless{}{-}}\NormalTok{ train\_data }\SpecialCharTok{\%\textgreater{}\%}
\NormalTok{  dplyr}\SpecialCharTok{::}\FunctionTok{mutate}\NormalTok{(}\FunctionTok{across}\NormalTok{(}\FunctionTok{c}\NormalTok{(Method, ICU, BURN, OBESE, SEX), as.factor))}

\CommentTok{\# Perform cross{-}validation for each complexity value}
\NormalTok{cv\_results }\OtherTok{\textless{}{-}} \FunctionTok{sapply}\NormalTok{(cp\_values, }\ControlFlowTok{function}\NormalTok{(cp) \{}
\NormalTok{  ensfit }\OtherTok{\textless{}{-}} \FunctionTok{rpart}\NormalTok{(}
\NormalTok{    Method }\SpecialCharTok{\textasciitilde{}}\NormalTok{ WT }\SpecialCharTok{+}\NormalTok{ CREAT }\SpecialCharTok{+}\NormalTok{ ICU }\SpecialCharTok{+}\NormalTok{ BURN }\SpecialCharTok{+}\NormalTok{ OBESE }\SpecialCharTok{+}\NormalTok{ AGE }\SpecialCharTok{+}\NormalTok{ SEX,}
    \AttributeTok{data =}\NormalTok{ train\_data,}
    \AttributeTok{method =} \StringTok{\textquotesingle{}class\textquotesingle{}}\NormalTok{,}
    \AttributeTok{parms =} \FunctionTok{list}\NormalTok{(}\AttributeTok{split =} \StringTok{"information"}\NormalTok{),}
    \AttributeTok{control =} \FunctionTok{rpart.control}\NormalTok{(}
      \AttributeTok{cp =}\NormalTok{ cp,}
      \AttributeTok{xval =} \DecValTok{10}\NormalTok{,}
      \AttributeTok{minsplit =} \DecValTok{4}\NormalTok{,}
      \AttributeTok{minbucket =} \DecValTok{4}
\NormalTok{    )}
\NormalTok{  )}
  \FunctionTok{min}\NormalTok{(ensfit}\SpecialCharTok{$}\NormalTok{cptable[, }\StringTok{"xerror"}\NormalTok{]) }
\NormalTok{\})}

\CommentTok{\# Get the minimal cross validation error}
\NormalTok{min\_error }\OtherTok{\textless{}{-}} \FunctionTok{min}\NormalTok{(cv\_results)}

\CommentTok{\# Choose largest complexity parameter among those achieving the minimum error}
\NormalTok{best\_cp }\OtherTok{\textless{}{-}} \FunctionTok{max}\NormalTok{(cp\_values[cv\_results }\SpecialCharTok{==}\NormalTok{ min\_error])}
\end{Highlighting}
\end{Shaded}

\begin{Shaded}
\begin{Highlighting}[]
\CommentTok{\# Refit the tree}
\NormalTok{ensfit }\OtherTok{\textless{}{-}} \FunctionTok{rpart}\NormalTok{(}
\NormalTok{  Method }\SpecialCharTok{\textasciitilde{}}\NormalTok{ WT }\SpecialCharTok{+}\NormalTok{ CREAT }\SpecialCharTok{+}\NormalTok{ ICU }\SpecialCharTok{+}\NormalTok{ BURN }\SpecialCharTok{+}\NormalTok{ OBESE }\SpecialCharTok{+}\NormalTok{ AGE }\SpecialCharTok{+}\NormalTok{ SEX,}
  \AttributeTok{data =}\NormalTok{ train\_data,}
  \AttributeTok{method =} \StringTok{\textquotesingle{}class\textquotesingle{}}\NormalTok{,}
  \AttributeTok{parms =} \FunctionTok{list}\NormalTok{(}\AttributeTok{split =} \StringTok{"information"}\NormalTok{),}
  \AttributeTok{control =} \FunctionTok{rpart.control}\NormalTok{(}
    \AttributeTok{cp =}\NormalTok{ best\_cp,}
    \AttributeTok{xval =} \DecValTok{10}\NormalTok{,}
    \AttributeTok{minsplit =} \DecValTok{4}\NormalTok{,}
    \AttributeTok{minbucket =} \DecValTok{4}
\NormalTok{  )}
\NormalTok{)}

\FunctionTok{saveRDS}\NormalTok{(ensfit, }\AttributeTok{file =} \StringTok{"ensfit.rds"}\NormalTok{)}
\end{Highlighting}
\end{Shaded}

The classification tree is plotted indicating the best-suited method for
each subject group.

\begin{Shaded}
\begin{Highlighting}[]
\CommentTok{\# Plot the tree}
\NormalTok{ens\_plot }\OtherTok{\textless{}{-}} \FunctionTok{rpart.plot}\NormalTok{(ensfit, }\AttributeTok{type =} \DecValTok{3}\NormalTok{, }\AttributeTok{extra =} \DecValTok{0}\NormalTok{, }\AttributeTok{cex.main =} \FloatTok{1.25}\NormalTok{, }\AttributeTok{cex =} \DecValTok{1}\NormalTok{, }\AttributeTok{box.palette =} \StringTok{"Blues"}\NormalTok{, }\AttributeTok{main =} \FunctionTok{paste}\NormalTok{(}\StringTok{"Method ensembling"}\NormalTok{, }\StringTok{"(cp ="}\NormalTok{, best\_cp, }\StringTok{")"}\NormalTok{))}
\end{Highlighting}
\end{Shaded}

\pandocbounded{\includegraphics[keepaspectratio]{MIPD/Precision_dosing_methods_files/figure-pdf/evaluate-ens-training-1.pdf}}

\begin{Shaded}
\begin{Highlighting}[]
\FunctionTok{print}\NormalTok{(ens\_plot)}
\end{Highlighting}
\end{Shaded}

\begin{verbatim}
$obj
n= 300 

node), split, n, loss, yval, (yprob)
      * denotes terminal node

1) root 300 246 XGB (0.14 0.13 0.14 0.057 0.093 0.12 0.14 0.18)  
  2) WT>=93.59753 86  60 CT (0.3 0.093 0.17 0.047 0.035 0.093 0.17 0.081) *
  3) WT< 93.59753 214 167 XGB (0.07 0.14 0.13 0.061 0.12 0.14 0.12 0.22) *

$snipped.nodes
NULL

$xlim
[1] 0 1

$ylim
[1] 0 1

$x
[1] 0.5181817 0.1180116 0.9183518

$y
[1] 1.09484081 0.04991622 0.04991622

$branch.x
       [,1]      [,2]      [,3]
x 0.5181817 0.1180116 0.9183518
         NA 0.1180116 0.9183518
         NA 0.5181817 0.5181817

$branch.y
      [,1]      [,2]      [,3]
y 1.094841 0.1280856 0.1280856
        NA 1.0948408 1.0948408
        NA 1.0948408 1.0948408

$labs
[1] NA    "CT"  "XGB"

$cex
[1] 1

$boxes
$boxes$x1
[1]         NA 0.07978804 0.86638118

$boxes$y1
[1]         NA 0.02385976 0.02385976

$boxes$x2
[1]        NA 0.1562351 0.9703224

$boxes$y2
[1]        NA 0.1280856 0.1280856


$split.labs
[1] ""

$split.cex
[1] 1 1 1

$split.box
$split.box$x1
[1] 0.03758334         NA         NA

$split.box$y1
[1] 0.9280795        NA        NA

$split.box$x2
[1] 0.1984398        NA        NA

$split.box$y2
[1] 1.032305       NA       NA
\end{verbatim}

\begin{Shaded}
\begin{Highlighting}[]
\FunctionTok{jpeg}\NormalTok{(}\FunctionTok{here}\NormalTok{(}\StringTok{"Amoxicillin/a\_priori/For\_publication/Figures/S14.jpg"}\NormalTok{), }\AttributeTok{width =} \DecValTok{8}\NormalTok{, }\AttributeTok{height =} \DecValTok{6}\NormalTok{, }\AttributeTok{units =} \StringTok{"in"}\NormalTok{, }\AttributeTok{res =} \DecValTok{400}\NormalTok{)}
\FunctionTok{rpart.plot}\NormalTok{(ensfit, }\AttributeTok{type =} \DecValTok{3}\NormalTok{, }\AttributeTok{extra =} \DecValTok{0}\NormalTok{, }\AttributeTok{cex =} \DecValTok{1}\NormalTok{, }\AttributeTok{cex.main =} \FloatTok{1.25}\NormalTok{, }\AttributeTok{box.palette =} \StringTok{"Blues"}\NormalTok{, }\AttributeTok{main =} \StringTok{"Method ensembling"}\NormalTok{)}
\FunctionTok{dev.off}\NormalTok{()}
\end{Highlighting}
\end{Shaded}

\begin{verbatim}
pdf 
  2 
\end{verbatim}

\begin{Shaded}
\begin{Highlighting}[]
\CommentTok{\# Get for which patient which method is the best}
\NormalTok{leaves }\OtherTok{\textless{}{-}} \FunctionTok{rownames}\NormalTok{(ensfit}\SpecialCharTok{$}\NormalTok{frame)[ensfit}\SpecialCharTok{$}\NormalTok{frame}\SpecialCharTok{$}\NormalTok{var }\SpecialCharTok{==} \StringTok{"\textless{}leaf\textgreater{}"}\NormalTok{]}
\NormalTok{paths }\OtherTok{\textless{}{-}} \FunctionTok{path.rpart}\NormalTok{(ensfit, leaves, }\AttributeTok{print.it =} \ConstantTok{FALSE}\NormalTok{)}
\NormalTok{leaf\_classes }\OtherTok{\textless{}{-}}\NormalTok{ ensfit}\SpecialCharTok{$}\NormalTok{frame}\SpecialCharTok{$}\NormalTok{yval[}\FunctionTok{which}\NormalTok{(ensfit}\SpecialCharTok{$}\NormalTok{frame}\SpecialCharTok{$}\NormalTok{var }\SpecialCharTok{==} \StringTok{"\textless{}leaf\textgreater{}"}\NormalTok{)]}
\NormalTok{leaf\_labels }\OtherTok{\textless{}{-}} \FunctionTok{as.character}\NormalTok{(}\FunctionTok{attr}\NormalTok{(ensfit, }\StringTok{"ylevels"}\NormalTok{)[leaf\_classes])}
\NormalTok{leaf\_summary }\OtherTok{\textless{}{-}} \FunctionTok{data.frame}\NormalTok{(}
  \AttributeTok{Method =}\NormalTok{ leaf\_labels,}
  \AttributeTok{Path =} \FunctionTok{sapply}\NormalTok{(paths, }\ControlFlowTok{function}\NormalTok{(path) }\FunctionTok{paste}\NormalTok{(path, }\AttributeTok{collapse =} \StringTok{" \& "}\NormalTok{))}
\NormalTok{)}
\FunctionTok{print}\NormalTok{(leaf\_summary)}
\end{Highlighting}
\end{Shaded}

\begin{verbatim}
  Method            Path
2     CT root & WT>=93.6
3    XGB root & WT< 93.6
\end{verbatim}

\begin{Shaded}
\begin{Highlighting}[]
\CommentTok{\# Make predictions for the test data}
\NormalTok{test\_data }\OtherTok{\textless{}{-}}\NormalTok{ test\_data }\SpecialCharTok{\%\textgreater{}\%}
  \FunctionTok{mutate}\NormalTok{(}\FunctionTok{across}\NormalTok{(}\FunctionTok{c}\NormalTok{(ICU, BURN, OBESE, SEX), as.factor))}

\NormalTok{test\_data}\SpecialCharTok{$}\NormalTok{Method }\OtherTok{\textless{}{-}} \FunctionTok{predict}\NormalTok{(ensfit, }\AttributeTok{newdata =}\NormalTok{ test\_data, }\AttributeTok{type =} \StringTok{"class"}\NormalTok{)}

\CommentTok{\# Add the dose prediction of the best predicted method}
\NormalTok{test\_data\_results }\OtherTok{\textless{}{-}}\NormalTok{ test\_data }\SpecialCharTok{\%\textgreater{}\%}
  \FunctionTok{rowwise}\NormalTok{() }\SpecialCharTok{\%\textgreater{}\%}
\NormalTok{  dplyr}\SpecialCharTok{::}\FunctionTok{mutate}\NormalTok{(}\AttributeTok{DOSE\_PRED =} \FunctionTok{cur\_data}\NormalTok{()[[Method]]) }\SpecialCharTok{\%\textgreater{}\%}
  \FunctionTok{ungroup}\NormalTok{()}
\end{Highlighting}
\end{Shaded}

\begin{Shaded}
\begin{Highlighting}[]
\NormalTok{results\_ens }\OtherTok{\textless{}{-}} \FunctionTok{explore\_predictions}\NormalTok{(test\_data\_results)}
\NormalTok{results\_ens}\SpecialCharTok{$}\NormalTok{target\_attainment }
\end{Highlighting}
\end{Shaded}

\pandocbounded{\includegraphics[keepaspectratio]{MIPD/Precision_dosing_methods_files/figure-pdf/evaluate-ens-predict-1.pdf}}

\begin{Shaded}
\begin{Highlighting}[]
\NormalTok{crcl\_plot\_ens }\OtherTok{\textless{}{-}}\NormalTok{ results\_ens}\SpecialCharTok{$}\NormalTok{crcl\_plot}
\NormalTok{crcl\_plot\_ens}
\end{Highlighting}
\end{Shaded}

\pandocbounded{\includegraphics[keepaspectratio]{MIPD/Precision_dosing_methods_files/figure-pdf/evaluate-ens-predict-2.pdf}}

\begin{Shaded}
\begin{Highlighting}[]
\NormalTok{results\_ens}\SpecialCharTok{$}\NormalTok{summary\_stats }
\end{Highlighting}
\end{Shaded}

\begin{verbatim}
# A tibble: 2 x 9
  Prediction_correctness mean_CREAT sd_CREAT mean_WT sd_WT mean_AGE sd_AGE Count
  <chr>                       <dbl>    <dbl>   <dbl> <dbl>    <dbl>  <dbl> <int>
1 Correct                     0.909    0.494    85.3  19.9     62.0   16.3   107
2 Incorrect                   0.876    0.475    84.3  18.0     58.8   15.3   193
# i 1 more variable: Proportion <dbl>
\end{verbatim}

\chapter{Machine learning dose
prediction}\label{machine-learning-dose-prediction}

\begin{Shaded}
\begin{Highlighting}[]
\CommentTok{\# Load openMIPD package from tar.gz file}
\FunctionTok{install.packages}\NormalTok{(}\StringTok{"stp{-}brioche{-}R{-}package/openMIPD\_0.0.6.tar.gz"}\NormalTok{, }\AttributeTok{repos =} \ConstantTok{NULL}\NormalTok{, }\AttributeTok{type =} \StringTok{"source"}\NormalTok{)}

\FunctionTok{library}\NormalTok{(openMIPD)}
\FunctionTok{library}\NormalTok{(dplyr)}
\FunctionTok{library}\NormalTok{(ggplot2)}
\FunctionTok{library}\NormalTok{(here)}
\FunctionTok{library}\NormalTok{(tidyr)}
\end{Highlighting}
\end{Shaded}

The data is preprocessed to have the log-transformed daily dose as
target variable which allows to reach the target concentration and the
data is stratified based on dosing scheme.

\begin{Shaded}
\begin{Highlighting}[]
\CommentTok{\# Import or generate training and test data}
\NormalTok{here}\SpecialCharTok{::}\FunctionTok{i\_am}\NormalTok{(}\StringTok{"Amoxicillin/a\_priori/For\_publication/MIPD/Machine\_Learning.qmd"}\NormalTok{)}

\CommentTok{\# Create folder to store published figures}
\ControlFlowTok{if}\NormalTok{ (}\SpecialCharTok{!}\FunctionTok{dir.exists}\NormalTok{(}\FunctionTok{here}\NormalTok{(}\StringTok{"Amoxicillin/a\_priori/For\_publication/Figures"}\NormalTok{))) \{}
  \FunctionTok{dir.create}\NormalTok{(}\FunctionTok{here}\NormalTok{(}\StringTok{"Amoxicillin/a\_priori/For\_publication/Figures"}\NormalTok{))}
\NormalTok{\}}

\NormalTok{AMOX\_CMIN\_TRAIN }\OtherTok{\textless{}{-}} \FunctionTok{read.csv}\NormalTok{(}\FunctionTok{here}\NormalTok{(}\StringTok{"Amoxicillin/a\_priori/For\_publication/Data/AMOX\_CMIN\_TRAIN.csv"}\NormalTok{), }\AttributeTok{quote =} \StringTok{""}\NormalTok{)}

\NormalTok{AMOX\_CMIN\_TEST }\OtherTok{\textless{}{-}} \FunctionTok{read.csv}\NormalTok{(}\FunctionTok{here}\NormalTok{(}\StringTok{"Amoxicillin/a\_priori/For\_publication/Data/AMOX\_CMIN\_TEST.csv"}\NormalTok{), }\AttributeTok{quote =} \StringTok{""}\NormalTok{)}

\CommentTok{\# The interdose interval column has to be called II}
\NormalTok{train }\OtherTok{\textless{}{-}}\NormalTok{ AMOX\_CMIN\_TRAIN }\SpecialCharTok{\%\textgreater{}\%}
\NormalTok{  dplyr}\SpecialCharTok{::}\FunctionTok{filter}\NormalTok{(REFERENCE }\SpecialCharTok{==} \DecValTok{1}\NormalTok{) }\SpecialCharTok{\%\textgreater{}\%}
  \FunctionTok{mutate}\NormalTok{(}\AttributeTok{II =}\NormalTok{ FREQ)}
  
\NormalTok{test }\OtherTok{\textless{}{-}}\NormalTok{ AMOX\_CMIN\_TEST }\SpecialCharTok{\%\textgreater{}\%}
\NormalTok{  dplyr}\SpecialCharTok{::}\FunctionTok{filter}\NormalTok{(REFERENCE }\SpecialCharTok{==} \DecValTok{1}\NormalTok{) }\SpecialCharTok{\%\textgreater{}\%}
  \FunctionTok{mutate}\NormalTok{(}\AttributeTok{II =}\NormalTok{ FREQ)}

\CommentTok{\# machine learning function from the package}
\NormalTok{train\_preprocessed }\OtherTok{\textless{}{-}} \FunctionTok{ml\_data\_preprocess}\NormalTok{(}\AttributeTok{data =}\NormalTok{ train, }\AttributeTok{target\_variable =} \StringTok{"CMIN"}\NormalTok{, }\AttributeTok{target\_concentration =} \DecValTok{60}\NormalTok{)}
\NormalTok{test\_preprocessed }\OtherTok{\textless{}{-}} \FunctionTok{ml\_data\_preprocess}\NormalTok{(}\AttributeTok{data =}\NormalTok{ test, }\AttributeTok{target\_variable =} \StringTok{"CMIN"}\NormalTok{, }\AttributeTok{target\_concentration =} \DecValTok{60}\NormalTok{)}
\end{Highlighting}
\end{Shaded}

\begin{Shaded}
\begin{Highlighting}[]
\NormalTok{explore\_predictions }\OtherTok{\textless{}{-}} \ControlFlowTok{function}\NormalTok{(data, }\AttributeTok{conc\_inf =} \DecValTok{40}\NormalTok{, }\AttributeTok{conc\_sup =} \DecValTok{80}\NormalTok{, DOSE\_PRED) \{}
  
\NormalTok{  dose\_pred\_col }\OtherTok{\textless{}{-}} \FunctionTok{sym}\NormalTok{(DOSE\_PRED)}

  \CommentTok{\# Calculate true dose range and prediction correctness}
\NormalTok{  data }\OtherTok{\textless{}{-}}\NormalTok{ data }\SpecialCharTok{\%\textgreater{}\%}
    \FunctionTok{mutate}\NormalTok{(}
      \AttributeTok{DOSE\_inf =}\NormalTok{ (conc\_inf }\SpecialCharTok{/}\NormalTok{ CMIN\_IND) }\SpecialCharTok{*}\NormalTok{ DOSE\_ADM,}
      \AttributeTok{DOSE\_sup =}\NormalTok{ (conc\_sup }\SpecialCharTok{/}\NormalTok{ CMIN\_IND) }\SpecialCharTok{*}\NormalTok{ DOSE\_ADM}
\NormalTok{    ) }\SpecialCharTok{\%\textgreater{}\%}
    \FunctionTok{mutate}\NormalTok{(}
      \AttributeTok{Prediction\_correctness =} \FunctionTok{ifelse}\NormalTok{(}
\NormalTok{        (}\SpecialCharTok{!!}\NormalTok{dose\_pred\_col }\SpecialCharTok{\textgreater{}=}\NormalTok{ DOSE\_inf }\SpecialCharTok{\&} \SpecialCharTok{!!}\NormalTok{dose\_pred\_col }\SpecialCharTok{\textless{}=}\NormalTok{ DOSE\_sup),}
        \StringTok{"Correct"}\NormalTok{, }\StringTok{"Incorrect"}
\NormalTok{      )}
\NormalTok{    ) }\SpecialCharTok{\%\textgreater{}\%}
    \FunctionTok{drop\_na}\NormalTok{(Prediction\_correctness) }\SpecialCharTok{\%\textgreater{}\%}
    \FunctionTok{mutate}\NormalTok{(}
      \AttributeTok{Dosing =} \FunctionTok{case\_when}\NormalTok{(}
\NormalTok{        Prediction\_correctness }\SpecialCharTok{==} \StringTok{"Correct"} \SpecialCharTok{\textasciitilde{}} \StringTok{"On target"}\NormalTok{,}
        \SpecialCharTok{!!}\NormalTok{dose\_pred\_col }\SpecialCharTok{\textless{}}\NormalTok{ DOSE\_inf }\SpecialCharTok{\textasciitilde{}} \StringTok{"Underdosed"}\NormalTok{,}
        \SpecialCharTok{!!}\NormalTok{dose\_pred\_col }\SpecialCharTok{\textgreater{}}\NormalTok{ DOSE\_sup }\SpecialCharTok{\textasciitilde{}} \StringTok{"Overdosed"}
\NormalTok{      )}
\NormalTok{    )}

  \CommentTok{\# Proportions of under{-} and overdosing}
\NormalTok{  dosing }\OtherTok{\textless{}{-}}\NormalTok{ data }\SpecialCharTok{\%\textgreater{}\%}
    \FunctionTok{count}\NormalTok{(Dosing) }\SpecialCharTok{\%\textgreater{}\%}
    \FunctionTok{mutate}\NormalTok{(}
      \AttributeTok{Proportion =}\NormalTok{ n }\SpecialCharTok{/} \FunctionTok{sum}\NormalTok{(n) }\SpecialCharTok{*} \DecValTok{100}\NormalTok{,}
      \AttributeTok{Dosing =} \FunctionTok{factor}\NormalTok{(Dosing, }\AttributeTok{levels =} \FunctionTok{c}\NormalTok{(}\StringTok{"Overdosed"}\NormalTok{, }\StringTok{"On target"}\NormalTok{, }\StringTok{"Underdosed"}\NormalTok{)),}
      \AttributeTok{Label =} \FunctionTok{paste0}\NormalTok{(Dosing, }\StringTok{"}\SpecialCharTok{\textbackslash{}n}\StringTok{"}\NormalTok{, }\FunctionTok{round}\NormalTok{(Proportion), }\StringTok{"\%"}\NormalTok{)}
\NormalTok{    )}

  \CommentTok{\# Over/underdosed graph}
\NormalTok{  p1 }\OtherTok{\textless{}{-}} \FunctionTok{ggplot}\NormalTok{(dosing, }\FunctionTok{aes}\NormalTok{(}\AttributeTok{x =} \StringTok{""}\NormalTok{, }\AttributeTok{y =}\NormalTok{ Proportion, }\AttributeTok{fill =}\NormalTok{ Dosing)) }\SpecialCharTok{+}
    \FunctionTok{geom\_bar}\NormalTok{(}\AttributeTok{stat =} \StringTok{"identity"}\NormalTok{, }\AttributeTok{width =} \FloatTok{0.5}\NormalTok{) }\SpecialCharTok{+}
    \FunctionTok{geom\_text}\NormalTok{(}\FunctionTok{aes}\NormalTok{(}\AttributeTok{label =}\NormalTok{ Label), }\AttributeTok{position =} \FunctionTok{position\_stack}\NormalTok{(}\AttributeTok{vjust =} \FloatTok{0.5}\NormalTok{), }\AttributeTok{color =} \StringTok{"white"}\NormalTok{, }\AttributeTok{size =} \DecValTok{5}\NormalTok{) }\SpecialCharTok{+}
    \FunctionTok{scale\_fill\_manual}\NormalTok{(}\AttributeTok{values =} \FunctionTok{c}\NormalTok{(}\StringTok{"Underdosed"} \OtherTok{=} \StringTok{"darkorange"}\NormalTok{, }\StringTok{"On target"} \OtherTok{=} \StringTok{"chartreuse4"}\NormalTok{, }\StringTok{"Overdosed"} \OtherTok{=} \StringTok{"\#A91A27"}\NormalTok{)) }\SpecialCharTok{+}
    \FunctionTok{labs}\NormalTok{(}\AttributeTok{y =} \StringTok{"\%"}\NormalTok{, }\AttributeTok{x =} \ConstantTok{NULL}\NormalTok{, }\AttributeTok{title =} \StringTok{"Target attainment"}\NormalTok{, }\AttributeTok{fill =} \StringTok{"Dosing Category"}\NormalTok{) }\SpecialCharTok{+}
    \FunctionTok{theme\_minimal}\NormalTok{() }\SpecialCharTok{+}
    \FunctionTok{theme}\NormalTok{(}
      \AttributeTok{axis.text.x =} \FunctionTok{element\_blank}\NormalTok{(),}
      \AttributeTok{axis.ticks.x =} \FunctionTok{element\_blank}\NormalTok{(),}
      \AttributeTok{plot.title =} \FunctionTok{element\_text}\NormalTok{(}\AttributeTok{size =} \DecValTok{20}\NormalTok{),}
      \AttributeTok{axis.text =} \FunctionTok{element\_text}\NormalTok{(}\AttributeTok{size =} \DecValTok{16}\NormalTok{),}
      \AttributeTok{axis.title =} \FunctionTok{element\_text}\NormalTok{(}\AttributeTok{size =} \DecValTok{20}\NormalTok{),}
      \AttributeTok{legend.position =} \StringTok{"none"}
\NormalTok{    ) }\SpecialCharTok{+}
    \FunctionTok{scale\_y\_continuous}\NormalTok{(}\AttributeTok{breaks =} \FunctionTok{seq}\NormalTok{(}\DecValTok{0}\NormalTok{, }\DecValTok{100}\NormalTok{, }\AttributeTok{by =} \DecValTok{10}\NormalTok{)) }\SpecialCharTok{+} 
        \FunctionTok{coord\_cartesian}\NormalTok{(}\AttributeTok{ylim =} \FunctionTok{c}\NormalTok{(}\DecValTok{0}\NormalTok{, }\DecValTok{100}\NormalTok{))}

  \CommentTok{\# Summary statistics}
\NormalTok{  summary\_stats }\OtherTok{\textless{}{-}}\NormalTok{ data }\SpecialCharTok{\%\textgreater{}\%}
    \FunctionTok{group\_by}\NormalTok{(Prediction\_correctness) }\SpecialCharTok{\%\textgreater{}\%}
    \FunctionTok{summarise}\NormalTok{(}
      \AttributeTok{Count =} \FunctionTok{n}\NormalTok{(),}
      \AttributeTok{Obese =} \FunctionTok{sum}\NormalTok{(WT }\SpecialCharTok{/}\NormalTok{ (HT }\SpecialCharTok{/} \DecValTok{100}\NormalTok{)}\SpecialCharTok{\^{}}\DecValTok{2} \SpecialCharTok{\textgreater{}} \DecValTok{30}\NormalTok{, }\AttributeTok{na.rm =} \ConstantTok{TRUE}\NormalTok{),}
      \AttributeTok{mean\_CREAT =} \FunctionTok{mean}\NormalTok{(CREAT, }\AttributeTok{na.rm =} \ConstantTok{TRUE}\NormalTok{),}
      \AttributeTok{sd\_CREAT =} \FunctionTok{sd}\NormalTok{(CREAT, }\AttributeTok{na.rm =} \ConstantTok{TRUE}\NormalTok{),}
      \AttributeTok{mean\_WT =} \FunctionTok{mean}\NormalTok{(WT, }\AttributeTok{na.rm =} \ConstantTok{TRUE}\NormalTok{),}
      \AttributeTok{sd\_WT =} \FunctionTok{sd}\NormalTok{(WT, }\AttributeTok{na.rm =} \ConstantTok{TRUE}\NormalTok{),}
      \AttributeTok{mean\_AGE =} \FunctionTok{mean}\NormalTok{(AGE, }\AttributeTok{na.rm =} \ConstantTok{TRUE}\NormalTok{),}
      \AttributeTok{sd\_AGE =} \FunctionTok{sd}\NormalTok{(AGE, }\AttributeTok{na.rm =} \ConstantTok{TRUE}\NormalTok{)}
\NormalTok{    ) }\SpecialCharTok{\%\textgreater{}\%}
    \FunctionTok{mutate}\NormalTok{(}\AttributeTok{Proportion =}\NormalTok{ Count }\SpecialCharTok{/} \FunctionTok{sum}\NormalTok{(Count))}

\NormalTok{  correct\_proportion }\OtherTok{\textless{}{-}}\NormalTok{ summary\_stats }\SpecialCharTok{\%\textgreater{}\%}
\NormalTok{   dplyr}\SpecialCharTok{::}\FunctionTok{filter}\NormalTok{(Prediction\_correctness }\SpecialCharTok{==} \StringTok{"Correct"}\NormalTok{) }\SpecialCharTok{\%\textgreater{}\%}
    \FunctionTok{pull}\NormalTok{(Proportion)}

  \FunctionTok{message}\NormalTok{(}\FunctionTok{sprintf}\NormalTok{(}\StringTok{"Proportion of \textquotesingle{}correct\textquotesingle{} predictions: \%.2f\%\%"}\NormalTok{, correct\_proportion }\SpecialCharTok{*} \DecValTok{100}\NormalTok{))}

  \FunctionTok{return}\NormalTok{(}\FunctionTok{list}\NormalTok{(}
    \AttributeTok{target\_attainment =}\NormalTok{ p1,}
    \AttributeTok{summary\_stats =}\NormalTok{ summary\_stats}
\NormalTok{  ))}
\NormalTok{\}}
\end{Highlighting}
\end{Shaded}

For ML, algorithms are trained on covariates to predict the daily dose
which allows to reach a concentration of 60 mg/L. The target dose is
obtained by linear extrapolation to attain 60 mg/L. For intermittent
infusion, the daily dose is fractioned by the number of administrations.
The predictors are the covariates (WT, CREAT, BURN, OBESE, ICU, SEX,
AGE) as well as the dosing scheme coded as the number of daily
administrations (INF - 1 for continuous infusion).

The \emph{tidymodels} workflow is used which is incorporated in a
fit-for-purpose way in our \emph{openMIPD} package.

\section{XGboost}\label{xgboost}

XGBoost is based on decision trees and the boosting algorithm,
developing trees sequentially to correct the prediction errors of
previous trees. Model overfitting is controlled by incorporating Lasso
(L1) and Ridge (L2) algorithms. XGBoost is faster and has better
computational efficiency compared to Random Forest. While XGBoost is
less interpretable, it is better suited for larger datasets than Random
Forest. Whereas Random Forest helps reduce variance, XGBoost primarily
reduces bias.

The hyperparameters to define are as follows:

\begin{itemize}
\item
  \textbf{η} -- learning rate (default: 0.3)
\item
  \textbf{γ} -- the minimum loss required for a new split; if increased,
  the algorithm becomes more conservative
\item
  \textbf{max depth} -- tree depth (typical values range from 3 to 10)
\item
  \textbf{minimum child weight} - minimum number of observations in the
  terminal node
\item
  \textbf{λ/α} -- regularization parameters for Ridge/Lasso
\item
  \textbf{max levels} -- maximum number of nodes
\end{itemize}

\begin{Shaded}
\begin{Highlighting}[]
\NormalTok{XGB\_results }\OtherTok{\textless{}{-}}\NormalTok{ openMIPD}\SpecialCharTok{::}\FunctionTok{xgb\_train}\NormalTok{(}\AttributeTok{train =}\NormalTok{ train\_preprocessed, }\AttributeTok{continuous\_cov =} \FunctionTok{c}\NormalTok{(}\StringTok{"WT"}\NormalTok{, }\StringTok{"CREAT"}\NormalTok{, }\StringTok{"AGE"}\NormalTok{, }\StringTok{"INF"}\NormalTok{), }\AttributeTok{categorical\_cov =} \FunctionTok{c}\NormalTok{(}\StringTok{"BURN"}\NormalTok{, }\StringTok{"OBESE"}\NormalTok{, }\StringTok{"SEX"}\NormalTok{, }\StringTok{"ICU"}\NormalTok{))}
\end{Highlighting}
\end{Shaded}

A variable importance plot is generated which show the contribution of
different predictors to the results. Also, the tuning and hyperparameter
optimization is visualized.

\begin{Shaded}
\begin{Highlighting}[]
\NormalTok{XGB\_results}\SpecialCharTok{$}\NormalTok{tune\_plot\_xgb }\CommentTok{\# tuning}
\end{Highlighting}
\end{Shaded}

\pandocbounded{\includegraphics[keepaspectratio]{MIPD/Machine_Learning_files/figure-pdf/evaluate-xgboost-train-1.pdf}}

\begin{Shaded}
\begin{Highlighting}[]
\NormalTok{XGB\_results}\SpecialCharTok{$}\NormalTok{final\_wf\_xgb }\CommentTok{\# optimized hyperparameters}
\end{Highlighting}
\end{Shaded}

\begin{verbatim}
== Workflow ====================================================================
Preprocessor: Recipe
Model: boost_tree()

-- Preprocessor ----------------------------------------------------------------
2 Recipe Steps

* step_mutate_at()
* step_dummy()

-- Model -----------------------------------------------------------------------
Boosted Tree Model Specification (regression)

Main Arguments:
  trees = 2000
  min_n = 31
  tree_depth = 4
  learn_rate = 0.0464158883361278

Computational engine: xgboost 
\end{verbatim}

\begin{Shaded}
\begin{Highlighting}[]
\NormalTok{XGB\_VIP }\OtherTok{\textless{}{-}}\NormalTok{ XGB\_results}\SpecialCharTok{$}\NormalTok{xgb\_vip }\CommentTok{\# variable importance plot}
\NormalTok{XGB\_VIP}
\end{Highlighting}
\end{Shaded}

\pandocbounded{\includegraphics[keepaspectratio]{MIPD/Machine_Learning_files/figure-pdf/evaluate-xgboost-train-2.pdf}}

\begin{Shaded}
\begin{Highlighting}[]
\FunctionTok{ggsave}\NormalTok{(}\AttributeTok{filename =} \FunctionTok{here}\NormalTok{(}\StringTok{"Amoxicillin/a\_priori/For\_publication/Figures/S12a.jpg"}\NormalTok{),}
       \AttributeTok{plot =}\NormalTok{ XGB\_VIP,}
       \AttributeTok{width =} \DecValTok{8}\NormalTok{, }\AttributeTok{height =} \DecValTok{6}\NormalTok{, }\AttributeTok{dpi =} \DecValTok{300}\NormalTok{)}

\NormalTok{final\_xgb\_fit }\OtherTok{\textless{}{-}}\NormalTok{ XGB\_results}\SpecialCharTok{$}\NormalTok{final\_xgb\_fit}
\end{Highlighting}
\end{Shaded}

\begin{Shaded}
\begin{Highlighting}[]
\NormalTok{test\_XGB }\OtherTok{\textless{}{-}}\NormalTok{ openMIPD}\SpecialCharTok{::}\FunctionTok{xgb\_test}\NormalTok{(}\AttributeTok{final\_fit =}\NormalTok{ final\_xgb\_fit, }\AttributeTok{test =}\NormalTok{ test\_preprocessed)}
\end{Highlighting}
\end{Shaded}

\begin{Shaded}
\begin{Highlighting}[]
\NormalTok{ta\_results\_XGB }\OtherTok{\textless{}{-}} \FunctionTok{explore\_predictions}\NormalTok{(}\AttributeTok{data =}\NormalTok{ test\_XGB, }\AttributeTok{DOSE\_PRED =} \StringTok{"XGBoost"}\NormalTok{)}
\NormalTok{ta\_results\_XGB}\SpecialCharTok{$}\NormalTok{target\_attainment}
\end{Highlighting}
\end{Shaded}

\pandocbounded{\includegraphics[keepaspectratio]{MIPD/Machine_Learning_files/figure-pdf/evaluate-xgboost-test-1.pdf}}

\begin{Shaded}
\begin{Highlighting}[]
\NormalTok{ta\_results\_XGB}\SpecialCharTok{$}\NormalTok{summary\_stats}
\end{Highlighting}
\end{Shaded}

\begin{verbatim}
# A tibble: 2 x 10
  Prediction_correctness Count Obese mean_CREAT sd_CREAT mean_WT sd_WT mean_AGE
  <chr>                  <int> <int>      <dbl>    <dbl>   <dbl> <dbl>    <dbl>
1 Correct                  216    84      0.950    0.475    84.9  19.3     62.3
2 Incorrect                384   133      0.950    0.630    84.1  18.3     58.6
# i 2 more variables: sd_AGE <dbl>, Proportion <dbl>
\end{verbatim}

\section{Random forest}\label{random-forest}

A decision tree is a model that represents possible decision paths in a
schematic tree-like structure. In regression trees, node splitting is
done to minimize intra-group variance. In classification trees, node
purity can be measured using the Gini index, where \emph{f} is the
frequency of observations:

\[
\text{Gini index} = 2 \cdot f \cdot (1 - f)
\]

Introduced in 2001, Random Forest (RF) is a machine learning method
based on an ensemble of decision trees. It uses the bootstrap
aggregating algorithm (commonly known as \emph{bagging}). The
performance of an RF model is measured by prediction error, which
corresponds to the mean squared error (MSE) for regression problems.
Random Forest helps reduce variance, making it suitable for unstable,
unbiased data.

Hyperparameters to define:

\begin{itemize}
\item
  \textbf{Number of trees} (typically around 400)
\item
  \textbf{mtry} -- the number of variables considered at each node.
  Default value is \(\sqrt{\text{number of predictors}}\)
\item
  \textbf{Number of observations in the terminal node (leaf)}
\end{itemize}

\begin{Shaded}
\begin{Highlighting}[]
\NormalTok{RF\_results }\OtherTok{\textless{}{-}}\NormalTok{ openMIPD}\SpecialCharTok{::}\FunctionTok{rf\_train}\NormalTok{(}\AttributeTok{train =}\NormalTok{ train\_preprocessed, }\AttributeTok{continuous\_cov =} \FunctionTok{c}\NormalTok{(}\StringTok{"WT"}\NormalTok{, }\StringTok{"CREAT"}\NormalTok{, }\StringTok{"AGE"}\NormalTok{, }\StringTok{"INF"}\NormalTok{), }\AttributeTok{categorical\_cov =} \FunctionTok{c}\NormalTok{(}\StringTok{"BURN"}\NormalTok{, }\StringTok{"OBESE"}\NormalTok{, }\StringTok{"SEX"}\NormalTok{, }\StringTok{"ICU"}\NormalTok{))}
\end{Highlighting}
\end{Shaded}

A variable importance plot is generated which show the contribution of
different predictors to the results. Also, the tuning and hyperparameter
optimization is visualized.

\begin{Shaded}
\begin{Highlighting}[]
\NormalTok{RF\_results}\SpecialCharTok{$}\NormalTok{tune\_plot\_rf }\CommentTok{\# tuning}
\end{Highlighting}
\end{Shaded}

\pandocbounded{\includegraphics[keepaspectratio]{MIPD/Machine_Learning_files/figure-pdf/evaluate-rf-train-1.pdf}}

\begin{Shaded}
\begin{Highlighting}[]
\NormalTok{RF\_results}\SpecialCharTok{$}\NormalTok{final\_wf\_rf }\CommentTok{\# optimized hyperparameters}
\end{Highlighting}
\end{Shaded}

\begin{verbatim}
== Workflow ====================================================================
Preprocessor: Recipe
Model: rand_forest()

-- Preprocessor ----------------------------------------------------------------
2 Recipe Steps

* step_mutate_at()
* step_dummy()

-- Model -----------------------------------------------------------------------
Random Forest Model Specification (regression)

Main Arguments:
  mtry = 3
  trees = 1777

Engine-Specific Arguments:
  importance = impurity

Computational engine: ranger 
\end{verbatim}

\begin{Shaded}
\begin{Highlighting}[]
\NormalTok{RF\_VIP }\OtherTok{\textless{}{-}}\NormalTok{ RF\_results}\SpecialCharTok{$}\NormalTok{rf\_vip }\CommentTok{\# variable importance plot}
\NormalTok{RF\_VIP}
\end{Highlighting}
\end{Shaded}

\pandocbounded{\includegraphics[keepaspectratio]{MIPD/Machine_Learning_files/figure-pdf/evaluate-rf-train-2.pdf}}

\begin{Shaded}
\begin{Highlighting}[]
\FunctionTok{ggsave}\NormalTok{(}\AttributeTok{filename =} \FunctionTok{here}\NormalTok{(}\StringTok{"Amoxicillin/a\_priori/For\_publication/Figures/S12b.jpg"}\NormalTok{),}
       \AttributeTok{plot =}\NormalTok{ RF\_VIP,}
       \AttributeTok{width =} \DecValTok{8}\NormalTok{, }\AttributeTok{height =} \DecValTok{6}\NormalTok{, }\AttributeTok{dpi =} \DecValTok{300}\NormalTok{)}
\NormalTok{final\_rf\_fit }\OtherTok{\textless{}{-}}\NormalTok{ RF\_results}\SpecialCharTok{$}\NormalTok{final\_rf\_fit}
\end{Highlighting}
\end{Shaded}

\begin{Shaded}
\begin{Highlighting}[]
\NormalTok{test\_RF }\OtherTok{\textless{}{-}}\NormalTok{ openMIPD}\SpecialCharTok{::}\FunctionTok{rf\_test}\NormalTok{(}\AttributeTok{final\_fit =}\NormalTok{ final\_rf\_fit, }\AttributeTok{test =}\NormalTok{ test\_preprocessed)}
\end{Highlighting}
\end{Shaded}

\begin{Shaded}
\begin{Highlighting}[]
\NormalTok{ta\_results\_RF }\OtherTok{\textless{}{-}} \FunctionTok{explore\_predictions}\NormalTok{(test\_RF, }\AttributeTok{DOSE\_PRED =} \StringTok{"RF"}\NormalTok{)}
\NormalTok{ta\_results\_RF}\SpecialCharTok{$}\NormalTok{target\_attainment}
\end{Highlighting}
\end{Shaded}

\pandocbounded{\includegraphics[keepaspectratio]{MIPD/Machine_Learning_files/figure-pdf/evaluate-rf-test-1.pdf}}

\begin{Shaded}
\begin{Highlighting}[]
\NormalTok{ta\_results\_RF}\SpecialCharTok{$}\NormalTok{summary\_stats}
\end{Highlighting}
\end{Shaded}

\begin{verbatim}
# A tibble: 2 x 10
  Prediction_correctness Count Obese mean_CREAT sd_CREAT mean_WT sd_WT mean_AGE
  <chr>                  <int> <int>      <dbl>    <dbl>   <dbl> <dbl>    <dbl>
1 Correct                  232    88      1.00     0.551    84.5  18.6     63.8
2 Incorrect                368   129      0.918    0.594    84.3  18.8     57.5
# i 2 more variables: sd_AGE <dbl>, Proportion <dbl>
\end{verbatim}

\section{Support Vector Machine}\label{support-vector-machine}

Introduced in 1995, Support Vector Machines (SVM) classify data by
finding a hyperplane that maximizes the distance between classes in a
multi-dimensional space. SVM regression predicts a continuous target by
fitting a function with a tolerance for small deviations from the true
values and penalizing large errors. SVM regression can capture
non-linear relationships and is less sensible to outliers compared to
linear regression. Unlike K-Nearest Neighbors (KNN), SVM is effective
for high-dimensional data.

\begin{itemize}
\item
  \textbf{Cost} -- regularization parameter that determines the weight
  given to classification errors. If cost increases, tolerance for
  classification errors decreases. A smaller cost improves
  generalizability.
\item
  \textbf{σ} -- controls the shape of the decision boundary. A smaller
  value captures local trends better but may lead to overfitting and
  reduced generalizability.
\end{itemize}

\begin{Shaded}
\begin{Highlighting}[]
\NormalTok{SVM\_results }\OtherTok{\textless{}{-}}\NormalTok{ openMIPD}\SpecialCharTok{::}\FunctionTok{svm\_train}\NormalTok{(}\AttributeTok{train =}\NormalTok{ train\_preprocessed, }\AttributeTok{continuous\_cov =} \FunctionTok{c}\NormalTok{(}\StringTok{"WT"}\NormalTok{, }\StringTok{"CREAT"}\NormalTok{, }\StringTok{"AGE"}\NormalTok{, }\StringTok{"INF"}\NormalTok{), }\AttributeTok{categorical\_cov =} \FunctionTok{c}\NormalTok{(}\StringTok{"BURN"}\NormalTok{, }\StringTok{"OBESE"}\NormalTok{, }\StringTok{"SEX"}\NormalTok{, }\StringTok{"ICU"}\NormalTok{))}
\end{Highlighting}
\end{Shaded}

\begin{Shaded}
\begin{Highlighting}[]
\NormalTok{SVM\_results}\SpecialCharTok{$}\NormalTok{tune\_plot\_svm }\CommentTok{\# tuning}
\end{Highlighting}
\end{Shaded}

\pandocbounded{\includegraphics[keepaspectratio]{MIPD/Machine_Learning_files/figure-pdf/evaluate-svm-train-1.pdf}}

\begin{Shaded}
\begin{Highlighting}[]
\NormalTok{SVM\_results}\SpecialCharTok{$}\NormalTok{final\_wf\_svm }\CommentTok{\# optimized hyperparameters}
\end{Highlighting}
\end{Shaded}

\begin{verbatim}
== Workflow ====================================================================
Preprocessor: Recipe
Model: svm_rbf()

-- Preprocessor ----------------------------------------------------------------
3 Recipe Steps

* step_mutate_at()
* step_dummy()
* step_normalize()

-- Model -----------------------------------------------------------------------
Radial Basis Function Support Vector Machine Model Specification (regression)

Main Arguments:
  cost = 0.0992125657480125
  rbf_sigma = 0.0774263682681128

Computational engine: kernlab 
\end{verbatim}

\begin{Shaded}
\begin{Highlighting}[]
\NormalTok{final\_svm\_fit }\OtherTok{\textless{}{-}}\NormalTok{ SVM\_results}\SpecialCharTok{$}\NormalTok{final\_svm\_fit}
\end{Highlighting}
\end{Shaded}

\begin{Shaded}
\begin{Highlighting}[]
\NormalTok{test\_SVM }\OtherTok{\textless{}{-}}\NormalTok{ openMIPD}\SpecialCharTok{::}\FunctionTok{svm\_test}\NormalTok{(}\AttributeTok{final\_fit =}\NormalTok{ final\_svm\_fit, }\AttributeTok{test =}\NormalTok{ test\_preprocessed)}
\end{Highlighting}
\end{Shaded}

\begin{Shaded}
\begin{Highlighting}[]
\NormalTok{ta\_results\_SVM }\OtherTok{\textless{}{-}} \FunctionTok{explore\_predictions}\NormalTok{(test\_SVM, }\AttributeTok{DOSE\_PRED =} \StringTok{"SVM"}\NormalTok{)}
\NormalTok{ta\_results\_SVM}\SpecialCharTok{$}\NormalTok{target\_attainment}
\end{Highlighting}
\end{Shaded}

\pandocbounded{\includegraphics[keepaspectratio]{MIPD/Machine_Learning_files/figure-pdf/evaluate-svm-test-1.pdf}}

\begin{Shaded}
\begin{Highlighting}[]
\NormalTok{ta\_results\_SVM}\SpecialCharTok{$}\NormalTok{summary\_stats}
\end{Highlighting}
\end{Shaded}

\begin{verbatim}
# A tibble: 2 x 10
  Prediction_correctness Count Obese mean_CREAT sd_CREAT mean_WT sd_WT mean_AGE
  <chr>                  <int> <int>      <dbl>    <dbl>   <dbl> <dbl>    <dbl>
1 Correct                  233    92      0.997    0.547    85.5  19.5     63.4
2 Incorrect                367   125      0.921    0.597    83.7  18.2     57.7
# i 2 more variables: sd_AGE <dbl>, Proportion <dbl>
\end{verbatim}

\section{K-nearest neighbors (KNN)}\label{k-nearest-neighbors-knn}

KNN is a simple, non-parametric algorithm that is less sensitive to
outliers and is used to create clusters and make predictions based on
the similarity of data points. The disadvantage of KNN is that the
algorithm can take a long time to run, and it is less suited for
high-dimensional data.

\begin{itemize}
\item
  The value of k is determined through 10-fold cross-validation.
\item
  The distance is calculated between all data points and the target
  point.
\item
  The k nearest neighbors are selected.
\item
  The average of the target values of these nearest neighbors will be
  the prediction for the given point.
\end{itemize}

The most commonly used distance metrics are as follows:

\begin{itemize}
\item
  Euclidean Distance: The most commonly used; a straight line connecting
  two points in vector space. This distance is well-suited for outliers
  and noise but less suitable for data with different scales or high
  dimensionality. \[
  d(x, y) = \sqrt{(x - y)^2}
  \]
\item
  Manhattan Distance: Well-suited for categorical data and
  high-dimensional data. It is less suitable for data with different
  scales.
\end{itemize}

\[
d(x, y) = |x - y|
\]

\begin{itemize}
\tightlist
\item
  Minkowski Distance: A generalization of the two previous distances. It
  has an additional parameter, p, that controls the importance given to
  differences between data points. If p = 2, it becomes the Euclidean
  distance, and if p = 1, it becomes the Manhattan distance. This
  distance is suitable for mixed data but is less interpretable, and the
  value of p needs to be defined.
\end{itemize}

\[
d(x, y) = \left| (x - y)^p \right|^{1/p}
\]

Euclidean distance is the default distance used and it is the most
interpretable and intuitive. Hyperparameters to define:

\begin{itemize}
\item
  k
\item
  p (if Minkowski distance is used)
\end{itemize}

\begin{Shaded}
\begin{Highlighting}[]
\NormalTok{KNN\_results }\OtherTok{\textless{}{-}}\NormalTok{ openMIPD}\SpecialCharTok{::}\FunctionTok{knn\_train}\NormalTok{(}\AttributeTok{train =}\NormalTok{ train\_preprocessed, }\AttributeTok{continuous\_cov =} \FunctionTok{c}\NormalTok{(}\StringTok{"WT"}\NormalTok{, }\StringTok{"CREAT"}\NormalTok{, }\StringTok{"AGE"}\NormalTok{, }\StringTok{"INF"}\NormalTok{), }\AttributeTok{categorical\_cov =} \FunctionTok{c}\NormalTok{(}\StringTok{"BURN"}\NormalTok{, }\StringTok{"OBESE"}\NormalTok{, }\StringTok{"SEX"}\NormalTok{, }\StringTok{"ICU"}\NormalTok{))}
\end{Highlighting}
\end{Shaded}

\begin{Shaded}
\begin{Highlighting}[]
\NormalTok{KNN\_results}\SpecialCharTok{$}\NormalTok{tune\_plot\_knn }\CommentTok{\# tuning}
\end{Highlighting}
\end{Shaded}

\pandocbounded{\includegraphics[keepaspectratio]{MIPD/Machine_Learning_files/figure-pdf/evaluate-knn-train-1.pdf}}

\begin{Shaded}
\begin{Highlighting}[]
\NormalTok{KNN\_results}\SpecialCharTok{$}\NormalTok{final\_wf\_knn }\CommentTok{\# optimized hyperparameters}
\end{Highlighting}
\end{Shaded}

\begin{verbatim}
== Workflow ====================================================================
Preprocessor: Recipe
Model: nearest_neighbor()

-- Preprocessor ----------------------------------------------------------------
3 Recipe Steps

* step_mutate_at()
* step_dummy()
* step_normalize()

-- Model -----------------------------------------------------------------------
K-Nearest Neighbor Model Specification (regression)

Main Arguments:
  neighbors = 15

Computational engine: kknn 
\end{verbatim}

\begin{Shaded}
\begin{Highlighting}[]
\NormalTok{final\_knn\_fit }\OtherTok{\textless{}{-}}\NormalTok{ KNN\_results}\SpecialCharTok{$}\NormalTok{final\_knn\_fit}
\end{Highlighting}
\end{Shaded}

\begin{Shaded}
\begin{Highlighting}[]
\NormalTok{test\_KNN }\OtherTok{\textless{}{-}}\NormalTok{ openMIPD}\SpecialCharTok{::}\FunctionTok{knn\_test}\NormalTok{(}\AttributeTok{final\_fit =}\NormalTok{ final\_knn\_fit, }\AttributeTok{test =}\NormalTok{ test\_preprocessed)}
\end{Highlighting}
\end{Shaded}

\begin{Shaded}
\begin{Highlighting}[]
\NormalTok{ta\_results\_KNN }\OtherTok{\textless{}{-}} \FunctionTok{explore\_predictions}\NormalTok{(test\_KNN, }\AttributeTok{DOSE\_PRED =} \StringTok{"KNN"}\NormalTok{)}
\NormalTok{ta\_results\_KNN}\SpecialCharTok{$}\NormalTok{target\_attainment}
\end{Highlighting}
\end{Shaded}

\pandocbounded{\includegraphics[keepaspectratio]{MIPD/Machine_Learning_files/figure-pdf/evaluate-knn-test-1.pdf}}

\begin{Shaded}
\begin{Highlighting}[]
\NormalTok{ta\_results\_KNN}\SpecialCharTok{$}\NormalTok{summary\_stats}
\end{Highlighting}
\end{Shaded}

\begin{verbatim}
# A tibble: 2 x 10
  Prediction_correctness Count Obese mean_CREAT sd_CREAT mean_WT sd_WT mean_AGE
  <chr>                  <int> <int>      <dbl>    <dbl>   <dbl> <dbl>    <dbl>
1 Correct                  227    91      1.00     0.595    85.0  19.6     62.7
2 Incorrect                373   126      0.918    0.567    84.0  18.1     58.2
# i 2 more variables: sd_AGE <dbl>, Proportion <dbl>
\end{verbatim}

\section{Stacking}\label{stacking}

The blend\_predictions function determines how member model output will
ultimately be combined in the final prediction by fitting a Lasso model
on the data stack, predicting the true assessment set outcome using the
predictions from each of the candidate members. Candidates with nonzero
stacking coefficients become members.

These plots the meta-learning model over a predefined grid of lasso
penalty values and uses an internal resampling method to determine the
best value. The autoplot() method, shown helps us understand if the
default penalization method was sufficient. It can also be used to
visualize the contribution of each model type.

\begin{Shaded}
\begin{Highlighting}[]
\NormalTok{amox\_cmin }\OtherTok{\textless{}{-}} \FunctionTok{stacks}\NormalTok{() }\SpecialCharTok{\%\textgreater{}\%}
    \FunctionTok{add\_candidates}\NormalTok{(XGB\_results}\SpecialCharTok{$}\NormalTok{tune\_res\_xgb) }\SpecialCharTok{\%\textgreater{}\%}
    \FunctionTok{add\_candidates}\NormalTok{(SVM\_results}\SpecialCharTok{$}\NormalTok{tune\_res\_svm) }\SpecialCharTok{\%\textgreater{}\%}
    \FunctionTok{add\_candidates}\NormalTok{(RF\_results}\SpecialCharTok{$}\NormalTok{tune\_res\_rf) }\SpecialCharTok{\%\textgreater{}\%}
    \FunctionTok{add\_candidates}\NormalTok{(KNN\_results}\SpecialCharTok{$}\NormalTok{tune\_res\_knn)}
  
\NormalTok{  conflicted}\SpecialCharTok{::}\FunctionTok{conflicts\_prefer}\NormalTok{(brulee}\SpecialCharTok{::}\NormalTok{coef)}
  
  \FunctionTok{set.seed}\NormalTok{(}\DecValTok{1234}\NormalTok{)}
\NormalTok{  amox\_cmin\_ens }\OtherTok{\textless{}{-}}\NormalTok{ amox\_cmin }\SpecialCharTok{\%\textgreater{}\%} \FunctionTok{blend\_predictions}\NormalTok{()}
  
  \CommentTok{\# fit ensembled members}
  \FunctionTok{set.seed}\NormalTok{(}\DecValTok{1234}\NormalTok{)}
\NormalTok{  amox\_cmin\_ens }\OtherTok{\textless{}{-}}\NormalTok{ amox\_cmin\_ens }\SpecialCharTok{\%\textgreater{}\%} \FunctionTok{fit\_members}\NormalTok{()}
\end{Highlighting}
\end{Shaded}

To evaluate training, the evolution of metrics with the number of
members and the contribution of different stacking members are plotted.

\begin{Shaded}
\begin{Highlighting}[]
  \CommentTok{\# Stacking plots}
\NormalTok{  stacking\_autoplot\_default }\OtherTok{\textless{}{-}} \FunctionTok{autoplot}\NormalTok{(amox\_cmin\_ens)}
\NormalTok{  stacking\_members\_plot }\OtherTok{\textless{}{-}} \FunctionTok{autoplot}\NormalTok{(amox\_cmin\_ens, }\AttributeTok{type =} \StringTok{"members"}\NormalTok{)}
\NormalTok{  stacking\_weights\_plot }\OtherTok{\textless{}{-}} \FunctionTok{autoplot}\NormalTok{(amox\_cmin\_ens, }\AttributeTok{type =} \StringTok{"weights"}\NormalTok{)}
  \FunctionTok{ggsave}\NormalTok{(}\AttributeTok{filename =} \FunctionTok{here}\NormalTok{(}\StringTok{"Amoxicillin/a\_priori/For\_publication/Figures/S13.jpg"}\NormalTok{),}
       \AttributeTok{plot =}\NormalTok{ stacking\_weights\_plot,}
       \AttributeTok{width =} \DecValTok{8}\NormalTok{, }\AttributeTok{height =} \DecValTok{6}\NormalTok{, }\AttributeTok{dpi =} \DecValTok{300}\NormalTok{)}
  
\NormalTok{  stacking\_autoplot\_default}
\end{Highlighting}
\end{Shaded}

\pandocbounded{\includegraphics[keepaspectratio]{MIPD/Machine_Learning_files/figure-pdf/stacking-fit-evaluation-1.pdf}}

\begin{Shaded}
\begin{Highlighting}[]
\NormalTok{  stacking\_members\_plot}
\end{Highlighting}
\end{Shaded}

\pandocbounded{\includegraphics[keepaspectratio]{MIPD/Machine_Learning_files/figure-pdf/stacking-fit-evaluation-2.pdf}}

\begin{Shaded}
\begin{Highlighting}[]
\NormalTok{  stacking\_weights\_plot}
\end{Highlighting}
\end{Shaded}

\pandocbounded{\includegraphics[keepaspectratio]{MIPD/Machine_Learning_files/figure-pdf/stacking-fit-evaluation-3.pdf}}

\begin{Shaded}
\begin{Highlighting}[]
  \CommentTok{\# Make predictions with stacking}
\NormalTok{  stack\_preds }\OtherTok{\textless{}{-}} \FunctionTok{predict}\NormalTok{(amox\_cmin\_ens, }\AttributeTok{new\_data =}\NormalTok{ test\_preprocessed) }\SpecialCharTok{\%\textgreater{}\%}
\NormalTok{    dplyr}\SpecialCharTok{::}\FunctionTok{pull}\NormalTok{(.pred)}
\end{Highlighting}
\end{Shaded}

Target attainment and corresponding statistics are calculated for the
predictions.

\begin{Shaded}
\begin{Highlighting}[]
  \CommentTok{\# Add stacking to the test table}
\NormalTok{  test\_STACK }\OtherTok{\textless{}{-}}\NormalTok{ test\_preprocessed }\SpecialCharTok{\%\textgreater{}\%}
\NormalTok{    dplyr}\SpecialCharTok{::}\FunctionTok{mutate}\NormalTok{(}
      \AttributeTok{STACK =} \FunctionTok{exp}\NormalTok{(stack\_preds) }\SpecialCharTok{*}\NormalTok{ (II }\SpecialCharTok{/} \DecValTok{24}\NormalTok{)}
\NormalTok{    )}
  
\NormalTok{ta\_results\_STACK }\OtherTok{\textless{}{-}} \FunctionTok{explore\_predictions}\NormalTok{(test\_STACK, }\AttributeTok{DOSE\_PRED =} \StringTok{"STACK"}\NormalTok{)}
\NormalTok{ta\_results\_STACK}\SpecialCharTok{$}\NormalTok{target\_attainment}
\end{Highlighting}
\end{Shaded}

\pandocbounded{\includegraphics[keepaspectratio]{MIPD/Machine_Learning_files/figure-pdf/stacking-prediction-evaluation-1.pdf}}

\begin{Shaded}
\begin{Highlighting}[]
\NormalTok{ta\_results\_STACK}\SpecialCharTok{$}\NormalTok{summary\_stats}
\end{Highlighting}
\end{Shaded}

\begin{verbatim}
# A tibble: 2 x 10
  Prediction_correctness Count Obese mean_CREAT sd_CREAT mean_WT sd_WT mean_AGE
  <chr>                  <int> <int>      <dbl>    <dbl>   <dbl> <dbl>    <dbl>
1 Correct                  176    51      1.01     0.555    79.8  16.3     67.6
2 Incorrect                424   166      0.926    0.587    86.3  19.3     56.7
# i 2 more variables: sd_AGE <dbl>, Proportion <dbl>
\end{verbatim}

\chapter*{References}\label{references-1}
\addcontentsline{toc}{chapter}{References}

\markboth{References}{References}

\phantomsection\label{refs}
\begin{CSLReferences}{1}{0}
\bibitem[\citeproctext]{ref-Agema2024-cf}
Agema, Bram C, Tolra Kocher, Ayşenur B Öztürk, Eline L Giraud, Nielka P
van Erp, Brenda C M de Winter, Ron H J Mathijssen, Stijn L W Koolen,
Birgit C P Koch, and Sebastiaan D T Sassen. 2024. {``Selecting the Best
Pharmacokinetic Models for a Priori Model-Informed Precision Dosing with
Model Ensembling.''} \emph{Clin. Pharmacokinet.} 63 (10): 1449--61.

\bibitem[\citeproctext]{ref-Carlier2013-bq}
Carlier, Mieke, Michaël Noë, Jan J De Waele, Veronique Stove, Alain G
Verstraete, Jeffrey Lipman, and Jason A Roberts. 2013. {``Population
Pharmacokinetics and Dosing Simulations of Amoxicillin/Clavulanic Acid
in Critically Ill Patients.''} \emph{J. Antimicrob. Chemother.} 68 (11):
2600--2608.

\bibitem[\citeproctext]{ref-fournier2018}
Fournier, Anne, Sylvain Goutelle, Yok-Ai Que, Philippe Eggimann, Olivier
Pantet, Farshid Sadeghipour, Pierre Voirol, and Chantal Csajka. 2018.
{``Population Pharmacokinetic Study of Amoxicillin-Treated Burn Patients
Hospitalized at a Swiss Tertiary-Care Center.''} \emph{Antimicrobial
Agents and Chemotherapy} 62 (9): e00505--18.
\url{https://doi.org/10.1128/AAC.00505-18}.

\bibitem[\citeproctext]{ref-Guilhaumou2019-eh}
Guilhaumou, Romain, Sihem Benaboud, Youssef Bennis, Claire
Dahyot-Fizelier, Eric Dailly, Peggy Gandia, Sylvain Goutelle, et al.
2019. {``Optimization of the Treatment with Beta-Lactam Antibiotics in
Critically Ill Patients-Guidelines from the French Society of
Pharmacology and Therapeutics (Soci{é}t{é} Fran{ç}aise de Pharmacologie
Et {Th{é}rapeutique-SFPT}) and the French Society of Anaesthesia and
Intensive Care Medicine (Soci{é}t{é} Fran{ç}aise d'anesth{é}sie Et
{R{é}animation-SFAR}).''} \emph{Crit. Care} 23 (1): 104.

\bibitem[\citeproctext]{ref-Johnson2023-zo}
Johnson, Alistair E W, Lucas Bulgarelli, Lu Shen, Alvin Gayles, Ayad
Shammout, Steven Horng, Tom J Pollard, et al. 2023. {``{MIMIC-IV}, a
Freely Accessible Electronic Health Record Dataset.''} \emph{Sci. Data}
10 (1): 1.

\bibitem[\citeproctext]{ref-mellon2020}
Mellon, G, K Hammas, C Burdet, X Duval, C Carette, N El-Helali, L
Massias, F Mentré, S Czernichow, and A -C Crémieux. 2020. {``Population
Pharmacokinetics and Dosing Simulations of Amoxicillin in Obese Adults
Receiving Co-Amoxiclav.''} \emph{Journal of Antimicrobial Chemotherapy}
75 (12): 3611--18. \url{https://doi.org/10.1093/jac/dkaa368}.

\bibitem[\citeproctext]{ref-rambaud2020}
Rambaud, Antoine, Benjamin Jean Gaborit, Colin Deschanvres, Paul Le
Turnier, Raphaël Lecomte, Nathalie Asseray-Madani, Anne-Gaëlle Leroy, et
al. 2020. {``Development and Validation of a Dosing Nomogram for
Amoxicillin in Infective Endocarditis.''} \emph{The Journal of
Antimicrobial Chemotherapy} 75 (10): 2941--50.
\url{https://doi.org/10.1093/jac/dkaa232}.

\end{CSLReferences}




\end{document}
